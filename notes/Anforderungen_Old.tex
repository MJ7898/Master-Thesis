 % \begin{table}[hbt!]
    %     \begin{center}
    %         \begin{tabular}{| p{7.0cm} | p{7.0cm} | p{1.4cm} | }
    %             \hline
    %                \textbf{Funktionale Anforderung} & \textbf{Nichtfunktionale Anforderungen} & \textbf{Quelle} \\
    %             \hline
    %                 Bereitstellung eines Konstruktes zur Implementierung eines Regelsets. & Zuverlässigkeit: Durch einen Auslöser gestartete Aktionen sollen bei gleichem Auslöser immer die gleiche Aktion anstoßen. & Ex.-Interv. \\ 
    %             \hline
    %                 MQTT Nachrichten werden verarbeitet und an einen Auslöser weitergeleitet. & Die Steuerzentrale muss eine Verfügbarkeit von 99,9\% vorweisen. & Ex.-Interv. \\ 
    %             \hline
    %                 Implementierung eines MQTT Subscriber and Producer. & Die Kommunikation erfolgt ausschließlich über MQTT. & AW-Fall \\ 
    %             \hline
    %                 Speicherung des Zustandsraumes in eine Datenbanken. & Single Point of Contact (Implementierung, Anpassung und Erweiterung der Regel und der Logik). & Zielgr.-Analyse \\
    %             \hline
    %                 Erweiterungen zu API Anbindungen per MQTT. & Sicherheit: Es werden nur Schnittstellen zur Datenabfrage erlaubt. & Ex.-Interv. \\ 
    %             \hline
    %                 Abbildung von Zustandsinformationen. & Die Steuerzentrale ist nur im eigenen Netz erreichbar. & Ex.-Interv. \\
    %             \hline
    %         \end{tabular}
    %     \end{center}
    %     \caption{Funktionale und Nichtfunktionale Anforderungen}
    %     \label{tab:FAandNFA}
    % \end{table}

        %                 Es soll eine Struktur vorgegeben werden, mit der einfach und formell Regeln und Prozesse implementiert werden können. & Essentiell & Experteninterview \\ 
    %             \hline
    %                 Das Konstrukt von Regeln soll allgemeingültig sein, dass der Entwickler die Logik (Bedingung, Aktion \& Zusätze) implementieren muss. & Essentiell & Experteninterview \\
    %             \hline
    %                 Nachrichten und Auslöser über Nachrichten sollen empfangen und verarbeitet werden können. & Essentiell & Experteninterview, Anwendungsfall \\
    %             \hline
    %                 Es soll eine allgemeingültige Programmiersprache verwendet werden. & Mittel & Experteninterview, Zielgruppenanalyse \\ 
    %             \hline
    %                 Komponenten müssen digital abgebildet werden. & Mittel & Experteninterview, Anwendungsfall \\ 
    %             \hline
    %                 Komponenten und dessen Zustände müssen abgebildet werden. & Hoch & Experteninterview, Anwendungsfall \\ 
    %             \hline
    %                 Aktionen und Regeln sollen nach einem bestimmten Auslöser (Trigger) ausgeführt werden. & Hoch & Experteninterview, Anwendungsfall \\
    %             \hline
    %                 Es sollen Auslöser gegeben sein, bzw. implementiert werden, die definierte Aktionen anstoßen & Essentiell & Experteninterview \\
    %             \hline
    %                 Aktionen sollen parallel ausgeführt werden können. & Mittel & Experteninterview \\ 
    %             \hline
    %                 Aktionen, welche die gleiche Komponente beanspruchen, sollen nacheinander ausgeführt werden können. & Mittel & Experteninterview \\
    %             \hline
    %                 Die Zustände der Komponenten sollten persistiert werden, damit bei einem Systemausfall oder -fehler diese nicht verloren gehen. & Mittel & Experteninterview \\ 
    %
    
     % \\
    % \begin{table}[hbt!]
    %     \begin{center}
    %         \begin{tabular}{| p{10.4cm} | p{1.8cm} | p{3.2cm} | }
    %             \hline
    %                \textbf{Anforderung} & \textbf{Priorität} & \textbf{Quelle} \\
    %             \hline
    %                 Als Programmiersprache soll Java verwendet werden. & Mittel & Experteninterview, Zielgruppenanalyse \\ 
    %             \hline
    %                 Damit Komponenten abgebildet werden können, muss ein Zustandsraum definiert sein. & Essentiell & Experteninterview, Anwendungsfall \\ 
    %             \hline
    %                 Der Zustandsraum muss zur Laufzeit immer zur Verfügung stehen. & Hoch & Experteninterview, Anwendungsfall \\ 
    %             \hline
    %                 Regeln und darauf ausgelöste Aktionen sollen zeitbasiert oder bspw. über MQTT ausgelöst werden. Diese dienen als Auslöser & Hoch & Experteninterview, Anwendungsfall \\
    %             \hline
    %                 Nutzung eines Thread Pools zur parallelen Ausführung von Regeln. & Mittel & Experteninterview \\ 
    %             \hline
    %                 Aktionen, welche die gleiche Komponente beanspruchen, sollen nacheinander ausgeführt werden können. & Mittel & Experteninterview \\
    %             \hline
    %         \end{tabular}
    %     \end{center}
    %     \caption{Konkretere Anforderungen}
    %     \label{tab:concretRequirements}
    % \end{table}
    % \\
