\section*{Use Case 2 - Notfall-Evakuierung}
\begin{table}[hbt!]
    \begin{center}
        \begin{tabular}{| p{3cm} | p{12.75cm} | }
            \hline
                \textbf{Task: Begrüßung / Check-In von Member und Gästen} \\
            \hline
                \textbf{Ziel:} & Einen Member (Mitarbeiter) oder einen Gast in der Lokation mit einem Service-Roboter zu begrüßen, wenn dieser eintreten möchte. Nach der Begrüßung findet der Check-In statt, bei dem der Member seinem Platz zugewiesen und der Gast zu seinem Ansprechpartner geleitet wird. \\
            \hline
                \textbf{Eingriffsmöglichkeiten:} & Auswahl der Situation, Interaktion mit dem Service-Roboter (Informieren, Begleiten)  \\
            \hline
                \textbf{Ursachen:} & Terminbuchung, Anmeldung, Klingeln an der Tür, Authentifizierung / Authentisierung an der Tür mittels Biometrische Eingangskontrolle \\
            \hline
                \textbf{Priorität:} & Hoch \\
            \hline
                \textbf{Durchführungsprofil:} & Nach Anmeldung oder Eintritt eines Gastes oder Members, Unterbrechungen möglich (Verbindungsfehler, Fehlerhafte Abfrage der Verfügbarkeiten von Platzreservierungen), mittlere Komplexität \\ 
            \hline
                \textbf{Vorbedingung:} & Platzbuchung / Reservierung und Einbuchung für ein Meeting fand im Vorfeld statt, Einverständniserklärung von den Personen liegt vor, Authentisierung über die Eingangskontrolle hat funktioniert / stattgefunden \\
            \hline 
                \textbf{Info-In:} & Die Person, die eintreten will, ist durch die Buchung oder Authentisierung bekannt, Buchung ist registriert und kann abgerufen werden \\
            \hline
                \textbf{Info-Out:} & Begrüßung Standardprotokoll mit Interaktionsmöglichkeiten und Kenntnis des Namens der Person, Check-In Informationen für die einzucheckende Person \\
            \hline
                \textbf{Ressourcen:} & Service-Roboter, Eintrittskontrolle per Biometrischer Authentisierung, Buchung der Person abrufbar über die API \\
            \hline
        \end{tabular}
    \end{center}
    \caption{Aufgabenbeschreibung UC 2 - Notfall-Evakuierung}
    \label{tab:evacuation}
\end{table}

\subsection*{UC 2 - Beschreibung}
    Steuerzentrale:
    \begin{itemize}
        \item Input: MQTT Nachricht angestoßen von einem Member (Aktion; Uhrzeit)
        \item Input: DoorJames API - Gebuchte Büroplätze und Personen die einen Termin vor Ort registriert haben
        \item -	Output: Zu durchlaufender Prozess (Evakuierung) und Name des Gastes / Members und Position der Plätze und Räume, die gebucht wurden. 
    \end{itemize}
	
    Service-Roboter: 
    \begin{itemize}
        \item Input: MQTT (Prozess-Trigger durch Steuerzentrale und Position gebuchter Plätze und Räume
        \item Input: Objekterkennung (Personenerkennung) durch die Kamera
        \item Output: (An Steuerzentrale) – Status / Prozess (erfolgreich/nicht erfolgreich) abgeschlossen
        \item -> Event-Möglichkeiten werden von Service-Roboter direkt verarbeitet. (Interaktion mit Person, Anweisungen an die Person) 
        \item -> Event-Möglichkeiten: 
        \item --> Kunde, Gast & Member: 
        \item ---> Alarmieren und bitten das Gebäude auf schnellstem Weg zu
    \end{itemize}

\subsection*{UC 2 - User Story}
    Als Gast, Member und Manager möchte ich bei einem Notfall alarmiert werden. Bei eintretendem Fall möchte ich 
    Anweisungen, die zu beachten und umzusetzen sind, erhalten. Ebenso ist ein Hinweis zu den Notausgängen sinnvoll 
    und hilft bei der Flucht, bzw. beim Verlassen des Gebäudes. Damit keine Personen im Büro bleiben, ist eine 
    Information über Anwesende von belangen. Der Member sollte die Möglichkeit haben, den Service-Roboter 
    aufzufordern nach Leuten zu rufen, die nicht in Sichtweite sind. Sobald alle Plätze abgefahren sind, bzw. 
    durch den Service-Roboter keine Personen mehr erkannt werden, soll eine Benachrichtigung erfolgen, dass alle 
    aus dem Gebäude sind, bzw. keine mehr gesichtet wurden. Diese Information hilft bei der anschließenden Kontrolle 
    über am Treffpunkt anwesender Personen. Abschließend soll ggf. Eine Information über die Platzbuchungen gezeigt 
    werden, damit bei der Kontrolle keine Fehler unterlaufen und keine Personen, die nicht gesehen werden, vergessen 
    werden. 