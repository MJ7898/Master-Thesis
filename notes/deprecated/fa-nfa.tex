\section{Funktionale Anforderungen (FA}
\label{sec:functionalrequirements}
%Definition des Systems, bzw. einer Systemkomponente, von bereitzustellenden Funktionen oder Diensten.
    \begin{itemize}
        \item Funktionen
        \item Geschäftsregeln
        \item Daten 
        \item Zustände
        \item Fehlerbehandlung
        \item Schnittstellen
    \end{itemize}

\section{Nichtfunktionale Anforderungen (NFA) - Qualitätsanforderungen}
\label{sec:nonfunctionalrequirements}
%Definition einer qualitativen Eigenschaft des Systems, bzw. einer Systemkomponente oder einer Funktion.
    \begin{itemize}
        \item Details zu Funktionen
        \item Sicherheit, Genauigkeit
        \item Zuverlässigkeit 
        \SubItem Vorhersagbarkeit (Predictability)
        \subitem Das System liefert bei identischen Eingaben immer das gleiche Ergebnis.
        \SubItem Robustheit (Robustness)
        \subitem Das System reagiert auf fehlerhafte oder ungenügende Eingaben mit einer definierten Antwort.
        \item Benutzbarkeit 
        \item Effizienz 
        \item Änderbarkeit 
        \SubItem Konfigurierbarkeit  
        \item Übertragbarkeit 
        \item Erweiterbarkeit
    \end{itemize}

\section{Rahmenbedingungen}
\label{sec:framingconditions}
% Organisatorische oder technologische Anforderungen, die die Art und Weise einschränken wie ein Produkt entwickelt wird. 
    \begin{itemize}
        \item Entwicklungsprozess 
        \item Budget 
        \item Termine
        \item Team 
        \item Gesetze, Normen, Standards
        \item Guidelines
        \item Betrieb
    \end{itemize}

\section{Definition of Done (DoD)}
\label{sec:definitionofdone}
    % Definition of Done (DoD) 