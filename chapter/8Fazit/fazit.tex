\chapter{Fazit}
\label{chap:fazit}
    Ziel dieser Arbeit war die Konzeption und prototypische Umsetzung einer Steuerzentrale für ein 
    smartes Büro mit dem Fokus einer einfachen Handhabung der formalisierten Interaktionen für Softwareentwickler. 
    Das in diesem Rahmen entstandene Framework bietet eine klare Struktur zur Implementierung von Anwendungsfällen, die der Anwender 
    im Umfeld eines smarten Büros umsetzen möchte. Durch den rein programmatischen Ansatz ist somit 
    dem Softwareentwickler ein System gegeben, das ausschließlich mit Java bedient wird. Verwendete 
    Entwurfsmuster schaffen eine wartbare, erweiterbare und modulare Architektur. Durch die vorgegebene Schablone der 
    Regel ist dem Entwickler zwar eine Struktur vorgeschrieben, die dennoch in der 
    Ausprägung des jeweiligen Anwendungsfalls alle Handlungsmöglichkeiten und Freiheiten offen lässt. 
    \\
    \linebreak 
    Für die Erreichung des Ziels wurden unter anderem Recherchen und Analysen durchgeführt, die den Stand 
    der Technik hervorbrachten, gefolgt von der Testung derzeitiger Softwarelösungen, um deren Möglichkeiten und Grenzen zu erfahren. 
    Auf dieser Grundlage wurden weitere Schritte eingeleitet, damit gezielte Anforderungen definiert werden konnten. Hierfür wurde 
    eine Markt- sowie Zielgruppenanalyse vorgenommen und konkrete Anwendungsfälle definiert. Mit diesen Hintergründen wurden zur 
    Auffassung möglichst vieler Anforderungen zusätzlich Experteninterviews durchgeführt, wodurch die Konzeption und Umsetzung 
    realisiert werden konnte. 
    \\
    \linebreak
    Unter Berücksichtigung der zu gewährleistenden Modularisierung, Erweiterbarkeit und Wartbarkeit wurden dementsprechende 
    Architekturentwurfsmuster verwendet. Zudem wurden einzelne Schichten des Systems entworfen, darunter zählt die Kommunikations-, sowie 
    Logik- und Persistenzschicht, wobei letztere nicht konkret konzipiert wurde.     
    \\
    Ein konkretes Konstrukt für die Regeldefinition wurde verwendet, damit der Anwender eine klare Struktur vorfindet und so die 
    einfache Handhabung der formalisierten Interaktionen gefördert wird. Darüber hinaus wurde eine Annotation ergänzt, die eine Regel als solche 
    deutlich markiert.
    \\
    \linebreak
    Zur Überprüfung und Bestätigung der Anforderungen, sowie des gestellten Ziels wurden anschließend für die Umsetzung Usability-Tests und zusätzliche 
    Experteninterviews durchgeführt. Diese halfen dabei, den erarbeiteten Prototypen und dessen Konzept zu evaluieren. Alle priorisierten 
    Anforderungen konnten erfüllt werden. Das Ergebnis der Usability-Tests, sowie der Interviews ergaben ein positives Feedback. Dennoch 
    gibt es Verbesserungs-, bzw. Anpassungspotentiale, die im Folgenden thematisiert werden. Erweiterungen und mögliche Weiterentwicklungen werden 
    ebenso im Ausblick (siehe Kapitel \ref{chap:ausblick}) aufgegriffen. 
    \\
    \linebreak 
    Zum aktuellen Zeitpunkt ist die Verwendung von Komponenten nicht vorgesehen, da diese im Zustandsraum derzeit keine Verwendung haben. Die 
    Gründe dafür sind unter anderem die Entscheidung der Verwendung der Java Reflection, sowie die aktuelle inverse Transformation, die jedoch in weiteren 
    Iterationen angepasst werden kann. Zusätzlich gibt es mögliche Optimierungen der Regelausführung, bzw. deren Management, um die Performanz 
    zu steigern und eine strukturierte Abfolge zu gewährleisten.  
    \\
    %\linebreak
    Dennoch liefert das Konzept, sowie der aktuelle Prototyp alle Grundvoraussetzungen, um das gewünschte Ziel zu 
    erreichen, bzw. Prozesse im intelligenten Büro zu automatisieren. Unter Berücksichtigung der Modellierungsgrenzen, als 
    auch der -vorgaben ist die Realisierung von komplexen Anwendungsfällen, bspw. die Verwendung eines 
    Service-Roboters, ohne weiteres möglich. 
    \\
    \linebreak
    Das Ziel dieser Master-Thesis, eine Steuerzentrale für ein intelligentes Büro zu konzipieren und entwickeln, welches den Fokus der 
    einfachen Handhabung der formalisierten Interaktionen für Softwareentwickler besitzt, wurde demnach erreicht: das hier erarbeitete 
    und vorgestellte Framework optimiert die Usability in Form der Programmierung und ermöglicht das Umsetzen und Ausführen von Regeln. 
