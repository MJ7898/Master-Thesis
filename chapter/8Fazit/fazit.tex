\chapter{Fazit}
\label{chap:fazit}
    Ziel dieser Arbeit war die Konzeption und prototypische Umsetzung einer Steuerzentrale für ein 
    smartes Büro mit dem Fokus einer einfachen Handhabung der formalisierten Interaktionen für Softwareentwickler. 
    Das in diesem Rahmen entstandene Framework bietet eine klare Struktur zur Implementierung von Anwendungsfällen, die 
    im Umfeld eines smarten Büros umgesetzt werden können. Durch den rein programmatischen Ansatz ist dem Anwender, somit 
    dem Softwareentwickler, ein System gegeben, dass ausschließlich mit Java bedient wird. Mithilfe von verwendeten 
    Softwarearchitekturmustern kann eine Ordnung geschaffen werden, die dennoch dem Entwickler in der 
    Ausprägung des jeweiligen Anwendungsfalls alle Freiheiten offen lässt. 
    \\
    \linebreak 
    Für die Erreichung des Ziels wurden unter anderem tiefgreifende Recherchen und Analysen durchgeführt, die den Stand 
    der Technik hervorbrachten, gefolgt von der Testung derzeitiger Softwarelösungen, um deren Möglichkeiten und Grenzen zu erfahren. 
    Auf dieser Grundlage wurden weitere Schritte eingeleitet, damit gezielte Anforderungen definiert werden konnten. Hierfür wurde 
    eine Markt- sowie Zielgruppenanalyse durchgeführt und konkrete Anwendungsfälle definiert. Mit diesen Hintergründen wurden zur 
    Auffassung möglichst vieler Anforderungen zusätzlich Experteninterviews durchgeführt, wodurch die Konzeption und Umsetzung 
    durchgeführt werden konnte. 
    \\
    \linebreak
    Unter Berücksichtigung der zu gewährleistenden Modularisierung, Erweiterbarkeit und Wartbarkeit wurden dementsprechende 
    Architekturentwurfsmuster verwendet. Zudem wurden einzelne Schichten des Systems entworfen, darunter zählt die Kommunikations-, sowie 
    Logik und Persistenzschicht, wobei letztere nicht konkret konzipiert wurde.     
    \\
    Damit der Anwender stets eine klare Struktur zur Regeldefinition auffindet und so die einfache Handhabung der formalisierten Interaktionen gefördert wird, wurde 
    ein konkretes Konstrukt verwendet, das diesen Leitfaden vorgibt. Darüber hinaus wurde der Regeldefinition eine Annotation ergänzt, die eine Regel als solche 
    deutlich markiert.
    \\
    \linebreak
    Zur Überprüfung und Bestätigung der Anforderungen, sowie des gestellten Ziels wurden anschließend zu der Umsetzung Usability-Tests und zusätzliche 
    Experteninterviews durchgeführt. Diese halfen dabei, den erarbeiteten Prototypen und dessen Konzept zu evaluieren. Alle hochpriorisierten 
    Anforderungen konnten erfüllt werden und das Ergebnis der Usability-Tests, sowie der Interviews ergaben ein überaus positives Feedback. Dennoch 
    gibt es Verbesserungs-, bzw. Anpassungspotentiale, die in folgendem thematisiert werden. Erweiterungen und mögliche Weiterentwicklungen werden 
    im Ausblick (siehe Kapitel \ref{chap:ausblick}) aufgegriffen. 
    \\
    \linebreak 
    Im aktuellen Stand des Konzeptes ist die Verwendung von Komponenten nicht notwendig, da diese im Zustandsraum derzeit keine Verwendung haben. Die 
    Gründe dafür ist die Entscheidung der Verwendung der Java Reflection, sowie die aktuelle inverse Transformation, die jedoch in weiteren 
    Iterationen angepasst werden kann. Des weiteren gibt es mögliche Optimierungen der Regelausführung, bzw. dessen Management, um die Performanz 
    zu steigern und eine strukturierte Abfolge zu gewährleisten.  
    \\
    %\linebreak
    Dennoch liefert das aktuelle Konzept, sowie der aktuelle Prototyp alle Grundvoraussetzungen, um das gewünschte Ziel zu 
    erreichen, bzw. Prozesse im intelligenten Büro zu automatisieren. Mit der Berücksichtigung der Modellierungsgrenzen, als 
    auch den -vorgaben ist die Realisierung von komplexen Anwendungsfällen, bspw. die Verwendung eines 
    Service-Roboters zur Interaktion mit Menschen, ohne weiteres möglich. 
    \\
    \linebreak
    Das Ziel dieser Master-Thesis, eine Steuerzentrale für ein intelligentes Büro zu konzipieren und entwickeln, welches den Fokus der 
    einfachen Handhabung der formalisierten Interaktionen für Softwareentwickler besitzt, wurde demnach erreicht: das hier erarbeitete 
    und vorgestellte Framework optimiert die Usability in Form der Programmierung und ermöglicht das Umsetzen und Ausführen von Regeln. 
