\chapter{Ausblick}
\label{chap:ausblick}
    Das Framework liefert aus Sicht der Usability ausgezeichnete Ergebnisse und gibt dem Anwender die zu implementierenden Elemente klar vor. 
    Vorausgesetzt werden hierbei jedoch Grundkenntnisse der Programmierung, sowie die Kenntnis über die zu implementierenden Elemente des Systems. 
    Zukünftige Weiterentwicklungen des im Rahmen dieser Arbeit vorgestellten Frameworks sollten demnach zum Ziel haben, das entworfene Konzept, 
    sowie den aktuellen Prototypen entsprechend zu erweitern. 
    \\
    \linebreak
    Für die Optimierung des Systems bietet primär der Algorithmus der Regelprüfung und -ausführung Potential. Dieser ließe sich beispielsweise 
    um einen \textit{Queuing}-Mechanismus ergänzen, damit alle eintreffenden Zustandsänderungen berücksichtigt und nacheinander abgearbeitet werden. 
    Kann eine Aktion auf Grundlage der Änderung nicht durchgeführt werden, würde sie zurückgehalten und die Bedingung immer wieder überprüft 
    werden, bis diese zutrifft und die Zustandsänderung eine entsprechende Aktion ausführt, sofern dies beabsichtigt ist. 
    \\
    \linebreak
    Des weiteren kann das derzeit statische \textit{JSON}-Konstrukt der inversen Transformation so optimiert werden, dass beliebige Konstrukte 
    als \acs{MQTT}-Nutzlast veröffentlicht werden können. Dadurch wäre eine Versendung mehrerer Informationen möglich. 
    \\
    Denkbar wäre eine Optimierung des Kopieralgorithmus, sodass verwendete Komponenten im Zustandsraum nach dem Kopiervorgang keine neue 
    Objektreferenz zur Programmlaufzeit erhalten. In Kombination dieser Schritte wäre nachfolgend eine Verwendung von Komponenten im Zustandsraum denkbar, wobei 
    dies nach wie vor keine konkreten Vorteile aufweist. Dadurch würde lediglich die Gestaltung des Zustandsraumes auf die Komponenten verlagert werden. 
    \\
    Die Java Reflection müsste dahingehend geringfügig angepasst werden, sodass Überprüfungen im Stile des absteigenden Baumgraphs entlang der Pfade stattfindet.
    \\
    \linebreak
    Neben den inhaltlichen Optimierungen gibt es ebenso Erweiterungsmöglichkeiten, um die formalisierten Interaktionen der Softwareentwickler zusätzlich zu vereinfachen und 
    weitere Funktionalitäten dem Framework zu übergeben. 
    \\
    Damit der Softwareentwickler zu den Annotationen und der vorgegebenen Regelstruktur dessen Richtigkeit überprüfen kann, wäre ein Debug-Mechanismus für die implementierten Regeln sinnvoll. 
    Mithilfe diesem könnte zusätzlich die Syntax und zu verwendende Struktur der Regel sichergestellt werden. Infolgedessen wäre ein Ansatz für ein Testframework zu konzipieren, das 
    ergänzend zu JUnit-Tests die Funktionalitäten des Systems überprüft. 
    \\
    \linebreak
    Eine weitere Maßnahme, mit der die formalisierten Interaktionen der Softwareentwickler künftig noch einfacher zu handhaben sind, wäre die Entwicklung eines \ac{CLI}s. Dadurch könnten mithilfe 
    dynamischer Quellcode-Generierungen die einzelnen Komponenten, darunter Attribute im Zustandsraum, leere und befüllte Transformationsobjekte und Regelklassen, erstellt werden. Als Grundlage dafür 
    würde beispielsweise das \acs{CLI} von \textit{Apache Maven}\footnote{Kommandozeilenwerkzeug aus der Kategorie der Build-Werkzeuge. \url{https://maven.apache.org/index.html} Besucht am 12.08.2022} 
    dienen. Über einen einfachen Befehl könnten alle Elemente, die für eine Regel notwendig sind, erstellt werden. Es entfiele z.B. die 
    eigenständige Erstellung der Regel. Wobei derzeit zu beachten ist, dass die richtige abstrakte Klasse verwendet wird und alle zu implementierenden Funktionen präsent sind. 
    \\
    Alternativ wäre die Entwicklung eines \ac{IDE} Plug-ins möglich, anhand dessen das Regelkonstrukt erzeugt werden kann. 
    \\
    \linebreak
    Zum stetigen Ausbau des Frameworks wäre eine Erweiterung zur Nutzung zusätzlicher Kommunikationsprotokolle durchaus denkbar. Die notwendigen Vorkehrungen sind durch die Architektur gegeben. 
    Ähnlich zu Home Assistant wäre die Bereitstellung einer \acs{MQTT}-Broker-Instanz möglich, sodass das Framework autark und unabhängiger verwendet werden kann, bzw. das Management der grundlegenden Funktionen 
    gebündelt wird. 
    \\
    \linebreak
    Zur Steigerung der Integrität und Konsistenz des Zustandsraumes kann mittels der Persistenzschicht beispielsweise eine Transaktionshistorie erstellt werden, die jede Zustandsänderung als Datenmodell, bzw. Objekt in eine 
    Datenbank speichert. So kann nach einem Ausfall der konkrete Zustandsraum wieder eingespielt und die eigentlichen Zustände verwendet werden. Aktuell müssten alle Geräte ihren derzeitigen Zustand erneut übermitteln, wodurch 
    die Steuerzentrale diese nochmals registrieren könnte. Durch das Einspielen einer Transaktion würden alle Zustände im Zustandsraum wieder vorhanden sein. Die Ausprägung dieser Schicht ist für weitere Entwicklungen offengehalten. 
    \\
    \linebreak
    Für die Fähigkeit der Überwachung von aktuell ablaufenden Regeln könnte ein sogenanntes \textit{Monitoring} entwickelt und beliebig ausgeprägt werden, sodass auch Prognosen erhoben und Interaktionspunkte geschaffen 
    werden könnten. Über einen separaten Monitor könnte beispielsweise im Büro angezeigt werden, welche Regel durchlaufen wird, bzw. wie die Zustände im Zustandsraum aussehen. 
    \\
    \linebreak
    Der Ansatz bietet allgemein eine Möglichkeit, in einer Umgebung neue Automationen und Prozesse schnellstmöglich umzusetzen. Durch das stetige Wachstum und die Akzeptanz des Bereiches \acl{SH} und dessen Spezifikation des 
    \textit{Smart Office} wäre das Vorantreiben dieser Idee durch ein Open-Source-Ansatz möglich. Die geschaffene Grundlage ermöglicht die Umsetzung aller Optimierungs-, sowie Erweiterungsmöglichkeiten. 
