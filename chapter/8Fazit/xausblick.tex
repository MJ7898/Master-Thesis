\chapter{Ausblick}
\label{chap:ausblick}
    Das Framework liefert aus Sicht der Usability gute Ergebnisse und gibt dem Anwender die zu implementierenden Elemente klar vor. 
    Vorausgesetzt werden hierbei jedoch Grundkenntnisse der Programmierung, sowie die Kenntnis über die zu implementierenden Elemente des Systems. 
    Zukünftige Weiterentwicklungen des im Rahmen dieser Arbeit vorgestellten Frameworks sollten demnach zum Ziel haben, das entworfene Konzept, 
    sowie den aktuellen Prototypen entsprechend zu erweitern. 
    \\
    \linebreak
    Für die Optimierung des Systems bietet primär der Algorithmus der Regelprüfung und -ausführung Potential. Dieser ließe sich beispielsweise 
    gut um einen \textit{Queuing}-Mechanismus ergänzen, damit alle eintreffenden Zustandsänderungen berücksichtigt werden, die zum 
    Zeitpunkt der Bedingungsüberprüfung eventuell nicht zutreffen. Diese würden dann zurückgehalten und immer wieder überprüft werden, bis eine 
    Bedingung zutrifft und die Zustandsänderung eine entsprechende Aktion ausführt. Sofern dies beabsichtigt ist. 
    \\
    \linebreak
    Des weiteren kann das derzeit statische \textit{JSON}-Konstrukt der inversen Transformation so optimiert werden, dass beliebige Konstrukte 
    als \acs{MQTT}-Nutzlast veröffentlicht werden können. Dadurch wäre eine Versendung mehrerer Informationen möglich. 
    \\
    Im Zuge dessen wäre eine Optimierung des Kopieralgorithmus angebracht, sodass verwendete Komponenten im Zustandsraum nach dem Kopiervorgang keine neue 
    Objektreferenz zur Programmlaufzeit erhalten. In Kombination dieser Schritte wäre nachfolgend eine Verwendung von Komponenten im Zustandsraum möglich, wobei 
    dies nach wie vor keine konkreten Vorteile aufweist. Dadurch würde lediglich die Gestaltung des Zustandsraumes auf die Komponenten verlagert werden. 
    \\
    Die Java Reflection müsste dahingehen geringfügig angepasst werden, sodass Überprüfungen im Stile des abfolgenden Baumgraphs, entlang der Pfade stattfindet.
    \\
    \linebreak
    Neben den inhaltlichen Optimierungen gibt es ebenso Erweiterungsmöglichkeiten, um die formalisierten Interaktionen der Softwareentwickler zusätzlich zu vereinfachen und 
    weitere Funktionalitäten dem Framework zu übergeben. Diese sind im Rahmen von Ergänzungen anzusehen.
    \\
    Damit der Softwareentwickler zu den Annotationen und der vorgegebenen Regelstruktur dessen Richtigkeit überprüfen kann, wäre ein Debug-Mechanismus für die implementierten Regeln sinnvoll. 
    Mithilfe dessen könnte zusätzlich die Syntax und zu verwendende Struktur der Regel sichergestellt werden. In Folge dessen könnte auch ein Ansatz konzipiert werden, wie ein Testframework 
    auf das System angewendet werden könnte, um die Funktionalität ergänzend zu den JUnit-Tests zu überprüfen. 
    \\
    \linebreak
    Eine weitere Maßnahme mit der die formalisierten Interaktionen der Softwareentwickler künftig noch einfacher zu handhaben sind, wäre die Entwicklung eines \ac{CLI}. Dadurch könnten mithilfe 
    von dynamischer Quellcode Generierung die einzelnen Komponenten für eine Regel, darunter Attribut im Zustandsraum, leeres Transformationsobjekt und Regelklasse, erstellt werden. Als Grundlage dafür 
    würde beispielsweise das \acs{CLI} von \textit{Apache Maven}\footnote{Kommandozeilenwerkzeug aus der Kategorie der Build-Werkzeuge. \url{https://maven.apache.org/index.html} Besucht am 12.08.2022} 
    dienen. Dadurch könnte über einen einfachen Befehl alle Elemente, die für eine Regel notwendig sind erstellt werden. Dadurch entfiele z.B. die 
    eigenständige Erstellung der Regel, wobei zu beachten ist, dass die richtige abstrakte Klasse verwendet wird und alle zu implementierenden Funktionen präsent sind. 
    \\
    Alternativ wäre die Entwicklung eines \ac{IDE} Plug-ins möglich, mithilfe dessen das Regelkonstrukt darüber erzeugt werden kann. 
    \\
    \linebreak
    Zum stetigen Ausbau des Frameworks wäre eine Erweiterung zur Nutzung zusätzlicher Kommunikationsprotokolle durchaus denkbar. Die notwendigen Vorkehrungen sind durch die Architektur durchaus gegeben. 
    Ähnlich zu Home Assistant wäre die Bereitstellung einer \acs{MQTT}-Broker-Instanz möglich, sodass das Framework autark und unabhängiger verwendet werden kann, bzw. das Management der grundlegenden Funktionen 
    gebündelt wird. 
    \\
    \linebreak
    Zur Steigerung der Integrität und Konsistenz des Zustandsraumes, kann mittels der Persistenzschicht beispielsweise eine Transaktionshitsorie erstellt werden, die jede Zustandsänderung als Datenmodell, bzw. Objekt in eine 
    Datenbank speichert. So kann nach einem Ausfall der konkrete Zustandsraum wieder eingespielt und die eigentlichen Zustände verwendet werden. Aktuell müssten alle Geräte ihren derzeitigen Zustand erneut übermitteln, wodurch 
    die Steuerzentrale diese erneut registrieren kann. Durch das Einspielen der letzten Transaktion würden alle Zustände wieder vorhanden sein. Die Ausprägung dieser Schicht ist für weitere Entwicklungen offengehalten. 
    \\
    \linebreak
    Für die Fähigkeit der Überwachung von aktuell ablaufenden Regeln könnte ein sogenanntes \textit{Monitoring} entwickelt und beliebig ausgeprägt werden, sodass auch Prognosen erhoben und Interaktionspunkte geschaffen 
    werden können. Hierbei könnte beispielsweise über einen separaten Monitor im Büro angezeigt werden, welche Regel durchlaufen wird, bzw. wie die Zustände im Zustandsraum aussehen. 
    \\
    \linebreak
    Der Ansatz bietet allgemein eine Möglichkeit, um in einer Umgebung neue Automationen und Prozesse schnellstmöglich umzusetzen. Durch das stetige Wachstum und die Akzeptanz des Bereiches \acl{SH} und dessen Spezifikation des 
    \textit{Smart Office} wäre das Vorantreiben dieser Idee durch ein Open-Source-Ansatz möglich. Durch die geschaffene Grundlage sind alle Optimierungs-, sowie Erweiterungsmöglichkeiten denkbar und ebenso umsetzbar. 
    %Monitoring
    %Open-source Projekt
    
    %Ergänzungen, bspw. 
        %Regeldebugger (Anwendung eines Testframeworks), 
        %Code Generator (CLI like Angular CLI), 
        %IDE Plugin zur Erstellung von Regelkonstrukten, 
        %Weitere KommunikationsProtokolle, Bereitstellung eigener Kommunikationskomponenten?
        % Transaktionshistorie
        %Monitoring von Prozessen und Regeln, um den Status des Systems überblicken zu können. (Welche Regel wird gerade ausgeführt und wie lange braucht diese?)
    % Optimierungen: 
        %Verbesserung der Regeldurchführung durch Queueing, 
        %inverse Transformation um zu sendendes JSON Konstrukt dynamisch zu erstellen je nach Bedarf, 
        %Nutzung von Komponenten im Zustandsraum 
    % Das Framework als open-source-Projekt pushen und anbieten


% Nutzung von Spring Dynamic Modules for OSGi Service Platforms -> https://de.wikipedia.org/wiki/Spring_(Framework)#Vergleich 
% -> Für die Realisierung eines weitreichenderen Frameworks, bzw. Applikation. 

% -> Framework als open-source projekt anbieten 