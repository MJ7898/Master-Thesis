\section{Smart Home}
\label{sec:smartHome}
\acl{SH} ist einer von mehreren bekannten Zweigen des \acs{IoT}s. Speziell diese Rubrik widmet sich explizit 
sämtlichen Haushaltsgeräten und -einrichtungen. Aus diesem Grund können viele Geräte ebenso intelligente Gegenstände 
der Rubrik \acl{SH} sein. Ein kleiner Ausschnitt solcher Nutzgegenstände sind unter anderem Lampe, Kontaktsensoren, 
Thermostate, Service-Roboter, Staubsauger-Roboter, Kühlschränke Geräte rundum die Haussicherheit. 
\\ 
Unter dem Oberbegriff \acl{SH} ist auch eine Weise zu verstehen, mit der die Erhöhung der Wohn- und Lebensqualität, 
effizienteren Energienutzung unter Verwendung vernetzter und fernsteuerbarer Geräten, Sicherheit sowie automatisierbaren 
Abläufe gesteigert werden kann. 
\\ 
Der Begriff intelligenten Zuhause wird auch schon verwendet, wenn die Haustechnik und Haushaltsgeräte unter einander  
vernetzt sind. Die Definition im Deutschen Gebrauch, welche nach (Strese et al. 2010) in der Untersuchung im Rahmen 
der wissenschaftlichen Begleitung zum Programm Next Generation Media (NGM) des Bundesministeriums für Wirtschaft und 
Technologie aufgegriffen wird, lautet wie folgt: 
\begin{quote}
    „Das Smart Home ist ein privat genutztes Heim (z.B. Eigenheim, Mietwohnung), in dem die zahlreichen Geräte der 
    Hausautomation (wie Heizung, Beleuchtung, Belüftung), Haushaltstechnik (wie z.B. Kühlschrank, Waschmaschine), 
    Konsumelektronik und Kommunikationseinrichtungen zu intelligenten Gegenständen werden, die sich an den 
    Bedürfnissen der Bewohner orientieren. Durch Vernetzung dieser Gegenstände untereinander können neue 
    Assistenzfunktionen und Dienste zum Nutzen des Bewohners bereitgestellt werden und einen Mehrwert 
    generieren, der über den einzelnen Nutzen der im Haus vorhandenen Anwendungen hinausgeht.“ \cite{strese.2010m}
\end{quote}
Eine vergleichbare Definition wurde zu späterem Zeitpunkt durch eine Literaturrecherche publiziert. Diese beschreibt 
die zugrundeliegende Thematik weniger aus Anwendersicht sonder widmet sich vielmehr dem System und der Technologie selbst. 

