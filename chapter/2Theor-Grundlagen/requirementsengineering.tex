\section{Requirements Engineering}
\label{sec:requirementsengineering}
    In der heutigen Softwareentwicklung ist zu Anfang jedes Projekts die Frage offen, welche Anforderungen soll das 
    Produkt erfüllen, sodass daraus eine gute, stabil und effiziente Software entsteht und der Umfang und das Ziel 
    klar gestaltet sind. Mit dem \acl{RE} wird der Gedanke verfolgt, die Bedürfnisse des Kunden, bzw. des Stakeholders 
    zu erfüllen und eine erfolgreiche, den Kunden zufriedenstellende Entwicklung des Systems zu erzielen. Ein Stakeholder 
    ist eine Person oder Organisation mit Einfluss auf die Anforderungen des Systems oder die Auswirkungen auf das System 
    hat \cite{pohl2021basiswissen}.
    Diese Anforderungen werden durch das \acl{RE} ermittelt und dokumentiert. 
    \\
    \linebreak
    Unter dem Begriff \ac{RE} ist folgendes zu verstehen: 
    \begin{quote}
        Das Requirements Engineering ist ein systematischer und disziplinierter Ansatz zur Spezifikation und zum 
        Management von Anforderungen mit dem Ziel, die Wünsche und Bedürfnisse der Stakeholder zu verstehen und 
        die Gefahr zu minimieren, ein System auszuliefern, das diese Wünsche und Bedürfnisse nicht erfüllt. \cite{pohl2021basiswissen}
    \end{quote}
    Oft wird darunter auch ein kooperativer und inkrementeller Prozess verstanden, dessen Ziele die Gewährleistung 
    folgender Punkte darstellt:
    \begin{itemize}
        \item Alle Anforderungen sind bekannt und werden auch in dem erforderlichen Detaillierungsgrad verstanden.
        \item Alle involvierten Stakeholder haben eine ausreichende Übereinstimmung über die bekannten Anforderungen erzielt.
        \item Alle Anforderungen sind zu den Dokumentationsvorschriften konform, bzw. zu den Spezifikationsvorschriften konform spezifiziert.
    \end{itemize}
    Damit ist von System-Design, Architekturen oder die nachfolgenden Tests zu differenzieren. 