\section{Requirements Engineering}
\label{sec:requirementsengineering}
    In der heutigen Softwareentwicklung ist zu Anfang jedes Projekts die Frage offen, welche Anforderungen soll das 
    Produkt erfüllen, sodass daraus eine gute, stabil und effiziente Software entsteht und der Umfang und das Ziel 
    klar gestaltet sind. Mit dem \acl{RE} wird der Gedanke verfolgt, die Bedürfnisse des Kunden, bzw. des Stakeholders 
    zu erfüllen und eine erfolgreiche, den Kunden zufriedenstellende Entwicklung des Systems zu erzielen. Ein Stakeholder 
    ist eine Person oder Organisation mit Einfluss auf die Anforderungen des Systems oder die Auswirkungen auf das System 
    hat \cite{pohl2021basiswissen}.
    Diese Anforderungen werden durch das \acl{RE} ermittelt und dokumentiert. 
    \\
    \linebreak
    Unter dem Begriff \ac{RE} ist folgendes zu verstehen: 
    \begin{quote}
        Das Requirements Engineering ist ein systematischer und disziplinierter Ansatz zur Spezifikation und zum 
        Management von Anforderungen mit dem Ziel, die Wünsche und Bedürfnisse der Stakeholder zu verstehen und 
        die Gefahr zu minimieren, ein System auszuliefern, das diese Wünsche und Bedürfnisse nicht erfüllt. \cite{pohl2021basiswissen}
    \end{quote}
    Oft wird darunter auch ein kooperativer und inkrementeller Prozess verstanden, dessen Ziele die Gewährleistung 
    folgender Punkte darstellt:
    \begin{itemize}
        \item Alle Anforderungen sind bekannt und werden auch in dem erforderlichen Detaillierungsgrad verstanden.
        \item Alle involvierten Stakeholder haben eine ausreichende Übereinstimmung über die bekannten Anforderungen erzielt.
        \item Alle Anforderungen sind zu den Dokumentationsvorschriften konform, bzw. zu den Spezifikationsvorschriften konform spezifiziert.
    \end{itemize}
    Damit ist von System-Design, Architekturen oder die nachfolgenden Tests zu differenzieren. 
    \\
    Dieser Prozessschritt deckt die Formalitäten ab, um allen beteiligten eine grundlegende Übersicht zu geben, welche 
    Ziele verfolgt werden, bzw. welche Anforderungen erfüllt werden sollen. Auch ist das \acl{RE} ein wesentlicher 
    Bestandteil des Entwicklungsprozesses nach Scrum. Unter dem Begriff Scrum ist ein Vorgehensmodell zu verstehen, 
    welches zum Projekt- und Produktmanagement in agilen Softwareentwicklungen eingesetzt wird. 
    \\
    \linebreak
    Innerhalb der Anforderungen, die im \acl{RE} ermittelt werden, wird zwischen drei Arten von Anforderungen unterschieden: 
    \begin{itemize}
        \item Funktionale Anforderungen: Diese legen die Funktionalität fest, die durch die Entwicklung des Systems 
            erreicht werden soll. Typischerweise werden diese in Funktions-, Verhaltens- und Strukturanforderungen unterteilt.
        \item Qualitätsanforderungen, sogenannten \ac{NFA}: Diese legen die Qualitäten des Systems fest und beeinflussen die Gestaltung der 
            Systemarchitektur in größerem Maß als die funktionalen Anforderungen.
        \item Rand- oder Rahmenbedingungen (Constraints): Diese legen die Randbedingungen des Systems fest. (Beispielsweise die zu nutzenden Frameworks, Programmiersprache, Konventionen, etc.)
    \end{itemize}
    \acp{NFA} werden über das Produktqualitätsmodell, im Englischen \ac{PQM}, nach der Norm ISO/IEC 25010:2011 abgebildet. \cite{ISO-IEC-PQM2011}
    \\
    \linebreak
    Ein wichtiger Bestandteil des \acl{RE} ist die Spezifikation der Anforderungen, welche in einem Dokument, der 
    sogenannten Anforderungsspezifikation, im Englischen \ac{SRS}, beschrieben sind. Dieses Dokument unterliegt der 
    Norm ISO/IEC 29148-2011 und deckt die Qualitätskriterien ab, damit die Anforderungen einem gewissen Standard 
    entsprechen. 
    \\
    \linebreak
    Das \acl{RE} spielt für die Anforderungsanalyse in Kapitel (\ref{chap:anforderungsanalyse}) und die darauf folgende 
    Konzeption in Kapitel (\ref{chap:konzept}) eine wichtige Rolle. Alle Anforderungen und Spezifikationen wurden 
    mit Zuhilfenahme der Standrads im \acl{RE} ermittelt. In den jeweiligen Kapiteln wird drauf genauestens eingegangen. 

