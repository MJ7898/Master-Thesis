\section{Home Assistant}
\label{sec:homeassistant}
    Eines der populärsten \acl{SH} Plattformen ist das sogenannte Home Assistant System. Die Open-Source-Software ist ein zentrales 
    Steuerungssystem von Heimautomationen und der Verwaltung von intelligenten Geräten mit dem Fokus der lokalen Steuerung und gesicherter 
    Privatsphäre. Der Zugriff kann über die Smartphone-App, jeweils verfügbar für iOS und Android, oder auch über die webbasierte 
    Benutzeroberfläche (Web-App) erfolgen. In dem lokalen System können auch Geräte die Steuerung per Sprachbefehlen ermöglichen. Kompatible 
    Plattformen sind unter anderem Google Assistant, Amazon Alexa und Apple HomeKit. Dies sind weitaus nur eine Selektion von bekannten 
    Herstellern. Home Assistant bietet eine weitaus vielfältigere Verknüpfung von Geräten, Services und Plattformen. Die zentrale Steuerung 
    unterstützt durch modulare Integrationskomponenten die einzelnen Geräte, Anwendungen und Services. Für die drahtlose Kommunikation 
    werden native Integrationskomponenten verwendet, darunter Bluetooth, ZigBee und Z-Wave. Diese werden verwendet, um lokale \ac{PAN} mit 
    Geräten mit geringem Stromverbrauch aufzubauen. Die Steuerung kann auch mit Proprietären Ökosystemen stattfinden, sofern diese eine offene 
    \acs{API} oder Anbindungen über \acs{MQTT} anbieten.\footnote{Grundlegende Ableitung der Definition von Home Assistant siehe \url{https://en.wikipedia.org/wiki/Home_Assistant} Abgerufen am 16.04.2022}
    \\
    Die Platform ist in Python geschrieben und wird aktiv instand gehalten und durch eine große Community unterstützt. Die Software ist allgemein unter 
    der Apache 2.0, veröffentlicht. Der folgende Abschnitt befasst sich in Kürze mit der Historie des Systems. 
    
    \subsection*{Historie}
    \label{sec:historyHOAS}
        Anfang des vierten Quartals im Jahr 2013 startete das Python-Projekt von Paulus Schoutsen und im November 2013 erstmals auf GitHub 
        veröffentlicht.\footnote{Anfänge von Home Assistant. \url{https://www.linux.com/topic/embedded-iot/home-assistant-python-approach-home-automation/} Abgerufen am 18.04.2022}
        \\
        \linebreak
        Vier Jahre nach den ersten Entwicklungen der \acl{SH} Plattform wurde im Juli 2017 ein verwaltetes Betriebssystem mit dem Namen 
        \textit{Hass.io} entwickelt.\footnote{Verkündungen von Home Assistant. \url{https://www.home-assistant.io/blog/categories/announcements/} Abgerufen am 18.04.2022} 
        Dadurch gelang der Durchbruch der vereinfachten Verwendung von der Home Assistant Plattform auf kleineren Computern, sogenannten 
        Einplatinencomputern, wie beispielsweise einem der Raspberry Pi Serie. In Zusammenhang mit dem Betriebssystem kam ein 
        \textit{Supervisor}-Verwaltungssystem hinzu, das den Benutzern die Verwaltung, Sicherung und Aktualisierung der lokalen Installation 
        ermöglicht. Ein weiteres Feature des Supervisors ist die Möglichkeit der Plattform über Add Ons weitere Funktionalitäten zu Verfügung zu 
        stellen.\footnote{Einstieg in das Hass.io Betriebssystem. \url{https://www.home-assistant.io/blog/2017/07/25/introducing-hassio/} Abgerufen am 18.04.2022}
        \\
        \linebreak
        Die Software wird stetig weiterentwickelt und verbessert. Mittlerweile gehört sie zu den am meist genutzten Open-Source-Plattformen 
        im Bereich \acl{SH}.
        

\subsection{Konzept und Architektur}
\label{sec:conceptArchitectureHAOS}
    % Beschreibung der Konzepte und Architektur von Home Assistant.
    %Bild der Architektur
    % Beschreibung der Komponenten, Aufbau etc von Home Assistant
%\subsection{Architektur}
\subsection{Ziele und Schwerpunkte}
\subsection{Stärken und Schwächen}
