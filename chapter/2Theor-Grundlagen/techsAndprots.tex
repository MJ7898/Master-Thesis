\section{Technologien}
\label{sec:technologien}
    Die bereits angesprochene Kommunikation und Vernetzung zwischen Geräten basiert im Allgemeinen auf 
    diversen Protokollen. Um diese Datenbewegung und Kommunikation besser verstehen zu können, werden im 
    Folgenden bekannte Protokolle erwähnt und aufgeführt und eines der meist verwendeten näher betrachtet. 
    Um einen Vergleich herzustellen, wird ein vergleichbares Protokoll betrachtet. Diese werden dann zum 
    Abschluss gegenübergestellt. 

    \subsection{Netzwerkprotokolle}
    \label{subsec:netzwerkprotokolle}
    Allgemein gibt es im Bereich des \acl{SH} mehrere Protokolle und Möglichkeiten, die Objekte miteinander vernetzt. 
    Unter dessen gehören Protokolle über Bluetooth, Ethernet, WLAN, Bussysteme, Funk und Stromleitung. 
    Diese werden Abhängig von den Herstellern eingesetzt. Proprietäre Systeme funktionieren nur über eine 
    Übertragungsmethode. So erzwingen die Hersteller die Nutzung einer Produktlinie, bzw. den Kauf einer 
    einheitlichen Lösung. Geräte die die Möglichkeiten besitzen über mehrere Protokolle 
    zu kommunizieren sind flexibler einsetzbar und mit mehreren Plattformen und Geräten kompatibel.
    Die populärsten werden in folgender Tabelle aufgelistet, jedoch in dieser Arbeit nicht weiter vertieft:
    

    \subsection{MQTT}
    \label{subsec:mqtt}


    \subsection{AMQP}
    \label{subsec:amqp}

    
    %\subsection{DDS}
    \subsection{Raspberry Pi}
    \label{subsec:raspberrypi} 
