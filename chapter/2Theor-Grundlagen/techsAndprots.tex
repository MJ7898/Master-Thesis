\section{Technologien}
\label{sec:technologien}
    Die bereits angesprochene Kommunikation und Vernetzung zwischen Geräten basiert im Allgemeinen auf 
    diversen Protokollen. Um diese Datenbewegung und Kommunikation besser verstehen zu können, werden im 
    Folgenden bekannte Protokolle erwähnt und aufgeführt und eines der meist verwendeten näher betrachtet. 
    Um einen Vergleich herzustellen, wird ein vergleichbares Protokoll betrachtet. Diese werden dann zum 
    Abschluss gegenübergestellt. 

    \subsection{Übertragungsmethoden}
    \label{subsec:netzwerkprotokolle}
    Allgemein gibt es im Bereich des \acl{SH} mehrere Methoden und Möglichkeiten, die Objekte miteinander zu vernetzen. 
    Unter dessen gehören Protokolle über Bluetooth, Ethernet, WLAN, Bussysteme, Funk und Stromleitung. 
    Diese werden Abhängig von den Herstellern eingesetzt. Proprietäre Systeme funktionieren nur über eine 
    Übertragungsmethode. So erzwingen die Hersteller die Nutzung einer Produktlinie, bzw. den Kauf einer 
    einheitlichen Lösung. Geräte die die Möglichkeiten besitzen über mehrere Protokolle 
    zu kommunizieren sind flexibler einsetzbar und mit mehreren Plattformen und Geräten kompatibel.
    Grundlegend werden mit diesen Übertragungsmethoden Netzwerke erstellt, über das die Geräte in einem \acl{SH} kommunizieren können.
    Die populärsten werden in folgender Tabelle aufgelistet: 
    \begin{table}[hbt!]
        \begin{center}
            \begin{tabular}{| p{3.25cm} | p{3.25cm} | p{3.25cm} | p{3.25cm} |}
                \hline
                    \textbf{Technologie} & \textbf{Übertragung} & \textbf{Frequenzbereich (Funk)} & \textbf{Proprietär} \\
                \hline
                    ZigBee & Funk & 2,4 GHz, 868 MHz & Nein \\ 
                \hline
                    Z-Wave & Funk & 868 MHz & Nein \\ 
                \hline
                    HomeMatic & Funk / Datenleitung & 868,3 MHz & Ja \\
                \hline
                    KNX & Funk / Strom- und Datenleitung & 868 MHz / - & Nein / Ja (Datenleitung als Gesamtsystem) \\
                \hline
                    Wi-Fi / WLAN & Funk & 2,4 - 5 GHz & Nein \\
                \hline 
                    Bluetooth & Funk & 2,4 GHz & Nein \\
                \hline
                    io-homecontrol & Funk & 868-870 MHz & Ja \\
                \hline
            \end{tabular}
        \end{center}
        \caption{Übertragungsmethoden des Smart Home}
        \label{tab:protocolsSH}
    \end{table}
    \\
    Die Auflistung der zum aktuellen Zeitpunkt am meist verwendeten Übertragungsmethoden dient 
    lediglich als Einblick, damit eine große Gesamtübersicht der Technologien und der Thematik \acl{SH} entsteht. 
    Demnach wird im Rahmen dieser Arbeit das Thema oberflächlich erläutert und nicht ausführlich vertieft.
    %ZigBee? Erläuterung etwas detaillierter?

    % https://www.bigdata-insider.de/so-laeuft-der-datenaustausch-zwischen-edge-und-cloud-a-1097887/ 

    \subsection{MQTT}
    \label{subsec:mqtt}
        Das \ac{MQTT}-Protokoll ist eines der ältesten offenen Netzwerk- und Nachrichtenprotokolle der 
        \ac{MtoM}-Kommunikation. 
        Dies wurde 1999 von IBM Mitarbeiter Andy Stanford-Clark\footnote{Informationen zu Herrn Stanford-Clark \url{https://stanford-clark.com} Abgerufen am 12.04.2022} 
        und von Cirrus Link Solutions Mitarbeiter Arlen Nipper\footnote{Informationen zu Herrn Nipper \url{https://www.inductiveautomation.com/resources/podcast/the-coinventor-of-mqtt-arlen-nipper-from-cirrus-link-solutions} Abgerufen am 12.04.2022} 
        entwickelt. Die Technologie ermöglicht die Übertragung von Messdaten, sogenannten Telemetriedaten, in Form von 
        Nachrichten zwischen Maschinen und Geräten. Die erzeugten Messdaten durch beispielsweise Sensoren und Aktoren 
        können durch ihre minimale Größe und die kompakte Form des Protokolls in einem kleinen Datenpaket auch bei 
        hoher Verzögerung oder bei beschränktem Netzwerk übertragen werden. 
        % Grunlegende Erläuterung zu MQTT:
        %\cite{Naik2017} und \cite{Hunkeler2008}
        % Publish / Subscribe Kommunikationmodell 



    \subsection{AMQP}
    \label{subsec:amqp}
        % Grunlegende Erläuterung zu AMQP:
        %\cite{Naik2017}
        % Vergleich zu MQTT

    
    %\subsection{DDS}
    \subsection{Raspberry Pi}
    \label{subsec:raspberrypi} 
