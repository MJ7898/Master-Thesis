\chapter{Grundlagen}
\label{chap:grundlagen}
    In diesem Kapitel werden die für diese Thesis relevanten Grundlagen geschaffen, um ein Grundverständnis und 
    fundiertes Wissen über verwendete Technologien zu erlangen und die nachfolgende Recherche, Konzeption und 
    Umsetzung besser verstehen zu können. 

    %%%%%%%%%%%%%%%%%%%%%%%%%%%%%%%%%%%%%%%%%%%%%%%%%%%%%%%%%%%%%%%%%%%%%%%%%%%%%%%
        %Das Internet der Dinge ist ein neuartiger Paradigmenwechsel in der IT-Arena. Der Begriff „Internet of Things“, 
        %kurz auch als IoT bekannt, setzt sich aus den beiden Wörtern zusammen, d.h. das erste Wort ist „Internet“ und das 
        %zweite Wort „Things“. Das Internet ist ein globales System miteinander verbundener Computernetzwerke, die die 
        %Standard-Internetprotokollsuite (TCP/IP) verwenden, um Milliarden von Benutzern weltweit zu dienen. Es ist ein 
        %Netzwerk von Netzwerken, das aus Millionen privater, öffentlicher, akademischer, geschäftlicher und staatlicher 
        %Netzwerke von lokaler bis globaler Reichweite besteht, die durch eine breite Palette elektronischer, drahtloser 
        %und optischer Netzwerktechnologien verbunden sind [3]. Heute sind mehr als 100 Länder über das Internet in den 
        %Austausch von Daten, Nachrichten und Meinungen eingebunden. Laut Internet World Statistics gab es zum 
        %31. Dezember 2011 weltweit schätzungsweise 2.267.233.742 Internetnutzer (aufgerufene Daten vom 06.06.2013: 
        %von der Universal Resource Location http://www.webopedia. com/TERM/I/Internet.html). Dies bedeutet, dass 32,7 \% 
        %der Gesamtbevölkerung der Welt das Internet nutzen. Sogar das Internet wird in den kommenden vierten Jahren durch 
        %das Internet Routing in Space (IRIS)-Programm von Cisco in den Weltraum gehen (Zugriff am 05.10.2012: 
        %(http://www.cisco.com/web/strategy/government/space -routing.html) 
        %Zu den Dingen, die beliebige Gegenstände oder Personen sein können, die von der realen Welt unterscheidbar sind, 
        %zählen nicht nur elektronische Geräte, denen wir täglich begegnen und die wir täglich verwenden, sondern auch 
        %technologisch fortschrittliche Produkte wie Geräte und Gadgets , sondern „Dinge“, die wir normalerweise überhaupt 
        %nicht als elektronisch betrachten – wie Lebensmittel, Kleidung und Möbel; Materialien, Teile und Ausrüstung, Waren 
        %und Spezialartikel; Sehenswürdigkeiten, Denkmäler und Kunstwerke und all das Verschiedenes aus Handel, Kultur und 
        %Raffinesse [4] Das heißt, hier können Dinge sowohl Lebewesen sein wie Menschen, Tiere – Kuh, Kalb, Hund, Tauben, 
        %Kaninchen usw., Pflanzen – Mangobaum, Jasmin, Banyan und so weiter und nicht - Lebewesen wie Stuhl, Kühlschrank, 
        %Röhrenlampe, Vorhang, Teller usw. jedes Haushaltsgerät es oder Industriegerät. An diesem Punkt sind die Dinge also 
        %reale Objekte in dieser physischen oder materiellen Welt.
    %%%%%%%%%%%%%%%%%%%%%%%%%%%%%%%%%%%%%%%%%%%%%%%%%%%%%%%%%%%%%%%%%%%%%%%%%%%%%%%%%%%%

    % Beschreibung
    % Definition
    % Gesamtbild 
    % Historische Entwicklung 

\section{Internet der Dinge}
\label{sec:iot}
    Das \ac{IdD}, im Englischen \ac{IoT}, zählt als eines der Schlagworte in der \ac{IT}. In der Domäne des \acs{IoT} bekommen 
    Gegenstände und Objekte eine eindeutige Identität, die eine Kommunikation miteinander als auch das Entgegennehmen von 
    Befehlen erlaubt. Mit dem \acl{IdD} lassen sich Anwendungen sowie Prozesse automatisieren und Aufgaben erledigen ohne das 
    von außen Eingegriffen werden muss \cite{bigdatainsider2016}. Die Prozessautomatisierung findet sich auch im Kontext des 
    \acl{SH} wieder, welches in nachfolgendem Kapitel genauer aufgegriffen wird. 
    \\
    \linebreak
    In der einschlägigen Literatur gibt es für das \acl{IoT} keine allgemeingültige Definition die alle Anwendungsbereiche abdeckt. 
    Die Definitionen und Auslegungen der Interpretation unterscheiden sich je nach Anwendungsgebiet. Demnach gibt es viele verschiedene 
    Forschungsgruppen, darunter Forscher, Akademiker, Innovatoren, Entwickler und Geschäftsleute, die den Begriff oder die zugrundeliegende 
    Problemstellung definiert haben. Die Ursprünge jedoch sind dem Experten für digitale Innovationen, Kevin Ashton\footnote{Britischer Technologie-Pionier, Mitgründer des Auto-ID Centers am Massachusetts Institute of Technology (MIT). \url{https://de.wikipedia.org/wiki/Kevin_Ashton} (Abgerufen am 22.03.2022)}, 
    zuzuschreiben. 
    \\ 
    Die in der Literatur auffindbaren Definitionen verfolgen zwei Sichtweisen, zum Einen aus Sicht des aktiven, dass die Daten von 
    Menschen erstellt wurde, zum Andere, aus Sicht des passiven, dass die Daten von Dingen, darunter die Sensoren und Aktoren, 
    erstellt wurde. \cite{Madakam2015} Eine aus dem Zusammenschluss hervorgehende Definition ist, dem wissenschaftlichen Artikel zufolge, folgende:
    % „Ein offenes und umfassendes Netzwerk intelligenter Objekte, die in der Lage sind, sich automatisch zu organisieren, 
    % Informationen, Daten und Ressourcen auszutauschen und auf Situationen und Veränderungen in der Umgebung zu reagieren 
    % und zu handeln.“
    \begin{quote}
        “An open and comprehensive network of intelligent objects that have the capacity to auto-organize, share information, data 
        and resources, reacting and acting in face of situations and changes in the environment” \cite{Madakam2015}
    \end{quote}
    Daraus kann abgeleitet werden, dass der Begriff des \acl{IdD} für die Vernetzung von Gegenständen im privaten Gebrauch oder 
    von Geräten und Maschinen im industriellen Umfeld über das Internet steht. Damit Geräte individuell angesprochen werden können, werden diese 
    mit einer eindeutigen Identität, genauer einer \ac{IP}-Adresse, im Netzwerk belegt und mit elektronischer Intelligenz ausgestattet \cite{bigdatainsider2016}.
    Darüber sind die Netzwerkteilnehmer im Stande, über das Internet zu kommunizieren Prozesse automatisiert zu erledigen. Die sogenannten 
    \textit{intelligenten Geräte} werden auch oft mit dem englischen Begriff, \textit{Smart Devices}, betitelt. 
    \\
    \linebreak
    Neben der Kommunikation der Geräte über das Netzwerk untereinander kann ebenso entweder durch das Gerät selbst oder eine zentrale 
    Schnittstelle über das Internet Interagiert werden. Dadurch sind Objekte und Gegenstände durch einen Benutzer von beliebigen Orten 
    auch außerhalb des Netzwerks erreichbar und können so bedient werden. Diese Art und Weise wird auch in dem zentralen Thema des 
    \acl{SH} verwendet. Die Funktion als auch die Umsetzung wird im Kapitel (\ref{sec:smartHome}) näher beleuchtet.

    %Das Internet der Dinge reift und ist nach wie vor das neueste und am meisten gehypte Konzept in der IT-Welt. 
    %In den letzten zehn Jahren hat der Begriff Internet of Things (IoT) Aufmerksamkeit erregt, indem er die Vision 
    %einer globalen Infrastruktur vernetzter physischer Objekte projizierte, die jederzeit und überall Konnektivität für 
    %alles und nicht nur für irgendjemanden ermöglicht [4]. Das Internet der Dinge kann auch als globales Netzwerk 
    %betrachtet werden, das die Kommunikation zwischen Mensch zu Mensch, Mensch zu Dingen und Dingen zu Dingen 
    %ermöglicht, was alles auf der Welt ist, indem es jedem Objekt eine einzigartige Identität verleiht [5]. 
    %IoT beschreibt eine Welt, in der so gut wie alles vernetzt werden kann und so intelligent kommuniziert wie nie zuvor. 
    %Die meisten von uns denken an „verbunden sein“ in Bezug auf elektronische Geräte wie Server, Computer, Tablets, Telefone 
    %und Smartphones. Im sogenannten Internet der Dinge werden Sensoren und Aktuatoren, die in physische Objekte eingebettet 
    %sind – von Straßen bis hin zu Herzschrittmachern – über kabelgebundene und drahtlose Netzwerke verbunden, wobei sie 
    %häufig dieselbe Internet-IP verwenden, die das Internet verbindet. Diese Netzwerke produzieren riesige Datenmengen, 
    %die zur Analyse an Computer fließen. Wenn Objekte sowohl die Umgebung wahrnehmen als auch kommunizieren können, 
    %werden sie zu Werkzeugen, um Komplexität zu verstehen und schnell darauf zu reagieren. Das Revolutionäre dabei ist, 
    %dass diese physischen Informationssysteme nun beginnen, eingesetzt zu werden, und einige von ihnen funktionieren sogar 
    %weitgehend ohne menschliches Zutun. Das „Internet der Dinge“ bezeichnet die Kodierung und Vernetzung von 
    %Alltagsgegenständen und Dingen, um sie im Internet individuell maschinenlesbar und rückverfolgbar zu machen [6]-[11]. 
    %Viele bestehende Inhalte im Internet der Dinge wurden durch codierte RFID-Tags und IP-Adressen erstellt, die mit einem 
    %EPC-Netzwerk (Electronic Product Code) verknüpft sind [12].




%\subsection{IIoT}
\subsection{Edge und Cloud Computing}

%Definition Edge Computing
%Definition Cloud Computing
% Unterschiede der beiden Ansätze und in Zusammenhang mit IoT (kurz)