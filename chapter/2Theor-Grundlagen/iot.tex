\chapter{Grundlagen}
\label{chap:grundlagen}
    In diesem Kapitel werden die für diese Thesis relevanten Grundlagen geschaffen, um ein Grundverständnis und 
    fundiertes Wissen über verwendete Technologien zu erlangen und die nachfolgende Recherche, Konzeption und 
    Umsetzung besser verstehen zu können. 

\section{IoT - Internet der Dinge}
\label{sec:iot}
    %Eines der Schlagworte in der Informationstechnologie ist das Internet der Dinge (IoT). Die Zukunft ist das 
    %Internet der Dinge, das die Objekte der realen Welt in intelligente virtuelle Objekte verwandeln wird. 
    %Das IoT zielt darauf ab, alles in unserer Welt unter einer gemeinsamen Infrastruktur zu vereinen und uns 
    %nicht nur die Kontrolle über die Dinge um uns herum zu geben, sondern uns auch über den Stand der Dinge 
    %auf dem Laufenden zu halten. Vor diesem Hintergrund befasst sich die vorliegende Studie mit IoT-Konzepten 
    %durch systematische Überprüfung von wissenschaftlichen Forschungsarbeiten, Unternehmens-Whitepapers, 
    %Fachdiskussionen mit Experten und Online-Datenbanken. Darüber hinaus konzentriert sich dieser Forschungsartikel 
    %auf Definitionen, Genese, grundlegende Anforderungen, Eigenschaften und Aliase des Internets der Dinge. 
    %Das Hauptziel dieses Papiers ist es, einen Überblick über das Internet der Dinge, Architekturen und wichtige
    %Technologien und ihre Verwendung in unserem täglichen Leben zu geben. Dieses Manuskript wird jedoch den neuen 
    %Forschern, die auf diesem Gebiet des Internets der Dinge (Technological GOD) forschen wollen, ein gutes 
    %Verständnis vermitteln und eine effiziente Wissensakkumulation ermöglichen.

    \ac{IoT}, im Deutschen \textit{\ac{IdD}},  

    Das Internet der Dinge ist ein neuartiger Paradigmenwechsel in der IT-Arena. Der Begriff „Internet of Things“, 
    kurz auch als IoT bekannt, setzt sich aus den beiden Wörtern zusammen, d.h. das erste Wort ist „Internet“ und das 
    zweite Wort „Things“. Das Internet ist ein globales System miteinander verbundener Computernetzwerke, die die 
    Standard-Internetprotokollsuite (TCP/IP) verwenden, um Milliarden von Benutzern weltweit zu dienen. Es ist ein 
    Netzwerk von Netzwerken, das aus Millionen privater, öffentlicher, akademischer, geschäftlicher und staatlicher 
    Netzwerke von lokaler bis globaler Reichweite besteht, die durch eine breite Palette elektronischer, drahtloser 
    und optischer Netzwerktechnologien verbunden sind [3]. Heute sind mehr als 100 Länder über das Internet in den 
    Austausch von Daten, Nachrichten und Meinungen eingebunden. Laut Internet World Statistics gab es zum 
    31. Dezember 2011 weltweit schätzungsweise 2.267.233.742 Internetnutzer (aufgerufene Daten vom 06.06.2013: 
    von der Universal Resource Location http://www.webopedia. com/TERM/I/Internet.html). Dies bedeutet, dass 32,7 \% 
    der Gesamtbevölkerung der Welt das Internet nutzen. Sogar das Internet wird in den kommenden vierten Jahren durch 
    das Internet Routing in Space (IRIS)-Programm von Cisco in den Weltraum gehen (Zugriff am 05.10.2012: 
    %(http://www.cisco.com/web/strategy/government/space -routing.html) 
    Zu den Dingen, die beliebige Gegenstände oder Personen sein können, die von der realen Welt unterscheidbar sind, 
    zählen nicht nur elektronische Geräte, denen wir täglich begegnen und die wir täglich verwenden, sondern auch 
    technologisch fortschrittliche Produkte wie Geräte und Gadgets , sondern „Dinge“, die wir normalerweise überhaupt 
    nicht als elektronisch betrachten – wie Lebensmittel, Kleidung und Möbel; Materialien, Teile und Ausrüstung, Waren 
    und Spezialartikel; Sehenswürdigkeiten, Denkmäler und Kunstwerke und all das Verschiedenes aus Handel, Kultur und 
    Raffinesse [4] Das heißt, hier können Dinge sowohl Lebewesen sein wie Menschen, Tiere – Kuh, Kalb, Hund, Tauben, 
    Kaninchen usw., Pflanzen – Mangobaum, Jasmin, Banyan und so weiter und nicht - Lebewesen wie Stuhl, Kühlschrank, 
    Röhrenlampe, Vorhang, Teller usw. jedes Haushaltsgerät es oder Industriegerät. An diesem Punkt sind die Dinge also 
    reale Objekte in dieser physischen oder materiellen Welt.

    Definitionen
    Es gibt keine eindeutige Definition für das Internet der Dinge, die von der weltweiten Benutzergemeinschaft 
    akzeptiert wird. Tatsächlich gibt es viele verschiedene Gruppen, darunter Akademiker, Forscher, Praktiker, 
    Innovatoren, Entwickler und Geschäftsleute, die den Begriff definiert haben, obwohl seine ursprüngliche 
    Verwendung Kevin Ashton, einem Experten für digitale Innovation, zugeschrieben wurde. Allen Definitionen 
    gemeinsam ist die Idee, dass es in der ersten Version des Internets um Daten ging, die von Menschen erstellt 
    wurden, während es in der nächsten Version um Daten ging, die von Dingen erstellt wurden. Die beste Definition 
    für das Internet der Dinge wäre:
    „Ein offenes und umfassendes Netzwerk intelligenter Objekte, die in der Lage sind, sich automatisch zu organisieren, 
    Informationen, Daten und Ressourcen auszutauschen und auf Situationen und Veränderungen in der Umgebung zu reagieren 
    und zu handeln.“
    Das Internet der Dinge reift und ist nach wie vor das neueste und am meisten gehypte Konzept in der IT-Welt. 
    In den letzten zehn Jahren hat der Begriff Internet of Things (IoT) Aufmerksamkeit erregt, indem er die Vision 
    einer globalen Infrastruktur vernetzter physischer Objekte projizierte, die jederzeit und überall Konnektivität für 
    alles und nicht nur für irgendjemanden ermöglicht [4]. Das Internet der Dinge kann auch als globales Netzwerk 
    betrachtet werden, das die Kommunikation zwischen Mensch zu Mensch, Mensch zu Dingen und Dingen zu Dingen 
    ermöglicht, was alles auf der Welt ist, indem es jedem Objekt eine einzigartige Identität verleiht [5]. 
    IoT beschreibt eine Welt, in der so gut wie alles vernetzt werden kann und so intelligent kommuniziert wie nie zuvor. 
    Die meisten von uns denken an „verbunden sein“ in Bezug auf elektronische Geräte wie Server, Computer, Tablets, Telefone 
    und Smartphones. Im sogenannten Internet der Dinge werden Sensoren und Aktuatoren, die in physische Objekte eingebettet 
    sind – von Straßen bis hin zu Herzschrittmachern – über kabelgebundene und drahtlose Netzwerke verbunden, wobei sie 
    häufig dieselbe Internet-IP verwenden, die das Internet verbindet. Diese Netzwerke produzieren riesige Datenmengen, 
    die zur Analyse an Computer fließen. Wenn Objekte sowohl die Umgebung wahrnehmen als auch kommunizieren können, 
    werden sie zu Werkzeugen, um Komplexität zu verstehen und schnell darauf zu reagieren. Das Revolutionäre dabei ist, 
    dass diese physischen Informationssysteme nun beginnen, eingesetzt zu werden, und einige von ihnen funktionieren sogar 
    weitgehend ohne menschliches Zutun. Das „Internet der Dinge“ bezeichnet die Kodierung und Vernetzung von 
    Alltagsgegenständen und Dingen, um sie im Internet individuell maschinenlesbar und rückverfolgbar zu machen [6]-[11]. 
    Viele bestehende Inhalte im Internet der Dinge wurden durch codierte RFID-Tags und IP-Adressen erstellt, die mit einem 
    EPC-Netzwerk (Electronic Product Code) verknüpft sind [12].




\subsection{IIoT}
\subsection{Cloud Computing}