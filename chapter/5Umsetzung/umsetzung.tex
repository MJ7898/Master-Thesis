\chapter{Umsetzung}
\label{chap:umsetzung}


\section{Auswahl der Frameworks}
\label{sec:frameworkauswahl}
    Im Bereich der Java-Entwicklung gibt es mittlerweile viele Möglichkeiten, um Applikationen, Anwendungen und Frameworks 
    zu entwickeln, die unterschiedliche Präferenzen und Einsatzmöglichkeiten bieten. Dadurch sind, auf den Einsatzbereich bezogen, 
    Vor- und Nachteile im Vergleich ähnlicher Systeme nicht auszuschließen. Eine kleine Auswahl an Systemen wurde getestet und auf deren 
    Brauchbarkeit analysiert und evaluiert. Für diese Evaluation wurden Kriterien ausgearbeitet, mit der die Auswahl des Frameworks 
    eingeschränkt und nach Möglichkeit das passende ergeben soll. Die Kriterien sind nach ihrer Relevanz aufgelistet: 
    \begin{enumerate}
        \item Freiheiten bei der Nutzung (Anwendung und Gewährleistung von Entwurfsmustern).
        \item Nutzung und Bereitstellung von Bibliotheken (Libraries).
        \item Bereitstellung einer Plattform (Full Stack).
        \item Aktive Community und stetige Weiterentwicklung des Systems.
        \item Nutzung des Frameworks in bereits bestehenden Projekten im Bereich Smart Home.
        \item Open Source-Projekt, um Flexibilität und weitestgehende Unabhängigkeit zu gewährleisten.
        \item Ausschluss von Frameworks, die ausschließlich für die Web-Entwicklung gedacht sind.
    \end{enumerate}  
    Aufgrund der Vielzahl an Frameworks war es im Rahmen dieser Arbeit nicht möglich, alle vorhandenen und in Betracht 
    gezogenen Systeme detailliert aufzuführen. Lediglich die engere Auswahl wird aufgegriffen. 
    Nach ausführlicher Recherche und unter Berücksichtigung von Frameworks, wie bspw. Grails, Quarkus, Blade, und Play, wurden 
    schließlich genau zwei Systeme genauer betrachtet, die die vorangestellten Kriterien in Gänze erfüllen. Viele in Betracht gezogenen Frameworks 
    finden überwiegend in der Web-Entwicklung Anwendung, bzw. basieren auf dem Spring Framework. Durch diese Erkenntnis wurden 
    viele Systeme nicht weiter berücksichtigt. Die beiden Kernsysteme, die ihren Einsatz und ihre Möglichkeiten rechtfertigen
    werden in den folgenden Abschnitten kurz erläutert.

    \subsection{OSGi}
    \label{subsec:osgiFramework}
        Das \ac{OSGI} Framework der \acs{OSGI}\footnote{Ursprung der OSGi Plattform. \url{https://www.osgi.org/about/} Abgerufen am 19.06.2022} 
        Alliance, welches in der openHAB Software eingesetzt wird, klassifiziert eine dynamische Softwareplattform, mit der die Modularisierung 
        und Verwaltung von Applikationen und den dazugehörigen Diensten mittels Komponentenmodell realisiert werden kann \cite{funke2009}. Bekannte 
        Produkte, die auf der \acs{OSGI} Plattform laufen, sind neben openHAB unter anderem die Entwicklungsumgebung Eclipse der Eclipse 
        Foundation, Produkte und Softwarelösungen von IBM, Oracle, Adobe und weitere. 
        \\
        Nennenswerte Eigenschaften und Vorteile der Software sind die Modularisierung und Versionierung, das zur Laufzeit organisierte 
        Abhängigkeitsmanagement, das Fernmanagement des laufenden Systems über sogenannte Management Agents und die Nutzung des 
        Serviceorientierten Programmiermodells, \ac{SOA}\footnote{Erläuterung des SOA Modells. \url{https://www.ibm.com/cloud/learn/soa} Abgerufen am 20.06.2022}. 
        An dieser Stelle wird das Framework nicht technisch vertieft. Die Funktionsweise und die technisch 
        fundierte Erläuterung kann dem Buch \cite{osgibuch}, sowie der Dokumentation \cite{osgipraesentation} eines ausgearbeiteten Workshops entnommen werden. 
        Nachteile der Plattform sind zum einen der weniger breite Einsatz des Frameworks und zum anderen die kleine Community. 
        Ebenso findet eine Weiterentwicklung der Plattform nur mäßig statt. Dies hat zur Folge, dass die Erweiterung, bzw. die Entwicklung des darauf 
        aufbauenden Systems träge vonstattengeht. 
        %Wieso Spring Boot und nicht OSGi?
        %Wieso Java und nicht Python oder andere?

    \subsection{Spring}
    \label{subsec:springBootFramework}
    Spring\footnote{Open-Source Framework Spring. \url{https://spring.io} Abgerufen am 02.07.2022} ist ein Open-Source Framework, welches auf der Java-Plattform aufbaut. 
    Das Ziel von Spring selbst ist die Vereinfachung und Förderung von Programmierpraktiken in der Java- und Java EE Entwicklung. Mit einem breiten Spektrum an 
    Funktionalitäten bietet das Framework eine ganzheitliche Lösung zur Entwicklung von Applikationen, Anwendungen und Frameworks. Im Vordergrund dabei steht immer die 
    Entkopplung von Applikationskomponenten. Das Spring Framework wurde im März 2004 offiziell Freigegeben und wird seitdem stetig weiterentwickelt. 
    Initiator des Frameworks ist der Author und Softwareentwickler Rod Johnson. Das Framework, welches in dem Buch des Experten Rod Johnson erläutert wird, basiert auf 
    den folgenden Prinzipien \cite{johnson2004expert}: 
    \\
    \linebreak
    Dependency Injection ist ein sehr bekanntes Entwurfsmuster. Dabei werden die Abhängigkeiten eines Objektes zur Laufzeit reglementiert. Unter 
    Verwendung des Frameworks werden den Objekten die benötigten Objekte und Ressourcen zugewiesen. Dadurch müssen diese nicht selbst gesucht werden. 
    \\
    \linebreak
    Ein weiterer Punkt ist die aspektorientierte Programmierung, durch die der Entwickler technische Aspekte voneinander isolieren und den eigentlichen Programmcode 
    von Transaktionen oder anderen Faktoren freizuhalten. 
    \\
    \linebreak
    Das dritte Prinzip ist das Bereitstellen von Vorlagen zur Vereinfachung der Nutzung von Schnittstellen, sog. \acs{API}s. Dadurch wird ein POJO-basiertes Modell 
    möglich. Unter POJO, \textit{Plain Old Java Object}, ist ein ganz normales Objekt in Java zu verstehen. 
    \\
    \linebreak
    Zusammenfassend sind beide Frameworks dazu geeignet, um das Konzept der Steuerzentrale unterstützend zu realisieren. 
    Ein direkter Vergleich zwischen den Frameworks kann nicht gezogen werden, da die Funktionalitäten, bzw. die aus der Nutzung 
    profitablen Eigenschaften nicht vergleichbar sind. Während sich Spring auf die Unterstützung zur Erstellung von Applikationen konzentriert, 
    legt \acs{OSGI} den Mehrwert auf die Modularisierung durch Kontainerisierung. Durchaus ist eine Kombination aus beiden 
    Frameworks möglich, um Vorteile beider zu vereinen. Fundamental bietet Spring jedoch weitaus mehr Möglichkeiten und Unterstützungen 
    zur Entwicklung von Applikationen und Frameworks, weshalb es im Rahmen dieser Arbeit eingesetzt wird. Mit Spring Boot, welches auf Spring 
    aufbaut können auch Model-View-Controller\footnote{Architekturmuster für Webapplikationen. \url{https://de.wikipedia.org/wiki/Model_View_Controller} Abgerufen am 02.07.2022} 
    Web-Architekturen und \acs{REST} \acs{API}s implementiert werden. So kann mit dieser Entscheidung 
    ein Vollumfängliches System erstellt werden. Unter zusätzlicher Verwendung von \acs{OSGI} wäre eine weiter Modularisierung und Kontainerisierung möglich. Dies 
    ist auch durch Spring mit Hilfe von Microservices in einer anderen Form realisierbar. 
    \\
    An der Stelle wird im Rahmen dieser Arbeit das Thema von Microservices und Kontainerisierung nicht weiter konkretisiert.
    \\
    \linebreak
    Die Entscheidung viel auf Spring, da dieses ein modernes und etabliertes Framework darstellt und die 
    Community erheblich größer ist, als die des \acs{OSGI} Frameworks. Hinzukommt, die Schaffung eines alternativen Ansatzes zu openHAB, da dieses System bereits 
    auf \acs{OSGI} aufbaut. 
    Dennoch unterscheidet sich die Steuerzentrale in den Funktionalitäten und Angeboten immens im Vergleich zu openHAB. Das System bietet weitaus mehr 
    Funktionalitäten, Erweiterungen und Ausprägungen. Lediglich kann zum Vergleich die Definition von Regeln herangezogen werden. 

\section{Implementierung}
\label{sec:implementation}

\subsection{Prototypische Implementierung des Anwendungsfalls}
%Erläuterung der Umsetzung des Anwendungsfalls (Mögliche Regeldefinition)
%-> Einbauen von Thread.sleeps um einen durchehenden Prozess zu durchlaufen oder 
% feingranulare Regeln, die immer wieder auf Rückmeldung über MQTT Topics warten und danach die Aktion auslösen


%\subsection{Aufbau der Architektur}
%\subsection{Einbindung der Funktionen abgeleitet von der Konzeption}
\section{Fazit der prototypischen Implementierung}