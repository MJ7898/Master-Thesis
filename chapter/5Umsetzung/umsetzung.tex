\chapter{Umsetzung}
\label{chap:umsetzung}


\section{Auswahl der Frameworks}
\label{sec:frameworkauswahl}
    Im Bereich der Java-Entwicklung gibt es mittlerweile viele Möglichkeiten, um Applikationen, Anwendungen und Frameworks 
    zu entwickeln, die unterschiedliche Präferenzen und Einsatzmöglichkeiten bieten. Dadurch sind, auf den Einsatzbereich bezogen, 
    Vor- und Nachteile im Vergleich ähnlicher Systeme nicht auszuschließen. Eine kleine Auswahl an Systemen wurde getestet und auf deren 
    Brauchbarkeit analysiert und evaluiert. Für diese Evaluation wurden Kriterien ausgearbeitet, mit der die Auswahl des Frameworks 
    eingeschränkt und nach Möglichkeit das passende ergeben soll. Diese sind nach ihrer Relevanz aufgelistet: 
    \begin{enumerate}
        \item Freiheiten bei der Nutzung (Anwendung und Gewährleistung von Entwurfsmustern).
        \item Nutzung und Bereitstellung von Bibliotheken (Libraries).
        \item Bereitstellung einer Plattform (Full Stack).
        \item Aktive Community und stetige Weiterentwicklung des Systems.
        \item Nutzung des Frameworks in bereits bestehenden Projekten im Bereich Smart Home.
        \item Open Source-Projekt, um Flexibilität und weitestgehende Unabhängigkeit zu gewährleisten.
        \item Ausschluss von Frameworks, die ausschließlich für die Web-Entwicklung gedacht sind.
    \end{enumerate}  
    Aufgrund der Vielzahl an Frameworks war es im Rahmen dieser Arbeit nicht möglich, alle vorhandenen und in Betracht 
    gezogenen Systeme detailliert aufzuführen. Lediglich die engere Auswahl wird aufgegriffen. 
    Nach ausführlicher Recherche und unter Berücksichtigung von Frameworks, wie bspw. Grails, Quarkus, Blade, und Play, wurden 
    schließlich genau zwei Systeme gegenübergestellt, die die vorangestellten Kriterien in Gänze erfüllen. Viele in Betracht gezogenen Frameworks 
    finden überwiegend in der Web-Entwicklung Anwendung, bzw. basieren auf dem Spring Framework. Durch diese Erkenntnis wurden 
    viele Systeme nicht weiter berücksichtigt und betrachtet. Die beiden Kernsysteme, die ihren Einsatz und ihre Möglichkeiten rechtfertigen
    werden in den folgenden Abschnitten kurz erläutert.

    \subsection{OSGi}
    \label{subsec:osgiFramework}
        Das \ac{OSGI} Framework der \acs{OSGI}\footnote{Ursprung der OSGi Plattform. \url{https://www.osgi.org/about/} Abgerufen am 19.06.2022} 
        Alliance, welches in der openHAB Software eingesetzt wird, klassifiziert eine dynamische Softwareplattform, mit der die Modularisierung 
        und Verwaltung von Applikationen und den dazugehörigen Diensten mittels Komponentenmodell realisiert werden kann \cite{funke2009}. Bekannte 
        Produkte, die auf der \acs{OSGI} Plattform laufen, sind neben openHAB unter anderem die Entwicklungsumgebung Eclipse der Eclipse 
        Foundation, Produkte und Softwarelösungen von IBM, Oracle, Adobe und weitere. 
        \\
        Nennenswerte Eigenschaften und Vorteile der Software sind die Modularisierung und Versionierung, das zur Laufzeit organisierte 
        Abhängigkeitsmanagement, das Fernmanagement des laufenden Systems über sogenannte Management Agents und die Nutzung des 
        Serviceorientierten Programmiermodells, \ac{SOA}\footnote{Erläuterung des SOA Modells. \url{https://www.ibm.com/cloud/learn/soa} Abgerufen am 20.06.2022}. 
        An dieser Stelle wird das Framework nicht technisch vertieft. Die Funktionsweise und die technisch 
        fundierte Erläuterung kann dem Buch \cite{osgibuch}, sowie der Dokumentation \cite{osgipraesentation} eines ausgearbeiteten Workshops entnommen werden. 
        Nachteile der Plattform sind zum einen der weniger breite Einsatz des Frameworks und zum anderen die kleine Community. 
        Ebenso findet eine Weiterentwicklung der Plattform nur mäßig statt. Dies hat zur Folge, dass die Erweiterung, bzw. die Entwicklung des darauf 
        aufbauenden Systems träge vonstattengeht. 
        %Wieso Spring Boot und nicht OSGi?
        %Wieso Java und nicht Python oder andere?

    \subsection{Spring Boot}
    \label{subsec:springBootFramework}

\section{Implementierung}
\label{sec:implementation}

\subsection{Aufbau der Architektur}
\subsection{Einbindung der Funktionen abgeleitet von der Konzeption}
\section{Fazit der prototypischen Implementierung}