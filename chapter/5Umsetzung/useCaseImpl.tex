\section{Fazit der prototypischen Implementierung}
    % Entscheidung für die Nutzung von Reflection. Begründung: Überprüfung der Werte im Zustandsraum, welche sich geändert haben. 
    %Und zur Nutzung der inversen Transformation, um an einer Stelle die Topics etc. zu definieren.

\subsection{Modellierungsvorgaben und -grenzen}
\label{subsec:modellierungsgrenzen}
%WICHTIG: ZUSTANDSRAUM MODELLIERUNG VON KOMPONENTEN, BZW. OBJEKTE SIND MÖGLICH, 
%Problematisch dabei ist die Prüfung der Zustandsänderung, da bei jeder Kopie eine neue 
%Referenz der Komponente erzeugt wird, die eine Zustandsänderung unter dem Feld simuliert (vorgibt) ohne das diese tatsächlich stattgefunden hat.
%UND DA DIE INVERSE TRANSFORMATION ZUM AUSLÖSEN VON AKTIONEN NUR DIE OBERSTE EBENE DES ZUSTANDSRAUMES ANSCHAUT UND ÜBERPRÜFT.
% An dieser Stelle müsste feingranularer überprüft werden, dass allerdings nicht funktioniert. Heißt: aktuell werden nur die 
%Änderungen auf der obersten Ebene überprüft das Absteigen in Objecte ist dabei nicht möglich, da das Object im Framework erst zur Laufzeit bekannt ist.

% IST ABER AUCH NICHT NÖTIG, DA ALLE 
%NOTWENDIGEN GEGENSTÄNDE ÜBER EINEN ODER MEHRERE ATTRIBUTE IM ZUSTANDSRAUM ABGEBILDET WERDEN KÖNNEN. BSP.? 
% UND DIE ATTRIBUTE, BZW. DIE FELDER ÜBER DIE TRANSFORMATION GEBUNDEN WERDEN. 

%Grenzen 
%Modellierungsempfehlungen - Regel, Bedingung und Zustandraum 
%Worauf man achten soll, wenn man ein neues Szenario abbildet. 
%Modellierungsmöglichkeiten des Frameworks.