\chapter{Einleitung}
\label{chap:Einleitung}
    Die folgende Master-Thesis befasst sich mit der Konzepterstellung einer zentralen Steuerzentrale, die 
    dem Entwickler die formalen Interaktionen, weitere Funktionen hinzuzufügen, erleichtern soll. Hierfür werden
    bereits bestehende Plattformen für \acl{SH} analysiert und daraus ein Konzept erstellt, die den Anforderungen 
    entsprechend einen größeren Mehrwert in der Weiterentwicklung der Plattform bietet. Die Umsetzung des ausgearbeiteten 
    Konzeptes wird nur in sehr geringem Maß behandelt.
    \\
    \linebreak
    In diesem Teil der Arbeit wird auf die Motivation des Themas eingegangen. Darüber hinaus 
    werden sowohl die Forschungsfragen als auch die Zielsetzung der Arbeit genauestens dargelegt. Darauf 
    folgend findet eine Übersicht über die Arbeit im Gesamten statt, mit der die Inhalte angerissen werden. 
    Eine nähere Betrachtung des Standes der Technik untermauert die Beweggründe dieser Themenwahl und 
    Ausarbeitung dessen. 

\section{Motivation}
\label{sec:Motivation}
    Jede neu entwickelte Technologie durchlebt im Laufe der Entstehung und Publikation ein enormes Aufsehen. 
    So lange bis diese Technik eine standardisierte Verwendung in der Gesellschaft findet oder sich als 
    unpraktikabel erweist und nicht weiter vorangetrieben oder eingestellt wird. Es wird in der Zeit des 
    Aufkommens und der Forschung viel darüber fantasiert, debattiert und geplant, ohne jedoch die Ausmaße und 
    Resultate der Forschungen und Praktiken abwägen zu können. Durch fehlende Erfahrung und nicht ausgereifte 
    Konzepte werden Höhepunkte und Illusionen erwartet, die zu diesem Zeitpunkt technisch nicht umsetzbar sind. 
    Um solche kühnen Versprechungen und Übertreibungen, sogenannte Hypes, die jede neue technologische Idee 
    mit sich bringt, von dem zu differenzieren was wirtschaftlich umsetzbar ist, werden bestimmte Phasen 
    der Entwicklung durchlaufen. \cite{gartner.2022m}
    \\
    \linebreak
    Die oben erwähnten Phasen der Entwicklung sind in einem sogenannten Hype-Zyklus, engl. Hype-Cycle, dargestellt. 
    Dieser Zyklus ist ein visualisiertes Modell, das die Entwicklung einer neuen Technologie von der Innovation 
    und Entstehung über die Forschung und Umsetzung bis hin zur ausgereiften Marktfähigkeit repräsentiert und so 
    diese Phasen der Entwicklung versinnbildlicht.  
    \\
    Entwickelt wurde der Hype Cycle von der Gartner Inc. Forschungsgruppe. Durch die Mitarbeiterin 
    Jackie Finn wurden die Definitionen der Entwicklungsphasen\footnote{Die Entwicklungsphasen der Gartner Inc. ist unter folgender URL zu finden: "\url{https://www.gartner.com/en/research/methodologies/gartner-hype-cycle}"} 
    geprägt. Diese sind wie folgt in fünf Phasen 
    dargestellt:
    \begin{enumerate}
        \item \textit{Innovationsauslöser, engl. Innovation Trigger}: Ein potentieller technologischer Durchbruch 
        löst die Dinge aus. Frühe Proof-of-Concept (PoC) Ansätze und ein großes Medieninteresse lösen eine 
        erhebliche Publizität aus. Oft gibt es keine brauchbaren Produkte und die Marktreife ist nicht 
        bewiesen. \cite{gartner.2022m}
        \item \textit{Höhepunkt überhöhter Erwartungen, engl. Peak of Inflated Expectations}: Frühe Publizität 
        bringt eine Reihe von Erfolgsgeschichten hervor – oft begleitet von zahlreichen Misserfolgen. 
        Einige Unternehmen ergreifen Maßnahmen; viele nicht. \cite{gartner.2022m}
        \item \textit{Trog der Ernüchterung, engl. Trough of Disillusionment}: Das Interesse schwindet, da 
        Experimente und Implementierungen nicht liefern. Hersteller der Technologie reißen es heraus 
        oder scheitern. Investitionen werden nur fortgesetzt, wenn die überlebenden Anbieter ihre Produkte 
        zur Zufriedenheit der frühzeitigen Anwender verbessern. \cite{gartner.2022m}
        \item \textit{Steigung der Erleuchtung, engl. Slope of Enlightenment}: Mehr Beispiele dafür, wie 
        die Technologie dem Unternehmen zugute kommen kann, beginnen sich zu herauszukristallisieren und 
        werden allgemeiner verstanden. Produkte der zweiten und dritten Generation erscheinen von den 
        Technologieanbietern. Mehr Unternehmen finanzieren Pilotprojekte; Konservative Unternehmen 
        bleiben vorsichtig. \cite{gartner.2022m}
        \item \textit{Plateau der Produktivität, engl. Plateau of Productivity}: Mainstream-Akzeptanz beginnt 
        sich abzuheben. Kriterien zur Bewertung der Lebensfähigkeit des Anbieters sind klarer definiert. 
        Die breite Markteinsetzbarkeit und Relevanz der Technologie zahlen sich eindeutig aus. \cite{gartner.2022m}
    \end{enumerate}
    Nachdem ein innovativer Gedanke den \textit{Höhepunkt überhöhter Erwartungen} passiert hat, z.B. die 
    Revolutionierung der Softwareentwicklung oder Szenarien, wie z.B. die Vollautomatisierung eines Gebäudes 
    oder Service-Roboter die uneingeschränkt interagieren können, die man in der Form nur aus Science-Fiction Filmen kennt, 
    folgt der \textit{Trog der Ernüchterung}. In Folge dessen wird festgestellt, dass die Erwartungen nicht 
    in Gänze übertragbar sind, bzw. nur zu einem geminderten Teil in die Realität umgesetzt werden können 
    und der verfolgte Gedanke an Interesse verliert. Nach erneutem Aufgriff der Technologie findet eine realistischere 
    Beurteilung der Innovation statt, die dazu beiträgt, dass die Technologie wieder an Interesse gewinnt. Die 
    objektive und realitätsnahe Betrachtungsweise formt ein neues und realistisches Bild der Potentiale, als auch 
    der Grenzen. Mit dem neu gewonnenen Maßstab geht die ehemals neue innovative Idee in eine routinierte Technologie über, 
    die an Anerkennung gewinnt und in der breiten Masse akzeptiert wird. Die Technologie erfährt mit steigender 
    Zuwendung eine stetigere Weiterentwicklung, die dann zu einer Community geformt wird. Mit der Erreichung dieses Status 
    befindet sich die Innovation, bezogen auf den Hype Cycle, in der letzten Phase, dem \textit{Plateau der Produktivität}, 
    und bestätigt so die Marktreife. Dieser Zeitpunkt löst die Zukunftsvision auf und es handelt sich um eine am Markt 
    etablierte Technologie.
    \\ 
    \linebreak
    Zum aktuellen Zeitpunkt befindet sich die Technologie rund um Plattformen für intelligente Geräte im privaten 
    Bereich, engl. \textit{\ac{SH}} oder \textit{Connected Home}, im Anfangsstadium der letzten Phase, dem sogenannten 
    \textit{Plateau der Produktivität}. Mit zunehmender Akzeptanz werden im Umfeld des Internets der Dinge, engl. 
    \textit{\acl{IoT}}, stetig Szenarien entwickelt, die das Wachstum und die Verwendung von solchen Plattformen vorantreibt. 
    Mit einer immer tiefer gehenden Forschung und Umsetzung von Anwendungsbeispielen werden Bereiche offenbart, die 
    eine solche Plattform im privaten als auch im geschäftlichen Umfeld immer attraktiver gestaltet. Mit steigender  
    Konnektivität und Kompatibilität mehreren Geräten und Gegenständen können Bereiche und Szenarien, wie die 
    Steuerung von Service-Robotern, umgesetzt werden. Der jetzige technologische Fortschritt und die über die Forschungsjahre 
    gesammelten Erfahrungen bringt das Segment der intelligenten Geräte der \acs{IoT} den ursprünglich angedachten 
    Visionen und Ideen näher, sodass ein weiterer Ausbau dieser Technologie und dessen Anwendung stattfindet und sich 
    vollständig in den Markt etabliert. Der finale Schritt der endgültigen Marktreife ist ein faszinierender und wichtiger Grund 
    für meine Motivation, mich dieser Technologie und der dahinterstehenden Theorie zu widmen.
    % !!! Weiterer Grund zur Motivation: Dem Entwickler mehrere Freiheiten bei der Umsetzung von Anwendungsfällen oder Regeln zu 
    % !!! schaffen und die Regeldefinition nicht über ein Zusätzliches Framework abzubilden, sondern eine klar Struktur 
    % !!! vorzugeben, die klar definiert, welche grundlegenden Komponenten zu Erweiterung notwendig sind. Weitere Anpassungen oder 
    % !!! Anwendungsfälle verantwortet der Entwickler selbst. (Einbindung von APIs o.ä.)
    %
    %\\
    %Mit der erzielten Marktreife entstehen Produkte und Lösungen, die bestimme Teile der anfänglichen Idee abdecken. Mit 
    %zunehmender Entwicklung und anfallenden Anforderungen, werden viele Produkte zu groß und haben dadurch die grundlegende 
    %Konzeption und Architektur nicht vorausschauend entwickelt. Daher ist die anfängliche Überlegung und Konzeption essentiell.
    %\\
    %Daher ist ein weiterer Punkt meiner Motivation den Schritt zu gehen, ein Konzept zu entwickeln, dass die Erweiterung eines 
    %solchen Systems basierend auf der Konzeptgrundlage vereinfacht und so die Nutzung für Entwickler zur Weiterentwicklung verbessert.
    %
    % !!! Ein weiterer Punkt meiner Motivation ist die Vereinfachung der Erweiterung einer solchen Zentrale. 
    %Somit soll dem Entwickler bei einer stetigen Erweiterung der Plattform Zeit und Aufwand erleichtert werden. Dadurch können weitere 
    %Anwendungsszenarien und Objekte integriert werden, ohne einen zu großen Entwicklungsaufwand zu erzeugen.   
    \\ 
    \linebreak
    Die Einsatzgebiete von intelligenten Geräten, beziehungsweise die Verwendung einer Kompaktlösung beschränkt sich räumlich 
    auf Gebäude, Häuser und Wohnungen, bieten trotz dessen viele Verwendungs- und Einsatzmöglichkeiten. Diese sehen wie folgt aus: 
    \begin{itemize}
        \item Komfort
        \item Entertainment
        \item Überwachung und Sicherheit
        \item Steuerung von Prozessen
        \item Management von Automationen
    \end{itemize}    
    Neben der Affinität von \acl{SH} zum \acl{IdD} und der damit einhergehenden Technologie bringt diese Vorteile mit sich, wie z.B. 
    die Modernisierung von Wohn- und Bürogebäuden und die angestrebte Verbesserung der Lebensqualität. 

\section{Forschungsfragen}
\label{sec:forschungsfragen}
    %-	Wie kann man die Usability von SmartHome-Plattformen verbessern, so dass die formalen Interaktionen der Softwareentwickler schneller sind?
        %o	Welche sind die Usability-Probleme der existierenden Plattformen? 
        %o	Welche Anforderungen würden für eine Verbesserung einer solchen Plattform gelten?
        % Nicht Bestandteil der eigentlichen Forschungsfrage: 
        % -Reicht eine Kommunikationsschnittstelle als Ausgangspunkt für alle Verbindungen?

    % - Wie kann man die Usability von Smart Home - Plattformen optimieren, sodass die formalen Interaktionen der Softwareentwickler schneller sind? 
    %  - Wie kann ein Framework vorgegeben werden, dass den Entwickler (Anwender) bei der Implementierung von Regeln (und der Erweiterung um 
    %    anwenderbasierte Funktionalitäten) nicht einschränkt?

\section{Zielsetzung der Arbeit}
\label{sec:zielsetzung}
    %TBD Wird zum Schluss geschrieben. 
    %Programmierseitige Möglichkeit um Regeln umzusetzen.
    Das Ziel der Arbeit ist die Erarbeitung eines Konzeptes und die prototypische Implementierung des Frameworks, welches dem Softwareentwickler das 
    Umsetzen von Regeln und zu automatisierende Prozesse im Rahmen der Programmierung erleichtert und in der Ausprägung 
    der Regeln nicht einschränkt. Dabei wird sich an bestehenden Produkten orientiert, die versuchen, diese 
    Automatisierungen für den Nutzer vereinfacht darzustellen. Die aktuellen Praktiken setzen trotz dessen eine starke Lernkurve voraus. 
    %und lässt den Nutzer Regeln und Prozesse nur beschränkt einfach abbilden. 
    Mit dem Framework soll dem Anwender eine Struktur an die Hand gegeben werden, mit der Regeln und Prozesse innerhalb eines 
    smarten Büros einfach abgebildet werden können. Die Abarbeitung von Regeln und Prozessen sollen über das Framework 
    koordiniert werden, sodass der Anwender sich um die Regeldefinition und um die abzubildende Umgebung durch Geräte, Zustände und weitere 
    potentielle Mittel innerhalb eines smarten Büros, kümmern muss. 
    %"Anleitung" dem Entwickler an die Hand zu geben, dass Prozesse schnell umgesetzt werden können und jedoch 
    % keinerlei Einschränkungen hat, da mit Java (Spring Boot) keine Grenzen gesetzt sind und mit dem Framework auch 
    %keine Grenzen gesetzt werden.

    %%%%%%%%%%%%%%%%%%%%%%%%%%%%%%%%%%%%%%%%%%%%%%%%%%%%%%%%%%%%%

    % ZIEL DES KONZEPTES: Ein Framework für Entwickler bereitzustellen, welches die Mächtigkeit für den Entwickler offen lässt, nicht einschränkt 
    % und dennoch Konfiguration und Ausführung umsetzt. Der Entwickler muss lediglich den Zustandsraum, die MQTT-Topics und die Regeln definieren.
    % Der Entwickler bekommt ein Framework in die Hand, welches die Umsetzung von Prozessen in einem smarten Büro ermöglicht. Das Framework kümmert sich um die 
    % Organisation und die Ausführung der Regeln. Die Richtigkeit der Regeln und des Zustandsraumes muss der Entwickler sicherstellen. 
    % Die Kommunikation über MQTT ist nur eine Möglichkeit. Des Setup wird wegabstrahiert 


    %%%%%%%%%%%%%%%%%%%%%%%%%%%%%%%%%%%%%%%%%%%%%%%%%%%%%%%%%%%%%

\section{Forschungsstrategie und Forschungsmethoden}
\label{sec:forschungsstrategie}
    Dieser Abschnitt der Arbeit widmet sich der Forschungsstrategien und der darauf angewendeten Forschungsmethoden. 
    Hierbei werden die Strategien kurz erläutert und die Methoden skizziert. Die Anwendung der Strategien findet zu 
    späterem Zeitpunkt statt. 
    \\
    Diese Arbeit ist in die drei Phasen aufgeteilt, Forschung, Konzeption und Evaluation. Zuerst erfolgt durch ein 
    systematisches Literaturreview eine Analyse des aktuellen Stands der Technik, welche die Forschungsfragen aufgreift und 
    anhand dessen den aktuellen Stand identifiziert. Schwerpunkt dabei liegt auf der einfachen Handhabung der formalisierten 
    Interaktionen eines Softwareentwicklers bei der Weiterentwicklung, bzw. bei der Ergänzung einer bestehenden 
    Softwarelösung, um weitere Anwendungsfälle abzudecken. 
    In der Phase der Anforderungserhebung werden Experteninterviews durchgeführt, mit denen die Anforderungen des Produktes, 
    für welches ein Konzept erstellt wird, identifiziert werden. Ergänzend dazu werden im Rahmen der Arbeit weitere 
    Anforderungen durch das \ac{RE} und der Anwendung des \textit{user-centered design}-Prinzips\footnote{Iterativer Prozess zur Ermittlung von Anforderungen, die nutzerorientiert aufgestellt werden. \url{https://www.interaction-design.org/literature/topics/user-centered-design} Abgerufen am 08.05.2022} 
    sowie des \textit{target group analysis}-Ansatzes (\ref{sec:zielgruppenanalyse}) ermittelt. Dabei wird die 
    Nutzerorientierung auf den Softwareentwickler ausgelegt. 
    Zur anschließenden Evaluation wird ein Usability Test durchgeführt und darauf erneut ein Experten Interview, damit 
    Eindrücke über das Konzept entstehen und die Erfahrungen der Experten während des Usability Tests 
    in die Beantwortung der Forschungsfrage mit einfließen.
    \\
    Die soeben genannten Forschungsmethoden werden im Nachfolgen- den näher erklärt.

    \subsection{Experteninterview}
    \label{subsec:experteninterview}
        Zu Anfang wird bei einem Experten Interview der Hintergrund des Interviews erläutert. Nachdem die interviewende 
        Person den Grund dargelegt hat, stellt ein Forscher einem Experten ausgewählte Interview-Fragen, die für die 
        Forschung relevant sind und dazu beitragen die Forschungsfrage zu beantworten.
        \\
        \linebreak
        Die Interview-Fragen können als offene oder geschlossene Fragen formuliert werden, je nachdem ist eine beliebige Reihe 
        an Antworten möglich oder nur eine begrenzte Anzahl an Antworten \cite{robson2002real}. Der Verlauf des Interviews 
        kann von dem Forscher selbst bestimmt werden. Sind nur bestimmte Antworten gewünscht, so können die Fragen strukturiert und 
        geschlossen formuliert werden. Ebenso gibt es den semi-strukturierten und den unstrukturierten Ansatz, bei dem das Interview 
        offen gestaltet werden kann. Der Semi-Strukturierte Ansatz eignet sich, wenn spezielle Fragen adressiert werden, bzw. eine 
        Reihenfolge festgelegt ist, sodass der Interviewer eine klare Reihenfolge hat. Mit dem unstrukturierten Vorgehen wird 
        ein völlig offenes Interview angestrebt, bei dem der Verlauf abhängig von dem Inhalt der Konversation ist.

    \subsection{Systematisches Literaturreview}
    \label{subsec:systematischesliteraturreview}
        Mit einem systematischen Literaturreview wird zur evidenzbasierten Identifizierung, Bewertung und Interpretation  
        von bestehender Literatur eine wissenschaftliche Methode angewendet, mit der Fragestellungen zu einer bestimmten Thematik 
        herausgearbeitet werden können. Mithilfe diesen Ansatzes soll die Erzielung einer Schlussfolgerung zu einem untersuchten Objekt 
        ermöglicht werden. 
        Die Methodik des systematischen Literaturreviews orientiert sich an den Richtlinen von Kitchenham et al.. Diese ist in drei 
        aufeinander aufbauende Schritte gegliedert. Zu Beginn erfolgt die Planung, anschließend die Durchführung und abschließend die 
        Dokumentation und Offenlegung der Ergebnisse \cite{Kitchenham2007}.

    \subsection{Usability-Test}
    \label{subsec:usabilitytests}
        Ein Usability-Test ist eine Methode zur Testung von Prototypen oder Arbeitsversionen von Computerschnittstellen \cite{LAZAR2017263}. 
        Das Testen der Nutzbarkeit kann durch vorab definierte Aufgaben, die durch Probanden und Benutzer in einer dafür vorgesehenen 
        Umgebung durchgeführt werden, erfolgen. Die Aufgaben können je nach Anwendungsfeld oder -kontext unterschiedlich ausfallen. 
        Ebenso können solche Tests auch dazu beitragen, um physische Interaktionen mit bspw. Geräten zu bewerten. 
        \\
        Alle Ansätze für Usability-Tests haben ein grundlegendes Ziel: Die Qualität einer Schnittstelle zu verbessern und den 
        dafür vorgesehenen Anwendern diese Benutzung so einfach wie möglich zu gestalten \cite{LAZAR2017263}. Während diese Tests optimaler 
        Weise Schnittstellenfehler aufdecken, die den Benutzern Schwierigkeiten bereiten, ist gleichzeitig rauszufinden, welches Schnittstellendesign 
        gut funktioniert, um dieses für Zukünftige Entwicklungen beizubehalten. 
        \\
        \linebreak
        Zur Kategorisierung von durchgeführten Tests, werden Werkzeuge verwendet, die das Verhalten und die Meinungen der Benutzer 
        messbar macht. Eines dieser Werzeuge ist das \textit{System Usability Scale (SUS)}\footnote{System Usability Scale. \url{https://www.usability.gov/how-to-and-tools/methods/system-usability-scale.html} Besucht am 03.07.2022}. 
        Dabei handelt es sich um einen einfachen 
        technologieunabhängigen Fragebogen, anhand dessen die Gebrauchstauglichkeit eines Systems bewertet werden kann. Dieser 
        ist eine Praktik zur quantitativen Analyse der Nutzbarkeit. Der Umfang umfasst zehn Fragen nach der Likert-Skala \cite{likert1932technique}.

\section{Aufbau der Arbeit}
\label{sec:aufbau}
    Die vorliegende Master-Thesis gliedert sich nach den soeben genannten einleitenden Information im Aufbau in insgesamt 
    zehn Kapitel. Das erste Kapitel (\ref{chap:Einleitung}) beschreibt die Motivation (\ref{sec:Motivation}), welche die 
    Intension kundtut, diese Thematik rund um \acs{IoT} und Smart Home zu bearbeiten. Darauf folgend werden die 
    Forschungsfragen (\ref{sec:forschungsfragen}), die im Rahmen der Thesis behandelt werden, erläutert. Nach der 
    Beschreibung der Forschungsfragen wird im anknüpfenden Abschnitt (\ref{sec:zielsetzung}) die Zielsetzung der 
    Arbeit erläutert. Hierbei werden zusätzliche Schwerpunkte und Ziele aufgegriffen. Abschließend wird das Unternehmen, in der 
    die Thesis geschrieben wird, hervorgehoben und deren Absichten in Verbindung mit Innovationen beleuchtet. 
    \\
    \linebreak
    Das Kapitel (\ref{chap:grundlagen}) widmet sich den essentiellen und wichtigen Grundlagen dieser Arbeit. Zu Anfang wird dem 
    Leser der Terminus des \acl{IoT} (\ref{sec:iot}) offenbart, um zum Teil den Kontext im Bezug zu dieser Arbeit zu begreifen, 
    gefolgt von einer Einführung in die Thematik des \acl{SH} (\ref{sec:smartHome}), der Problematik der Begriffsdefinition, der 
    historischen und kontinuierlichen Entwicklung und mit den Zielen, die mit der Verwendung einer Smart Home Lösung bewältigt 
    werden sollen. Mit dem Verständnis der übergeordneten Begriffe, \acs{IoT} und \acl{SH}, werden Technologien 
    (\ref{sec:technologien}) aufgegriffen, die im Rahmen dieser Arbeit erwähnenswert sind und verwendet werden. %Um auf die Vielfältigkeit 
    %von der Umsetzung eines \acl{SH} einzugehen und einen Teilaspekt der Anforderungen Abzudecken, wird ebenso auf Service-Roboter 
    %(\ref{sec:roboter}) eingegangen. 
    Abschließend werden in Kapitel (\ref{chap:grundlagen}) die Softwarelösungen, Home Assistant 
    und openHAB (\ref{sec:homeassistant} \& \ref{sec:openhab}), dargestellt. Diese dienen zur Grundlage für die Evaluation als 
    auch zur Gegenüberstellung der Lösungen in Kapitel (\ref{chap:evaluation}) Diskussion und Evaluation. 
    \\
    \linebreak
    Die theoretischen und methodischen Hintergründe sowie den Stand der Technik wird in Kapitel (\ref{chap:technikStand} )
    angesprochen. Dieser Teil enthält Beschreibungen, Forschungen und aktuelle Erkenntnisse über Technologien, die im Umfeld der 
    Smart Home Anwendungen innerhalb des \acs{IoT} verwendet werden. Zudem werden in Zusammenhang der Erkenntnisse und 
    Möglichkeiten der Technologie die Szenarien dargestellt. 
    \\
    \linebreak
    Kapitel (\ref{chap:anforderungsanalyse}) befasst sich mit den Anforderungen, engl. Requirements, die für die 
    eigentliche Konzeption relevant sind. Innerhalb dieses Kapitels wird anhand von Informationen und den umzusetzenden 
    Szenarien die Anforderungen für die Konzeption erarbeitet. Hierbei werden aus der Praxis bekannte Verfahren verwenden, um 
    die Anforderungen zu definieren. Mittels den zugrundeliegenden Anforderungen wird im nachfolgenden Schritt die eigentliche 
    Konzeption dargelegt.
    \\
    \linebreak 
    Nach Aufbereitung der Anforderungen durch das sogenannte Anforderungsmanagement, engl. \textit{Requirements Engineering},
    wird in Kapitel (\ref{chap:konzept}) das Konzept erarbeitet, welches als Grundlage für die prototypische Implementierung und 
    Umsetzung des Konzepts dient. Das Konzept befasst sich mit den Anforderungen und setzt diese ein, um die Organisation des 
    Systems in Komponenten, deren Beziehungen zueinander und zur Umgebung sowie deren Prinzipien zu definieren. Zum Ende des 
    Konzepts steht eine Architektur, die sich aus den Anforderungen und auch aus den Analysen der eigentlichen Forschungsfrage 
    abzeichnet.
    \\
    \linebreak
    In Kapitel (\ref{chap:umsetzung}) wird die Umsetzung des Konzepts skizziert. Darunter welche Problem während der Implementierung 
    auftraten als auch deren Lösungsfindung. Ebenso wird hier aus praktischer Sicht auf die Architektur geschaut, welche Komponenten,  
    Bibliotheken und zusätzliche Systeme, engl. \textit{Frameworks}, verwendet wurden. Das erzielte Ergebnis und Resultat wird 
    abschließend zusammengefasst und neutral betrachtet.
    %\\ 
    %\linebreak
    %Das Ergebnis wird aus objektiver Sicht in dem darauf folgenden Kapitel (\ref{chap:ergebnis}) erläutert. 
    \\
    \linebreak
    Nach Abschluss der Umsetzung und dessen Ergebnisdokumentation befasst sich das Kapitel (\ref{chap:evaluation}) mit der Diskussion 
    und Evaluation. Hier findet eine Analyse des Konzepts sowie deren Umsetzung und objektive Betrachtung statt. Anschließend werden 
    Vergleiche zwischen der Eigenentwicklung und bereits bestehender Softwarelösungen, die im Grundlagenkapitel aufgefasst werden, 
    aufgestellt und bewertet. %Testdurchführung und anschließende Evaluation (Experten Interview)
    \\
    \linebreak
    Im vorletzten Teil, Kapitel (\ref{chap:fazit}), wird ein Fazit aus den Erkenntnissen und Ergebnissen gezogen. Dieses Schlussresümee 
    führt nochmals die Höhepunkte sowie eine eigene Einschätzung auf. 
    \\
    \linebreak
    Zum Abschluss der Thesis wird in Kapitel (\ref{chap:ausblick}) ein Ausblick gegeben. Dieser gibt Aufschluss darüber, welche 
    Erweiterungsmöglichkeiten es für die in dieser Thesis erfolgten Arbeit gibt und wie innovativ sich dieser Grundbaustein in Zukunft 
    erweisen könnte. 

%\section{CGI}
%\label{sec:cgi}