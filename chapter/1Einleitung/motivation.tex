\chapter{Einleitung}
\label{chap:Einleitung}
    In diesem Teil der Master-Thesis wird auf die Motivation des Themas eingegangen. Darüber hinaus 
    werden sowohl die Forschungsfragen als auch die Zielsetzung der Arbeit genauestens dargelegt. Darauf 
    folgend findet eine Übersicht über die Arbeit im Gesamten statt, mit der die Inhalte angerissen werden. 
    Eine nähere Betrachtung des Standes der Technik untermauert die Beweggründe dieser Themenwahl und 
    Ausarbeitung dessen. 

\section{Motivation}
\label{sec:Motivation}
    Jede neu entwickelte Technologie durchlebt im Laufe der Entstehung und Publikation ein enormes Aufsehen. 
    So lange bis diese Technik eine standardisierte Verwendung in der Gesellschaft findet oder sich als 
    unpraktikabel erweist und nicht weiter vorangetrieben oder eingestellt wird. Es wird in der Zeit des 
    Aufkommens und der Forschung viel darüber fantasiert, debattiert und geplant, ohne jedoch die Ausmaße und 
    Resultate der Forschungen und Praktiken abwägen zu können. Durch fehlende Erfahrung und nicht ausgereifte 
    Konzepte werden Höhepunkte und Illusionen erwartet, die zu diesem Zeitpunkt technisch nicht umsetzbar sind. 
    Um solche kühnen Versprechungen und Übertreibungen, sogenannte Hypes, die jede neue technologische Idee 
    mit sich bringt, von dem zu differenzieren was wirtschaftlich umsetzbar ist, werden bestimmte Phasen 
    der Entwicklung durchlaufen. \cite{gartner.2022m}
    \\
    \linebreak
    Die oben erwähnten Phasen der Entwicklung sind in einem sogenannten Hype-Zyklus, engl. Hype-Cycle, dargestellt. 
    Dieser Zyklus ist ein visualisiertes Modell, das die Entwicklung einer neuen Technologie von der Innovation 
    und Entstehung über die Forschung und Umsetzung bis hin zur ausgereiften Marktfähigkeit repräsentiert und so 
    diese Phasen der Entwicklung versinnbildlicht.  

