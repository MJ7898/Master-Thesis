\chapter{Einleitung}
\label{chap:Einleitung}
    Die folgende Masterthesis befasst sich mit der Konzepterstellung einer Steuerzentrale, die 
    dem Entwickler die formalen Interaktionen, wie z.B. weitere Funktionen hinzuzufügen, erleichtern sollen. Hierfür werden
    bereits bestehende Plattformen für \acl{SH} analysiert und daraus ein Konzept erstellt, das den Anforderungen 
    entsprechend das Implementieren von weiteren Anwendungsfällen einfach handhaben lässt. 
    \\
    \linebreak
    In diesem Teil der Arbeit wird auf die Motivation des Themas eingegangen. Darüber hinaus 
    werden sowohl die Forschungsfragen als auch die Zielsetzung der Arbeit genauestens dargelegt. Darauf 
    folgend findet eine Übersicht über die Arbeit im Gesamten statt, mit der die Inhalte angerissen werden. 
    Eine nähere Betrachtung des Standes der Technik untermauert die Beweggründe für diese Themenwahl und 
    ihre Ausarbeitung. 

\section{Motivation}
\label{sec:Motivation}
    Jede neu entwickelte Technologie durchlebt im Laufe der Entstehung und Publikation ein enormes Aufsehen, 
    so lange, bis diese Technik eine standardisierte Verwendung in der Gesellschaft findet oder sich als 
    unpraktikabel erweist und nicht weiter vorangetrieben oder eingestellt wird. Zu Beginn des Aufkommens der Idee 
    und der Forschung wird viel darüber fantasiert, debattiert und geplant, ohne jedoch die Ausmaße, Resultate und 
    Umsetzungen abwägen zu können. Durch fehlende Erfahrungen und nicht ausgereifte 
    Konzepte werden Höhepunkte und Illusionen erwartet, die zu diesem Zeitpunkt technisch womöglich nicht umsetzbar sind. 
    Solche Versprechungen und Übertreibungen, sogenannte Hypes, durchlaufen bestimmte Phasen in ihrer Entwicklung, wonach 
    sich eine aussichtsreiche technologische Idee von einer wirtschaftlich nicht umsetzbaren differenziert. 
    \\
    \linebreak
    Die oben erwähnten Phasen der Entwicklung sind in einem sogenannten Hype-Zyklus, engl. Hype Cycle \cite{gartner.2022m}, dargestellt. 
    Dieser Zyklus ist ein visualisiertes Modell für den Verlauf der Entwicklung einer neuen Technologie von der Innovation 
    und Entstehung bis hin zur Forschung, Umsetzung und zur ausgereiften Marktfähigkeit. 
    \\
    Erfunden wurde der Hype Cycle von der Gartner Inc. Forschungsgruppe. Die Definitionen der 
    Entwicklungsphasen\footnote{Die Entwicklungsphasen der Gartner Inc. ist unter folgender URL zu finden: "\url{https://www.gartner.com/en/research/methodologies/gartner-hype-cycle}"} 
    wurden durch die Mitarbeiterin Jackie Finn geprägt. Diese sind wie folgt in fünf Phasen untergliedert: 
    \begin{enumerate}
        \item \textit{Innovationsauslöser, engl. Innovation Trigger}: Ein potenzieller technologischer Durchbruch 
        löst die Dinge aus. Frühe Proof-of-Concept (PoC) Ansätze und ein großes Medieninteresse lösen eine 
        erhebliche Publizität aus. Oft gibt es keine brauchbaren Produkte und die Marktreife ist nicht 
        bewiesen. \cite{gartner.2022m}
        \item \textit{Höhepunkt überhöhter Erwartungen, engl. Peak of Inflated Expectations}: Frühe Publizität 
        bringt eine Reihe von Erfolgsgeschichten hervor, – oft begleitet von zahlreichen Misserfolgen. 
        Einige Unternehmen ergreifen Maßnahmen; viele nicht. \cite{gartner.2022m}
        \item \textit{Trog der Ernüchterung, engl. Trough of Disillusionment}: Das Interesse schwindet, wenn 
        Experimente und Implementierungen die Erwartungen nicht erfüllen. Hersteller der Technologie schaffen den Durchbruch 
        oder scheitern. Investitionen werden nur fortgesetzt, wenn die übrig gebliebenen Anbieter ihre Produkte 
        zur Zufriedenheit der Kunden verbessern. \cite{gartner.2022m}
        \item \textit{Steigung der Erleuchtung, engl. Slope of Enlightenment}: Mehr Beispiele dafür, wie 
        die Technologie dem Unternehmen zugutekommen kann, beginnen sich herauszukristallisieren und 
        werden allgemeiner verstanden. Produkte der zweiten und dritten Generation erscheinen von den 
        Technologieanbietern. Mehr Unternehmen finanzieren Pilotprojekte; konservative Unternehmen 
        bleiben vorsichtig. \cite{gartner.2022m}
        \item \textit{Plateau der Produktivität, engl. Plateau of Productivity}: Mainstream-Akzeptanz beginnt 
        sich abzuheben. Kriterien zur Bewertung der Rentabilität des Anbieters sind klarer definiert. 
        Die breite Markteinsetzbarkeit und Relevanz der Technologie zahlen sich eindeutig aus. \cite{gartner.2022m}
    \end{enumerate}
    Nachdem ein innovativer Gedanke, z.B. die 
    Vollautomatisierung eines Gebäudes 
    oder eines uneingeschränkt interagierenden Service-Roboters, den \textit{Höhepunkt überhöhter Erwartungen} passiert hat, 
    folgt der \textit{Trog der Ernüchterung}. In Folge dessen wird festgestellt, dass die Erwartungen zum aktuellen Zeitpunkt nicht 
    vollständig bzw. nur zu einem geminderten Teil in die Realität umgesetzt werden können 
    und dadurch an Interesse verlieren. Wird die Technologie erneut aufgegriffen, findet eine realitätsbezogenere 
    Beurteilung der Innovation statt, die dazu beitragen kann, dass die Technologie wieder an Interesse gewinnt. Die 
    objektive und realitätsnahe Betrachtungsweise kann ein neues und realistischeres Bild der Potenziale, aber auch 
    der Grenzen aufzeigen. Mit dem neu gewonnenen Maßstab kann die ehemals neue innovative Idee in eine routinierte Technologie übergehen, 
    die eventuell an Anerkennung gewinnt und in der breiten Masse akzeptiert wird. Die Technologie erfährt mit steigender 
    Zuwendung eine stetigere Weiterentwicklung, die dann zu einer Community geformt wird. Mit der Erreichung dieses Status 
    befindet sich die Innovation, bezogen auf den Hype Cycle, in der letzten Phase, dem \textit{Plateau der Produktivität}, 
    und bestätigt so die Marktreife. Dieser Zeitpunkt löst die Zukunftsvision auf und es handelt sich um eine am Markt 
    etablierte Technologie. 
    \\ 
    \linebreak
    Zum aktuellen Zeitpunkt befindet sich die Technologie rund um Plattformen für intelligente Geräte im privaten als auch im unternehmerischen 
    Bereich, engl. \textit{\ac{SH}} oder \textit{Connected Home}, im Anfangsstadium der letzten Phase, dem sogenannten 
    \textit{Plateau der Produktivität}. Mit zunehmender Akzeptanz werden im Umfeld des Internets der Dinge, engl. 
    \textit{\acl{IoT}}, stetig Szenarien entwickelt, die das Wachstum und die Verwendung von solchen Plattformen vorantreibt. 
    Mit einer immer tiefer gehenden Forschung und Umsetzung von Anwendungsbeispielen werden Bereiche eröffnet, die 
    eine solche Plattform im privaten als auch im geschäftlichen Umfeld immer attraktiver werden lässt. Mit steigender  
    Konnektivität und Kompatibilität mehrerer Geräte und Gegenstände können Szenarien und Prozessautomatisierungen, wie die 
    Steuerung von Service-Robotern, umgesetzt werden. Der jetzige technologische Fortschritt und die über die Forschungsjahre 
    gesammelten Erfahrungen bringen das Segment der intelligenten Geräte des \acs{IoT} den ursprünglich angedachten 
    Visionen und Ideen näher, sodass ein weiterer Ausbau dieser Technologie und dessen Anwendung stattfindet und sich 
    vollständig in den Markt etabliert. Der finale Schritt der endgültigen Marktreife ist ein faszinierender und wichtiger Beweggrund 
    für meine Motivation, mich dieser Technologie und der dahinterstehenden Theorie zu widmen. Ein weiterer dazu beitragender Aspekt ist 
    die Möglichkeit der Steuerung und Verknüpfung von Geräten sowie die Automatisierung von Prozessen innerhalb eines intelligenten Büros. 
    Stetig hinzukommende Anwendungsfälle abzudecken und einfach implementieren zu können, einen Ansatz zu schaffen, der einem 
    Softwareentwickler formalisierte Interaktionen ermöglicht und trotzdem Handlungsspielraum für Prozessabbildungen lässt, sind 
    zusätzliche Impulse, warum ich mich dieser Thematik zuwende. 
    % !!! Weiterer Grund zur Motivation: Dem Entwickler mehrere Freiheiten bei der Umsetzung von Anwendungsfällen oder Regeln zu 
    % !!! schaffen und die Regeldefinition nicht über ein Zusätzliches Framework abzubilden, sondern eine klar Struktur 
    % !!! vorzugeben, die klar definiert, welche grundlegenden Komponenten zu Erweiterung notwendig sind. Weitere Anpassungen oder 
    % !!! Anwendungsfälle verantwortet der Entwickler selbst. (Einbindung von APIs o.ä.)
    %
    %\\
    %Mit der erzielten Marktreife entstehen Produkte und Lösungen, die bestimme Teile der anfänglichen Idee abdecken. Mit 
    %zunehmender Entwicklung und anfallenden Anforderungen, werden viele Produkte zu groß und haben dadurch die grundlegende 
    %Konzeption und Architektur nicht vorausschauend entwickelt. Daher ist die anfängliche Überlegung und Konzeption essentiell.
    %\\
    %Daher ist ein weiterer Punkt meiner Motivation den Schritt zu gehen, ein Konzept zu entwickeln, dass die Erweiterung eines 
    %solchen Systems basierend auf der Konzeptgrundlage vereinfacht und so die Nutzung für Entwickler zur Weiterentwicklung verbessert.
    %
    % !!! Ein weiterer Punkt meiner Motivation ist die Vereinfachung der Erweiterung einer solchen Zentrale. 
    %Somit soll dem Entwickler bei einer stetigen Erweiterung der Plattform Zeit und Aufwand erleichtert werden. Dadurch können weitere 
    %Anwendungsszenarien und Objekte integriert werden, ohne einen zu großen Entwicklungsaufwand zu erzeugen.   
    \\ 
    \linebreak
    Die Einsatzgebiete von Kompaktlösungen, Plattformen und Steuereinheiten beziehungsweise von Gateways und intelligenten Geräten zielen räumlich 
    auf Gebäude, Häuser und Wohnungen ab und bieten viele Verwendungs- und Einsatzmöglichkeiten, die stetig ausgebaut und optimiert werden. 
    Diese sehen wie folgt aus: 
    \begin{itemize}
        \item Komfort
        \item Entertainment
        \item Überwachung und Sicherheit
        \item Steuerung von Prozessen
        \item Management von Automatisierungen (Automationen)
    \end{itemize}    
    Neben der Affinität von \acl{SH} zum \acl{IdD} und der damit einhergehenden Technologie bringt diese Vorteile mit sich, wie z.B. 
    die Modernisierung von Wohn- und Bürogebäuden und die routinierte Ausführung und Abarbeitung von Prozessen, die dem Nutzer 
    Aufgaben und Arbeiten abnehmen.
\pagebreak
\section{Forschungsfrage}
\label{sec:forschungsfragen}
    Die in der Arbeit zentral behandelte Forschungsfrage ist wie folgt definiert: 
    \\
    \linebreak 
    \textit{F: Wie kann man die Usability von Prozessautomatisierungen und Regeldefinitionen von SmartHome-Plattformen optimieren, sodass die formalisierten Interaktionen für den Softwareentwickler einfacher in der Handhabung sind?}
    %\\
    %\linebreak
    %Eine Konkretisierung identifiziert konkret die Herausforderungen existierender Plattformen in Bezug auf die Definition von Regeln und Automatisierungen:
    %\\
    %\linebreak
    %\textit{F1.1: Welche sind die Usability-Herausforderungen existierender Plattformen bzw. welche Anforderungen würden für eine Optimierung gelten?}
    %-	Wie kann man die Usability von Prozessautomatisierungen und Regeldefinitionen von SmartHome-Plattformen optimieren, so dass die formalen Interaktionen der Softwareentwickler schneller sind?
        %o	Welche sind die Usability-Probleme der existierenden Plattformen? 
        %o	Welche Anforderungen würden für eine Verbesserung einer solchen Plattform gelten?
        % Nicht Bestandteil der eigentlichen Forschungsfrage: 
        % -Reicht eine Kommunikationsschnittstelle als Ausgangspunkt für alle Verbindungen?
    % - Wie kann man die Usability von Smart Home - Plattformen optimieren, sodass die formalen Interaktionen der Softwareentwickler schneller sind? 
    %  - Wie kann ein Framework vorgegeben werden, dass den Entwickler (Anwender) bei der Implementierung von Regeln (und der Erweiterung um 
    %    anwenderbasierte Funktionalitäten) nicht einschränkt?

\section{Zielsetzung der Arbeit}
\label{sec:zielsetzung}
    %TBD Wird zum Schluss geschrieben. 
    %Programmierseitige Möglichkeit um Regeln umzusetzen.
    Das Ziel dieser Thesis ist die Erarbeitung eines Konzeptes und die prototypische Implementierung des Frameworks, welches dem Softwareentwickler das 
    Umsetzen von Regeln sowie zu automatisierende Prozesse im Rahmen der Programmierung erleichtert, jedoch in der Ausprägung 
    der Regeln nicht einschränkt. Dabei wird sich an bestehenden Produkten orientiert, die versuchen, diese 
    Automatisierungen für den Nutzer vereinfacht darzustellen. Voraussetzung ist trotz allem eine intensive Auseinandersetzung mit der Materie. 
    Mit dem Framework soll dem Anwender eine Struktur an die Hand gegeben werden, mit der Regeln und Prozesse innerhalb eines 
    smarten Büros einfach abgebildet werden können. Die Abarbeitung von Regeln und Prozessen sollen über das Framework 
    koordiniert werden, sodass der Anwender sich ausschließlich um die Regeldefinition und um die abzubildende Umgebung durch Geräte, Zustände und weitere 
    potenzielle Mittel innerhalb eines smarten Büros kümmern muss. 
    %"Anleitung" dem Entwickler an die Hand zu geben, dass Prozesse schnell umgesetzt werden können und jedoch 
    % keinerlei Einschränkungen hat, da mit Java (Spring Boot) keine Grenzen gesetzt sind und mit dem Framework auch 
    %keine Grenzen gesetzt werden.

    %%%%%%%%%%%%%%%%%%%%%%%%%%%%%%%%%%%%%%%%%%%%%%%%%%%%%%%%%%%%%

    % ZIEL DES KONZEPTES: Ein Framework für Entwickler bereitzustellen, welches die Mächtigkeit für den Entwickler offen lässt, nicht einschränkt 
    % und dennoch Konfiguration und Ausführung umsetzt. Der Entwickler muss lediglich den Zustandsraum, die MQTT-Topics und die Regeln definieren.
    % Der Entwickler bekommt ein Framework in die Hand, welches die Umsetzung von Prozessen in einem smarten Büro ermöglicht. Das Framework kümmert sich um die 
    % Organisation und die Ausführung der Regeln. Die Richtigkeit der Regeln und des Zustandsraumes muss der Entwickler sicherstellen. 
    % Die Kommunikation über MQTT ist nur eine Möglichkeit. Des Set-up wird wegabstrahiert 


    %%%%%%%%%%%%%%%%%%%%%%%%%%%%%%%%%%%%%%%%%%%%%%%%%%%%%%%%%%%%%

\section{Forschungsstrategie und Forschungsmethoden}
\label{sec:forschungsstrategie}
    Dieser Abschnitt der Arbeit widmet sich der Forschungsstrategien und der darauf angewendeten Forschungsmethoden. 
    Hierbei werden die Strategien kurz erläutert und die Methoden skizziert. Die Anwendung findet zu 
    späterem Zeitpunkt statt. 
    \\
    Die Struktur ist in vier Phasen Forschung, Konzeption, prototypische Umsetzung und Evaluation aufgeteilt. Zuerst erfolgt durch ein 
    systematisches Literaturreview eine Analyse zum aktuellen Stand der Technik in Bezug auf die Forschungsfrage. 
    Schwerpunkt dabei liegt auf der einfachen Handhabung der formalisierten 
    Interaktionen eines Softwareentwicklers bei der Umsetzung bzw. Ergänzung einer bestehenden 
    Softwarelösung zur Abdeckung weiterer Anwendungsfälle. 
    In der Phase der Anforderungserhebung werden Experteninterviews durchgeführt, mit denen die Anforderungen an das Produkt, 
    für welches ein Konzept erstellt wird, identifiziert werden. Ergänzend dazu werden im Rahmen der Arbeit zusätzliche 
    Anforderungen durch das \ac{RE} und der Anwendung des \textit{user-centered design}-Prinzips\footnote{Iterativer Prozess zur Ermittlung von Anforderungen, die nutzerorientiert aufgestellt werden. \url{https://www.interaction-design.org/literature/topics/user-centered-design} Besucht am 08.05.2022} 
    sowie des \textit{target group analysis}-Ansatzes (\ref{sec:zielgruppenanalyse}) ermittelt. Dabei wird die 
    Nutzerorientierung auf den Softwareentwickler ausgelegt. Basierend auf den Ergebnissen wird das Konzept erstellt und daraufhin die Umsetzung eingeleitet. 
    Zur anschließenden Evaluation wird ein Usability-Test und ein erneutes Experteninterview durchgeführt, damit 
    Eindrücke über das Konzept entstehen und die Erfahrungen der Experten während des Usability-Tests 
    in die Beantwortung der Forschungsfrage mit einfließen.
    \\
    Die soeben genannten Forschungsmethoden werden nachfolgend näher erklärt.

    \subsection{Experteninterview}
    \label{subsec:experteninterview}
        Zu Anfang wird dem Experten der Hintergrund des Interviews erläutert und ihm zum Sachverhalt relevante Fragen gestellt. 
        \\
        \linebreak
        Die Interviewfragen können als offene oder geschlossene Fragen formuliert werden, eine beliebige Reihenfolge an Antworten ist 
        möglich oder nur eine begrenzte Anzahl \cite{robson2002real}. Der Verlauf des Interviews 
        kann von dem Forscher selbst bestimmt werden. Sind nur bestimmte Antworten erwünscht, so können die Fragen strukturiert und 
        geschlossen formuliert werden. 
        \\
        Ebenso gibt es den semi-strukturierten und den unstrukturierten Ansatz, bei dem das Interview 
        offen gestaltet werden kann. Der semi-strukturierte Ansatz eignet sich, um spezielle Fragen zu adressieren bzw. eine 
        Reihenfolge festgelegt ist. Mit dem unstrukturierten Vorgehen wird 
        ein völlig offenes Interview angestrebt, bei dem der Verlauf abhängig vom Inhalt der Konversation ist.
        \\
        Die Durchführung von Experteninterviews zählt zu den qualitativen Forschungsansätzen zum Sammeln von Daten in bestimmten Kontexten. 

    \subsection{Systematisches Literaturreview}
    \label{subsec:systematischesliteraturreview}
        Mit einem systematischen Literaturreview wird zur evidenzbasierten Identifizierung, Bewertung und Interpretation  
        von bestehender Literatur eine wissenschaftliche Methode angewendet, mit der Fragestellungen zu einer bestimmten Thematik 
        herausgearbeitet werden können. Mithilfe dieses Ansatzes soll die Erzielung einer Schlussfolgerung zu einem untersuchten Objekt 
        ermöglicht werden. 
        Die Methodik des systematischen Literaturreviews orientiert sich an den Richtlinien von Kitchenham et al.. Diese ist in drei 
        aufeinander aufbauende Schritte gegliedert. Zu Beginn erfolgt die Planung, anschließend die Durchführung und abschließend die 
        Dokumentation und Offenlegung der Ergebnisse \cite{Kitchenham2007}.

    \subsection{Usability-Test}
    \label{subsec:usabilitytests}
        Ein Usability-Test ist eine Methode zur Testung von Prototypen oder Arbeitsversionen von Computerschnittstellen \cite{LAZAR2017263}. 
        Das Testen der Nutzbarkeit kann durch vorab definierte Aufgaben, die durch Probanden und Benutzer in einer dafür vorgesehenen 
        Umgebung durchgeführt werden, erfolgen. Diese können je nach Anwendungsfeld oder -kontext unterschiedlich ausfallen. 
        Ebenso können solche Tests auch dazu beitragen, physische Interaktionen mit bspw. Geräten zu bewerten. 
        \\
        Alle Ansätze für Usability-Tests haben ein grundlegendes Ziel: die Qualität einer Schnittstelle zu verbessern und sie anwenderfreundlich zu 
        gestalten \cite{LAZAR2017263}. Während diese Tests optimalerweise Schnittstellenfehler aufdecken, die den Benutzern Schwierigkeiten bereiten 
        können, ist gleichzeitig eine Modifizierung des Schnittstellendesigns für zukünftige Entwicklungen. 
        \\
        \linebreak
        \linebreak
        Zur Kategorisierung von durchgeführten Tests werden Werkzeuge verwendet, die das Verhalten und die Meinungen der Benutzer 
        messbar machen. Eines davon ist das \textit{System Usability Scale (SUS)}\footnote{System Usability Scale. \url{https://www.usability.gov/how-to-and-tools/methods/system-usability-scale.html} Besucht am 03.07.2022}. 
        Dabei handelt es sich um einen einfachen 
        technologieunabhängigen Fragebogen, anhand dessen die Gebrauchstauglichkeit eines Systems bewertet werden kann. Dieser 
        ist eine Praktik zur quantitativen Analyse der Nutzbarkeit. Der Umfang umfasst zehn Fragen nach der Likert-Skala \cite{likert1932technique}.

\section{Aufbau der Arbeit}
\label{sec:aufbau}
    Die vorliegende Masterthesis gliedert sich nach den soeben genannten einleitenden Informationen im Aufbau in insgesamt 
    neun Kapitel. Das erste Kapitel (\ref{chap:Einleitung}) beschreibt die Motivation (\ref{sec:Motivation}) zur Bearbeitung 
    dieser Thematik rundum \acs{IoT} und Smart Home. Darauffolgend wird die 
    Forschungsfrage (\ref{sec:forschungsfragen}) im Rahmen der Thesis definiert. Nach der 
    Beschreibung der Forschungsfrage wird im anknüpfenden Abschnitt (\ref{sec:zielsetzung}) die Zielsetzung der 
    Arbeit erläutert. Hierbei werden zusätzliche Schwerpunkte aufgegriffen. 
    \\
    \linebreak
    Das Kapitel (\ref{chap:grundlagen}) widmet sich den essenziellen und wichtigen Grundlagen dieser Arbeit. Zu Anfang wird dem 
    Leser der Terminus des \acl{IoT} (\ref{sec:iot}) zum besseren Verständnis näher gebracht, 
    gefolgt von einer Einführung in die Thematik des \acl{SH} (\ref{sec:smartHome}), der 
    historischen und kontinuierlichen Entwicklung und den Zielen, die mit der Verwendung verfolgt 
    werden sollen. Mit dem Verständnis der übergeordneten Begriffe, \acs{IoT} und \acl{SH}, werden Technologien 
    (\ref{sec:technologien}) aufgegriffen, die im Rahmen dieser Arbeit erwähnenswert sind und verwendet werden. %Um auf die Vielfältigkeit 
    %von der Umsetzung eines \acl{SH} einzugehen und einen Teilaspekt der Anforderungen Abzudecken, wird ebenso auf Service-Roboter 
    %(\ref{sec:roboter}) eingegangen. 
    Abschließend werden im Kapitel (\ref{chap:grundlagen}) die Softwarelösungen, Home Assistant 
    und openHAB (\ref{sec:homeassistant} \& \ref{sec:openhab}), dargestellt. Diese dienen als Grundlage für die Evaluation und werden im Kapitel 
    (\ref{chap:evaluation}) der im Rahmen dieser Arbeit konzipierten Lösung gegenübergestellt. 
    \\
    \linebreak
    Die theoretischen und methodischen Hintergründe sowie der Stand der Technik wird in Kapitel (\ref{chap:technikStand})
    beleuchtet. Dieser Teil enthält Beschreibungen, Forschungen und aktuelle Erkenntnisse über Technologien, die im Umfeld der 
    \acl{SH} Anwendungen innerhalb des \acs{IoT} eingesetzt werden. 
    \\
    \linebreak
    Kapitel (\ref{chap:anforderungsanalyse}) befasst sich mit den Anforderungen, engl. Requirements, an das zu 
    entwickelnde System, die aus der Markt- und Zielgruppenanalyse, den Anwendungsfällen und den Experteninterviews generiert werden. 
    \\
    \linebreak 
    Nach Aufbereitung der Anforderungen durch das sogenannte Anforderungsmanagement, engl. \textit{Requirements Engineering},
    wird in Kapitel (\ref{chap:konzept}) das Konzept erarbeitet, welches als Grundlage für die prototypische Implementierung dient. 
    \\
    \linebreak
    In Kapitel (\ref{chap:umsetzung}) wird die Umsetzung des Konzepts skizziert, Herausforderungen aufgegriffen und Lösungen dargestellt. 
    Es wird aus praktischer Sicht die Architektur und deren Komponenten betrachtet. Das erzielte Ergebnis und Resultat wird 
    abschließend zusammengefasst.
    \\
    \linebreak
    Die Evaluation (\ref{chap:evaluation}) behandelt die objektive Betrachtung und Bewertung des entstandenen Prototyps 
    anhand von Usability-Tests und damit zusammenhängenden Experteninterviews. Zudem wird 
    die Forschungsfrage evaluiert und beantwortet. 
    \\
    \linebreak
    Im vorletzten Teil, Kapitel (\ref{chap:fazit}), wird ein Fazit aus den Erkenntnissen und Ergebnissen gezogen. Dieses Schlussresümee 
    führt nochmals die Höhepunkte sowie eine eigene Einschätzung auf. 
    \\
    \linebreak
    Zum Abschluss der Thesis wird in Kapitel (\ref{chap:ausblick}) ein Ausblick aufgezeigt, der Aufschluss über die Erweiterungsmöglichkeiten 
    und der Innovation der in dieser Thesis erfolgten Arbeit gibt. 
