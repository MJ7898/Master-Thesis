\chapter{Einleitung}
\label{chap:Einleitung}
    Die folgende Master-Thesis befasst sich mit der Konzepterstellung einer zentralen Steuerzentrale, die 
    dem Entwickler die formalen Interaktionen, weitere Funktionen hinzuzufügen, erleichtern soll. Hierfür werden
    bereits bestehende Plattformen für \acl{SH} analysiert und daraus ein Konzept erstellt, die den Anforderungen 
    entsprechend einen größeren Mehrwert in der Weiterentwicklung der Plattform bietet. Die Umsetzung des ausgearbeiteten 
    Konzeptes wird nur in sehr geringem Maß behandelt.
    \\
    \linebreak
    In diesem Teil der Arbeit wird auf die Motivation des Themas eingegangen. Darüber hinaus 
    werden sowohl die Forschungsfragen als auch die Zielsetzung der Arbeit genauestens dargelegt. Darauf 
    folgend findet eine Übersicht über die Arbeit im Gesamten statt, mit der die Inhalte angerissen werden. 
    Eine nähere Betrachtung des Standes der Technik untermauert die Beweggründe dieser Themenwahl und 
    Ausarbeitung dessen. 

\section{Motivation}
\label{sec:Motivation}
    Jede neu entwickelte Technologie durchlebt im Laufe der Entstehung und Publikation ein enormes Aufsehen. 
    So lange bis diese Technik eine standardisierte Verwendung in der Gesellschaft findet oder sich als 
    unpraktikabel erweist und nicht weiter vorangetrieben oder eingestellt wird. Es wird in der Zeit des 
    Aufkommens und der Forschung viel darüber fantasiert, debattiert und geplant, ohne jedoch die Ausmaße und 
    Resultate der Forschungen und Praktiken abwägen zu können. Durch fehlende Erfahrung und nicht ausgereifte 
    Konzepte werden Höhepunkte und Illusionen erwartet, die zu diesem Zeitpunkt technisch nicht umsetzbar sind. 
    Um solche kühnen Versprechungen und Übertreibungen, sogenannte Hypes, die jede neue technologische Idee 
    mit sich bringt, von dem zu differenzieren was wirtschaftlich umsetzbar ist, werden bestimmte Phasen 
    der Entwicklung durchlaufen. \cite{gartner.2022m}
    \\
    \linebreak
    Die oben erwähnten Phasen der Entwicklung sind in einem sogenannten Hype-Zyklus, engl. Hype-Cycle, dargestellt. 
    Dieser Zyklus ist ein visualisiertes Modell, das die Entwicklung einer neuen Technologie von der Innovation 
    und Entstehung über die Forschung und Umsetzung bis hin zur ausgereiften Marktfähigkeit repräsentiert und so 
    diese Phasen der Entwicklung versinnbildlicht.  
    \\
    Entwickelt wurde der Hype Cycle von der Gartner Inc. Forschungsgruppe. Durch die Mitarbeiterin 
    Jackie Finn wurden die Definitionen der Entwicklungsphasen\footnote{Die Entwicklungsphasen der Gartner Inc. ist unter folgender URL zu finden: "\url{https://www.gartner.com/en/research/methodologies/gartner-hype-cycle}"} 
    geprägt. Diese sind wie folgt in fünf Phasen 
    dargestellt:
    \begin{enumerate}
        \item \textit{Innovationsauslöser, engl. Innovation Trigger}: Ein potentieller technologischer Durchbruch 
        löst die Dinge aus. Frühe Proof-of-Concept (PoC) Ansätze und ein großes Medieninteresse lösen eine 
        erhebliche Publizität aus. Oft gibt es keine brauchbaren Produkte und die Marktreife ist nicht 
        bewiesen. \cite{gartner.2022m}
        \item \textit{Höhepunkt überhöhter Erwartungen, engl. Peak of Inflated Expectations}: Frühe Publizität 
        bringt eine Reihe von Erfolgsgeschichten hervor – oft begleitet von zahlreichen Misserfolgen. 
        Einige Unternehmen ergreifen Maßnahmen; viele nicht. \cite{gartner.2022m}
        \item \textit{Trog der Ernüchterung, engl. Trough of Disillusionment}: Das Interesse schwindet, da 
        Experimente und Implementierungen nicht liefern. Hersteller der Technologie reißen es heraus 
        oder scheitern. Investitionen werden nur fortgesetzt, wenn die überlebenden Anbieter ihre Produkte 
        zur Zufriedenheit der frühzeitigen Anwender verbessern. \cite{gartner.2022m}
        \item \textit{Steigung der Erleuchtung, engl. Slope of Enlightenment}: Mehr Beispiele dafür, wie 
        die Technologie dem Unternehmen zugute kommen kann, beginnen sich zu herauszukristallisieren und 
        werden allgemeiner verstanden. Produkte der zweiten und dritten Generation erscheinen von den 
        Technologieanbietern. Mehr Unternehmen finanzieren Pilotprojekte; Konservative Unternehmen 
        bleiben vorsichtig. \cite{gartner.2022m}
        \item \textit{Plateau der Produktivität, engl. Plateau of Productivity}: Mainstream-Akzeptanz beginnt 
        sich abzuheben. Kriterien zur Bewertung der Lebensfähigkeit des Anbieters sind klarer definiert. 
        Die breite Markteinsetzbarkeit und Relevanz der Technologie zahlen sich eindeutig aus. \cite{gartner.2022m}
    \end{enumerate}
    Nachdem ein innovativer Gedanke den \textit{Höhepunkt überhöhter Erwartungen} passiert hat, z.B. die 
    Revolutionierung der Softwareentwicklung oder Szenarien, wie z.B. die Vollautomatisierung eines Gebäudes 
    oder Service-Roboter die uneingeschränkt interagieren können, die man in der Form nur aus Science-Fiction Filmen kennt, 
    folgt der \textit{Trog der Ernüchterung}. In Folge dessen wird festgestellt, dass die Erwartungen nicht 
    in Gänze übertragbar sind, bzw. nur zu einem geminderten Teil in die Realität umgesetzt werden können 
    und der verfolgte Gedanke an Interesse verliert. Nach erneutem Aufgriff der Technologie findet eine realistischere 
    Beurteilung der Innovation statt, die dazu beiträgt, dass die Technologie wieder an Interesse gewinnt. Die 
    objektive und realitätsnahe Betrachtungsweise formt ein neues und realistisches Bild der Potentiale, als auch 
    der Grenzen. Mit dem neu gewonnenen Maßstab geht die ehemals neue innovative Idee in eine routinierte Technologie über, 
    die an Anerkennung gewinnt und in der breiten Masse akzeptiert wird. Die Technologie erfährt mit steigender 
    Zuwendung eine stetigere Weiterentwicklung, die dann zu einer Community geformt wird. Mit der Erreichung dieses Status 
    befindet sich die Innovation, bezogen auf den Hype Cycle, in der letzten Phase, dem \textit{Plateau der Produktivität}, 
    und bestätigt so die Marktreife. Dieser Zeitpunkt löst die Zukunftsvision auf und es handelt sich um eine am Markt 
    etablierte Technologie.
    \\ 
    \linebreak
    Zum aktuellen Zeitpunkt befindet sich die Technologie rund um Plattformen für intelligente Geräte im privaten 
    Bereich, engl. \textit{Smart Home} oder \textit{Connected Home}, im Anfangsstadium der letzten Phase, dem sogenannten 
    \textit{Plateau der Produktivität}. Mit zunehmender Akzeptanz werden im Umfeld des Internets der Dinge, engl. 
    \textit{\acl{IoT}}, stetig Szenarien entwickelt, die das Wachstum und die Verwendung von solchen Plattformen vorantreibt. 
    Mit einer immer tiefer gehenden Forschung und Umsetzung von Anwendungsbeispielen werden Bereiche offenbart, die 
    eine solche Plattform im privaten als auch im geschäftlichen Umfeld immer attraktiver gestaltet. Mit steigender  
    Konnektivität und Kompatibilität mehreren Geräten und Gegenständen können Bereiche und Szenarien, wie die 
    Steuerung von Service-Robotern, umgesetzt werden. Der jetzige technologische Fortschritt und die über die Forschungsjahre 
    gesammelten Erfahrungen bringt das Segment der intelligenten Geräte der \acs{IoT} den ursprünglich angedachten 
    Visionen und Ideen näher, sodass ein weiterer Ausbau dieser Technologie und dessen Anwendung stattfindet und sich 
    vollständig in den Markt etabliert. Der finale Schritt der endgültigen Marktreife ist ein faszinierender und wichtiger Grund 
    für meine Motivation, mich dieser Technologie und der dahinterstehenden Theorie zu widmen. 
    \\
    Ein weiterer Punkt meiner Motivation ist die Vereinfachung der Erweiterung einer solchen Zentrale. Somit soll dem 
    Entwickler bei einer stetigen Erweiterung der Plattform Zeit und Aufwand erleichtert werden. So können weiter 
    Anwendungsszenarien und Objekte integriert werden, ohne einen zu großen Entwicklungsaufwand zu erzeugen.   
    \\ 
    \linebreak
    Die Einsatzgebiete von intelligenten Geräten, beziehungsweise die damit zu verwendende Steuerzentrale beschränkt sich räumlich 
    auf Gebäude, Häuser und Wohnungen, bieten trotz dessen viele Verwendungs- und Einsatzmöglichkeiten. Diese sehen wie folgt aus: 
    \begin{itemize}
        \item Komfort
        \item Entertainment
        \item Überwachung und Sicherheit
        \item Steuerung von Prozessen
        \item Management von Automationen
    \end{itemize}    


\section{Forschungsfragen}
\label{sec:forschungsfragen}

\section{Zielsetzung der Arbeit}
\label{sec:zielsetzung}

\section{Aufbau der Arbeit}
\label{sec:aufbau}
    Die vorliegende Master-Thesis gliedert sich nach den soeben genannten einleitenden Information im Aufbau in insgesamt 
    zehn Kapitel. Das erste Kapitel (\ref{chap:Einleitung}) beschreibt die Motivation (\ref{sec:Motivation}), welche die 
    Intension kundtut, diese Thematik rund um \acs{IoT} und Smart Home zu bearbeiten. Darauf folgend werden die 
    Forschungsfragen (\ref{sec:forschungsfragen}), die im Rahmen der Thesis behandelt werden, erläutert. Nach der 
    Beschreibung der Forschungsfragen wird im anknüpfenden Abschnitt (\ref{sec:zielsetzung}) die Zielsetzung der 
    Arbeit erläutert. Hierbei werden zusätzliche Schwerpunkte und Ziele aufgegriffen. Abschließend wird das Unternehmen, in der 
    die Thesis geschrieben wird, hervorgehoben und deren Absichten in Verbindung mit Innovationen beleuchtet. 
    \\
    \linebreak
    Das Kapitel (\ref{chap:grundlagen}) widmet sich den essentiellen und wichtigen Grundlagen dieser Arbeit. Zu Anfang wird dem 
    Leser der Terminus des \acl{IoT} (\ref{sec:iot}) offenbart, um zum Teil den Kontext im Bezug zu dieser Arbeit zu begreifen, 
    gefolgt von einer Einführung in die Thematik des \acl{SH}s (\ref{sec:smartHome}), der Problematik der Begriffsdefinition, der 
    historischen und kontinuierlichen Entwicklung und mit den Zielen, die mit der Verwendung einer Smart Home Lösung bewältigt 
    werden sollen. Mit dem Verständnis der übergeordneten Begriffe, \acs{IoT} und \acs{SH}, werden Technologien 
    (\ref{sec:technologien}) aufgegriffen, die im Rahmen dieser Arbeit erwähnenswert sind und verwendet werden. Um auf die Vielfältigkeit 
    von der Umsetzung eines \acl{SH}s einzugehen und einen Teilaspekt der Anforderungen Abzudecken, wird ebenso auf Service-Roboter 
    (\ref{sec:roboter}) eingegangen. Abschließend werden in Kapitel (\ref{chap:grundlagen}) die Softwarelösungen, Home Assistant 
    und openHAB (\ref{sec:homeassistant} \& \ref{sec:openhab}), dargestellt. Diese dienen zur Grundlage für die Evaluation als 
    auch zur Gegenüberstellung der Lösungen in Kapitel (\ref{chap:evaluation}) Diskussion und Evaluation. 
    \\
    \linebreak
    Die theoretischen und methodischen Hintergründe sowie den Stand der Technik wird in Kapitel (\ref{chap:technikStand} )
    angesprochen. Dieser Teil enthält Beschreibungen, Forschungen und aktuelle Erkenntnisse über Technologien, die im Umfeld der 
    Smart Home Anwendungen innerhalb des \acs{IoT}s verwendet werden. Zudem werden in Zusammenhang der Erkenntnisse und 
    Möglichkeiten der Technologie die Szenarien dargestellt. 
    \\
    \linebreak
    Kapitel (\ref{chap:anforderungsanalyse}) befasst sich mit den Anforderungen, engl. Requirements, die für die 
    eigentliche Konzeption relevant sind. Innerhalb dieses Kapitels wird anhand von Informationen und den umzusetzenden 
    Szenarien die Anforderungen für die Konzeption erarbeitet. Hierbei werden aus der Praxis bekannte Verfahren verwenden, um 
    die Anforderungen zu definieren. Mittels den zugrundeliegenden Anforderungen wird im nachfolgenden Schritt die eigentliche 
    Konzeption dargelegt.
    \\
    \linebreak 
    Nach Aufbereitung der Anforderungen durch das sogenannte Anforderungsmanagement, engl. \textit{Requirements Engineering},
    wird in Kapitel (\ref{chap:konzept}) das Konzept erarbeitet, welches als Grundlage für die prototypische Implementierung und 
    Umsetzung des Konzepts dient. Das Konzept befasst sich mit den Anforderungen und setzt diese ein, um die Organisation des 
    Systems in Komponenten, deren Beziehungen zueinander und zur Umgebung sowie deren Prinzipien zu definieren. Zum Ende des 
    Konzepts steht eine Architektur, die sich aus den Anforderungen und auch aus den Analysen der eigentlichen Forschungsfrage 
    abzeichnet.
    \\
    \linebreak
    In Kapitel (\ref{chap:umsetzung}) wird die Umsetzung des Konzepts skizziert. Darunter welche Problem während der Implementierung 
    auftraten als auch deren Lösungsfindung. Ebenso wird hier aus praktischer Sicht auf die Architektur geschaut, welche Komponenten,  
    Bibliotheken und zusätzliche Systeme, engl. \textit{Frameworks}, verwendet wurden. 
    \\ 
    \linebreak
    Das Ergebnis wird aus objektiver Sicht in dem darauf folgenden Kapitel (\ref{chap:ergebnis}) erläutert. 
    \\
    \linebreak
    Nach Abschluss der Umsetzung und dessen Ergebnisdokumentation befasst sich das Kapitel (\ref{chap:evaluation}) mit der Diskussion 
    und Evaluation. Hier findet eine Analyse des Konzepts sowie deren Umsetzung und objektive Betrachtung statt. Anschließend werden 
    Vergleiche zwischen der Eigenentwicklung und bereits bestehender Softwarelösungen, die im Grundlagenkapitel aufgefasst werden, 
    aufgestellt und bewertet.
    \\
    \linebreak
    Im vorletzten Teil, Kapitel (\ref{chap:fazit}), wird ein Fazit aus den Erkenntnissen und Ergebnissen gezogen. Dieses Schlussresümee 
    führt nochmals die Höhepunkte sowie eine eigene Einschätzung auf. 
    \\
    \linebreak
    Zum Abschluss der Thesis wird in Kapitel (\ref{chap:ausblick}) ein Ausblick gegeben. Dieser gibt Aufschluss darüber, welche 
    Erweiterungsmöglichkeiten es für die in dieser Thesis erfolgten Arbeit gibt und wie innovativ sich dieser Grundbaustein in Zukunft 
    erweisen könnte. 

\section{CGI}
\label{sec:cgi}