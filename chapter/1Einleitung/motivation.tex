\chapter{Einleitung}
\label{chap:Einleitung}
    Die folgende Master-Thesis befasst sich mit der Konzepterstellung einer zentralen Steuerzentrale, die 
    dem Entwickler die formalen Interaktionen, weitere Funktionen hinzuzufügen, erleichtern soll. Hierfür werden
    bereits bestehende Plattformen für Smart Home analysiert und daraus ein Konzept erstellt, die den Anforderungen 
    entsprechend einen größeren Mehrwert in der Weiterentwicklung der Plattform bietet. Die Umsetzung des ausgearbeiteten 
    Konzeptes wird nur in sehr geringem Maß behandelt.
    \\
    \linebreak
    In diesem Teil der Arbeit wird auf die Motivation des Themas eingegangen. Darüber hinaus 
    werden sowohl die Forschungsfragen als auch die Zielsetzung der Arbeit genauestens dargelegt. Darauf 
    folgend findet eine Übersicht über die Arbeit im Gesamten statt, mit der die Inhalte angerissen werden. 
    Eine nähere Betrachtung des Standes der Technik untermauert die Beweggründe dieser Themenwahl und 
    Ausarbeitung dessen. 

\section{Motivation}
\label{sec:Motivation}
    Jede neu entwickelte Technologie durchlebt im Laufe der Entstehung und Publikation ein enormes Aufsehen. 
    So lange bis diese Technik eine standardisierte Verwendung in der Gesellschaft findet oder sich als 
    unpraktikabel erweist und nicht weiter vorangetrieben oder eingestellt wird. Es wird in der Zeit des 
    Aufkommens und der Forschung viel darüber fantasiert, debattiert und geplant, ohne jedoch die Ausmaße und 
    Resultate der Forschungen und Praktiken abwägen zu können. Durch fehlende Erfahrung und nicht ausgereifte 
    Konzepte werden Höhepunkte und Illusionen erwartet, die zu diesem Zeitpunkt technisch nicht umsetzbar sind. 
    Um solche kühnen Versprechungen und Übertreibungen, sogenannte Hypes, die jede neue technologische Idee 
    mit sich bringt, von dem zu differenzieren was wirtschaftlich umsetzbar ist, werden bestimmte Phasen 
    der Entwicklung durchlaufen. \cite{gartner.2022m}
    \\
    \linebreak
    Die oben erwähnten Phasen der Entwicklung sind in einem sogenannten Hype-Zyklus, engl. Hype-Cycle, dargestellt. 
    Dieser Zyklus ist ein visualisiertes Modell, das die Entwicklung einer neuen Technologie von der Innovation 
    und Entstehung über die Forschung und Umsetzung bis hin zur ausgereiften Marktfähigkeit repräsentiert und so 
    diese Phasen der Entwicklung versinnbildlicht.  
    \\
    Entwickelt wurde der Hype Cycle von der Gartner Inc. Forschungsgruppe. Durch die Mitarbeiterin 
    Jackie Finn wurden die Definitionen der Entwicklungsphasen geprägt. Diese sind wie folgt in fünf Phasen 
    dargestellt:
    \begin{enumerate}
        \item \textit{Innovationsauslöser, engl. Innovation Trigger}: Ein potentieller technologischer Durchbruch 
        löst die Dinge aus. Frühe Proof-of-Concept (PoC) Ansätze und ein großes Medieninteresse lösen eine 
        erhebliche Publizität aus. Oft gibt es keine brauchbaren Produkte und die Marktreife ist nicht 
        bewiesen. \cite{gartner.2022m}
        \item \textit{Höhepunkt überhöhter Erwartungen, engl. Peak of Inflated Expectations}: Frühe Publizität 
        bringt eine Reihe von Erfolgsgeschichten hervor – oft begleitet von zahlreichen Misserfolgen. 
        Einige Unternehmen ergreifen Maßnahmen; viele nicht. \cite{gartner.2022m}
        \item \textit{Trog der Ernüchterung, engl. Trough of Disillusionment}: Das Interesse schwindet, da 
        Experimente und Implementierungen nicht liefern. Hersteller der Technologie reißen es heraus 
        oder scheitern. Investitionen werden nur fortgesetzt, wenn die überlebenden Anbieter ihre Produkte 
        zur Zufriedenheit der frühzeitigen Anwender verbessern. \cite{gartner.2022m}
        \item \textit{Steigung der Erleuchtung, engl. Slope of Enlightenment}: Mehr Beispiele dafür, wie 
        die Technologie dem Unternehmen zugute kommen kann, beginnen sich zu herauszukristallisieren und 
        werden allgemeiner verstanden. Produkte der zweiten und dritten Generation erscheinen von den 
        Technologieanbietern. Mehr Unternehmen finanzieren Pilotprojekte; Konservative Unternehmen 
        bleiben vorsichtig. \cite{gartner.2022m}
        \item \textit{Plateau der Produktivität, engl. Plateau of Productivity}: Mainstream-Akzeptanz beginnt 
        sich abzuheben. Kriterien zur Bewertung der Lebensfähigkeit des Anbieters sind klarer definiert. 
        Die breite Markteinsetzbarkeit und Relevanz der Technologie zahlen sich eindeutig aus. \cite{gartner.2022m}
    \end{enumerate}
    Nachdem ein innovativer Gedanke den \textit{Höhepunkt überhöhter Erwartungen} passiert hat, z.B. die vollständige 
    Revolutionierung der Softwareentwicklung oder Szenarien, wie z.B. die Vollautomatisierung eines Gebäudes 
    oder Roboter die uneingeschränkt interagieren können, die man in der Form nur aus Science-Fiction Filmen kennt, 
    folgt der \textit{Trog der Ernüchterung}. In Folge dessen wird festgestellt, dass die Erwartungen nicht 
    in Gänze übertragbar sind, bzw. nur zu einem geminderten Teil in die Realität umgesetzt werden können 
    und der verfolgte Gedanke an Interesse verliert. Nach erneutem Aufgriff der Technologie findet eine realistischere 
    Beurteilung der Innovation statt, die dazu beiträgt, dass die Technologie wieder an Interesse gewinnt. Die 
    objektive und realitätsnahe Betrachtungsweise formt ein neues und realistisches Bild der Potentiale, als auch 
    der Grenzen. Mit dem neu gewonnenen Maßstab geht die ehemals neue innovative Idee in eine routinierte Technologie über, 
    die an Anerkennung gewinnt und in der breiten Masse akzeptiert wird. Die Technologie erfährt mit steigender 
    Zuwendung eine stetigere Weiterentwicklung, die dann zu einer Community geformt wird. Mit der Erreichung dieses Status 
    befindet sich die Innovation, bezogen auf den Hype Cycle, in der letzten Phase, dem \textit{Plateau der Produktivität}, 
    und bestätigt so die Marktreife. Dieser Zeitpunkt löst die Zukunftsvision auf und es handelt sich um eine am Markt 
    etablierte Technologie.
    \\ 
    \linebreak
    