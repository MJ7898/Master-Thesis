\chapter{Stand der Technik}
\label{chap:technikStand}
%\section{Theorien}
%\section{Techniken}
    Zur Analyse des aktuellen Standes der Technik und Forschung in Bezug zur Konzeption von Software-Lösungen, mit denen 
    die formalisierten Interaktionen der Softwareentwickler vereinfacht werden können, erfolgt in diesem 
    Kapitel ein systematisches Literaturreview. Die Literaturüberprüfung wird gemäß den Richtlinien, die in der Publikation 
    von %TBD \cite{}
    vorgeschlagen werden, durchgeführt. 

    %\section{Methodiken}
    %\label{sec:methodiken}
    %    Im Rahmen dieser Arbeit wurden zur Informationserhebung zuerst ein systematisches Literaturreview durchgeführt und

    \section{Systematisches Literaturreview}
    \label{subsec:systematischesLiteraturReview}
        Die Thematik des systematischen Literaturreviews wurde bereits in den einleitenden Kapiteln erwähnt. 
        Dieser Abschnitt widmet sich ausschließlich der Anwendung der Richtlinien und dem daraus abgeleiteten Stand der Technik 
        abhängig zu dem in der Arbeit behandelten Thema. 

    \subsection{Ziele des Systematischen Literaturreviews}
        Das Ziel dieses systematischen Literaturreviews ist es, die aktuellen Fortschritte von Smart Home Plattformen und 
        Gateways in Richtung der entwicklerseitigen Benutzerfreundlichkeit zu recherchieren. Dabei liegt der Schwerpunkt 
        auf der Usability und der einfachen Handhabung der formalisierten Interaktionen der Softwareentwickler. Es gilt zu 
        analysieren, ob es in diesem Themenbereich bereits Publikationen und Forschungen gibt und welche Entscheidungen 
        getroffen werden müssen, um die Weiterentwicklung eines Systems nicht je nach hinzukommender Funktionalität oder 
        auch Bedingung komplexer werden zu lassen. Die Ergebnisse dieses systematischen Literaturreviews sollen daraufhin 
        als Grundlage der Konzeption einer solchen Plattform dienen und mit einfließen. 

    \subsection{Suchstrategie- und anfragen}
    \begin{table}[hbt!]
        \begin{center}
            \begin{tabular}{| p{2.8cm} | p{1.9cm} | p{1.7cm} | p{1.9cm} | p{1.9cm} | p{1.8cm} | p{1.8cm} | }
                \hline
                   \textbf{Suchanfrage} & \textbf{Datum} & \textbf{Filter} & \textbf{Plattform} & \textbf{Ergebnisse} & \textbf{Gesehene} & \textbf{Relevant} \\
                \hline
                    &  &  &  &  &  &  \\ 
                \hline
            \end{tabular}
        \end{center}
        \caption{Suchprotokoll des Systematischen Literaturreviews}
        \label{tab:slr}
    \end{table}