\section*{Use Case 1 - Check-in mit einem Service-Roboter}
\subsection*{Aufgabenbeschreibung UC 1}
\begin{table}[hbt!]
    \begin{center}
        \begin{tabular}{| p{3cm} | p{12.75cm} | }
            \hline
                \textbf{Task:} & \textbf{Begrüßung / Check-In von Member und Gästen} \\
            \hline
                \textbf{Ziel:} & Einen Member (Mitarbeiter) oder einen Gast in der Lokation mit einem Service-Roboter zu begrüßen, wenn dieser eintreten möchte. Nach der Begrüßung findet der Check-In statt, bei dem der Member seinem Platz zugewiesen und der Gast zu seinem Ansprechpartner geleitet wird. \\
            \hline
                \textbf{Eingriffs-möglichkeiten:} & Auswahl der Situation, Interaktion mit dem Service-Roboter (Informieren, Begleiten). \\
            \hline
                \textbf{Ursachen:} & Terminbuchung, Anmeldung, Klingeln an der Tür, Authentifizierung / Authentisierung an der Tür mittels Biometrische Eingangskontrolle. \\
            \hline
                \textbf{Priorität:} & Hoch \\
            \hline
                \textbf{Durchführungs-profil:} & Nach Anmeldung oder Eintritt eines Gastes oder Members, Unterbrechungen möglich (Verbindungsfehler, Fehlerhafte Abfrage der Verfügbarkeiten von Platzreservierungen), mittlere Komplexität. \\ 
            \hline
                \textbf{Vorbedingung:} & Platzbuchung / Reservierung und Einbuchung für ein Meeting fand im Vorfeld statt, Einverständniserklärung von den Personen liegt vor, Authentisierung über die Eingangskontrolle hat funktioniert / stattgefunden. \\
            \hline 
                \textbf{Info-In:} & Die Person, die eintreten will, ist durch die Buchung oder Authentisierung bekannt, Buchung ist registriert und kann abgerufen werden. \\
            \hline
                \textbf{Info-Out:} & Begrüßung Standardprotokoll mit Interaktionsmöglichkeiten und Kenntnis des Namens der Person, Check-In Informationen für die einzucheckende Person. \\
            \hline
                \textbf{Ressourcen:} & Service-Roboter, Eintrittskontrolle per Biometrischer Authentisierung, Buchung der Person abrufbar über die API. \\
            \hline
        \end{tabular}
    \end{center}
    \caption{Aufgabenbeschreibung UC 1 - Begrüßung / Check-in}
    \label{tab:checkIn}
\end{table}

%\subsection*{UC 1 - Beschreibung}
%    Steuerzentrale:
%    \begin{itemize}
%        \item Input: MQTT Nachricht von UC 1 (Verifizierung ok?, Name des Gastes / Members und Prozessanstoß (Check In))
%        \item Input: Terminregistrierung durch DoorJames (Übertragung der Termindaten) 
%        \item Output: Zu durchlaufender Prozess (CheckIn) und Name des Gastes / Members
%        \subItem (CheckIn) Genauer Begrüssungsprozess? Aktiv / Passiv 
%        \subsubItem Zuständigkeit Service-Roboter / Member / Gast / Steuerzentrale (Zentral / dezentral)
%        \subsubItem Begrüßungsprozess durch Service-Roboter selbst (Möglichkeiten sind im Service-Roboter implementiert) Steuerzentrale dient nur als Event-Auslöser 
%    \end{itemize}
	
%    Service-Roboter: 
%    \begin{itemize}
%        \item Input: MQTT (Prozess-Trigger durch Steuerzentrale und Name des Gastes / Members)
%        \item Input: Mitarbeiter lädt Kunde ein / Prepare / Nachweis der Platzbuchungssoftware
%        \item Output: (An Steuerzentrale) – Prozess erfolgreich abgeschlossen
%        \subItem Event-Möglichkeiten werden von Service-Roboter direkt verarbeitet. (Interaktion mit Gast, Ausrichten mit Hilfe Objekterkennung zur Person) 
%        \subItem Event-Möglichkeiten: 
%        \subsubItem Kunde: 
%        \subsubsubItem Begrüßen von Kunde und einchecken mit QR / Wenn eingeladen, Service-Roboter wartet schon vorher 5-15min vor der Tür
%        \subsubsubItem Zum Einladenden führen oder einen Call starten
%        \subsubItem Mitarbeiter: 
%        \subsubsubItem Anbieten eines Kaffees (zum Arbeitsplatz?) / oder zur Kaffee Maschine führen
%    \end{itemize}

\subsection*{UC 1 - User Story}
\textbf{Gast:}
\\
Als Gast möchte ich für den Zugang in das Gebäude ein Bild für den Anmeldeprozess hinterlegen können. Vor Ort möchte ich über die Kamera 
am Eingang mein Gesicht Authentisieren, bei nicht Hinterlegung eines Bildes bei der Anmeldung, möchte ich klingeln. Nach der 
Authentisierung bzw. nach Betätigung der Klingel soll sich die Tür öffnen. Beim Hineintreten in das Gebäude nehme ich einen Service-Roboter 
(Service-Roboter) wahr. Dieser begrüßt mich und startet eine Konversation mit mir. Er bietet mir an die Registrierung bzw. das Antreffen im Gebäude 
per QR Code über den Service-Roboter-Bildschirm durchzuführen. Somit werde ich als anwesend markiert und der Mitarbeiter, mit dem ich einen Termin 
habe, wird informiert. Nach erfolgreicher Registrierung führt mich der Service-Roboter zum Mitarbeiter mit dem ich einen Termin vereinbart habe.
\\
\linebreak
\textbf{Member:}\\

Als Member möchte ich die Buchung eines Arbeitsplatzes wahrnehmen können. Um in das Gebäude eintreten zu können, ist eine Authentifizierung durch 
die Gesichtserkennung oder das Abscannen der Chipkarte notwendig. Nach erfolgreicher Autorisierung werde ich von Service-Roboter bei Eintritt des Gebäudes 
begrüßt. Service-Roboter gibt mir den Arbeitsplatz bekannt und fragt, ob ich zum Platz begleitet werden soll. 
(oder Service-Roboter mir einen Kaffee an den Arbeitsplatz bringt.) -> (Erweiterung des Use Cases in Planung)
%Kunden zum Ausgangführen/Checkout anderer use case

\subsubsection*{Akzeptanzkriterien}
\begin{itemize}
    \item AK-01: Eine MQTT-Message mit dem Namen der Authentisierten Person wird empfangen.
    \item AK-02: Die MQTT-Message wird gelesen und mittels den Informationen können Abfragen an die DoorJames API erfolgen.
    \item AK-03: Informationen der DoorJames API werden empfangen und können weiterverarbeitet werden. 
    \item AK-04: Die benötigten Informationen können an den Service-Roboter (Temi) weitergeleitet werden.
    \item Ak-05: Eine MQTT-Message wird empfangen, sobald der Service-Roboter alle Tasks (ToDo's)  des Prozesses abgearbeitet hat. 
\end{itemize}
