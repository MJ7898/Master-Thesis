\section*{Use Case 2 - Notfall-Evakuierung}
\begin{table}[hbt!]
    \begin{center}
        \begin{tabular}{| p{3cm} | p{12.75cm} | }
            \hline
                \textbf{Task:} & \textbf{Begrüßung / Check-In von Member und Gästen} \\
            \hline
                \textbf{Ziel:} & Einen Notfall zu erkennen und darauf hin alle Member und Gäste, die sich im Büro befinden, informieren und bitten das Gebäude zu verlassen. \\
            \hline
                \textbf{Eingriffs-möglichkeiten:} & Keine, (Evtl. direkte Interaktion mit dem Service-Roboter). \\
            \hline
                \textbf{Ursachen:} & Aktivierung des Sensors durch äußerliche Einwirkung. \\
            \hline
                \textbf{Priorität:} & Hoch \\
            \hline
                \textbf{Durchführungs-profil:} & nach Meldung eines Notfalls durch Sensoren und Melder \\ 
            \hline
                \textbf{Vorbedingung:} & Ein Melder kann Nachrichten veröffentlichen, Topics für die Melder und Empfänger sind definiert. \\
            \hline 
                \textbf{Info-In:} & Meldung eines Sensors (Melder) über einen Notfall (Rauch, Feuer, Wasser, Gas). \\
            \hline
                \textbf{Info-Out:} & Kenntnis der gebuchten Plätze, Interaktionsmöglichkeiten mit dem Service-Roboter, Benachrichtigung bei Beendung des Prozesses. \\
            \hline
                \textbf{Ressourcen:} & Service-Roboter, Sensor (Rauch-, Feuer- oder Gas-Melder), Information über aktuelle Buchungen, Zuordnung von Positionen von Büroplätzen. \\
            \hline
        \end{tabular}
    \end{center}
    \caption{Aufgabenbeschreibung UC 2 - Notfall-Evakuierung}
    \label{tab:evacuation}
\end{table}

%\subsection*{UC 2 - Beschreibung}
%%    Steuerzentrale:
%    \begin{itemize}
%        \item Input: MQTT Nachricht angestoßen von einem Member (Aktion; Uhrzeit)
%        \item Input: DoorJames API - Gebuchte Büroplätze und Personen die einen Termin vor Ort registriert haben
%        \item -	Output: Zu durchlaufender Prozess (Evakuierung) und Name des Gastes / Members und Position der Plätze und Räume, die gebucht wurden. 
%    \end{itemize}
	
%    Service-Roboter: 
%    \begin{itemize}
%        \item Input: MQTT (Prozess-Trigger durch Steuerzentrale und Position gebuchter Plätze und Räume
%        \item Input: Objekterkennung (Personenerkennung) durch die Kamera
%        \item Output: (An Steuerzentrale) – Status / Prozess (erfolgreich/nicht erfolgreich) abgeschlossen
%        \item -> Event-Möglichkeiten werden von Service-Roboter direkt verarbeitet. (Interaktion mit Person, Anweisungen an die Person) 
%        \item -> Event-Möglichkeiten: 
%        \item --> Kunde, Gast & Member: 
%        \item ---> Alarmieren und bitten das Gebäude auf schnellstem Weg zu
%    \end{itemize}

\subsection*{UC 2 - User Story}
\textbf{Gast:}
\\
    Als Gast, Member und Manager möchte ich bei einem Notfall alarmiert werden. Bei eintretendem Fall möchte ich 
    Anweisungen, die zu beachten und umzusetzen sind, erhalten. Ebenso ist ein Hinweis zu den Notausgängen 
    sinnvoll und hilft bei der Flucht bzw. beim Verlassen des Gebäudes. Damit keine Personen im Büro 
    bleiben, ist eine Information über Anwesende von belangen. Der Member sollte die Möglichkeit haben, den 
    Service-Roboter aufzufordern nach Leuten zu rufen, die nicht in Sichtweite sind. Sobald alle Plätze 
    abgefahren sind bzw. durch den Service-Roboter keine Personen mehr erkannt werden, soll eine Benachrichtigung 
    erfolgen, dass alle aus dem Gebäude sind bzw. keine mehr gesichtet wurden. Diese Information hilft bei der 
    anschließenden Kontrolle über am Treffpunkt anwesender Personen. Abschließend soll ggf. Eine Information über die 
    Platzbuchungen gezeigt werden, damit bei der Kontrolle keine Fehler unterlaufen und keine Personen, die nicht 
    gesehen werden, vergessen werden.

\subsection*{Akzeptanzkriterien}
\begin{itemize}
    \item AK-01: Eine MQTT-Message des Rauchmelders (Sensors) wird von der Steuerzentrale empfangen.
    \item AK-02: Die MQTT-Message wird gelesen und Abfragen an die DoorJames API erfolgen (Abfrage der Personen zum Zeitpunk im Bürogebäude)
    \item AK-03: Informationen der DoorJames API werden empfangen und können weiterverarbeitet werden. 
    \item AK-04: Die benötigten Informationen können an den Service-Roboter (Temi) weitergeleitet werden.
    \item AK-05: Temi wird losgeschickt, um nach Personen im Gebäude zu suchen.
    \item AK-06: Die Steuerzenrale schickt eine SMS, E-Mail an die Gebäudeverantwortliche Person, dass ein Notfall losgetreten wurde.
    \item Ak-07: Eine MQTT-Message wird empfangen, sobald der Service-Roboter alle Tasks (ToDo's)  des Prozesses abgearbeitet hat. 
\end{itemize}