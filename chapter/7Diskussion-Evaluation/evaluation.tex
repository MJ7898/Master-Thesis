\chapter{Evaluation}
\label{chap:evaluation}
In diesem Abschnitt wird das erzielte Ergebnis des erarbeiteten Konzeptes (siehe Kapitel \ref{chap:konzept}) und des 
Prototypen analysiert. Anhand von ausgewählten Forschungsmethoden werden die vorab identifizierten Anforderungen überprüft 
und evaluiert. Mithilfe der erhobenen Informationen wird zum Ende des Kapitels die Forschungsfrage, sowie das
Ziel der Arbeit abschließend evaluiert. 

\section{Usability-Test}
\label{sec:usabilitytest}
    Im Rahmen der Evaluation wurde ein Usability-Test durchgeführt, um bestimmte, für die Nutzbarkeit aufgestellte Anforderungen zu 
    überprüfen.
    Das angestrebte Ziel und die Ergebnisse der Tests werden in dem folgenden Abschnitt erläutert:
    
    \subsection{Ziel des Usability-Tests}
        Anhand des Usability-Tests sind die Anforderungen zu prüfen, die den Bereich der Nutzbarkeit adressieren. 
        Hierfür werden die Anforderungen, die den Tabellen (\ref{tab:functionalRequirements}, \ref{tab:notfunctionalRequirements} und \ref{tab:furtherRequirements}) 
        zu entnehmen sind, betrachtet. Die dafür berücksichtigen Anforderungen sind konkret die Folgenden: NFA1, NFA2, NFA3 und NFA4 (siehe \ref{tab:notfunctionalRequirements})
        \\
        Damit die Erfüllung der Anforderungen überprüfbar ist, wird einer Auswahl an Anwendern der Zielgruppe im Rahmen der Nutzbarkeit 
        eine Aufgabe gestellt, mit der das Framework getestet wird und dabei die oben genannten Anforderungen genauestens betrachtet werden. 
        Ziel dabei ist die Sicherstellung der Erfüllung von obigen Anforderungen, bzw. die Identifizierung von Verbesserungspotentialen. 

    \subsection{Durchführung des Usability-Tests}
        Für die Durchführung der Tests wurde vorab eine Aufgabe erstellt. Diese 
        dient als Grundlage zur Identifizierung der Usability-Anforderungen. Der Aufbau und das Vorgehen der Testung wurde für jeden Teilnehmer 
        ähnlich gestaltet. Zu Anfang, bevor der Test durchgeführt wurde, war der Kontext sowie das Framework zu erläutern, damit 
        die Informationen und Vorbedingungen für jeden Teilnehmer identisch waren. Vorab war die Erläuterung der 
        Kernkomponenten des Frameworks, sowie die Rahmenbedingungen und die Aufgabenstellung selbst notwendig. So waren für jeden Teilnehmer 
        die Informationen zur Durchführung des Tests klar. Die Realisierung des Tests wurde 
        an unterschiedlichen Orten durchgeführt. Für die Variation des Durchführungsortes war eine Aufgabe zu erstellen, die umgebungsunabhängig 
        erfüllt werden konnte. Da die Tests ausschließlich auf die Nutzung des Frameworks abzielten, war die Umgebung sowie die 
        Anbindung verfügbarer Geräte zu vernachlässigen. Damit der Testumfang nicht zu komplex wurde, war der Anwendungsfall der 
        umzusetzenden Aufgabe trivial. Für die Ausführung des Tests war eine Voraussetzung, dass bereits ein \acs{MQTT}-Broker 
        aufgesetzt war und für die Aufgabe zur Verfügung stand. Durch die Fehlende Anbindung von reellen Geräten, wurden die ausgehenden 
        \acs{MQTT}-Nachrichten durch eine Simulation, konkret die Simulation eines Service-Roboters, verarbeitet. Eingehende Nachrichten 
        wurden simuliert, indem durch den Broker manuell \acs{MQTT}-Nachrichten veröffentlicht werden konnten.
        \\
        Die von den Anwendern durchzuführende Aufgabe ist wie folgt definiert: 
        \\
        \linebreak
        Es soll ein Anwendungsfall implementiert werden, bei dem ein Service-Roboter einen Mitarbeiter oder Gast, der an der Tür steht, 
        empfangen und begrüßen soll. Eingangsbedingungen sind, dass eine Kamera zur Authentifizierung für die Anwendung simuliert wird. 
        Die \acs{MQTT}-Nachricht wird im Rahmen der Aufgabe manuell erzeugt und losgelöst. Auch die Definition des \acs{MQTT}-Topics ist vorgegeben. 
        Nachdem die simulierte Authentifizierung ausgelöst wurde, soll über die Steuerzentrale eine Regel ausgeführt werden, anhand dessen der simulierte 
        Service-Roboter gesteuert wird. Dafür sind folgende Anforderungen zu erfüllen: 
        \begin{itemize}
            \item Nach eingehendem Topic soll der Service-Roboter angesteuert und an die Tür geschickt werden. 
            (Die Ansteuerung des Service-Roboters erfolgt ebenso über \acs{MQTT}. Da in den meisten Fällen kein Roboter verfügbar ist, wird auch 
            diese Kommunikation mittels \acs{MQTT} simuliert.)
            \item Ist der Roboter an der Tür, soll er die Begrüßung starten. (Die folgende Interaktion wird im Rahmen des Test ebenso mittels der Simulation durchgeführt. 
            Wird die gesendete Nachricht über den Service-Roboter-\textit{Mock} ausgegeben, so gilt die Aufgabe als erledigt.)
        \end{itemize}
        Für die Aufgabe sind folgende Punkte zu berücksichtigen:
        \begin{itemize}
            \item Die Kenntnis über \acs{MQTT}-Topics, die für die Kommunikation benötigt werden.
            \item Ein Zustandsraum, der alle benötigten Komponenten abbildet.
            \item Der Service-Roboter als Komponente, sodass dieser bei Ausführung einer Aufgabe für weiter Aufgaben geblockt werden kann. 
            \item Der Auslöser, welcher durch ein \acs{MQTT} (PlugIn) abgebildet wird.
            \item Die Transformation der eingehenden \acs{MQTT}-Topics zu Zustandsänderungen, auf die Regeln ausgeführt werden sollen.
        \end{itemize}
        Dem Anhang (siehe \ref{appendix:usabilitytestpaper}) ist das Dokument zur Durchführung des Usability-Test beigefügt. 
        \\
        \linebreak
        Nach dem Abschluss der Aufgabe wurden die Probanden gebeten einen Fragebogen, welcher sich an dem \textit{System Usability Scale (SUS)} 
        Template orientiert, auszufüllen, um damit ihre Erkenntnisse und Meinungen mitzuteilen. Dieser Fragebogen ist ebenso dem Anhang (siehe \ref{appendix:usabilitytestpaper}) zu entnehmen Auf die Auswertung des Fragebogens wird in der 
        Zusammenfassung (\ref{subsec:usabilityFazit}) des Usability-Tests eingegangen. 

    \subsection{Fazit}
    \label{subsec:usabilityFazit}
        Die Durchführung der quantitativen Forschung kann nicht als repräsentativ eingestuft werden. Die Ergebnisse der 
        Usability-Tests zeigen im Rahmen einer kleinen Auswahl von Experten eine Tendenz, die zeigt, welche Anforderungen abgedeckt, bzw. 
        welche Verbesserungspotentiale identifiziert wurden. Der Usability-Test konnte erfolgreich durchgeführt werden. 
        Die Ergebnisse der Probanden waren bis auf die Namensgebung der Attribute im Zustandsraum, sowie der Objekte ähnlich. 
        Nachdem die praktische Aufgabe abgeschlossen war, wurden die Probanden im Anschluss gebeten, den Fragebogen des \textit{SUS} zu 
        beantworten. Folgender Auflistung ist der durchschnittliche Wert der Antworten zu entnehmen:
        \begin{enumerate}
            \item \textit{Ich denke, dass ich dieses System gerne öfter nutzen würde.} (80 Punkte)
            \item \textit{Ich fand das System unnötig komplex.} (8 Punkte)
            \item \textit{Ich fand das System einfach zu bedienen.} (76 Punkte)
            \item \textit{Ich denke, dass ich die Unterstützung einer technischen Person benötigen würde, um dieses System nutzen zu können.} (9 Punkte)
            \item \textit{Ich fand, dass die verschiedenen Funktionen in diesem System gut integriert waren.} (86 Punkte)
            \item \textit{Ich dachte, es gäbe zu viele Inkonsistenzen in diesem System.} (0 Punkte)
            \item \textit{Ich könnte mir vorstellen, dass die meisten Leute sehr schnell lernen würden, dieses System zu benutzen.} (89 Punkte)
            \item \textit{Ich fand das System sehr umständlich zu bedienen.} (5 Punkte)
            \item \textit{Ich fühlte mich sehr sicher mit dem System.} (76 Punkte)
            \item \textit{Ich musste viele Dinge lernen, bevor ich mit diesem System loslegen konnte.} (19 Punkte)
        \end{enumerate}
        Im Rahmen der Bewertung, unter Berücksichtigung, dass 100 die volle Zustimmung und 0 den Widerspruch darstellt, ist das Ergebnis 
        zufriedenstellend und zeigt das Potential des Frameworks. 
        \\
        Nachdem der Test abgeschlossen und der Fragebogen beantwortet war, wurde die Befragte Person abschließend noch interviewt, damit 
        Eindrücke über den Test hinaus gesammelt werden konnten. Diese Interviews werden in folgendem Abschnitt resümiert. 
        %1.: - 80 = 100, 75, 60, 80, 85 
        %2.: - 8 = 0, 15, 10, 10, 5 
        %3.: - 76 = 100, 80, 75, 75, 50 
        %4.: - 9 = 0, 5, 5, 10, 25 
        %5.: - 86 = 100, 90, 80, 85, 75 
        %6.: - 0 = 0, 0, 0, 0, 0 
        %7.: - 89 = 100, 100, 90, 85, 70
        %8.: - 5 = 0, 0, 5, 5, 15 
        %9.: - 76 = 85, 80, 75, 90, 50
        %10.: - 19 = 0, 15, 15, 20, 45 

\section{Experteninterview}
        Zur umfangreichen Informationsgewinnung wurde anschließend zu dem Usability-Test ein Interview mit dem jeweiligen Probanden 
        durchgeführt. Hierbei wurden sowohl die gesammelten Erkenntnisse und Erfahrungen während der Aufgabe erfragt, sowie auf die Anforderungen 
        (siehe Kapitel \ref{subsec:experteninterview}) eingegangen, die durch Experteninterviews erhoben wurden. Ein Anteil der Probanden wurde bereits 
        bei der Erhebung von Anforderungen durch Experteninterviews befragt. Somit konnte ein Bild geschaffen werden, wie die Anforderungen 
        umgesetzt wurden, bzw. diese von den Experten empfunden wurden.
    
    \subsection{Ziele des Experteninterviews}
        Das Ziel des abschließenden Experteninterviews ist es, die Erfahrungswerte der Probanden zu sammeln, um diese den Anforderungen gegenüberzustellen. Bei 
        Experten, die bereits zu der Anforderungserhebung interviewt wurden, kann konkret darauf eingegangen werden, wie die Kriterien und Anforderungen 
        umgesetzt wurden. Ebenso ist es wünschenswert ein abschließendes Feedback zu erhalten und ggf. Herausforderungen für den Anwender und Schwachstellen 
        des Systems zu identifizieren. 

    \subsection{Fazit}
        Das Experteninterview wurde, vergleichbar zu dem vorherigen Interview (siehe \ref{subsec:experteninterview}), mit einem semi-strukturierten Ansatz durchgeführt. 
        Anfangs wurde an den Fragebogen angeknüpft, um eine nachträgliche Zusammenfassung der Erkenntnisse jedes einzelnen zu erfahren. Sofern die Zusammenfassung bereits 
        die notwendigen Informationen enthielt, so wurde ein unstrukturierter Ansatz verfolgt und ein offener Dialog geführt. War die Zusammenfassung jedoch nicht 
        so aufschlussreich, so wurden anschließend konkret die 
        folgenden Fragen gestellt: 
        \begin{enumerate}
            \item \textit{„Wie ist Ihr erster Eindruck des Systems?“}
            \item \textit{„Gibt es aus Ihrer Sicht (Sicht des Anwenders) Mängel, die Ihnen die Nutzung des Frameworks erschweren?“}
            \item \textit{„Gab es Unklarheiten während der Anwendung des Frameworks? - Wenn ja, welche?“}
            \item \textit{„Haben Sie eine Idee oder Lösung diese Unklarheit zu beseitigen?“}
            \item \textit{„Gibt es allgemein Anregungen, bzw. Vorschläge für Verbesserungen?“}
            \item \textit{„Was gefällt Ihnen an dem Framework am meisten? - Was gefällt Ihnen nicht?“}
        \end{enumerate} 
        Rekapitulierend wurde ein überwiegend lehrreiches Ergebnis erzielt. Die Antworten gaben eine klare Meinung und Resonanz der Probanden wieder. 
        In Summe wurde ein erfreuliches Ergebnis erzielt. Aus Sicht des Anwenders gab es wenige Anmerkungen, da die Strukturen und Anforderungen dem Anwender 
        gegenüber deutlich sind. Mit der konkreten Anweisung, welcher Anwendungsfall umgesetzt werden soll, bzw. welche Randbedingungen dafür gelten, kann 
        schnell ein Ergebnis erzielt werden. 
        \\
        Eine erwähnenswerte Anmerkung war folgende:
        \begin{quote}
            „Wenn man wenig bis keine Erfahrung mit der Java Entwicklung hat, gibt es klar die ein oder andere Wissenslücke, die zuerst beseitigt werden muss. 
            Durch die wenigen Handlungsschritte, die jedoch viel Interpretationsfreiheiten bieten, besonders bei der Regeldefinition, ist dies doch überschaubar. 
            Bekommt man dementsprechend eine Hilfestellung, bzw. eine detailliertes Beispiel an die Hand, so können kleinere Schritte und Anwendungsfälle 
            relativ schnell umgesetzt werden.“
        \end{quote}
        Ein Ergänzungs-, bzw. Verbesserungsvorschlag, der während eines Interviews erwähnt wurde, ist die Optimierung zum Anlegen von sämtlichen Objekten. 
        Hierfür wurde vorgeschlagen, das eine Art \textit{\ac{CLI}}, wie bspw. die von Angular\footnote{TypeSkript basiertes Web Application Framework. \url{https://angular.io/} Besucht am 05.08.2022} 
        implementiert wird. Diesbezüglich war die Idee, über die Kommandozeile Befehle geben zu können, um bspw. Attribute im Zustandsobjekt anlegen zu können, leere Transformationen zu erzeugen, bzw. mit der 
        Übergabe von den dafür vorgesehenen Parameterwerten diese auch vollends zu erzeugen und das Erstellen eines leeren Regelkonstruktes. Dadurch müsste sich der Entwickler 
        ausschließlich um das Implementieren der Regel kümmern. Dadurch würde der notwendige Code generiert und dem Anwender zur Verfügung gestellt werden.
        
\section{Evaluation der Anforderungen}
    Im Rahmen der Arbeit werden die Anforderungen (siehe Kapitel \ref{sec:requirementsFinal}) abschließend anhand von dafür vorgesehenen Praktiken evaluiert. 
    Die Reihenfolge entspricht der tabellarischen Aufstellung der Anforderungen. Zuerst erfolgt die Abarbeitung der \textit{funktionalen Anforderungen}, 
    anschließend die der \textit{nicht funktionalen Anforderungen} und zum Schluss die der \textit{zusätzlichen Anforderungen}, die ebenso Randbedingungen adressieren.
    
    \subsection*{Funktionale Anforderungen}
        Während der Konzeption und Entwicklung des Frameworks wurden die funktionalen Anforderungen (FA1 - FA6, siehe Tabelle \ref{tab:functionalRequirements}) stets 
        berücksichtigt. Diese stellen die wichtigsten Anhaltspunkte dar und die grundlegende Funktionen bereit. Deren Erfüllung konnte durch die abschließenden 
        Experteninterviews identifiziert werden. Die Resonanz der Experten gab auch deutlich wieder, dass auf Grundlage der Anforderungen der Prototyp aufbaut. Die Struktur 
        des Regelobjektes (FA1) wird gegeben, indem eine Klasse von der Oberklasse erbt und diese erweitert. Umgesetzt wurde diese Anforderung mithilfe des 
        \textit{Template Method} Architekturmuster, auf das auch bereits in der Konzeption (siehe Kapitel \ref{chap:konzept}) eingegangen wird. Die Anforderung (FA2) ist 
        umgesetzt, indem der Regel über eine Methode Namens \textit{ruleTrigger} ein Wert eines \textit{Enum Types}\footnote{Spezieller Datentyp. \url{https://docs.oracle.com/javase/tutorial/java/javaOO/enum.html} Besucht am 05.08.2022} 
        zugeordnet wird. Ein unumgänglicher Teil des Frameworks ist der vom Anwender implementierte Zustandsraum, welcher über eine separate Klasse vorgegeben wird. Dadurch 
        ist auch die Anforderung (FA3) als umgesetzt anzusehen. Die Individualität der Implementierung des Zustandsraume ist durch die frameworkseitige Verwendung der 
        Java \textit{Generics} gegeben. Dadurch arbeitet das Framework mit dem vom Entwickler übergebenen Klassenobjekt. 
        \\
        \linebreak
        Zu den Regeln sind die Bedingungen zur Ausführung des Prozesses unabdingbar. Diese sind durch eine eigenständige Funktion der 
        Regelklasse gegeben und müssen vom Anwender aufgestellt werden (FA4). Die darauf folgende Forderung (FA5) ist prinzipiell gegeben, wird allerdings durch die Einschränkung im 
        Zustandsraum durch die Nutzung von Java \textit{Reflection}, bzw. die Umsetzung der inversen Transformation aktuell nicht als geeignet angesehen. Deshalb wird derzeit von der Einbindung von 
        Komponenten in das Zustandsobjekt abgeraten. Das Erstellen von Komponenten ist dennoch möglich und kann im Regelkontext angewendet werden, sofern bestimmte Eigenschaften für eine Regel relevant sind. Der 
        letzte Punkt der funktionalen Anforderungen (FA6) ist ebenso ermöglicht, da innerhalb der Regel keine Einschränkungen greifen, bis auf die Voraussetzung, dass die Regelbedingung, sowie der -prozess 
        implementiert werden. Zusätzlich können bspw. sämtliche \acs{HTTP}-Abfragen getätigt werden. Im Falle des Use Cases, Check-in mit einem Service-Roboter, werden \acs{HTTP}-Abfragen ausgeführt, damit 
        Informationen der Büroplatzbuchungssoftware abgerufen werden können. 

    \subsection*{Nicht funktionale Anforderungen}
        Zur Überprüfung der \textit{nicht funktionalen Anforderungen} wurden weitere Methoden zur Überprüfung verwendet. Zusätzlich zu den verifizierten \textit{User Stories} mittels der Usability-Tests 
        konnten ebenso die aufgestellten Forderungen, die allgemein die Benutzerfreundlichkeit (BF) adressieren, überprüft werden. Der durch die Experteninterviews aufgestellte Anspruch (NFA1) konnte 
        durch die Usability-Tests erfüllt werden, da die Probanden überwiegend einen Schritt benötigten, um die Regel den Framework zu übergeben. Grund dafür ist die Funktion der Steuerzentrale 
        \textit{addRule(...)}. Der Anwender kann die Funktion schnell zuordnen und bekommt mitgeteilt, welche Übergabeparameter die Funktion erwartet.
        \\
        \linebreak 
        Im Rahmen des überschaubaren, jedoch nicht repräsentativen Usability-Tests waren alle Teilnehmer in der Lage eine Regel zu erstellen und diese dem Regelwerk hinzuzufügen (NFA2). Durch die 
        rein programmatische Anwendung des Frameworks, kann der Entwickler zu jedem Zeitpunkt auf die Funktionen der Steuerzentrale zugreifen (NFA3). Dies wurde ebenso von den Testteilnehmern im anschließenden 
        Experteninterview bestätigt. Zu aktuellem Zeitpunkt sind die nutzbaren Funktionen sehr überschaubar und decken die zum derzeitigen Stand notwendigsten Funktion ab, darunter das Hinzufügen von Regeln, 
        das Erstellen aller notwendigen Objekte und Komponenten, das Aufsetzen der notwendigen \acs{MQTT}-Konfiguration, das Ergänzen von Transformationen, das Starten des \acs{MQTT}-Clients und das Erweitern um 
        zeitbasierte Regelauslöser. Die Anforderung (NFA4) ist ebenso gegeben, da der Anwender sich ausschließlich um die Definition der Regeln kümmern braucht. Dennoch gibt es zur weiteren Vereinfachung der 
        Regeldefinition zusätzliche Möglichkeiten, die im Rahmen dieser Arbeit nicht umgesetzt wurden. Diese werden im Ausblick aufgegriffen, um zu zeigen, welche potentiellen Erweiterungen stattfinden können. 
        \\
        \linebreak
        Ein weiterer Aspekt ist die allgemeine Zuverlässigkeit des Systems. Zur Überprüfung der Zuverlässigkeitsanforderung (NFA5) wurden im Rahmen der Entwicklung JUnit-Test implementiert, die 
        einen bestimmten Sachverhalt überprüfen. Demnach wird ein Zustand vorausgesetzt, der nach dem Auslösen einer Regel eintreffen sollte. Somit kann die Zuverlässigkeit überprüft werden. 
        Dennoch gibt es eine starke Abhängigkeit zur  der Regelbedingung, die der Anwender definiert. Ist diese nicht deutlich genug, kann es passieren, dass Regeln ungewollt ausgeführt werden. 
        Deshalb ist die Validierung der Regelbedingung durch den Entwickler essentiell. Somit ist die Zuverlässigkeit abhängig von den Interaktionen und der Implementierung des Anwenders. Grundsätzlich gibt es 
        systemseitig keine Auswirkung, die fälschlicherweise die Zuordnung und Ausführung gefährdet. 
        \\
        \linebreak
        Durch die Verwendung von \textit{Threads} werden Regel, deren Bedingung im Vorfeld zutreffen, asynchron ausgeführt. Dadurch können voneinander unabhängige Regel parallel abgearbeitet werden. Somit kann 
        die Performanz des Systems erhöht werden (NFA6). Die Reaktionszeit und die Performanz der Kommunikation ist sowohl abhängig von den jeweiligen Protokollen und Technologien, die eingesetzt werden, 
        sowie auch die dafür verwendete Hardware und Infrastruktur (NFA7). Eine konkrete Überprüfung der Anforderungen wurde im Rahmen der Arbeit nicht durchgeführt, da diese zu aktuellem Zeitpunkt keine 
        Auswirkungen auf das System selbst haben. Dennoch ist unter Verwendung der derzeitigen Technologien eine gewisse Performanz vorauszusetzen und zu erwarten. 

        %BF durch die Usability-Tests und Interviews
        % Testung mit Regel, die über bspw. einen cronjob ausgelöst werden sollen.
        %JUnit Test, die immer den gleichen Aufruf auslösen und ein bestimmtes Ergebnis erwarten
        % Perfomance Theading -> asnchrones ausführen von Regeln
        % Performance wird auf mqtt abgeladen
        % Verfügbarkeit ist aktuell abhängig von dem MQTT broker, da ohne Verbindung die Steuerzentrale nicht funktioniert, bzw. terminiert
        % Fehlertoleranz Fehlerhafte Regel werden ausgeführt, jedoch führen diese Zu Fehlern und ggf. zum abstürtzen. Dafür wäre evtl. eine Validierung der Regelsyntax vorhrer von Vorteil 
        % Kontrollierbarkeit ist einfach nachzuziehen. 
    
    \subsection*{Zusätzliche Anforderungen (Randbedingungen)}
        % Randbedingungen, die eingehalten wurden (Ja -> Wie?)
        % Kommunikation über MQTT -> Check
        % SPO in gewisser Maßen 


\section{Evaluation der Forschungsfrage}