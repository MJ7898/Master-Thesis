\chapter{Evaluation}
\label{chap:evaluation}
In diesem Abschnitt wird das erzielte Ergebnis des erarbeiteten Konzeptes (siehe Kapitel \ref{chap:konzept}) und des 
Prototypen analysiert. Anhand von ausgewählten Forschungsmethoden werden die vorab identifizierten Anforderungen überprüft 
und evaluiert. Mithilfe der erhobenen Informationen wird zum Ende des Kapitels die Forschungsfrage sowie das
Ziel der Arbeit abschließend betrachtet. 

\section{Usability-Test}
\label{sec:usabilitytest}
    Im Rahmen der Evaluation wurde ein Usability-Test durchgeführt, um bestimmte, für die Nutzbarkeit aufgestellte Anforderungen zu 
    überprüfen.
    Das angestrebte Ziel und die Ergebnisse der Tests werden in dem folgenden Abschnitt erläutert:
    
    \subsection{Ziel des Usability-Tests}
        Anhand des Usability-Tests sind die Anforderungen zu prüfen, die den Bereich der Nutzbarkeit abdecken. 
        Hierfür werden die Anforderungen, die den Tabellen (\ref{tab:functionalRequirements}, \ref{tab:notfunctionalRequirements} und \ref{tab:furtherRequirements}) 
        zu entnehmen sind, betrachtet. Die dafür berücksichtigten Anforderungen sind konkret die Folgenden: NFA1, NFA2, NFA3 und NFA4 (siehe \ref{tab:notfunctionalRequirements}).
        \\
        Damit die Erfüllung der Anforderungen überprüfbar ist, wird ausgewählten Anwendern der Zielgruppe %im Rahmen der Nutzbarkeit 
        eine Aufgabe gestellt, mit der das Framework getestet wird und dabei die oben genannten Anforderungen genauestens betrachtet werden. 
        Ziel dabei ist die Überprüfung der Erfüllung der oben genannten Anforderungen an das System, bzw. Verbesserungen zu eruieren. 

    \subsection{Durchführung des Usability-Tests}
        Für die Durchführung der Tests wurde vorab eine Aufgabe erstellt. Diese 
        dient als Grundlage zur Identifizierung der Usability-Anforderungen. Der Aufbau und das Vorgehen der Testung wurde für jeden Teilnehmer 
        identisch gestaltet. Vor Beginn wurde der Kontext sowie das Framework und dessen Kernkomponenten, die Rahmenbedingungen und die Aufgabenstellung selbst erläutert. 
        Die Aufgabe war so zu erstellen, dass sie umgebungsunabhängig erfüllt werden konnte. 
        Da die Tests ausschließlich auf die Nutzung des Frameworks abzielten, war die Umgebung sowie die 
        Anbindung verfügbarer Geräte zu vernachlässigen. Voraussetzung war die Verfügbarkeit eines \acs{MQTT}-Brokers. 
        Durch die fehlende Anbindung von reellen Geräten wurden die ausgehenden 
        \acs{MQTT}-Nachrichten durch eine Simulation, konkret die eines Service-Roboters, verarbeitet. Eingehende Nachrichten 
        wurden simuliert, indem \acs{MQTT}-Nachrichten manuell veröffentlicht wurden. 
        \\
        Die von den Anwendern durchzuführende Aufgabe ist wie folgt definiert: 
        \\
        \linebreak
        Es soll ein Anwendungsfall implementiert werden, bei dem ein Service-Roboter einen Mitarbeiter oder Gast, der an der Tür steht, 
        empfangen und begrüßen soll. Die Eingangsbedingung ist die Authentifizierung über eine Kamera. Der ganze Vorgang wird simuliert, indem die \acs{MQTT}-Nachricht manuell erzeugt und ausgelöst wird. 
        Auch die Definition des \acs{MQTT}-Topics ist vorgegeben. 
        Danach soll über die Steuerzentrale eine Regel ausgeführt werden, anhand derer der simulierte 
        Service-Roboter gesteuert wird. Dafür sind folgende Anforderungen zu erfüllen: 
        \begin{itemize}
            \item Nach eingehendem Topic soll der Service-Roboter angesteuert und an die Tür geschickt werden. 
            (Die Ansteuerung des Service-Roboters erfolgt ebenso über \acs{MQTT}. Da in den meisten Fällen kein Roboter verfügbar ist, wird auch 
            diese Kommunikation mittels \acs{MQTT} simuliert.)
            \item Ist der Roboter an der Tür, soll er die Begrüßung starten. (Die folgende Interaktion wird im Rahmen des Test ebenso mittels der Simulation durchgeführt. 
            Wird die gesendete Nachricht über den Service-Roboter-\textit{Mock} auf der Konsole ausgegeben, so gilt die Aufgabe als erledigt.)
        \end{itemize}
        Für die Aufgabe sind folgende Punkte zu erfüllen:
        \begin{itemize}
            \item Die Kenntnis über \acs{MQTT}-Topics, die für die Kommunikation benötigt werden.
            \item Ein Zustandsraum, der alle benötigten Komponenten abbildet.
            \item Der Service-Roboter als Komponente, sodass dieser bei Ausführung einer Aufgabe für weitere Aufgaben gesperrt werden kann. 
            \item Der Auslöser, welcher durch ein \acs{MQTT} (PlugIn) abgebildet wird.
            \item Die Transformation der eingehenden \acs{MQTT}-Topics zu Zustandsänderungen, auf die Regeln ausgeführt werden sollen.
            \item Die Regel, die den Vorgang der Begrüßung und des Check-ins vorgibt.
        \end{itemize}
        Dem Anhang (siehe \ref{appendix:usabilitytestpaper}) ist das Dokument zur Durchführung des Usability-Test beigefügt. 
        \\
        \linebreak
        Nach dem Abschluss der Aufgabe wurden die Probanden um ihre Meinung mithilfe eines Fragebogens, der sich an dem \textit{System Usability Scale (SUS)} 
        Template orientiert, gebeten. Dieser Fragebogen ist ebenso dem Anhang (siehe \ref{appendix:usabilitytestpaper}) zu entnehmen. Auf die Auswertung wird in der 
        Zusammenfassung (\ref{subsec:usabilityFazit}) des Usability-Tests eingegangen. 

    \subsection{Fazit}
    \label{subsec:usabilityFazit}
        Der Usability-Test konnte erfolgreich durchgeführt werden.
        Das Ergebnis der quantitativen Forschung kann nicht als repräsentativ eingestuft werden, da es sich um eine kleine Auswahl an Experten handelt. Es zeigt dennoch eine Tendenz, 
        welche Anforderungen abgedeckt sind, bzw. welche Verbesserungspotentiale in dem Framework stecken.  
        Das Resultat der Aufgabe ähnelte sich bei den Probanden mit Ausnahme der Namensgebung der Attribute im Zustandsraum. 
        \\
        Folgender Auflistung ist der durchschnittliche Wert der Antworten zu entnehmen:
        \begin{enumerate}
            \item \textit{Ich denke, dass ich dieses System gerne öfter nutzen würde.} (80 Punkte)
            \item \textit{Ich fand das System unnötig komplex.} (8 Punkte)
            \item \textit{Ich fand das System einfach zu bedienen.} (76 Punkte)
            \item \textit{Ich denke, dass ich die Unterstützung einer technischen Person benötigen würde, um dieses System nutzen zu können.} (9 Punkte)
            \item \textit{Ich fand, dass die verschiedenen Funktionen in diesem System gut integriert waren.} (86 Punkte)
            \item \textit{Ich dachte, es gäbe zu viele Inkonsistenzen in diesem System.} (0 Punkte)
            \item \textit{Ich könnte mir vorstellen, dass die meisten Leute sehr schnell lernen würden, dieses System zu benutzen.} (89 Punkte)
            \item \textit{Ich fand das System sehr umständlich zu bedienen.} (5 Punkte)
            \item \textit{Ich fühlte mich sehr sicher mit dem System.} (76 Punkte)
            \item \textit{Ich musste viele Dinge lernen, bevor ich mit diesem System loslegen konnte.} (19 Punkte)
        \end{enumerate}
        Die obigen Fragen aus dem \textit{SUS} \cite{brook1995} (siehe Kapitel \ref{subsec:usabilitytests}) wurden verwendet und ins Deutsche übersetzt. 
        \\
        Im Rahmen der Bewertung, bei der 100 die volle Zustimmung und 0 die Ablehnung darstellt, ist das Ergebnis 
        sehr positiv und zeigt das Potential des Frameworks. 
        \\
        Nachdem der Test abgeschlossen und der Fragebogen beantwortet war, wurde die befragte Person abschließend noch interviewt, damit 
        Eindrücke über den Test hinaus gesammelt werden konnten. Diese Interviews werden im folgenden Abschnitt resümiert. 
        %1.: - 80 = 100, 75, 60, 80, 85 
        %2.: - 8 = 0, 15, 10, 10, 5 
        %3.: - 76 = 100, 80, 75, 75, 50 
        %4.: - 9 = 0, 5, 5, 10, 25 
        %5.: - 86 = 100, 90, 80, 85, 75 
        %6.: - 0 = 0, 0, 0, 0, 0 
        %7.: - 89 = 100, 100, 90, 85, 70
        %8.: - 5 = 0, 0, 5, 5, 15 
        %9.: - 76 = 85, 80, 75, 90, 50
        %10.: - 19 = 0, 15, 15, 20, 45 

\section{Experteninterview}
        Zur umfangreichen Informationsgewinnung wurde anschließend zu dem Usability-Test ein Interview mit dem jeweiligen Probanden 
        durchgeführt. Hierbei wurden sowohl die gesammelten Erkenntnisse und Erfahrungen während der Aufgabe erfragt sowie auf die Anforderungen 
        (siehe Kapitel \ref{sec:requirementsFinal}) eingegangen. Ein Teil der Probanden wurde dafür bereits 
        bei der Erhebung von Anforderungen befragt. Somit konnte sich ein Bild verschafft werden, wie die Anforderungen 
        umgesetzt, bzw. diese von den Experten empfunden wurden.
    
    \subsection{Ziele des Experteninterviews}
        Das Ziel des abschließenden Experteninterviews ist es, die Erfahrungswerte der Probanden zu sammeln, um diese den Anforderungen gegenüberzustellen. Bei 
        Experten, die bereits zu der Anforderungserhebung interviewt wurden, konnte konkret auf die Umsetzung der Kriterien und Anforderungen eingegangen werden. 
        Ebenso ist ein abschließendes Feedback wünschenswert, sowie ein Identifizieren der Herausforderungen für den Anwender und der eventuellen Schwachstellen 
        des Systems. 

    \subsection{Fazit}
        Das Experteninterview wurde, vergleichbar zu dem vorherigen Interview (siehe \ref{subsec:experteninterview}), mit einem semi-strukturierten Ansatz durchgeführt. 
        Anfangs wurde an den Fragebogen angeknüpft, um eine nachträgliche Zusammenfassung der Erkenntnisse jedes einzelnen zu erfahren. Sofern die Zusammenfassung bereits 
        die notwendigen Informationen enthielt, wurde ein unstrukturierter Ansatz verfolgt und ein offener Dialog geführt. War die Zusammenfassung jedoch nicht 
        aufschlussreich, so wurden anschließend konkret die 
        folgenden Fragen gestellt: 
        \begin{enumerate}
            \item \textit{„Wie ist Ihr erster Eindruck des Systems?“}
            \item \textit{„Gibt es aus Ihrer Sicht (Sicht des Anwenders) Mängel, die Ihnen die Nutzung des Frameworks erschweren?“}
            \item \textit{„Gab es Unklarheiten während der Anwendung des Frameworks? - Wenn ja, welche?“}
            \item \textit{„Haben Sie eine Idee oder Lösung, um diese Unklarheit zu beseitigen?“}
            \item \textit{„Gibt es allgemein Anregungen, bzw. Vorschläge für Verbesserungen?“}
            \item \textit{„Was gefällt Ihnen an dem Framework am meisten? - Was gefällt Ihnen nicht?“}
        \end{enumerate} 
        Rekapitulierend wurde ein aufschlussreiches und erfreuliches Ergebnis erzielt. Die Antworten gaben ein klares Meinungsbild von der Einfachheit der 
        Handhabung der formalisierten Interaktionen der Steuerzentrale wieder und bestätigen die Erfüllung des Ziels dieser Arbeit. 
        Aus Sicht des Anwenders gab es wenige Anmerkungen zum Aufbau des Frameworks. 
        \\
        Eine erwähnenswerte Anmerkung war folgende:
        \begin{quote}
            „Wenn man wenig bis keine Erfahrung mit der Java Entwicklung hat, gibt es klar die ein oder andere Wissenslücke, die zuerst beseitigt werden muss. 
            Durch die wenigen Handlungsschritte, die jedoch viel Interpretationsfreiheiten bieten, besonders bei der Regeldefinition, ist dies doch überschaubar. 
            Bekommt man dementsprechend eine Hilfestellung, bzw. eine detailliertes Beispiel an die Hand, so können kleinere Schritte und Anwendungsfälle 
            relativ schnell umgesetzt werden.“
        \end{quote}
        Ein Ergänzungs-, bzw. Verbesserungsvorschlag, der während eines Interviews erwähnt wurde, ist die Optimierung des Anlegens von sämtlichen Objekten. 
        Hierfür wurde vorgeschlagen, eine Art \textit{\ac{CLI}}, wie bspw. die von Angular\footnote{TypeSkript basiertes Web Application Framework. \url{https://angular.io/} Besucht am 05.08.2022}, 
        zu implementieren. Die Idee war Befehle über die Kommandozeile geben zu können, bspw. das Anlegen von Attributen im Zustandsobjekt, das Erzeugen leerer, bzw. mit den dafür 
        vorgesehenen Parameterwerten befüllter Transformationen, und das Erstellen eines leeren Regelkonstruktes. Dadurch würde der notwendige Code generiert und dem Entwickler zur Verfügung gestellt werden, wodurch dieser 
        sich ausschließlich um das Implementieren der Regeln kümmern muss. 
        \\
        \linebreak
        Abschließend lässt sich anmerken, dass mit der konkreten Anweisung, welcher Anwendungsfall umgesetzt werden soll, bzw. welche Randbedingungen 
        dafür gelten, schnell ein Ergebnis erzielt werden kann. 
        
\section{Evaluation der Anforderungen}
    Im Rahmen der Arbeit werden die Anforderungen (siehe Kapitel \ref{sec:requirementsFinal}) zum Abschluss anhand von dafür vorgesehenen Praktiken evaluiert. 
    Die Reihenfolge entspricht der tabellarischen Aufstellung der Anforderungen. Zuerst erfolgt die Abarbeitung der \textit{funktionalen Anforderungen}, 
    anschließend die der \textit{nicht funktionalen Anforderungen} und zum Ende hin die der \textit{zusätzlichen Anforderungen}. 
    
    \subsection*{Funktionale Anforderungen}
        Während der Konzeption und Entwicklung des Frameworks wurden die funktionalen Anforderungen (FA1 - FA6, siehe Tabelle \ref{tab:functionalRequirements}) 
        berücksichtigt. Sie stellen den wichtigsten Kernpunkt dar. Deren Erfüllung konnte durch die abschließenden 
        Experteninterviews identifiziert werden. Die Resonanz der Experten gab auch deutlich wieder, dass auf Grundlage der Anforderungen der Prototyp aufbaut. Die Struktur 
        des Regelobjektes (FA1) wird gegeben, indem eine Klasse von der Oberklasse erbt und diese erweitert. Umgesetzt wurde diese Anforderung mithilfe des 
        \textit{Template Method} Architekturmusters, auf das auch bereits in der Konzeption (siehe Kapitel \ref{chap:konzept}) eingegangen wurde. 
        \\
        \linebreak
        Die Anforderung (FA2) ist 
        umgesetzt, indem der Regel mit der \textit{ruleTrigger}-Methode ein Wert eines \textit{Enum Types}\footnote{Spezieller Datentyp. \url{https://docs.oracle.com/javase/tutorial/java/javaOO/enum.html} Besucht am 05.08.2022} 
        zugeordnet wird. Ein essentieller Teil des Frameworks ist der vom Anwender zu implementierende Zustandsraum, welcher über eine separate Klasse vorgegeben wird. Dadurch 
        ist auch die Anforderung (FA3) als umgesetzt anzusehen. Die Individualität der Implementierung des Zustandsraume ist durch die frameworkseitige Verwendung der 
        Java \textit{Generics} gegeben. Demzufolge arbeitet das Framework mit dem vom Entwickler übergebenen Klassenobjekt. 
        \\
        \linebreak
        Die Ausprägungen der Bedingungen zur Ausführung der Regelprozesse sind unverzichtbar. Sie müssen vom Anwender aufgestellt werden (FA4). Damit eine Aufstellung erfolgen kann, ist in der Regelklasse eine 
        Funktion gegeben, die das Implementieren der Bedingung erzwingt. 
        Die darauffolgende Forderung (FA5) ist prinzipiell gegeben, wird allerdings durch die Einschränkung im 
        Zustandsraum aktuell nicht als geeignet angesehen. Die Nutzung von Java \textit{Reflection}, bzw. die Umsetzung der inversen Transformation erschwert dies. Derzeit wird von der Komponenteneinbindung 
        in das Zustandsobjekt abgeraten. Das Erstellen von Komponenten ist dennoch möglich und kann im Regelkontext angewendet werden, sofern bestimmte Eigenschaften für eine Regel relevant sind. 
        \\
        \linebreak
        Der letzte Punkt (FA6) ist ebenso umgesetzt, da innerhalb der Regel bis auf die Voraussetzung der Implementierung der Regelbedingung sowie des -prozesses keine Einschränkungen bestehen. 
        Zusätzlich können bspw. sämtliche \acs{HTTP}-Abfragen getätigt werden. Im Falle des Use Cases "Check-in mit einem Service-Roboter" werden \acs{HTTP}-Abfragen ausgeführt, damit 
        Informationen der Büroplatzbuchungssoftware abgerufen werden können. 

    \subsection*{Nicht funktionale Anforderungen}
        Zur Überprüfung der \textit{nicht funktionalen Anforderungen} wurden Methoden für deren Überprüfung verwendet. Zusätzlich zu den 
        verifizierten \textit{User Stories} mittels der Usability-Tests konnten ebenso die aufgestellten Forderungen, die allgemein 
        die Benutzerfreundlichkeit (BF) umfassen, überprüft werden. Der durch die Experteninterviews aufgestellte Anspruch (NFA1) konnte 
        durch die Usability-Tests erfüllt werden, da die Probanden selten mehr als eine Aktion benötigten, um die bereits erstellte Regel 
        dem Framework zu übergeben. Grund dafür ist die Funktion der Steuerzentrale \textit{addRule(...)}. Der Anwender kann die Aufgabe der Funktion 
        schnell zuordnen und bekommt über die Entwicklungsumgebung die erwarteten Übergabeparameter mitgeteilt. 
        \\
        \linebreak 
        Im Rahmen des Usability-Tests waren alle Teilnehmer in der Lage eine Regel zu erstellen 
        und diese dem Regelwerk hinzuzufügen (NFA2). Ausgehend von dieser Anforderung wurden 
        während der Experteninterviews Anregungen angebracht, die gegebenenfalls eine weitere Vereinfachung für den Entwickler darstellen. 
        Diese werden im Ausblick (siehe Kapitel \ref{chap:ausblick}) aufgegriffen. 
        \\
        Durch die rein programmatische Anwendung des Frameworks kann der Entwickler zu jedem Zeitpunkt auf die Funktionen der Steuerzentrale zugreifen (NFA3). 
        Dies wurde ebenso von den Testteilnehmern im anschließenden Experteninterview bestätigt. Zum aktuellen Zeitpunkt sind die nutzbaren Funktionen sehr 
        überschaubar und decken die notwendigsten Funktion ab, darunter das Hinzufügen von Regeln, 
        Erstellen aller wesentlichen Objekte und Komponenten, Aufsetzen der erforderlichen \acs{MQTT}-Konfiguration, Ergänzen von Transformationen, Starten 
        des \acs{MQTT}-Clients und das Erweitern um zeitbasierte Regelauslöser. 
        \\
        \linebreak 
        Die Anforderung (NFA4) ist ebenso gegeben, da der Anwender sich 
        ausschließlich um die Definitionen der Regeln, der Transformationen und des Zustandsraumes kümmern muss. Dennoch gibt es zur weiteren 
        Vereinfachung der Regeldefinition zusätzliche Möglichkeiten, die im Rahmen dieser Arbeit nicht umgesetzt wurden. Diese werden im Ausblick 
        aufgegriffen, um weitere Potentiale aufzuzeigen. 
        \\
        \linebreak
        Ein nennenswerter Aspekt ist die allgemeine Zuverlässigkeit des Systems. Zur Überprüfung der Zuverlässigkeitsanforderung (NFA5) wurden im Rahmen der 
        Entwicklung \textit{JUnit-Tests}\footnote{Test Framework für Java. \url{https://junit.org/junit5/} Besucht am 08.08.2022} implementiert, die 
        einen bestimmten Sachverhalt auslösen und überprüfen. Demnach wird ein Zustand erwartet, der nach dem Auslösen einer Regel eintreffen sollte. Anschließend 
        wird der Ist-Zustand mit dem Soll-Zustand abgeglichen und somit die Zuverlässigkeit überprüft. 
        Dennoch gibt es eine starke Abhängigkeit zur Regelbedingung, die der Anwender definiert. Ist diese 
        nicht deutlich genug, kann es passieren, dass Regeln ungewollt ausgeführt werden. Deshalb ist die Validierung der Regelbedingung durch den 
        Entwickler essentiell. Somit ist die Zuverlässigkeit abhängig von den Interaktionen und der Implementierung des Anwenders. Grundsätzlich gibt es 
        systemseitig keine negativen Auswirkungen von mehrdeutigen Bedingungen, die fälschlicherweise die Zuordnung und Ausführung einer Regel gefährden. Wird jedoch durch die 
        Regel der Zustand an einer anderen Stelle erneut geändert, ohne dass sich der vorherige Wert und dessen Bedingung wieder negiert hat, so würde das 
        Framework eine Endlosschleife und Rückkopplung bilden und dadurch einen \textit{StackOverflowError} erzeugen. Demnach ist die Auswahl der Bedingung einer der wichtigsten 
        Punkte, die zu berücksichtigen sind.
        \\
        \linebreak
        Durch die Verwendung von \textit{Threads} werden Regeln, deren Bedingungen im Vorfeld zutreffen, asynchron ausgeführt. Es können dadurch voneinander 
        unabhängige Regel parallel abgearbeitet werden. Somit ist die Performanz des Systems erhöht (NFA6). Die Reaktionszeit und die Performanz der 
        Kommunikation ist sowohl abhängig von den jeweiligen Protokollen und Technologien, die eingesetzt werden, sowie von der dafür verwendeten Hardware 
        und Infrastruktur (NFA7). Eine konkrete Überprüfung der Anforderungen wurde im Rahmen der Arbeit nicht durchgeführt, da diese zum aktuellen Zeitpunkt keine 
        Auswirkungen auf das System selbst haben. Dennoch ist unter Verwendung der derzeitigen Technologien eine gewisse Performanz vorauszusetzen und zu erwarten. 
        \\
        \linebreak
        Die \textit{nicht funktionalen Anforderungen}, die allgemein die Verfügbarkeit des Systems beschreiben (NFA8 \& NFA9), sind aktuell zu vernachlässigen und wurden im Rahmen dieser 
        Arbeit nicht überprüft. Das Framework selbst produziert von sich aus keine Downtime. Entscheidend sind dabei die Richtigkeit der Regeldefinition, 
        Verbindungen zum \acs{MQTT}-Broker und Internet, sowie die Stromversorgung, da das Framework lokal auf einem \textit{Raspberry Pi} betrieben wird. 
        \\
        \linebreak
        Zur Überprüfung der Fehlertoleranz (NFA10) wurden die \textit{JUnit-Tests} um weitere Testfälle ergänzt. Hierbei wurde eine fehlerhafte Regel definiert, die innerhalb des 
        Tests ausgelöst und ausgeführt wurde. Die Resultate erzielten eine gewisse Fehlertoleranz, die durch eine Weiterentwicklung stets verbessert werden kann.
        \\
        \linebreak
        Der letzte Punkt (NFA11) der Tabelle (\ref{tab:notfunctionalRequirements}) zählt als Erweiterung des Frameworks, sodass ein ständiger Informationsaustausch stattfindet und der 
        Entwickler das System überwachen kann und zu jedem Zeitpunkt einen Einblick in die Prozesse erlangt. Die Integration solch einer Lösung in das System ist ohne weiteres möglich.
    
    \subsection*{Zusätzliche Anforderungen (Randbedingungen)}
        Die Tabelle (\ref{tab:furtherRequirements}) beinhaltet weitere Anforderungen und Randbedingungen, die mithilfe der verwendeten 
        Erhebungstechniken identifiziert wurden. 
        \\
        Die Randbedingung (ZAF1), Nutzung der Programmiersprache Java, wurde eingehalten. Das Framework ist ausschließlich in Java entwickelt. 
        \\
        Für den Zustandsraum ist eine gewissen Modellierungsempfehlung vorgegeben, die keine Objekte als Felder im Zustandsraum zulässt. Die 
        Abbildung von reellen Zuständen findet 
        ausschließlich über Attribute statt. Somit ist die Mindestanforderung für (ZAF2) erfüllt. 
        \\
        \linebreak
        Die Anforderung (ZAF3) ist abgedeckt, da die Regelprüfung uns -ausführung über einen \textit{Thread Pool} in einen eigenen \textit{Thread} ausgelagert wird. Ebenso 
        ist (ZAF4) erfüllt, da die Kommunikation zum aktuellen Zeitpunkt ausschließlich über \acs{MQTT} erfolgt. Das Auslösen von Regeln über definierte Zeitpunkte kann durch 
        \textit{Cronjobs} oder durch die Annotation (\textit{\@ Scheduled}) des Spring Frameworks stattfinden. Demnach stellen diese beide Möglichkeiten die einzigen Optionen eines 
        Auslösers dar. In Folge dessen gilt auch die Forderung (ZAF5) als erledigt.
        \\
        \linebreak
        Durch die Verwendung der generischen Programmierung in Java ist sichergestellt, dass das Zustandsobjekt zur Laufzeit dem Framework übergeben wird (ZAF6). %Zusätzlich zum generischen 
        %Typ wird ebenfalls eine Objektinstanz erzeugt, die den Zustandsraum wiedergibt. 
        Somit kann das Framework mit dem Objekt arbeiten, unabhängig von der Struktur der Klasse.
        \\
        \linebreak
        Anhand der Anforderung (ZAF7) findet die Kommunikation überwiegend über \acs{MQTT} statt. Ein Punkt, der nicht vom Framework abgedeckt ist, ist die Sicherheit der 
        Kommunikation. Der Anwender muss sicherstellen, dass dementsprechende Vorkehrungen getroffen sind, damit der \acs{MQTT}-Broker und die darüber laufende Kommunikation abgeschirmt und für 
        Dritte nicht zugänglich sind (ZAF8). Im Rahmen der Arbeit ist der \acs{MQTT}-Broker ausschließlich im lokalen Netzwerk verfügbar und erlaubt den Zugriff nur für bestimmte Nutzer und Kommunikationsteilnehmer. 
        \\
        \linebreak
        Der \textit{\acl{SPC}} ist gegeben, da die Einstellungen und das Hinzufügen von Regeln und Transformationen ausschließlich über eine einzige Klasse funktionieren, die \textit{InnovationLogicHubApplication}. 
        \\
        \linebreak
        Nachdem die aufgestellten Anforderungen evaluiert wurden, wird im folgenden Abschnitt auf die Evaluation der Forschungsfrage eingegangen.
        \pagebreak

\section{Evaluation der Forschungsfrage}
    %Wie kann man die Usability von Prozessautomatisierungen und Regeldefinitionen von SmartHome-Plattformen optimieren, sodass die formalen Interaktionen für den Softwareentwickler einfacher in der Handhabung sind?
    Der in dieser Arbeit verfolgte Ansatz bietet dem Softwareentwickler eine Möglichkeit, mit wenigen Schritten eine 
    Prozessautomatisierung oder Regeldefinition basierend auf einem Anwendungsfall im Bereich des intelligenten 
    Büros, bzw. \textit{Smart Office} oder \textit{\acl{SH}}, umzusetzen. Hierbei wird im Vergleich zu anderen 
    Lösungen, darunter \acs{OPENHAB} und Home Assistant, nicht der Weg über eine Integration angestrebt. 
    Die Gestaltung solcher Komponenten, die über Integrationen abgebildet werden, findet in dem konzipierten 
    Framework über den Zustandsraum statt. Er lässt dem Entwickler die Ausprägung offen. Zusätzlich 
    wird darauf verzichtet eine Benutzeroberfläche bereitzustellen, mit der die Interaktionen gegebenenfalls 
    aufwändig, bzw. mehrschrittig sein könnten. Durch die alleinige Nutzung der Programmiersprache Java ist innerhalb des 
    Frameworks kein Kontextwechsel, bzw. Syntaxwechsel erforderlich, wodurch die Interaktionen ebenso vereinfacht bleiben. 
    Mithilfe der reduzierten Interaktionspunkte für den Entwickler ist gleichermaßen eine Formalisierung der Interaktionen 
    gewährleistet. So ist lediglich pro Anwendungsfall eine klar vorgegebene Struktur zu implementieren. Diese besteht aus mindestens einer 
    Transformation, einer Regel und einem Attribut im Zustandsraum. 
    \\
    \linebreak
    Die Usability-Tests sowie die Experteninterviews zeigen das Potential dieses Systems, welches bei Weitem nicht ausgeschöpft ist 
    und in Zukunft vorangetrieben werden kann. 
    \\
    \linebreak
    Durch die einfache Erweiterbarkeit, die durch die Architektur gewährleistet ist, können nützliche Funktionen hinzugefügt 
    werden. 
    \\
    In der Gesamtheit wurde das System als überaus hilfreich und entwicklungsfähig eingestuft. In dieser Arbeit stehen ausschließlich die 
    Konzeption, Grundlagenschaffung und prototypische Entwicklung zur Erreichung der Zielsetzung im Fokus. Die Grundlagen und -funktionen 
    wurden implementiert und zur Verfügung gestellt.
    \\
    \linebreak
    Das Framework kann dennoch nicht direkt mit den bestehenden Softwarelösungen, bspw. Home Assistant und \acs{OPENHAB}, verglichen 
    werden, da diese weitaus mehr Funktionalitäten anbieten, die im Rahmen dieser Arbeit nicht vorgesehen waren. 
    \\
    \linebreak
    Zusammenfassend lässt sich sagen, dass die Usability von Prozessautomatisierungen und Regeldefinitionen durch das Framework 
    optimiert werden können. Dieses bietet klare Strukturen und zeigt alle Handlungsschritte auf. Die Softwareentwickler erfahren 
    durch das Framework die einfache Handhabung der formalisierten Interaktionen. Das System bietet einen möglichen Lösungsweg, 
    der die Forschungsfrage beantwortet. 