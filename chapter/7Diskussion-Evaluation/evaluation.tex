\chapter{Evaluation}
\label{chap:evaluation}
In diesem Abschnitt wird das erzielte Ergebnis des erarbeiteten Konzeptes (siehe Kapitel \ref{chap:konzept}) und des 
Prototypen analysiert. Anhand von ausgewählten Forschungsmethoden werden die vorab identifizierten Anforderungen überprüft 
und evaluiert. Mithilfe der erhobenen Informationen wird zum Ende des Kapitels die Forschungsfrage, sowie das
Ziel der Arbeit abschließend evaluiert. 

\section{Usability-Test}
    Im Rahmen der Evaluation wurde ein Usability-Test durchgeführt, um bestimmte, für die Nutzbarkeit aufgestellte Anforderungen zu 
    überprüfen.
    Das angestrebte Ziel und die Ergebnisse der Tests werden in dem folgenden Abschnitt erläutert:
    
    \subsection{Ziel des Usability-Tests}
        Anhand des Usability-Tests sind die Anforderungen zu prüfen, die den Bereich der Nutzbarkeit adressieren. 
        Hierfür werden die Anforderungen, die den Tabellen (\ref{tab:functionalRequirements}, \ref{tab:notfunctionalRequirements} und \ref{tab:furtherRequirements}) 
        zu entnehmen sind, betrachtet. Die dafür berücksichtigen Anforderungen sind konkret die Folgenden: NFA1, NFA2, NFA3 und NFA4 (siehe \ref{tab:notfunctionalRequirements})
        \\
        Damit die Erfüllung der Anforderungen überprüfbar ist, wird einer Auswahl an Anwendern der Zielgruppe im Rahmen der Nutzbarkeit 
        eine Aufgabe gestellt, mit der das Framework getestet wird und dabei die oben genannten Anforderungen genauestens betrachtet werden. 
        Ziel dabei ist die Sicherstellung der Erfüllung von obigen Anforderungen, bzw. die Identifizierung von Verbesserungspotentialen. 

    \subsection{Durchführung des Usability-Tests}
        Für die Durchführung der Tests wurde vorab eine Aufgabe erstellt. Diese 
        dient als Grundlage zur Identifizierung der Usability-Anforderungen. Der Aufbau und das Vorgehen der Testung wurde für jeden Teilnehmer 
        ähnlich gestaltet. zu Anfang, bevor der Test durchgeführt wurde, war der Kontext sowie das Framework zu erläutern, damit 
        die Informationen und Vorbedingungen für jeden Teilnehmer zu vergleichen waren. Vorab war die Erläuterung der 
        Kernkomponenten des Frameworks sowie die Rahmenbedingungen und die Aufgabenstellung selbst notwendig. So war für jeden Teilnehmer 
        klar, welche Informationen zum Durchführen des Tests erforderlich waren. Die Realisierung der Tests wurde 
        an unterschiedlichen Orten durchgeführt. Für die Variation des Durchführungsortes war eine Aufgabe zu erstellen, die umgebungsunabhängig 
        erfüllt werden konnte. Da die Tests ausschließlich auf die Nutzung des Frameworks abzielten, war die Umgebung sowie die 
        Anbindung verfügbarer Geräte zu vernachlässigen. Damit der Testumfang nicht zu komplex wurde, war der Anwendungsfall der 
        umzusetzenden Aufgabe trivial. Für die Ausführung des Tests war eine Voraussetzung, dass bereits ein \acs{MQTT}-Broker 
        aufgesetzt war und für die Aufgabe zur Verfügung stand. Durch die Fehlende Anbindung von Geräten, wurden die ausgehenden 
        \acs{MQTT} Nachrichten nicht für die weitere Verarbeitung benötigt. Eingehende Nachrichten wurden simuliert, indem durch den Broker 
        manuell \acs{MQTT} Nachrichten veröffentlicht werden konnten.
        \\
        Die von den Anwendern durchzuführende Aufgabe ist wie folgt definiert: 
        \\
        \linebreak
        TBD - Als PDF oder als text?   
        

    \subsection{Fazit}
        Die Durchführung der quantitativen Forschung kann nicht als repräsentativ eingestuft werden. Die Ergebnisse der 
        Usability-Tests zeigen im Rahmen einer kleinen Auswahl von Experten eine Tendenz, die zeigt, welche Anforderungen abgedeckt, bzw. 
        welche Verbesserungspotentiale identifiziert wurden. 
        %TODO: ZUSAMMENFASSUNG DER DURCHGEFÜHRTEN TESTS (GROB)
        Nachdem die Tests durchgeführt wurden, wurde 
        mit den jeweiligen Experten ein abschließendes Interview geführt. Der Aufbau, die Durchführung und die Ergebnisse der Interviews 
        werden in folgendem Abschnitt dargelegt.

\section{Experten Interview}
    \subsection{Ziele des Experten Interviews}
    \subsection{Fazit}

\section{Evaluation der Anforderungen}
%\section{Analyse des Konzepts der Eigenentwicklung}
    % Nicht repräsentativ durch die fehlenden Probanden. Wird lediglich mit einem Usability Test und einem anschließenden 
    %Interview mit max. 3 Personen begründen und analysiert.

\section{Evaluation der Forschungsfrage}