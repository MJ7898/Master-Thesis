\chapter{Evaluation}
\label{chap:evaluation}
In diesem Abschnitt wird das erzielte Ergebnis des erarbeiteten Konzeptes (siehe Kapitel \ref{chap:konzept}) und des 
Prototypen analysiert. Anhand von ausgewählten Forschungsmethoden werden die vorab identifizierten Anforderungen überprüft 
und evaluiert. Mithilfe der erhobenen Informationen wird zum Ende des Kapitels die Forschungsfrage, sowie das
Ziel der Arbeit abschließend evaluiert. 

\section{Usability-Test}
\label{sec:usabilitytest}
    Im Rahmen der Evaluation wurde ein Usability-Test durchgeführt, um bestimmte, für die Nutzbarkeit aufgestellte Anforderungen zu 
    überprüfen.
    Das angestrebte Ziel und die Ergebnisse der Tests werden in dem folgenden Abschnitt erläutert:
    
    \subsection{Ziel des Usability-Tests}
        Anhand des Usability-Tests sind die Anforderungen zu prüfen, die den Bereich der Nutzbarkeit adressieren. 
        Hierfür werden die Anforderungen, die den Tabellen (\ref{tab:functionalRequirements}, \ref{tab:notfunctionalRequirements} und \ref{tab:furtherRequirements}) 
        zu entnehmen sind, betrachtet. Die dafür berücksichtigen Anforderungen sind konkret die Folgenden: NFA1, NFA2, NFA3 und NFA4 (siehe \ref{tab:notfunctionalRequirements})
        \\
        Damit die Erfüllung der Anforderungen überprüfbar ist, wird einer Auswahl an Anwendern der Zielgruppe im Rahmen der Nutzbarkeit 
        eine Aufgabe gestellt, mit der das Framework getestet wird und dabei die oben genannten Anforderungen genauestens betrachtet werden. 
        Ziel dabei ist die Sicherstellung der Erfüllung von obigen Anforderungen, bzw. die Identifizierung von Verbesserungspotentialen. 

    \subsection{Durchführung des Usability-Tests}
        Für die Durchführung der Tests wurde vorab eine Aufgabe erstellt. Diese 
        dient als Grundlage zur Identifizierung der Usability-Anforderungen. Der Aufbau und das Vorgehen der Testung wurde für jeden Teilnehmer 
        ähnlich gestaltet. Zu Anfang, bevor der Test durchgeführt wurde, war der Kontext sowie das Framework zu erläutern, damit 
        die Informationen und Vorbedingungen für jeden Teilnehmer identisch waren. Vorab war die Erläuterung der 
        Kernkomponenten des Frameworks, sowie die Rahmenbedingungen und die Aufgabenstellung selbst notwendig. So waren für jeden Teilnehmer 
        die Informationen zur Durchführung des Tests klar. Die Realisierung des Tests wurde 
        an unterschiedlichen Orten durchgeführt. Für die Variation des Durchführungsortes war eine Aufgabe zu erstellen, die umgebungsunabhängig 
        erfüllt werden konnte. Da die Tests ausschließlich auf die Nutzung des Frameworks abzielten, war die Umgebung sowie die 
        Anbindung verfügbarer Geräte zu vernachlässigen. Damit der Testumfang nicht zu komplex wurde, war der Anwendungsfall der 
        umzusetzenden Aufgabe trivial. Für die Ausführung des Tests war eine Voraussetzung, dass bereits ein \acs{MQTT}-Broker 
        aufgesetzt war und für die Aufgabe zur Verfügung stand. Durch die Fehlende Anbindung von reellen Geräten, wurden die ausgehenden 
        \acs{MQTT}-Nachrichten durch eine Simulation, konkret die Simulation eines Service-Roboters, verarbeitet. Eingehende Nachrichten 
        wurden simuliert, indem durch den Broker manuell \acs{MQTT}-Nachrichten veröffentlicht werden konnten.
        \\
        Die von den Anwendern durchzuführende Aufgabe ist wie folgt definiert: 
        \\
        \linebreak
        Es soll ein Anwendungsfall implementiert werden, bei dem ein Service-Roboter einen Mitarbeiter oder Gast, der an der Tür steht, 
        empfangen und begrüßen soll. Eingangsbedingungen sind, dass eine Kamera zur Authentifizierung für die Anwendung simuliert wird. 
        Die \acs{MQTT}-Nachricht wird im Rahmen der Aufgabe manuell erzeugt und losgelöst. Auch die Definition des \acs{MQTT}-Topics ist vorgegeben. 
        Nachdem die simulierte Authentifizierung ausgelöst wurde, soll über die Steuerzentrale eine Regel ausgeführt werden, anhand dessen der simulierte 
        Service-Roboter gesteuert wird. Dafür sind folgende Anforderungen zu erfüllen: 
        \begin{itemize}
            \item Nach eingehendem Topic soll der Service-Roboter angesteuert und an die Tür geschickt werden. 
            (Die Ansteuerung des Service-Roboters erfolgt ebenso über \acs{MQTT}. Da in den meisten Fällen kein Roboter verfügbar ist, wird auch 
            diese Kommunikation mittels \acs{MQTT} simuliert.)
            \item Ist der Roboter an der Tür, soll er die Begrüßung starten. (Die folgende Interaktion wird im Rahmen des Test ebenso mittels der Simulation durchgeführt. 
            Wird die gesendete Nachricht über den Service-Roboter-\textit{Mock} ausgegeben, so gilt die Aufgabe als erledigt.)
        \end{itemize}
        Für die Aufgabe sind folgende Punkte zu berücksichtigen:
        \begin{itemize}
            \item Die Kenntnis über \acs{MQTT}-Topics, die für die Kommunikation benötigt werden.
            \item Ein Zustandsraum, der alle benötigten Komponenten abbildet.
            \item Der Service-Roboter als Komponente, sodass dieser bei Ausführung einer Aufgabe für weiter Aufgaben geblockt werden kann. 
            \item Der Auslöser, welcher durch ein \acs{MQTT} (PlugIn) abgebildet wird.
            \item Die Transformation der eingehenden \acs{MQTT}-Topics zu Zustandsänderungen, auf die Regeln ausgeführt werden sollen.
        \end{itemize}
        Dem Anhang (siehe \ref{appendix:usabilitytestpaper}) ist das Dokument zur Durchführung des Usability-Test beigefügt. 
        \\
        \linebreak
        Nach dem Abschluss der Aufgabe wurden die Probanden gebeten einen Fragebogen, welcher sich an dem \textit{System Usability Scale (SUS)} 
        Template orientiert, auszufüllen, um damit ihre Erkenntnisse und Meinungen mitzuteilen. Dieser Fragebogen ist ebenso dem Anhang (siehe \ref{appendix:usabilitytestpaper}) zu entnehmen Auf die Auswertung des Fragebogens wird in der 
        Zusammenfassung (\ref{subsec:usabilityFazit}) des Usability-Tests eingegangen. 

    \subsection{Fazit}
    \label{subsec:usabilityFazit}
        Die Durchführung der quantitativen Forschung kann nicht als repräsentativ eingestuft werden. Die Ergebnisse der 
        Usability-Tests zeigen im Rahmen einer kleinen Auswahl von Experten eine Tendenz, die zeigt, welche Anforderungen abgedeckt, bzw. 
        welche Verbesserungspotentiale identifiziert wurden. Der Usability-Test konnte erfolgreich durchgeführt werden. 
        Die Ergebnisse der Probanden waren bis auf die Namensgebung der Attribute im Zustandsraum, sowie der Objekte ähnlich. 
        Nachdem die praktische Aufgabe abgeschlossen war, wurden die Probanden im Anschluss gebeten, den Fragebogen des \textit{SUS} zu 
        beantworten. Folgender Auflistung ist der durchschnittliche Wert der Antworten zu entnehmen:
        \begin{enumerate}
            \item Frage: \textit{Ich denke, dass ich dieses System gerne öfter nutzen würde.} (80 Punkte)
            \item Frage: \textit{Ich fand das System unnötig komplex.} (8 Punkte)
            \item Frage: \textit{Ich fand das System einfach zu bedienen.} (76 Punkte)
            \item Frage: \textit{Ich denke, dass ich die Unterstützung einer technischen Person benötigen würde, um dieses System nutzen zu können.} (9 Punkte)
            \item Frage: \textit{Ich fand, dass die verschiedenen Funktionen in diesem System gut integriert waren.} (86 Punkte)
            \item Frage: \textit{Ich dachte, es gäbe zu viele Inkonsistenzen in diesem System.} (0 Punkte)
            \item Frage: \textit{Ich könnte mir vorstellen, dass die meisten Leute sehr schnell lernen würden, dieses System zu benutzen.} (89 Punkte)
            \item Frage: \textit{Ich fand das System sehr umständlich zu bedienen.} (5 Punkte)
            \item Frage: \textit{Ich fühlte mich sehr sicher mit dem System.} (76 Punkte)
            \item Frage: \textit{Ich musste viele Dinge lernen, bevor ich mit diesem System loslegen konnte.} (19 Punkte)
        \end{enumerate}
        Im Rahmen der Bewertung, unter Berücksichtigung, dass 100 die volle Zustimmung und 0 den Widerspruch darstellt, ist das Ergebnis 
        zufriedenstellend und zeigt das Potential des Frameworks. 
        \\
        Nachdem der Test abgeschlossen und der Fragebogen beantwortet war, wurde die Befragte Person abschließend noch interviewt, damit 
        Eindrücke über den Test hinaus gesammelt werden konnten. Diese Interviews werden in folgendem Abschnitt resümiert. 
        %1.: - 80 = 100, 75, 60, 80, 85 
        %2.: - 8 = 0, 15, 10, 10, 5 
        %3.: - 76 = 100, 80, 75, 75, 50 
        %4.: - 9 = 0, 5, 5, 10, 25 
        %5.: - 86 = 100, 90, 80, 85, 75 
        %6.: - 0 = 0, 0, 0, 0, 0 
        %7.: - 89 = 100, 100, 90, 85, 70
        %8.: - 5 = 0, 0, 5, 5, 15 
        %9.: - 76 = 85, 80, 75, 90, 50
        %10.: - 19 = 0, 15, 15, 20, 45 

\section{Experteninterview}
        Zur umfangreichen Informationsgewinnung wurde anschließend zu dem Usability-Test ein Interview mit dem jeweiligen Probanden 
        durchgeführt. Hierbei wurden sowohl die Erfahrungen während der Aufgabe erfragt, sowie auf die durch Experteninterviews 
        erhobenen Anforderungen (siehe Kapitel \ref{subsec:experteninterview}) eingegangen. Ein Anteil der Probanden wurde bereits 
        bei der Erhebung von Anforderungen durch Experteninterviews befragt. Somit konnte ein Bild geschaffen werden, wie die Anforderungen 
        umgesetzt wurden, bzw. diese von den Experten empfunden wurden.
    \subsection{Ziele des Experteninterviews}
        Das Ziel des abschließenden Experteninterviews ist es, die Erfahrungswerte der Probanden zu sammeln, um diese den Anforderungen gegenüberzustellen. Bei 
        Experten, die bereits zu der Anforderungserhebung interviewt wurden, kann konkret darauf eingegangen werden, wie die Kriterien und Anforderungen 
        umgesetzt wurden. Ebenso ist es wünschenswert ein abschließendes Feedback zu erhalten und ggf. Herausforderungen und Schwachstellen zu identifizieren. 

    \subsection{Fazit}

\section{Evaluation der Anforderungen}
%\section{Analyse des Konzepts der Eigenentwicklung}
    % Nicht repräsentativ durch die fehlenden Probanden. Wird lediglich mit einem Usability Test und einem anschließenden 
    %Interview mit max. 3 Personen begründen und analysiert.

\section{Evaluation der Forschungsfrage}