\chapter{Anforderungsanalyse}
\label{chap:anforderungsanalyse}
    Dieses Kapitel befasst sich im Allgemeinen mit der Anforderungsanalyse und -erhebung. Hierbei wird eine
    Marktanalyse repräsentiert, um das Potential rundum \acl{SH} aufzuzeigen und ein Gefühl 
    dafür zu vermitteln, welche Anforderungen dabei entstehen können, bzw. bereits bestehen. Die Analyse ist 
    mit repräsentativen Statistiken, Studien und Umfragen belegt. Hauptsächlich wird im Rahmen der Anforderungserhebung 
    auf die Methodiken und Techniken eingegangen, die verwendet werden, um 
    Anforderungen zu identifizieren. Diese dienen als Grundlage für die Konzeption der Steuerungs-Plattform. Bestandteile der 
    Anforderungserhebung sind unter anderem zentrale Prozesse des \acl{RE}, ein \textit{user-centered Design}, im Deutschen 
    nutzerzentriertes Design, eine \textit{Target Group Analysis}, im Deutschen Zielgruppenanalyse, und die Durchführung von 
    Experten Interviews. Diese Interviews sind nicht repräsentativ und dienen lediglich der weiträumigeren Informationsgewinnung. 
    \\
    Vorab wird sichergestellt, dass im Rahmen des benutzerzentrierten Designs der Softwareentwickler als Nutzer im Vordergrund 
    steht, da dieser die Plattform betreibt, bzw. für die Erweiterung der Software als auch für die 
    Anpassungen auf die eigenen Anwendungsfälle zuständig ist.
    \\
    \linebreak
    Damit ein Eindruck entsteht, welches Marktpotential \acl{SH} Anwendungen haben und damit einhergehende Anforderungen mit 
    sich bringen, wird basierend auf gegebenen Studien, Statistiken und Umfragen eine Marktanalyse durchgeführt.
\section{Marktanalyse}
\label{sec:marktanalyse}
    Der Markt rundum \acl{SH} nimmt immer weiter zu. Sei es die Entwicklung von neuen intelligenten Geräten, die 
    Massentauglichkeit von Geräten %, die nicht lange am Markt bestehen 
    oder die stetig wachsende Abdeckung von Anwendungsfällen und Übernahme von Aufgaben und Prozessen. Durch die Vielzahl an 
    Produktanbietern und diversen Kommunikationsmöglichkeiten, ist es schwierig eine Lösung für alle Alternativen und 
    Produktausprägungen anzubieten. Hersteller versuchen mit der angebotenen Produktpalette ihr eigenes Ökosystem im Bereich 
    \acl{SH} zu erstellen, um die Nutzer abhängig zu machen. Der repräsentativen Studie von Deloitte zufolge ist jedoch eine 
    Insellösung bei den Nutzern in Deutschland nicht gefragt \cite{deloitte2018}. Befragt wurden 2000 Konsumenten zwischen 
    19 und 75 Jahren. Einem geringen Anteil von 22 Prozent der 
    Befragten ist die Erweiterbarkeit des Systems mit Produkten anderer Hersteller weniger, bzw. nicht wichtig. Im Gegensatz dazu 
    empfinden 43 Prozent der Befragten die Erweiterbarkeit als wichtig und 28 Prozent als sehr wichtig \cite{deloitte2018}. 
    Demnach müssen die Hersteller eine flexiblere Einsetzbarkeit gewährleisten, damit solche Systeme den Marktdurchbruch 
    erlangen. Dadurch wird die Entwicklung von Plattformen komplexer und umfangreicher. Beispielsweise sind die am weit 
    verbreitetsten Open Source Plattformen, openHAB und Home Assistant, sehr komplex und bilden ein großes Ökosystem ab, da 
    stetig der Zuwachs an integrierbaren Geräten zunimmt und damit der Funktionsumfang steigt.  

    \subsection{Allgemeine Marktsituation und Marktprognose}
    %Anbieter, Plattformen, Geräte
        Derzeit gibt es viele Anbieter für intelligente Produkte. Diese bieten zum einen einzelne Geräte an, die in 
        beliebige Plattformen integriert werden können und zum anderen ein eigenes Ökosystem, sofern der Anwender 
        mehrere Produkte des Anbieters nutzen möchte. Dennoch ist in den meisten Fällen die Konfiguration der Geräte nur auf 
        den hauseigenen Plattformen möglich. Somit kann der Nutzer nicht alle Komponenten nur über eine Plattform 
        konfigurieren und steuern. 
        \\
        \linebreak
        Eine repräsentative Umfrage der \ac{SGCS} mit 1384 Teilnehmern, welche im April 2022 veröffentlicht wurde, zeigt, welche 
        Anbieter in Deutschland am meist verbreitetsten sind, bzw. welche die Nutzer am häufigsten kaufen. An oberster Stelle 
        steht Philips und Samsung mit jeweils 25 Prozent und an dritter Stelle Bosch mit 23 Prozent. Weitere Anbieter können dem 
        Diagramm im Anhang (siehe \ref{appendix:brandings}) entnommen werden. Dabei sind jedoch weitaus nicht alle Hersteller und 
        Anbieter aufgelistet. Detaillierter wird an dieser Stelle jedoch nicht eingegangen. 
        %alle anbieter ein eigene ökosystem 
        %forecast und revenue allgemeiner smart home consum statista 
        %
 
\section{Zielgruppenanalyse}
\label{sec:zielgruppenanalyse}
    \subsection{Ziel der Zielgruppenanalyse}
        %Softwareentwickler, die das System betreiben und für ihre Bedürfnisse anpassen.

\section{Use Cases}
\label{sec:usecases}
    Für die Erhebung und Ausarbeitung aller Anforderungen an das System, werden Anwendungsfälle, sogenannte Use Cases, 
    definiert, bewertet und auf ihre Funktionalität geprüft und dokumentiert. Zum besseren Verständnis wird auf diese in den 
    folgenden Kapiteln eingegangen. 
\subsection{Check in mit Temi}
\subsection{Notfallevakuierung mit Temi}
\section{Experten Interview}
\label{sec:experteninterviewReqirements}
    %Die Experten Interviews sind eine Möglichkeit, die Anforderungen zu erfassen, die durch die Entwicklung des Systems 
    %nicht durch eine einfache Marktanalyse erfasst werden konnten. Diese Interviews sind nicht repräsentativ und dienen lediglich 
    %der weiträumigeren Informationsgewinnung.
\subsection{Ziele des Experten Interviews}
\section{Anforderungen}
