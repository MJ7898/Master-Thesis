\chapter{Anforderungsanalyse}
\label{chap:anforderungsanalyse}
    Dieses Kapitel befasst sich im Allgemeinen mit der Anforderungsanalyse und -erhebung. Hierbei wird eine
    Marktanalyse repräsentiert, um das Potential rundum \acl{SH} aufzuzeigen und ein Gefühl 
    dafür zu geben, welche Anforderungen dabei entstehen können, bzw. bereits bestehen. Die Analyse ist 
    mit repräsentativen Statistiken, Studien und Umfragen belegt. Hauptsächlich wird im Rahmen der Anforderungserhebung 
    auf die Methodiken und Techniken eingegangen, die verwendet werden, um 
    Anforderungen zu identifizieren. Diese dienen als Grundlage für die Konzeption der Plattform. Bestandteile der 
    Anforderungserhebung sind unter anderem zentrale Prozesse des \acl{RE}, ein \textit{user-centered Design}, im Deutschen nutzerzentriertes Design, eine 
    \textit{Target Group Analysis}, im Deutschen Zielgruppenanalyse, und die Durchführung von Experten Interviews.
    \\
    Vorab wird sichergestellt, dass im Rahmen des benutzerzentrierten Designs der Entwickler als Nutzer im Vordergrund steht.
\section{Marktanalyse}
\label{sec:marktanalyse}
    Der Markt rundum \acl{SH} nimmt immer weiter zu. Sei es die Entwicklung von neuen intelligenten Geräten, die 
    Massentauglichkeit von Geräten, die nicht lange am Markt bestehen oder die stetig wachsende Abdeckung von 
    Use Cases und Übernahme von Aufgaben und Prozessen.

    \subsection{Allgemeine Marktsituation und Marktprognose}
    %Anbieter, Plattformen, Geräte

    \subsection{Zielgruppenanalyse}


\section{Use Cases}
    Für die Erhebung und Ausarbeitung aller Anforderungen an das System, werden Anwendungsfälle, sogenannte Use Cases, 
    definiert, bewertet und auf ihre Funktionalität geprüft und dokumentiert. Zum besseren Verständnis wird auf diese in den 
    folgenden Kapiteln eingegangen. 
\subsection{Check in mit Temi}
\subsection{Notfallevakuierung mit Temi}
\section{Experten Interview}
\subsection{Ziele des Experten Interviews}
\section{Anforderungen}
