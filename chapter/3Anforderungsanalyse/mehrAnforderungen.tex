\subsubsection*{}
Ein Ausschnitt an Anforderungen, die ergänzend aus der Zielgruppenanalyse und den Anwendungsfällen entstanden, die sowohl die funktionalen 
(\ref{tab:functionalRequirements}) als auch die nicht funktionalen Anforderungen (\ref{tab:notfunctionalRequirements}) erweitern, ist der folgenden 
Tabelle zu entnehmen. Diese sind als Zusätzliche Anforderungen (ZAF) gekennzeichnet: 
\begin{table}[hbt!]
    \begin{center}
        \begin{tabular}{ | p{1.0cm} | p{9.2cm} | p{1.6cm} | p{3.1cm} | }
            \hline
                \textbf{} & \textbf{Anforderung} & \textbf{Priorität} & \textbf{Quelle} \\
            \hline
                ZAF1 & Zur Entwicklung des Frameworks soll Java als Programmiersprache verwendet werden. & Hoch & Experteninterview, Zielgruppenanalyse \\ 
            \hline
                ZAF2 & Die Abbildung von Komponenten soll über einen Zustandsraum dargestellt werden. & Essentiell & Anwendungsfall, Experteninterview \\ 
            \hline
                ZAF3 & Zur parallelen Ausführung von Regeln soll ein Thread Pool eingesetzt werden, der die Regeln jeweils als eigenen Thread laufen lässt. & Hoch & Anwendungsfall, Experteninterview \\ 
            \hline
                ZAF4 & Regeln sollen über zeitbasierte oder MQTT basierte Auslöser gestartet werden, nachdem die dafür vorgesehene Bedingung zutrifft. (Im Rahmen der Anwendungsfälle wird MQTT als Kommunikationsprotokoll und Auslöser gewählt.) & Hoch & Anwendungsfall \\
            \hline
                ZAF5 & Regeln und damit einhergehende Aktionen dürfen erst nach aktivieren eines bestimmten Auslösers (Triggers) ausgeführt werden. & Essentiell & Experteninterview \\
            \hline
                ZAF6 & Der Zustandsraum muss zur Laufzeit zur Verfügung stehen & Hoch & Experteninterview \\ 
            \hline
                ZAF7 & Die Kommunikation erfolgt überwiegend durch MQTT & Hoch & Anwendungsfall \\ 
            \hline
                ZAF8 & Sicherheit: Die Kommunikation soll über einen MQTT Broker im eigenen Netzwerk erfolgen. Das System soll nach außen nicht erreichbar sein. & Hoch & Experteninterview \\
            \hline
                ZAF9 & Single Point of Contact (SPC) (Implementierung, Anpassung und Erweiterung der Regel und der Logik). & Hoch & Zielgruppenanalyse, Experteninterview \\ 
            \hline
                ZAF10 & Die Zustände der Komponenten sollen persistiert werden, damit bei einem Systemausfall die letzten Transaktionen und Änderungen nachvollzogen werden können. (Ist zur Konzeption zu berücksichtigen, Priorität aus dem Proof of Concept (PoC) jedoch gering.) & Niedrig & Experteninterview \\ 
            \hline 
        \end{tabular}
    \end{center}
    \caption{Zusätzliche Anforderungen}
    \label{tab:furtherRequirements}
\end{table} 
\\
Alle relevanten Anforderungen, die mithilfe den genannten und durchgeführten Erhebungsmethoden erhoben wurden, fließen in die Konzeption 
des Frameworks mit ein. Diese bilden die Grundlage des Konzeptes und ergänzen den Kontext der zu erzielenden Lösung.
% Alle aufgeführten Anforderungen fließen mit in die Konzeption ein und bilden die Grundlage für die Ausarbeitung des 
% Konzeptes. Das folgende Kapitel (\ref{chap:konzept}) widmet sich dem Konzept.
