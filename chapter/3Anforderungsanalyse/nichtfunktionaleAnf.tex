\begin{table}[hbt!]
    \begin{center}
        \begin{tabular}{ | p{1.0cm} | p{9.7cm} | p{1.6cm} | p{2.6cm} | }
            \hline
                \textbf{} & \textbf{Anforderung} & \textbf{Priorität} & \textbf{Quelle} \\
            \hline
                NFA1 & Benutzerfreundlichkeit (BF): Der Anwender soll nach Implementierung bzw. Definition einer Regel, diese mit max. 3 Schritten in das Regelwerk einfügen können. & Hoch & Experten-interview \\
            \hline
                NFA2 & BF: Mehr als 90\% der Anwender sollten bei erster Verwendung des Frameworks in der Lage sein, Regeln anzulegen und diese dem Regelwerk hinzuzufügen. & Essentiell & Zielsetzung der Arbeit, Experteninterview \\ 
            \hline
                NFA3 & BF: Der Anwender soll die Möglichkeit haben, zu dem Zeitpunkt der Verwendung des Frameworks alle verfügbaren Funktionen nutzen und anwenden zu können. & Mittel & Experten-interview \\ 
            \hline
                NFA4 & BF: Der Anwender braucht sich zu keinem Zeitpunkt um das Management bzw. um den Ablauf und Durchführung von Regeln kümmern. Er muss nur sicherstellen, dass die Regeln valide sind und dem Kontext passende Bedingungen und Regelausführungen festlegt. & Essentiell & Experten-interview \\ 
            \hline
            %    NFA6 & Zuverlässigkeit (Z): Zeitlich gebundene Regeln sollen immer zu der vorgegebenen Uhrzeit ausgelöst werden. & Hoch & Experten-interview \\
            %\hline
                NFA5 & Zuverlässigkeit (Z): Als Anwender möchte ich, dass zu 100\% die Regel ausgeführt wird, die der Auslöser ansteuert. (Bei der Ausführung soll immer die Regel gestartet werden, die durch den Auslöser ausgelöst wurde.) & Essentiell & Experten-interview \\
            \hline
                NFA6 & Performanz (P): Ausgelöste Regeln, die beide voneinander unabhängige Ressourcen oder Attribute im Zustandsraum belegen, sollen parallel ausgeführt werden. & Hoch & Anwendungsfall, Experten-interview \\ 
            \hline
                NFA7 & P: Regeln, die über Kommunikationsprotokolle ausgelöst werden, sollen unter einer Sekunde starten. Die Zeit der Durchführung ist abhängig von der Regel selbst. & Mittel & Anwendungsfall, Experten-interview \\ 
            \hline
                NFA8 & Verfügbarkeit (V): Die Steuerzentrale muss eine Verfügbarkeit von 99,9\% vorweisen. (Zum aktuellen Zeitpunkt ist diese Anforderung zu vernachlässigen.) & Niedrig & Experten-interview \\ 
            \hline
                NFA9 & V: Die Steuerzentrale muss während der Arbeitszeiten zwischen 7-18 Uhr ohne Ausfälle verfügbar sein. (Zum aktuellen Zeitpunkt ist diese Anforderung zu vernachlässigen.) & Niedrig & Experten-interview \\ 
            \hline
                NFA10 & Fehlertoleranz: Syntaktisch oder inhaltlich fehlerhafte Regeln sollen nicht zum Absturz der Steuerzentrale führen. & Mittel & Experten-interview \\ 
            \hline  
                NFA11 & Kontrollierbarkeit, Beobachtbarkeit: Prozesse und Zustände der Steuerzentrale müssen eingesehen werden können, bspw. durch Monitoring oder über Oberflächen. (Zum aktuellen Zeitpunkt zu vernachlässigen.) & Niedrig & Experten-interview \\
            \hline
        \end{tabular}
    \end{center}
    \caption{Nicht funktionale Anforderungen (NFAs)}
    \label{tab:notfunctionalRequirements}
\end{table}