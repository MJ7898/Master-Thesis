\section{Experteninterview}
\label{sec:experteninterviewReqirements}
    Zur Analyse und Erhebung von Anforderungen, die sich an das System richten, werden Experteninterviews durchgeführt. 
    Dabei wird sich, wie in dem Abschnitt (\ref{subsec:experteninterview}) beschrieben, an dem unstrukturierten Ansatz der 
    Führung eines solchen Interviews orientiert \cite{robson2002real}. Deshalb werden keine konkreten offenen oder 
    geschlossenen Fragen gestellt. Der Aufbau des Interviews wird in weiteren Schritten aufgegriffen. 
    \\
    Die Ergebnisse der Interviews sind nicht repräsentativ, da sie im Rahmen dieser Arbeit mit einigen wenigen Experten durchgeführt wurden. Sie 
    dienen lediglich der Informationsgewinnung und dem Einholen mehrerer Meinungsbilder, um aus allen 
    Ideen, Gedankenanstößen und Ansichten ein Gesamtbild zu erzeugen und daraus mehrere Anforderungen zu gewinnen 
    und abzuleiten. 

\subsection{Ziel des Experteninterviews}
    Ziel des Experteninterviews ist die Informationsgewinnung aus bestimmten Sachverhalten, für die es 
    noch keine repräsentativen Umfragen gibt. Der Experte kann seine Sicht auf den Sachverhalt 
    wiedergeben und neue Blickwinkel eröffnen. Dadurch können Rückschlüsse gezogen und Erfahrungen ausgetauscht werden. Diese sind oft wichtige 
    Informationen zur Erhebung von Anforderungen zu einem bestimmten Sachverhalt. 

\subsection{Aufbau des Experteninterviews}
    Die Experteninterviews wurden im Gesamten als unstrukturierte Interviews durchgeführt. Lediglich die Rahmenbedingungen sowie 
    der Einstieg in das Interview waren vorgegeben und ähneln sich bei jeder interviewten Person. Zu Anfang des Gesprächs wurde 
    der Kontext und die Intension erläutert, damit der Experte die Situation und die Absichten kennenlernt und die 
    eigentliche Herausforderung 
    erkennt. Grundlage dafür war die Erläuterung der Zielsetzung der Arbeit (\ref{sec:zielsetzung}) zur Verdeutlichung der 
    Intension. Mit den identifizierten Anwendungsfällen (\ref{sec:usecases}), die als Basis 
    zur Extraktion von Anforderungen und als potenziell umsetzbare Funktionalitäten gelten, wurden Szenarien veranschaulicht, 
    die dabei halfen, das Anwendungsumfeld zu konkretisieren.  
    Nach der Schilderung des Kontextes und der zugrunde liegenden Ausgangspunkte wurde das Gespräch in Richtung Anforderungen 
    gelenkt. Hierbei lag der Fokus auf dem Sammeln von Ideen, Sichtweisen und Meinungen, die als Grundlage 
    dienten oder gar direkte Anforderungen an das System ergaben. Dabei war der Ausgang des Gesprächs offen. Falls während 
    eines Gespräches der Fokus verloren ging bzw. Exkurse ein zu großes Ausmaß annahmen, wurde wieder auf die 
    vorliegende Sachlage aufmerksam gemacht und der Fokus erneut auf die Anforderungen geworfen. 
    Schwerpunkte bei den Interviews waren zum einen die Anforderungen, welche für eine einfache Handhabung der 
    formalisierten Interaktionen für den Softwareentwickler gelten, und zum anderen die Funktionalitäten, die der 
    Experte dem System gegenüber sieht, um ein Regelwerk für ein intelligentes Büro zu erstellen. 
    Ebenso wurden nicht funktionale Anforderungen, die ein solches System mit sich bringen soll, adressiert. 
    Die zusammengefassten Anforderungen und die daraus abgeleiteten Bedingungen sind dem Abschnitt (\ref{sec:requirementsFinal}) 
    zu entnehmen.
     
\subsection{Zusammenfassung der Experteninterviews}
    In Summe wurden insgesamt fünf Experteninterviews durchgeführt. Jedes Gespräch war individuell und hatte dementsprechend 
    einen anderen Verlauf bzw. ein anderes Ergebnis. Dennoch konnte der inhaltliche Fokus gewahrt und verschiedene 
    Meinungsbilder eingeholt werden. Jeder befragte Experte konnte zu dem anliegenden Sachverhalt seine Meinung äußern und 
    wichtige Informationen und Sichtweisen mitteilen. Die erhobenen Informationen wurden analysiert, aufbereitet und 
    als Anforderungen dokumentiert. Die dabei entstandenen Informationen und Anforderungen werden in 
    folgendem Abschnitt aufgezeigt. 
    
%\pagebreak
\section{Anforderungen}
\label{sec:requirementsFinal}
    In Folge der vorangestellten Literaturrecherche, der Markt- und Zielgruppenanalyse, der entwickelten Anwendungsfälle und 
    der durchgeführten Experteninterviews ergaben sich Anforderungen an das System, 
    die in der Konzeption und der anschließenden prototypischen Implementierung zu berücksichtigen sind. 
    Zur Dokumentation der nicht funktionalen Anforderungen wird sich an dem 
    \textit{Software Produkt Qualitätsmodell} nach der ISO Norm 25010\footnote{Qualitätscharakteristiken nach ISO. \url{https://iso25000.com/index.php/en/iso-25000-standards/iso-25010} Besucht am 24.06.2022.} 
    orientiert. 
    \\
    %\linebreak
    Der folgenden Tabelle (\ref{tab:functionalRequirements}) sind die funktionalen Anforderungen (FA) zu entnehmen: 
    %\pagebreak
    \begin{table}[hbt!]
        \begin{center}
            \begin{tabular}{ | p{0.6cm} | p{9.5cm} | p{1.6cm} | p{3.1cm} | }
                \hline
                    \textbf{} & \textbf{Anforderung (User Story)} & \textbf{Priorität} & \textbf{Quelle} \\
                \hline
                    FA1 & Als Softwareentwickler möchte ich mit einer vorgegebenen Struktur Regeln definieren, die zur Laufzeit ausgeführt werden können. & Essentiell & Zielsetzung der Arbeit \\  %, wenn das zutreffende Ereignis eintrifft.
                \hline
                    FA2 & Als Softwareentwickler möchte ich, dass alle Regeln, die ich definiert habe, zu bestimmtem Auslöser gestartet werden. & Essentiell & Experteninterview \\
                \hline
                    FA3 & Als Softwareentwickler möchte ich einen Zustandsraum erstellen, der individuell implementiert werden kann. & Essentiell & Experteninterview \\ 
                \hline
                    FA4 & Als Softwareentwickler möchte ich Bedingungen definieren und abfragen können, die basierend auf dem Ereignis die dazugehörige Regel ausführen. & Essentiell & Experteninterview \\ 
                \hline
                    FA5 & Als Softwareentwickler möchte ich Komponenten anlegen können, die reelle Gegenstände und dessen Zustände abbilden. & Hoch & Experteninterview \\
                \hline
                    FA6 & Als Softwareentwickler möchte ich das vorgegebene Framework individuell nutzen können, indem ich bei der Definition der Regeln und der Informationsbeschaffung nicht eingeschränkt bin. & Essentiell & Zielsetzung der Arbeit, Experteninterview \\
                \hline
                %    FA7 & Als Softwareentwickler möchte ich Auslöser aktivieren bzw. definieren können, nach denen die Regeln ausgelöst werden. & Essentiell & Experteninterview \\ 
                %\hline
            \end{tabular}
        \end{center}
        \caption{Funktionale Anforderungen (FAs)}
        \label{tab:functionalRequirements}
    \end{table}
    %\\
    %\pagebreak
    %Nach der Aufstellung der funktionalen Anforderungen folgt die der nicht funktionalen Anforderungen (NFA):
   