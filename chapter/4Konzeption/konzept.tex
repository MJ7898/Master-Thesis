\chapter{Konzeption}
\label{chap:konzept}
    In diesem Kapitel wird das erarbeitete Konzept dieser Arbeit dargelegt. Basierend auf den 
    Anforderungen, die aus den Anwendungsfällen, Experteninterviews und der Zielgruppenanalyse 
    erhoben wurden, werden die daraus generierten Überlegungen und Entscheidungen transparent 
    dargestellt. Durch die bereits erfolgten Schritte der Anforderungsanalyse (\ref{chap:anforderungsanalyse})
    sind erste Aufgaben der Konzeption abgeschlossen. 
    \\
    Zu Anfang des Kapitels wird das allgemeine Ziel eines Konzeptes, sowie das konkrete Ziel dieser Arbeit 
    erläutert. Anschließend 
    wird auf das Anwendungsumfeld des Systems, bzw. des Frameworks, (\ref{sec:anwendungsumfeld}) eingegangen. 
    % die abzudeckenden Funktionalitäten (\ref{sec:konzeptfunktionalitaet}), die 
    %aus den Anforderungen ermittelt wurden, eingegangen. 
    Darauf folgend wird anhand den 
    zugrundeliegenden Informationen das Architekturkonzept (\ref{sec:architekturkonzept}), sowie das 
    Softwarekonzept (\ref{subsec:softwarekonzept}) erläutert. Die Darstellung des Konzeptes stützt sich 
    mit Diagrammen und Veranschaulichungen.
    %Des Weiteren werden die Hintergründe der 
    %Wahl des Frameworks (\ref{sec:frameworkauswahl}) aufgezeigt. %Abschließend wird das konzipierte 
    % und prototypische Datenmodell \ref{sec:...} dargelegt. 

\section{Ziel der Konzeption}
\label{sec:konzeptziele}
    Das Ziel einer Konzeption ist die Veranschaulichung von abstrakten Ideen, geistiger Entwürfe und Leitideen. 
    Hierbei werden aus den zugrundeliegenden Problemstellungen, Szenarien und Anforderungen Entwürfe und 
    Lösungsmöglichkeiten erarbeitet und identifiziert. Diese helfen bei der Aufstellung von notwendigen Schritten 
    und dienen als Grundlage zur Untermauerung und Darlegung von Entscheidungen. Somit wird Dritten der Kontext, die 
    Domäne und das zu lösende Problem, bzw. die Lösung dargestellt. 
    \\
    \linebreak
    Im Rahmen dieser Arbeit ist das Ziel des Konzeptes die Veranschaulichung der Entscheidungsfindung zur Lösung des vorliegenden 
    Sachverhaltes. Das Konzept 
    erarbeitet eine Lösung zur Implementierung einer Anwendung zur Koordination von Regeln und Prozessen innerhalb eines 
    Firmenbüros, mit dem Fokus der einfachen Nutzung des Frameworks durch den Anwender und die uneingeschränkte Ausprägungsvielfalt von Regelprozessen. 
    Hierbei wird der allgemeine Aufbau der Architektur skizziert und demonstriert, wie eine solche Lösung aussehen kann. 
    Unter Berücksichtigung der Forschungsfrage (siehe Abschnitt \ref{sec:forschungsfragen}) wird eine Möglichkeit offengelegt, mit der 
    ein Softwareentwickler neue Regeln entwickeln und dem System hinzufügen kann, ohne ein weiteres zu erlernendes Framework zu verwenden. 
    Dabei sollen die notwendigen Schritte und Interaktionen formalisiert und für den Entwickler vereinfacht werden. Mit der 
    Definition der Zielsetzung der Arbeit (siehe Abschnitt\ref{sec:zielsetzung}) wird die Abgrenzung deutlich. In folgender 
    Darlegung %des Konzeptes 
    wird nochmals konkreter auf den Kontext als auch auf die Intension der Arbeit eingegangen. 
    % Das Ziel des Konzeptes ist es, dem Entwickler den Aufwand zur Erweiterung des Systems zu minimieren durch weitere 
    % Regeln und Abdeckung von Anwendungsfällen (Use Cases) und eine Struktur vorgeben. (ToDo's, Flexibilität in der 
    % Umsetzung (nicht wie Home Assistant und openHAB eher eingeschränkt)) 

%\section{Abzudeckende Funktionalitäten}
%\label{sec:konzeptfunktionalitaet}
    % Was soll der Entwickler machen können? 
    % Welche Grundlagen braucht er, um eine Regel implementieren zu können?
    % Welche Funktionen müssen gegeben sein, um die Struktur vorzugeben? 
    % Reicht ein Hinweis weöche Stellen angepackt werden müssen, um eine Regel hinzuzufügen? 
    
    %%%%%%%%%%%%%%%%%%%%%%%%%%%%%%%%%%%%%%%%%%%%%%%%%%%%%%%%%%%%%

    % ZIEL DES KONZEPTES: Ein Framework für Entwickler bereitzustellen, welches die Mächtigkeit für den Entwickler offen lässt, nicht einschränkt 
    % und dennoch Konfiguration und Ausführung umsetzt. Der Entwickler muss lediglich den Zustandsraum, die MQTT-Topics und die Regeln definieren.
    % Der Entwickler bekommt ein Framework in die Hand, welches die Umsetzung von Prozessen in einem smarten Büro ermöglicht. Das Framework kümmert sich um die 
    % Organisation und die Ausführung der Regeln. Die Richtigkeit der Regeln und des Zustandsraumes muss der Entwickler sicherstellen. 
    % Die Kommunikation über MQTT ist nur eine Möglichkeit. Des Setup wird wegabstrahiert 

    %%%%%%%%%%%%%%%%%%%%%%%%%%%%%%%%%%%%%%%%%%%%%%%%%%%%%%%%%%%%%

\section{Anwendungsumfeld}
\label{sec:anwendungsumfeld}
    Grundsätzlich ist der Einsatzort des Frameworks variabel, da die eigentliche Implementierung und Nutzung der Regeln und Prozesse stark 
    abhängig von den Anwendern ist. Dadurch kann sowohl in privatem \acl{SH} Umfeld als auch in Büroräumen eine solche Instanz mittels dem 
    Framework erstellt werden. Basierend auf den vorangestellten Tätigkeiten, darunter die Anforderungsanalyse und die Eingrenzung auf den 
    Einsatz im Smart Office, liegt der Schwerpunkt auf dem Einsatz in einem smarten Büro. 

\section{Konzept}
\label{sec:concept}
    Die Intension, die hinter der Ausarbeitung dieses Konzeptes steht, ist zum einen die einfache Handhabung der 
    formalisierten Interaktionen für Softwareentwickler während der Verwendung des Frameworks und zum anderen die 
    offene Gestaltung von Regelprozessen. In vergleichbaren Softwareprodukten, die im Rahmen dieser Arbeit erläutert 
    wurden (siehe Kapitel \ref{sec:homeassistant} und \ref{sec:openhab}), ist die Vielfalt der Regelausprägung auf 
    den Kontext des Systems eingeschränkt. Dies bedeutet, dass 
    Regeln und Prozesse nur mit Komponenten und Informationen innerhalb des Systems arbeiten können, bzw. benötigte 
    Informationen erst durch eine systemseitige Erweiterung durch Plugins verfügbar sind. Diese Auswirkung ist der 
    Tatsache geschuldet, dass mit den bestehenden Lösungen versucht wird, die Regeldefinition für Endnutzer so 
    einfach wie möglich zu gestalten.
    \\
    Um dennoch dem Anwender eine Struktur vorzugeben, mit der Regeln definiert und zur Laufzeit der Anwendung ausgeführt 
    werden können, soll mit diesem Konzept ein Framework erarbeitet werden, dass diese Herausforderung löst. Hierfür soll dem 
    Anwender die Komplexität der Regelverwaltung und deren Durchführung nicht vorgehalten werden. Dieser ist lediglich in der 
    Verantwortung, die für ihn notwendigen Regeln und Prozesse zu definieren und dem Framework bereitzustellen. Dadurch soll der 
    Person, die das Framework verwendet, die Möglichkeit geboten werden, das zur Verfügung gestellte System mit individuellen 
    Regeln, dafür vorgesehene Bedingungen, Komponenten und dessen Zustände zu implementieren.



\section{Architekturkonzept}
\label{sec:architekturkonzept}
    % Es wird alles abgebildet über einen Zustandsraum, der sich aus den Dingen (Gegenständen) und Zuständen der Anwendung ergibt.
    % Der Zustandsraum wird verändert, wenn eine Aktion durchgeführt wird, bzw. durch eine Trigger angestoßen. 
    % Bzw. speichert den aktuellen Zustand des Gegenstandes 
    % (lightBulb = true/false, personOnDoor = null/Mikka, booking = stringBooking, temiAktive = true/false, 
    % temiPosition = stringKoordinates)
    % Zustandsraum muss von dem Entwickler definiert werden. 
    % MQTT Broker über Home Assistant, bzw. losgelöster Broker
    % Anbindung von APIs auch Entwickler-Sache. Kann ich das vereinfachen, sodass die Integration einfacher wird?

    \subsection{Überlegungen, Anstöße und Herausforderungen}
    % Regeln über Thread abbilden? Ja, Nein? - Nein, wieso? Da Durch die MQTT Message mehrere Regeln ausgeführt 
    %werden können. -> Lediglich den Zustand der Komponenten locken.
    % KEIN THREAD (wird schon abgebildet durch die Services und die Auslöser durch MQTT), Falls eine Komponente 
    %  doppelt beansprucht wird, ist der Zustand der Komponenten zu locken und ein 
    % Thread.sleep einzurichten. Abfrage, ob der Wert, bzw. die Komponente wieder freigegeben wurde. 

    % Zustandsraum -> Abbildung aller notwendigen Komponenten 
    % Bei Bearbeitung einer Regeln die Komponenten Locken, sodass nur die einzelne Komponenten (deren Zustand) gelockt ist 
    % und nicht der ganze Zustandsraum, somit können mehrere Komponenten und Aktionen ausführen zu können. 

    %Was brauche ich für Funktionen und Werte in einer Regel?

    % Ein Zustandsraum (Objekt) für alles oder ein Globales, welches die die Komponenten enthält? - Begründung für die Auswahl.
    
    \subsection{Schnittstellen}
        % Kommunikation mit API's je nach Use Case und Gebrauch zur Datenabfrage
    
    \subsection{Datenbanken}
        % Datenbanken je nach Use Case und Gebrauch zur Datenabfrage

\subsection{Softwarekonzept}
\label{subsec:softwarekonzept}



