\chapter{Konzeption}
\label{chap:konzept}
    In diesem Kapitel wird das erarbeitete Konzept dieser Arbeit dargelegt. Basierend auf den 
    Anforderungen, die aus den Anwendungsfällen, Experteninterviews und der Zielgruppenanalyse 
    erhoben wurden, werden die daraus generierten Überlegungen und Entscheidungen transparent 
    dargestellt. Durch die bereits erfolgten Schritte der Anforderungsanalyse (\ref{chap:anforderungsanalyse})
    sind erste Aufgaben der Konzeption abgeschlossen. 
    \\
    Zu Anfang des Kapitels wird das allgemeine Ziel eines Konzeptes erläutert. Anschließend 
    wird auf die abzudeckenden Funktionalitäten (\ref{sec:konzeptfunktionalitaet}), die 
    aus den Anforderungen ermittelt wurden, eingegangen. Darauf folgend wird anhand den 
    zugrundeliegenden Informationen das Architekturkonzept (\ref{sec:architekturkonzept}), sowie das 
    Softwarekonzept (\ref{sec:softwarekonzept}) erläutert. Des Weiteren werden die Hintergründe der 
    Wahl des Frameworks (\ref{sec:frameworkauswahl}) aufgezeigt. %Abschließend wird das konzipierte 
    % und prototypische Datenmodell \ref{sec:...} dargelegt. 

\section{Ziel der Konzeption}
\label{sec:konzeptziele}
    % Das Ziel des Konzeptes ist es, dem Entwickler den Aufwand zur Erweiterung des Systems zu minimieren durch weitere 
    % Regeln und Abdeckung von Anwendungsfällen (Use Cases) und eine Struktur vorgeben. (ToDo's, Flexibilität in der 
    % Umsetzung (nicht wie Home Assistant und openHAB eher eingeschränkt)) 

\section{Abzudeckende Funktionalitäten}
\label{sec:konzeptfunktionalitaet}
    % Was soll der Entwickler machen können? 
    % Welche Grundlagen braucht er, um eine Regel implementieren zu können?
    % Welche Funktionen müssen gegeben sein, um die Struktur vorzugeben? 
    % Reicht ein Hinweis weöche Stellen angepackt werden müssen, um eine Regel hinzuzufügen? 

\section{Architekturkonzept}
\label{sec:architekturkonzept}
    % Es wird alles abgebildet über einen Zustandsraum, der sich aus den Dingen (Gegenständen) und Zuständen der Anwendung ergibt.
    % Der Zustandsraum wird verändert, wenn eine Aktion durchgeführt wird, bzw. durch eine Trigger angestoßen. 
    % Bzw. speichert den aktuellen Zustand des Gegenstandes 
    % (lightBulb = true/false, personOnDoor = null/Mikka, booking = stringBooking, temiAktive = true/false, 
    % temiPosition = stringKoordinates)
    % Zustandsraum muss von dem Entwickler definiert werden. 
    % MQTT Broker über Home Assistant, bzw. losgelöster Broker
    % Anbindung von APIs auch Entwickler-Sache. Kann ich das vereinfachen, sodass die Integration einfacher wird?

    \subsection{Überlegungen, Anstöße und Herausforderungen}
    % Regeln über Thread abbilden? Ja, Nein? - Nein, wieso? Da Durch die MQTT Message mehrere Regeln ausgeführt 
    %werden können. -> Lediglich den Zustand der Komponenten locken.
    % KEIN THREAD (wird schon abgebildet durch die Services und die Auslöser durch MQTT), Falls eine Komponente 
    %  doppelt beansprucht wird, ist der Zustand der Komponenten zu locken und ein 
    % Thread.sleep einzurichten. Abfrage, ob der Wert, bzw. die Komponente wieder freigegeben wurde. 

    % Zustandsraum -> Abbildung aller notwendigen Komponenten 
    % Bei Bearbeitung einer Regeln die Komponenten Locken, sodass nur die einzelne Komponenten (deren Zustand) gelockt ist 
    % und nicht der ganze Zustandsraum, somit können mehrere Komponenten und Aktionen ausführen zu können. 

    %Was brauche ich für Funktionen und Werte in einer Regel?

    % Ein Zustandsraum (Objekt) für alles oder ein Globales, welches die die Komponenten enthält? - Begründung für die Auswahl.
    
    \subsection{Schnittstellen}
        % Kommunikation mit API's je nach Use Case und Gebrauch zur Datenabfrage
    
    \subsection{Datenbanken}
        % Datenbanken je nach Use Case und Gebrauch zur Datenabfrage

\section{Softwarekonzept}
\label{sec:softwarekonzept}

\section{Auswahl des Frameworks}
\label{sec:frameworkauswahl}

    \subsection{OSGi}
    \label{subsec:osgiFramework}

    \subsection{Spring Boot}
    \label{subsec:springBootFramework}



