\chapter{Konzept}
\label{chap:konzept}
    

\section{Ziel des Konzeptes}
    % Das Ziel des Konzeptes ist es, dem Entwickler den Aufwand zur Erweiterung des Systems zu minimieren durch weitere 
    % Regeln und Abdeckung von Anwendungsfällen (Use Cases) und eine Struktur vorgeben. (ToDo's, Flexibilität in der 
    % Umsetzung (nicht wie Home Assistant und openHAB eher eingeschränkt)) 

\section{Abzudeckende Funktionen}
    % Was soll der Entwickler machen können? 
    % Welche Grundlagen braucht er, um eine Regel implementieren zu können?
    % Welche Funktionen müssen gegeben sein, um die Struktur vorzugeben? 
    % Reicht ein Hinweis weöche Stellen angepackt werden müssen, um eine Regel hinzuzufügen? 

\section{Architekturkonzept}
    % Es wird alles abgebildet über einen Zustandsraum, der sich aus den Dingen (Gegenständen) und Zuständen der Anwendung ergibt.
    % Der Zustandsraum wird verändert, wenn eine Aktion durchgeführt wird, bzw. durch eine Trigger angestoßen. 
    % Bzw. speichert den aktuellen Zustand des Gegenstandes 
    % (lightBulb = true/false, personOnDoor = null/Mikka, booking = stringBooking, temiAktive = true/false, 
    % temiPosition = stringKoordinates)
    % Zustandsraum muss von dem Entwickler definiert werden. 
    % MQTT Broker über Home Assistant, bzw. losgelöster Broker
    % Anbindung von APIs auch Entwickler-Sache. Kann ich das vereinfachen, sodass die Integration einfacher wird?
    
    \subsection{Schnittstellen}
        % Kommunikation mit API's je nach Use Case und Gebrauch zur Datenabfrage
    
    \subsection{Datenbanken}
        % Datenbanken je nach Use Case und Gebrauch zur Datenabfrage



