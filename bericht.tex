%%%%%%%%%%%%%%%%%%%%%%%%%%%%%%%%%%%%%%%%%%%%%%%%%%%%%%%%%%%%%%%%%%%%%%%%%%%%%%%
%% Descr:       Vorlage für die Master-Thesis an der KFRU 
%% Author:      Mikka Jenne, mikka.jenne@cgi.com
%%%%%%%%%%%%%%%%%%%%%%%%%%%%%%%%%%%%%%%%%%%%%%%%%%%%%%%%%%%%%%%%%%%%%%%%%%%%%%%

\documentclass[
  ngerman           % neue deutsche Rechtschreibung
% ,a4paper          % Papiergrösse
  ,twoside          % Zweiseitiger Druck (rechts/links)
% ,10pt             % Schriftgrösse
  ,11pt
% ,12pt
  ,pdftex
%  ,disable         % Todo-Markierungen auschalten
]{report}

% Bitte die Codierung Ihrer Dateien auswählen:
% \usepackage[latin1]{inputenc}    % Für UNIX mit ISO-LATIN-codierten Dateien
% \usepackage[applemac]{inputenc}  % Für Apple Mac
% \usepackage[ansinew]{inputenc}   % Für Microsoft Windows
\usepackage[utf8]{inputenc}        % UTF-8 codierte Dateien
                                   % Dieses Dokument ist unter Unix erstellt, daher
                                   % wird diese Input-Codierung benutzt.
\usepackage[english,german]{babel}
\usepackage{bericht}
\usepackage{caption}
\usepackage{subcaption}
%\usepackage{subfigure}
\usepackage{amsmath}
%\usepackage{blindtext}
\usepackage{footnote}
% This package is responsible for the "draft" watermark on the background of the pdf
\usepackage{background}
% This removes the watermark 
%\backgroundsetup{contents={}}
%Is used to get Pages at landscape format
\usepackage{pdflscape}
%\usepackage{lscape}% works

%%%%%%%%%%%%%%%%%%%%%%%%%%%%%%%%%%%%%%%%%%%%%%%%%%%%%%%%%%%%%%%%%%%%%%%%%%%%%%%
%% Angaben zur Arbeit
%%%%%%%%%%%%%%%%%%%%%%%%%%%%%%%%%%%%%%%%%%%%%%%%%%%%%%%%%%%%%%%%%%%%%%%%%%%%%%%

\newcommand{\Autor}{Mikka Jenne}
\newcommand{\MatrikelNummer}{800864}
\newcommand{\Kursbezeichnung}{PSEJG20}
\newcommand{\FirmenName}{Reutlingen}
\newcommand{\FirmenStadt}{Karlsruhe}
\newcommand{\cgiFirmenLogo}{\includegraphics[]{CGI_LOGO}}
\newcommand{\FirmenLogoDeckblatt}{}
\newcommand{\BetreuerFirma}{Dr. Robin Braun}
\newcommand{\BetreuerDHBW}{Prof. Dr. Natividad Martinez Madrid}

%%%%%%%%%%%%%%%%%%%%%%%%%%%%%%%%%%%%%%%%%%%%%%%%%%%%%%%%%%%%%%%%%%%%%%%%%%%%%%%%%%%%%

\newcommand{\Was}{Master-Thesis}

%%%%%%%%%%%%%%%%%%%%%%%%%%%%%%%%%%%%%%%%%%%%%%%%%%%%%%%%%%%%%%%%%%%%%%%%%%%%%%%%%%%%%

\newcommand{\Titel}{Konzeption und prototypische Umsetzung einer Steuerzentrale eines smarten Büros mit dem Fokus einer einfachen Handhabung der formalisierten Interaktionen für Softwareentwickler}
\newcommand{\AbgabeDatum}{31. August 2022}
\newcommand{\Dauer}{24 Wochen}
\newcommand{\Abschluss}{Master of Science}
\newcommand{\Studiengang}{Professional Software Engineering}

\hypersetup{%%
  pdfauthor={\Autor},
  pdftitle={\Titel},
  pdfsubject={\Was}
}

%%%%%%%%%%%%%%%%%%%%%%%%%%%%%%%%%%%%%%%%%%%%%%%%%%%%%%%%%%%%%%%%%%%%%%%%%%%%%%%

\bibliography{bericht}

\begin{document}

%%%%%%%%%%%%%%%%%%%%%%%%%%%%%%%%%%%%%%%%%%%%%%%%%%%%%%%%%%%%%%%%%%%%%%%%%%%%%%%%

\begin{titlepage}
  \begin{center}
    \vspace*{-3cm}
    
    \cgiFirmenLogo
    \FirmenLogoDeckblatt\hfill\includegraphics[]{kfru}\\[2cm]
    
    {\Huge \Titel}\\[1.5cm]
    {\Huge\scshape \Was}\\[1cm]
    {\large für die Prüfung zum}\\[0.5cm]
    {\Large \Abschluss}\\[0.5cm]
    {\large des Studienganges \Studiengang}\\[0.5cm]
    {\large an der}\\[0.5cm]
    {\large Knowledge Foundation @ Reutlingen University}\\[0.5cm]
    {\large von}\\[0.5cm]
    {\large\bfseries \Autor}\\[1cm]
    {\large Abgabedatum \AbgabeDatum}
    
    \vfill
  \end{center}
  \begin{tabular}{l@{\hspace{4cm}}l}

    Bearbeitungszeitraum	         & \Dauer 			    \\
    Teilnehmernummer	             & \MatrikelNummer	\\
    Kurs			                     & \Kursbezeichnung	\\
    %Standort der Universität	     & \FirmenName			\\
    %Standort der Firma			       & \FirmenStadt			\\
    ErstprüferIn	                 & \BetreuerDHBW		\\
    ZweitprüferIn	                 & \BetreuerFirma		\\
  \end{tabular}

  % Fußzeile enthält die Anschrift und das Logo der Reutlingen University
  \backgroundsetup{
    scale=1,
    color=black,
    opacity=1,
    angle=0,
    position=current page.south,
    vshift=60pt,
    hshift=-200pt,
    contents={%
      \begin{minipage}{.18\textwidth}
      \includegraphics[width=1000pt,height=70pt,keepaspectratio]{images/FHRTFooter.png}
      \end{minipage}%
    }
  }
\end{titlepage}

\newpage
\selectlanguage{german}
\begin{abstract}
  Augmented Reality ist eine Technologie, die dem Nutzer ein visuelles 
  Erlebnis mit einer angereicherten Welt voller virtueller Objekte ermöglicht. Das Resultat, eine 
  Kombination aus Realität und Virtualität, bietet dem Benutzer eine neue Art der Wahrnehmung der Gegenwart. %Realitätswahrnehmung.  
  \\ 
  \linebreak
  Diese Bachelorarbeit befasst sich mit der Konzeption und Umsetzung eines industriellen Assistenzsystems unter Verwendung der Augmented Reality Technologie. Dabei soll die Umgebung 
  mit Hilfe des SLAM Verfahrens analysiert werden, um auf dieser Basis dreidimensionale Objekte als Referenz zu realen Objekten im Raum virtuell platzieren zu können. 
  Durch die entstehende Visualisierung können Informationen zu den jeweiligen Objekten in eine Datenbank eingetragen und angezeigt werden, dadurch kann das Überwachen von 
  Industriemaschinen vereinfacht werden. 
  \\ 
  \linebreak
  Zu dem Konzept gehört sowohl die Ausarbeitung der grundlegenden Softwarearchitektur, als auch ein allgemein-gültiges Datenmodell zur 
  Persistierung der generierten Daten. Für die bestmögliche Umsetzung der Augmented Reality Experience werden hierzu bereits schon bestehende Frameworks und Software 
  Development Kits, beispielsweise Google ARCore, verwendet. 
  \\ 
  Der entstandene Prototyp ist ein eigenständiges System. Die Architektur ist modular aufgebaut, um eine stetige Weiterentwicklung zu gewährleisten.  
\end{abstract}

\newpage
\selectlanguage{english}
\begin{abstract}
  Augmented Reality is a technology that enables the user to have a visual 
  experience with an enriched world full of virtual objects. The result offers the user a new 
  way of perceiving surrounding as an Combination of reality and virtuality.
  \\
  \linebreak
  This bachelor thesis deals with the conception and implementation of an industrial assistance system using augmented reality technology. The environment %should
  can be analyzed with the help of the SLAM method in order to be able to place three-dimensional objects virtually as a reference to real objects in space.
  The resulting visualization enables information on the respective objects to be entered in a database and displayed, which enables the simplified 
  monitoring of Industrial machines.
  \\
  \linebreak
  The concept includes the development of the basic software architecture as well as a generally applicable data model for
  saving the generated data. Already existing frameworks and software development kits where used for the best possible implementation of the 
  augmented reality experience as Google ARCore.
  \\
  The created prototype is an standalone system. The architecture is modular in order to ensure continuous further development.
\end{abstract}

\newpage
\selectlanguage{german}
\pagestyle{plain}
\chead{\headmark}
\cfoot{\pagemark}
%\pagestyle{headings}
\pagenumbering{Roman}
\tableofcontents           % Inhaltsverzeichnis hier ausgeben

\newpage
\pagenumbering{arabic}
\chapter{Einleitung}
\label{chap:Einleitung}
    Die folgende Master-Thesis befasst sich mit der Konzepterstellung einer zentralen Steuerzentrale, die 
    dem Entwickler die formalen Interaktionen, weitere Funktionen hinzuzufügen, erleichtern soll. Hierfür werden
    bereits bestehende Plattformen für \acl{SH} analysiert und daraus ein Konzept erstellt, die den Anforderungen 
    entsprechend einen größeren Mehrwert in der Weiterentwicklung der Plattform bietet. Die Umsetzung des ausgearbeiteten 
    Konzeptes wird nur in sehr geringem Maß behandelt.
    \\
    \linebreak
    In diesem Teil der Arbeit wird auf die Motivation des Themas eingegangen. Darüber hinaus 
    werden sowohl die Forschungsfragen als auch die Zielsetzung der Arbeit genauestens dargelegt. Darauf 
    folgend findet eine Übersicht über die Arbeit im Gesamten statt, mit der die Inhalte angerissen werden. 
    Eine nähere Betrachtung des Standes der Technik untermauert die Beweggründe dieser Themenwahl und 
    Ausarbeitung dessen. 

\section{Motivation}
\label{sec:Motivation}
    Jede neu entwickelte Technologie durchlebt im Laufe der Entstehung und Publikation ein enormes Aufsehen. 
    So lange bis diese Technik eine standardisierte Verwendung in der Gesellschaft findet oder sich als 
    unpraktikabel erweist und nicht weiter vorangetrieben oder eingestellt wird. Es wird in der Zeit des 
    Aufkommens und der Forschung viel darüber fantasiert, debattiert und geplant, ohne jedoch die Ausmaße und 
    Resultate der Forschungen und Praktiken abwägen zu können. Durch fehlende Erfahrung und nicht ausgereifte 
    Konzepte werden Höhepunkte und Illusionen erwartet, die zu diesem Zeitpunkt technisch nicht umsetzbar sind. 
    Um solche kühnen Versprechungen und Übertreibungen, sogenannte Hypes, die jede neue technologische Idee 
    mit sich bringt, von dem zu differenzieren was wirtschaftlich umsetzbar ist, werden bestimmte Phasen 
    der Entwicklung durchlaufen. \cite{gartner.2022m}
    \\
    \linebreak
    Die oben erwähnten Phasen der Entwicklung sind in einem sogenannten Hype-Zyklus, engl. Hype-Cycle, dargestellt. 
    Dieser Zyklus ist ein visualisiertes Modell, das die Entwicklung einer neuen Technologie von der Innovation 
    und Entstehung über die Forschung und Umsetzung bis hin zur ausgereiften Marktfähigkeit repräsentiert und so 
    diese Phasen der Entwicklung versinnbildlicht.  
    \\
    Entwickelt wurde der Hype Cycle von der Gartner Inc. Forschungsgruppe. Durch die Mitarbeiterin 
    Jackie Finn wurden die Definitionen der Entwicklungsphasen\footnote{Die Entwicklungsphasen der Gartner Inc. ist unter folgender URL zu finden: "\url{https://www.gartner.com/en/research/methodologies/gartner-hype-cycle}"} 
    geprägt. Diese sind wie folgt in fünf Phasen 
    dargestellt:
    \begin{enumerate}
        \item \textit{Innovationsauslöser, engl. Innovation Trigger}: Ein potentieller technologischer Durchbruch 
        löst die Dinge aus. Frühe Proof-of-Concept (PoC) Ansätze und ein großes Medieninteresse lösen eine 
        erhebliche Publizität aus. Oft gibt es keine brauchbaren Produkte und die Marktreife ist nicht 
        bewiesen. \cite{gartner.2022m}
        \item \textit{Höhepunkt überhöhter Erwartungen, engl. Peak of Inflated Expectations}: Frühe Publizität 
        bringt eine Reihe von Erfolgsgeschichten hervor – oft begleitet von zahlreichen Misserfolgen. 
        Einige Unternehmen ergreifen Maßnahmen; viele nicht. \cite{gartner.2022m}
        \item \textit{Trog der Ernüchterung, engl. Trough of Disillusionment}: Das Interesse schwindet, da 
        Experimente und Implementierungen nicht liefern. Hersteller der Technologie reißen es heraus 
        oder scheitern. Investitionen werden nur fortgesetzt, wenn die überlebenden Anbieter ihre Produkte 
        zur Zufriedenheit der frühzeitigen Anwender verbessern. \cite{gartner.2022m}
        \item \textit{Steigung der Erleuchtung, engl. Slope of Enlightenment}: Mehr Beispiele dafür, wie 
        die Technologie dem Unternehmen zugute kommen kann, beginnen sich zu herauszukristallisieren und 
        werden allgemeiner verstanden. Produkte der zweiten und dritten Generation erscheinen von den 
        Technologieanbietern. Mehr Unternehmen finanzieren Pilotprojekte; Konservative Unternehmen 
        bleiben vorsichtig. \cite{gartner.2022m}
        \item \textit{Plateau der Produktivität, engl. Plateau of Productivity}: Mainstream-Akzeptanz beginnt 
        sich abzuheben. Kriterien zur Bewertung der Lebensfähigkeit des Anbieters sind klarer definiert. 
        Die breite Markteinsetzbarkeit und Relevanz der Technologie zahlen sich eindeutig aus. \cite{gartner.2022m}
    \end{enumerate}
    Nachdem ein innovativer Gedanke den \textit{Höhepunkt überhöhter Erwartungen} passiert hat, z.B. die 
    Revolutionierung der Softwareentwicklung oder Szenarien, wie z.B. die Vollautomatisierung eines Gebäudes 
    oder Service-Roboter die uneingeschränkt interagieren können, die man in der Form nur aus Science-Fiction Filmen kennt, 
    folgt der \textit{Trog der Ernüchterung}. In Folge dessen wird festgestellt, dass die Erwartungen nicht 
    in Gänze übertragbar sind, bzw. nur zu einem geminderten Teil in die Realität umgesetzt werden können 
    und der verfolgte Gedanke an Interesse verliert. Nach erneutem Aufgriff der Technologie findet eine realistischere 
    Beurteilung der Innovation statt, die dazu beiträgt, dass die Technologie wieder an Interesse gewinnt. Die 
    objektive und realitätsnahe Betrachtungsweise formt ein neues und realistisches Bild der Potentiale, als auch 
    der Grenzen. Mit dem neu gewonnenen Maßstab geht die ehemals neue innovative Idee in eine routinierte Technologie über, 
    die an Anerkennung gewinnt und in der breiten Masse akzeptiert wird. Die Technologie erfährt mit steigender 
    Zuwendung eine stetigere Weiterentwicklung, die dann zu einer Community geformt wird. Mit der Erreichung dieses Status 
    befindet sich die Innovation, bezogen auf den Hype Cycle, in der letzten Phase, dem \textit{Plateau der Produktivität}, 
    und bestätigt so die Marktreife. Dieser Zeitpunkt löst die Zukunftsvision auf und es handelt sich um eine am Markt 
    etablierte Technologie.
    \\ 
    \linebreak
    Zum aktuellen Zeitpunkt befindet sich die Technologie rund um Plattformen für intelligente Geräte im privaten 
    Bereich, engl. \textit{\ac{SH}} oder \textit{Connected Home}, im Anfangsstadium der letzten Phase, dem sogenannten 
    \textit{Plateau der Produktivität}. Mit zunehmender Akzeptanz werden im Umfeld des Internets der Dinge, engl. 
    \textit{\acl{IoT}}, stetig Szenarien entwickelt, die das Wachstum und die Verwendung von solchen Plattformen vorantreibt. 
    Mit einer immer tiefer gehenden Forschung und Umsetzung von Anwendungsbeispielen werden Bereiche offenbart, die 
    eine solche Plattform im privaten als auch im geschäftlichen Umfeld immer attraktiver gestaltet. Mit steigender  
    Konnektivität und Kompatibilität mehreren Geräten und Gegenständen können Bereiche und Szenarien, wie die 
    Steuerung von Service-Robotern, umgesetzt werden. Der jetzige technologische Fortschritt und die über die Forschungsjahre 
    gesammelten Erfahrungen bringt das Segment der intelligenten Geräte der \acs{IoT} den ursprünglich angedachten 
    Visionen und Ideen näher, sodass ein weiterer Ausbau dieser Technologie und dessen Anwendung stattfindet und sich 
    vollständig in den Markt etabliert. Der finale Schritt der endgültigen Marktreife ist ein faszinierender und wichtiger Grund 
    für meine Motivation, mich dieser Technologie und der dahinterstehenden Theorie zu widmen.
    \\
    Mit der erzielten Marktreife entstehen Produkte und Lösungen, die bestimme Teile der Anfänglichen Idee abdecken. Mit 
    zunehmender Entwicklung und anfallenden Anforderungen, werden viele Produkte zu groß und haben dadurch die grundlegende 
    Konzeption und Architektur nicht vorausschauend entwickelt. Daher ist die anfängliche Überlegung und Konzeption essentiell.
    \\
    Daher ist ein weiterer Punkt meiner Motivation den Schritt zu gehen, ein Konzept zu entwickeln, dass die Erweiterung eines 
    solchen Systems basierend auf der Konzeptgrundlage vereinfacht und so die Nutzung für Entwickler zur Weiterentwicklung verbessert.
    %welches anhand bestehender Produkte  
    %Ein weiterer Punkt meiner Motivation ist die Vereinfachung der Erweiterung einer solchen Zentrale. 
    Somit soll dem Entwickler bei einer stetigen Erweiterung der Plattform Zeit und Aufwand erleichtert werden. Dadurch können weitere 
    Anwendungsszenarien und Objekte integriert werden, ohne einen zu großen Entwicklungsaufwand zu erzeugen.   
    \\ 
    \linebreak
    Die Einsatzgebiete von intelligenten Geräten, beziehungsweise die Verwendung einer Kompaktlösung beschränkt sich räumlich 
    auf Gebäude, Häuser und Wohnungen, bieten trotz dessen viele Verwendungs- und Einsatzmöglichkeiten. Diese sehen wie folgt aus: 
    \begin{itemize}
        \item Komfort
        \item Entertainment
        \item Überwachung und Sicherheit
        \item Steuerung von Prozessen
        \item Management von Automationen
    \end{itemize}    
    Neben der Affinität von \acl{SH} zum \acl{IdD} und der damit einhergehenden Technologie bringt diese Vorteile mit sich, wie z.B. 
    die Modernisierung von Wohngebäuden und die angestrebte Verbesserung der Lebensqualität. 

\section{Forschungsfragen}
\label{sec:forschungsfragen}
    %-	Wie kann man die Usability von SmartHome-Plattformen verbessern, so dass die formalen Interaktionen der Softwareentwickler schneller sind?
        %o	Welche sind die Usability-Probleme der existierenden Plattformen? 
        %o	Welche Anforderungen würden für eine Verbesserung einer solchen Plattform gelten?
        % Nicht Bestandteil der eigentlichen Forschungsfrage: 
        % -Reicht eine Kommunikationsschnittstelle als Ausgangspunkt für alle Verbindungen?


\section{Zielsetzung der Arbeit}
\label{sec:zielsetzung}
    %TBD Wird zum Schluss geschrieben. 

\section{Forschungsstrategie und Forschungsmethoden}
\label{sec:forschungsstrategie}
    %TBD Wird zum Schluss geschrieben.
    Dieser Abschnitt der Arbeit widmet sich der Forschungsstrategien und der darauf angewendeten Forschungsmethoden. 
    Hierbei werden die Strategien kurz erläutert und die Methoden skizziert. 

    %Diese Arbeit ist gemäß des Grundgedanken des Design Science in die drei Phasen Analy- se, Design und Evaluation aufgeteilt. 
    %Zuerst erfolgt durch ein Systematisches Literaturreview Systematisches Literaturreview (SLR) eine Analyse des aktuellen 
    %Stands der Technik zum RE bei der Entwicklung von Features einer Software, von Software-Modulen oder sogar Software 
    %Produktlinien. Darauf wird das RE beim Referenz-KIS-Hersteller anhand einer Fallstudie zu dieser Arbeit analysiert. 
    %Für diese Analysephase werden die Ergebnisse, der vorangegangenen Forschungsarbeit [Loh20] berücksichtigt. Darauf wird in 
    %einer Design-Phase ein Auswahlfra- mework für die systematische Auswahl des geeigneten RE für die Weiterentwicklung eines 
    %KIS entwickelt und eingeführt. Die Design-Phase orientiert sich am Vorgehen nach Wieringa [Wie14] gemäß des 
    %problemlösungsorientierten Ansatzes der Design Science Methode. In der letzten Phase der Arbeit wird das selbst entwickelte 
    %Auswahlframework durch eine Experten- interview mit Mitgliedern des Internationalen Requirements Engineerings Boards 
    %International Requirements Engineering Board (IREB) evaluiert und anschließend optimiert. Das IREB wur- de 2007 von 
    %RE-Experten weltweit gegründet, um eine einheitliche Lehre des RE zu etablieren, zu verbreiten und weiter zu entwickeln [PR15]. 
    %Das Experteninterview wird anhand des GQM- Ansatzes [LL10] erstellt. Die soeben genannten Forschungsmethoden werden im 
    %Nachfolgen- den näher erklärt.

    
    \subsection{Experten Interview}
    \label{subsec:experteninterview}
        %Beim Experten Interview wird vorab der Hintergrund für das Interview kurz erklärt. Darauf stellt ein Forscher einem 
        %Experten ausgewählte Interview-Fragen, die für das Thema der Forschung relevant sind und sich aus den Forschungsfragen 
        %ableiten lassen.
    
        %Die Interview-Fragen können als offene oder geschlossene Fragen formuliert werden, je nachdem ist eine beliebige Reihe 
        %an Antworten möglich oder nur eine begrenzte Anzahl an Antworten [Rob02].
        %Experten Interviews könnten strukturiert, semi-strukturiert oder unstrukturiert sein, wobei bei etwa zu Fallstudien sich 
        %die semi-strukturierte Form bewährt hat. Bei einem semi-strukturierten Experten Interview werden die Interview-Fragen vor 
        %der Durchführung des Interviews sowie deren Reihenfolge oder Themenbereiche geplant. Die Interview-Fragen können dem zu 
        %Inter- viewenden orientiert an der geplanten Reihenfolge oder den definierten Themenbereichen ge- stellt werden. Bei 
        %Bedarf kann jedoch bzgl. der Reihenfolge oder Formulierung von offenen bzw. geschlossenen Interview-Fragen improvisiert 
        %werden [HA 2].

    \subsection{Qualitative Forschung}
    \label{subsec:qualitativeforschung}

    \subsection{Systematisches Literatur Review}
    \label{subsec:systematischesliteraturreview}
        %Das SLR ist eine wissenschaftliche Methode zur Evidenz-basierten Identifizierung, Bewertung und Interpretation der 
        %relevanten, bestehenden Literatur, zur einer bestimmten Fragestellung, zu einem bestimmten Thema oder einem bestimmten 
        %Phänomen. Dadurch soll das Erzielen einer Schlussfolgerung zum untersuchten Objekt ermöglicht werden, um Lücken der 
        %aktuellen Forschung aufzuzeigen, weiteren Forschungsbedarf vorzuschlagen oder für die Positionierung der erhobenen 
        %Forschungsergebnisse.
        %Ein SLR beinhaltet, orientiert an den Richtlinien von Kitchenham et al., drei aufeinander aufbau- ende Schritte. Zuerst 
        %erfolgt die Planung, anschließende die Durchführung und abschließende Berichterstattung zum SLR [B. 07].

    \subsection{Usability-Tests}
    \label{subsec:usabilitytests}
    % Methodik zur Testung und Evaluierung der Benutzerfreundlichkeit

\section{Aufbau der Arbeit}
\label{sec:aufbau}
    Die vorliegende Master-Thesis gliedert sich nach den soeben genannten einleitenden Information im Aufbau in insgesamt 
    zehn Kapitel. Das erste Kapitel (\ref{chap:Einleitung}) beschreibt die Motivation (\ref{sec:Motivation}), welche die 
    Intension kundtut, diese Thematik rund um \acs{IoT} und Smart Home zu bearbeiten. Darauf folgend werden die 
    Forschungsfragen (\ref{sec:forschungsfragen}), die im Rahmen der Thesis behandelt werden, erläutert. Nach der 
    Beschreibung der Forschungsfragen wird im anknüpfenden Abschnitt (\ref{sec:zielsetzung}) die Zielsetzung der 
    Arbeit erläutert. Hierbei werden zusätzliche Schwerpunkte und Ziele aufgegriffen. Abschließend wird das Unternehmen, in der 
    die Thesis geschrieben wird, hervorgehoben und deren Absichten in Verbindung mit Innovationen beleuchtet. 
    \\
    \linebreak
    Das Kapitel (\ref{chap:grundlagen}) widmet sich den essentiellen und wichtigen Grundlagen dieser Arbeit. Zu Anfang wird dem 
    Leser der Terminus des \acl{IoT} (\ref{sec:iot}) offenbart, um zum Teil den Kontext im Bezug zu dieser Arbeit zu begreifen, 
    gefolgt von einer Einführung in die Thematik des \acl{SH} (\ref{sec:smartHome}), der Problematik der Begriffsdefinition, der 
    historischen und kontinuierlichen Entwicklung und mit den Zielen, die mit der Verwendung einer Smart Home Lösung bewältigt 
    werden sollen. Mit dem Verständnis der übergeordneten Begriffe, \acs{IoT} und \acl{SH}, werden Technologien 
    (\ref{sec:technologien}) aufgegriffen, die im Rahmen dieser Arbeit erwähnenswert sind und verwendet werden. %Um auf die Vielfältigkeit 
    %von der Umsetzung eines \acl{SH} einzugehen und einen Teilaspekt der Anforderungen Abzudecken, wird ebenso auf Service-Roboter 
    %(\ref{sec:roboter}) eingegangen. 
    Abschließend werden in Kapitel (\ref{chap:grundlagen}) die Softwarelösungen, Home Assistant 
    und openHAB (\ref{sec:homeassistant} \& \ref{sec:openhab}), dargestellt. Diese dienen zur Grundlage für die Evaluation als 
    auch zur Gegenüberstellung der Lösungen in Kapitel (\ref{chap:evaluation}) Diskussion und Evaluation. 
    \\
    \linebreak
    Die theoretischen und methodischen Hintergründe sowie den Stand der Technik wird in Kapitel (\ref{chap:technikStand} )
    angesprochen. Dieser Teil enthält Beschreibungen, Forschungen und aktuelle Erkenntnisse über Technologien, die im Umfeld der 
    Smart Home Anwendungen innerhalb des \acs{IoT} verwendet werden. Zudem werden in Zusammenhang der Erkenntnisse und 
    Möglichkeiten der Technologie die Szenarien dargestellt. 
    \\
    \linebreak
    Kapitel (\ref{chap:anforderungsanalyse}) befasst sich mit den Anforderungen, engl. Requirements, die für die 
    eigentliche Konzeption relevant sind. Innerhalb dieses Kapitels wird anhand von Informationen und den umzusetzenden 
    Szenarien die Anforderungen für die Konzeption erarbeitet. Hierbei werden aus der Praxis bekannte Verfahren verwenden, um 
    die Anforderungen zu definieren. Mittels den zugrundeliegenden Anforderungen wird im nachfolgenden Schritt die eigentliche 
    Konzeption dargelegt.
    \\
    \linebreak 
    Nach Aufbereitung der Anforderungen durch das sogenannte Anforderungsmanagement, engl. \textit{Requirements Engineering},
    wird in Kapitel (\ref{chap:konzept}) das Konzept erarbeitet, welches als Grundlage für die prototypische Implementierung und 
    Umsetzung des Konzepts dient. Das Konzept befasst sich mit den Anforderungen und setzt diese ein, um die Organisation des 
    Systems in Komponenten, deren Beziehungen zueinander und zur Umgebung sowie deren Prinzipien zu definieren. Zum Ende des 
    Konzepts steht eine Architektur, die sich aus den Anforderungen und auch aus den Analysen der eigentlichen Forschungsfrage 
    abzeichnet.
    \\
    \linebreak
    In Kapitel (\ref{chap:umsetzung}) wird die Umsetzung des Konzepts skizziert. Darunter welche Problem während der Implementierung 
    auftraten als auch deren Lösungsfindung. Ebenso wird hier aus praktischer Sicht auf die Architektur geschaut, welche Komponenten,  
    Bibliotheken und zusätzliche Systeme, engl. \textit{Frameworks}, verwendet wurden. 
    \\ 
    \linebreak
    Das Ergebnis wird aus objektiver Sicht in dem darauf folgenden Kapitel (\ref{chap:ergebnis}) erläutert. 
    \\
    \linebreak
    Nach Abschluss der Umsetzung und dessen Ergebnisdokumentation befasst sich das Kapitel (\ref{chap:evaluation}) mit der Diskussion 
    und Evaluation. Hier findet eine Analyse des Konzepts sowie deren Umsetzung und objektive Betrachtung statt. Anschließend werden 
    Vergleiche zwischen der Eigenentwicklung und bereits bestehender Softwarelösungen, die im Grundlagenkapitel aufgefasst werden, 
    aufgestellt und bewertet.
    \\
    \linebreak
    Im vorletzten Teil, Kapitel (\ref{chap:fazit}), wird ein Fazit aus den Erkenntnissen und Ergebnissen gezogen. Dieses Schlussresümee 
    führt nochmals die Höhepunkte sowie eine eigene Einschätzung auf. 
    \\
    \linebreak
    Zum Abschluss der Thesis wird in Kapitel (\ref{chap:ausblick}) ein Ausblick gegeben. Dieser gibt Aufschluss darüber, welche 
    Erweiterungsmöglichkeiten es für die in dieser Thesis erfolgten Arbeit gibt und wie innovativ sich dieser Grundbaustein in Zukunft 
    erweisen könnte. 

%\section{CGI}
%\label{sec:cgi}
%\include{chapter/1Einleitung/forschungsfragen}
%\section{Zielsetzung der Arbeit}
Zielsetzung / Zielerreichung
%\section{Aufbau der Arbeit}
Hallo Test
\chapter{Grundlagen}
\section{Internet der Dinge}
%\section{Smart Home}
\label{sec:smartHome}
\begin{table}[hbt!]
    \begin{center}
        \begin{tabular}{| p{3.25cm} | p{3.25cm} | p{3.25cm} | p{3.25cm} |}
            \hline
                \textbf{Technologie} & \textbf{Übertragung} & \textbf{Frequenzbereich (Funk)} & \textbf{Proprietär} \\
            \hline
                ZigBee & Funk & 2,4 GHz, 868 MHz & Nein \\ 
            \hline
                Z-Wave & Funk & 868 MHz & Nein \\ 
            \hline
                HomeMatic & Funk / Datenleitung & 868,3 MHz & Ja \\
            \hline
                KNX & Funk / Strom- und Datenleitung & 868 MHz / - & Nein / Ja (Datenleitung als Gesamtsystem) \\
            \hline
                Wi-Fi / WLAN & Funk & 2,4 - 5 GHz & Nein \\
            \hline 
                Bluetooth & Funk & 2,4 GHz & Nein \\
            \hline
                io-homecontrol & Funk & 868-870 MHz & Ja \\
            \hline
        \end{tabular}
    \end{center}
    \caption{Übertragungsmethoden des Smart Home}
    \label{tab:protocolsSH}
\end{table}

    \subsection{MQTT}
    \label{subsec:mqtt}
        Das \ac{MQTT}-Protokoll ist eines der ältesten offenen Netzwerk- und Nachrichtenprotokolle der 
        \ac{MtoM}-Kommunikation. 
        Dies wurde 1999 von IBM Mitarbeiter Andy Stanford-Clark\footnote{Informationen zu Herrn Stanford-Clark \url{https://stanford-clark.com} Abgerufen am 12.04.2022} 
        und von Cirrus Link Solutions Mitarbeiter Arlen Nipper\footnote{Informationen zu Herrn Nipper \url{https://www.inductiveautomation.com/resources/podcast/the-coinventor-of-mqtt-arlen-nipper-from-cirrus-link-solutions} Abgerufen am 12.04.2022} 
        entwickelt. Die Technologie ermöglicht die Übertragung von Messdaten, sogenannten Telemetriedaten, in Form von 
        Nachrichten zwischen Maschinen und Geräten. Die erzeugten Messdaten durch beispielsweise Sensoren und Aktoren 
        können durch ihre minimale Größe und die kompakte Form des Protokolls in einem kleinen Datenpaket auch bei 
        hoher Verzögerung oder bei beschränktem Netzwerk übertragen werden \cite{Naik2017}. \acs{MQTT} ist ein 
        klassisches Client-Server-Protokoll, welches nach dem \textit{Publish/Subscribe} Kommunikationsmodell 
        entwickelt wurde. Ein \acs{MQTT}-Client veröffentlicht Nachrichten an einen \acs{MQTT}-Server, den sogenannten 
        \acs{MQTT}-Broker. Diese können von anderen Clients abonniert oder auf dem Broker für zukünftige Abonnements 
        aufbewahrt werden. Jede erzeugte Nachricht wird an eine Adresse veröffentlicht, die als Thema, im Englischen \textit{Topic}, 
        bezeichnet wird \cite{Naik2017}. \acs{MQTT}-Clients können mehrere Topics abonnieren und erhalten jede Nachricht, die an 
        das jeweilig abonnierte Topic gesendet wird. 
        \\
        Die Leichtgewichtigkeit des Protokolls ermöglicht es, die Nachrichten bei eingeschränkter Netzwerkverfügbarkeit zu 
        übermitteln. Ausschlaggebend dafür ist das binär-basierte Protokoll, welches normalerweise einen festen Header von 
        zwei Bytes mit kleinen Nachrichtennutzlasten von maximal bis zu einer Größe von 256 MB \cite{Naik2017} enthält. Grundlegend 
        ist \acs{MQTT} auf der Basis des Transportprotokolls \ac{TCP} aufgebaut und nutzt zur Verstärkung der Sicherheit die 
        \ac{TLS}/\ac{SSL}-Verschlüsselung. Dadurch sind Client und Broker mit ihrer Kommunikation verbindungsorientiert. 
        Die Anwendung von \acs{MQTT} im Bereich des \acl{SH} zeichnet sich durch die Möglichkeit aus, große Netzwerke mit 
        vielen kleineren Geräten, die von einem Backend-Server bzw. dem Backend-System überwacht und gesteuert werden müssen, zu 
        betreiben. Dennoch ist es nicht für Multicast-Daten oder Übertragungen von Gerät zu Gerät ausgelegt. Die Nutzbarkeit 
        von \acs{MQTT} ist aufgrund der wenigen Steueroptionen und der Einfachheit des Messaging-Protokolls sehr simpel und 
        leichtgewichtig \cite{Naik2017}. 
        \\
        \linebreak
        Ein interessanter Aspekt des \acs{MQTT}-Brokers ist, dass er die gesamte Datenlage seiner Kommunikationspartner aufbewahrt und 
        so die Option bietet, zusätzlich als Zustandsdatenbank betrieben werden zu können. Weniger leistungsfähige Geräte können dadurch  
        mit einem \acs{MQTT}-Broker verbunden werden und Befehle und Daten entgegennehmen, zugleich entsteht aber ein 
        komplexeres Lagebild auf dem Broker. So können Daten an einen leistungsfähigeren Kommunikationspartner 
        weitergeleitet und dort ausgewertet werden.\footnote{\url{https://mqtt.org} Abgerufen am 13.04.2022}
        \\
        Auf diese Weise können durch das \acl{MQTT}-Protokoll Automatisierungslösungen geschaffen werden, wodurch im Segment 
        \acs{IoT} und \acl{SH} eine einfache Verwendung ermöglicht wird. Dementsprechend fand bis heute eine 
        schnelle Verbreitung statt. 
        \\
        \linebreak
        Zusammengefasst ist der \acs{MQTT}-Broker die Kommunikationsschnittstelle der \acl{SH}-Geräte und der \acl{SH}-Plattform. 
        Alle Kommunikationspartner können so Informationen (Nachrichten / Messages) auf bestimmte Topics (Endpunkten) senden und 
        diese abonnieren (publish / subscribe). 

        \subsubsection*{Publish/Subscribe Kommunikationsmodell}
        \label{subsubsec:pubsub}
            Das Prinzip des Publish/Subscribe Kommunikationsmodells besteht darin, dass Komponenten, die an vorgegebenen Topics 
            interessiert sind, bestimmte Informationen konsumieren und ihr Interesse anmelden \cite{Hunkeler2008}. Dieser Vorgang wird als 
            Abonnement (subscription) bezeichnet. Geräte, die an dem Vorgang oder an bestimmten Informationen interessiert sind, 
            werden als Abonnenten (subscriber) definiert. Im Gegenzug können Geräte und Komponenten bestimmte Informationen produzieren und 
            diese veröffentlichen (publish) und über den Markler (broker) an bestimmte Abonnenten weitergeben. Diese Vermittlung der Informationen zwischen 
            Herausgeber (publisher) und Abonnent (subscriber) erfolgt über den Markler, welcher sämtliche Abonnements koordiniert. 
            Alle Abonnenten müssen sich explizit bei dem Broker anmelden, um die Informationen zu erhalten \cite{Hunkeler2008}. 
            \\
            \linebreak
            \pagebreak
            \begin{figure}[hbt!]
                \centering
                \includegraphics[width=12cm,height=12cm,keepaspectratio]{images/sub-model.drawio.png}
                \caption{Themenbasiertes Pub/Sub Kommunikationsmodell \cite{Hunkeler2008}}
                \label{pic:pub-sub-model}
            \end{figure}
            \\
            Dieses Prinzip ist die Grundlage des \acs{MQTT}-Protokolls. Im Folgenden wird ein Beispiel aufgezeigt, welches 
            die Kommunikation über das Publish/Subscribe Modell darstellt.
            
            \subsubsection*{Beispiel}
            \label{subsubsec:pubsub-example}
            Damit am Beispiel der Steuerzentrale, die im Rahmen dieser Arbeit konzipiert und prototypisch implementiert wird, 
            ein Prozess gestartet werden kann, müssen bestimmte Ereignisse durch \acs{MQTT}-Nachrichten eintreten. Das hier 
            verwendete Beispiel ist das Öffnen einer Büroeingangstür. Hierbei wird von einem Relais an der Tür, welches das 
            Türschloss steuert, eine Nachricht an den \textit{Broker} herausgegeben:
            \\
            \begin{lstlisting}[language=Java, frame=lines, xleftmargin=\parindent, style=algoBericht, label={code:pubmqtt}, captionpos=b, caption={Erzeugung und Veröffentlichung einer Nachricht}]
                publish -topic: Buero/Tuer/Zustand -message: "offen"
            \end{lstlisting}
            Diese Nachricht wird von der Steuerzentrale konsumiert und weiterverarbeitet. Damit der Informationskanal (topic) 
            über die Steuerzentrale verfügbar ist, muss dieser Kanal konsumiert werden: 
            \\
            \begin{lstlisting}[language=Java, frame=lines, xleftmargin=\parindent, style=algoBericht, label={code:submqtt}, captionpos=b, caption={Empfang und Konsum einer Nachricht}]
                subscribe -topic: Buero/Tuer/Zustand
            \end{lstlisting}
            Mit der empfangenen Nachricht kann dann über die Steuerzentrale ein Prozess, welcher vorab als Regel implementiert wurde, ausgelöst werden, 
            beispielsweise eine Durchsage im Büro.
    \pagebreak
    \subsection{AMQP}
    \label{subsec:amqp}
        Das \ac{AMQP}-Protokoll ist ebenso ein leichtgewichtiges \acs{MtoM}-Protokoll, welches im Jahre 2003 von John O\'Hara JPMorgan Chase 
        in London, Großbritannien, entwickelt wurde \cite{Naik2017}. Der Fokus dieses Protokolls liegt auf der Unternehmens-Messaging-Ebene. 
        Es wird hohen Wert auf die Zuverlässigkeit, Sicherheit, Bereitstellung und Interoperabilität der Kommunikation gelegt. \acs{AMQP} 
        unterstützt neben der Publish/Subscribe- auch die Request/Response Architektur. Es bietet eine Bandbreite an 
        Funktionen im Zusammenhang mit Messaging, wie z. B. Queuing, themenbasiertes Publish-and-Subscribe-Messaging, 
        flexibles Routing und Transaktionen \cite{Naik2017}. Das Kommunikationsmodell nach dem \acs{AMQP} Standard erfordert, dass der 
        Herausgeber (publisher) oder der Empfänger (subscriber) einen \textit{Austausch (exchange)} mit einem bestimmten Namen generiert 
        und diesen dann sendet \cite{Naik2017}. Die beiden Komponenten, Empfänger und Herausgeber, nutzen den Namen zum Austausch und zum 
        Verbindungsaufbau. Der Empfänger erstellt darauf eine Warteschlange (queue) und hängt diese an den Austausch an. Nachrichten, die 
        über diese Verbindung ausgetauscht werden, müssen über einen gesonderten Prozess (binding) mit der Warteschlange abgeglichen werden 
        \cite{Naik2017}.
        \\ 
        Das binäre Protokoll \acs{AMQP} erfordert einen Header von acht Byte mit Nachrichtennutzlasten. Die Größe der Nachricht ist abhängig 
        von dem Broker bzw. dem Server. Die verbindungsorientierte Kommunikation von \acs{AMQP} baut auf dem Standard-Transportprotokoll 
        \acs{TCP} auf. Durch das \acs{TLS}/\acs{SSL}-Protokoll wird die Kommunikationssicherheit gewährleistet. Ein Wesensmerkmal des \acs{AMQP} 
        Kommunikationsmodells ist die Zuverlässigkeit \cite{Naik2017}. 
        \\
        \linebreak
        Der direkte Vergleich der beiden Protokolle, \acs{MQTT} und \acs{AMQP}, ist dem Anhang (\ref{appendix:mqtt-amqp}) beigefügt.
        \\
        \linebreak
        Neben den beiden aufgeführten Kommunikationsmodellen gibt es noch weitere, darunter das klassische \ac{HTTP} und das \ac{COAP}, die allerdings 
        im Rahmen dieser Arbeit nicht weiter ausgeführt werden. 
        \\
        \linebreak
        Der Fokus liegt in dieser Arbeit auf dem \acs{MQTT}-Protokoll. 

%\section{Roboter}
\subsection{Serviceroboter}
\subsection{Temi - Roboter}
\section{Home Assistant}
\label{sec:homeassistant}
\subsection{Konzept}
\subsection{Architektur}
\subsection{Ziele und Schwerpunkte}
\subsection{Stärken und Schwächen}

\section{openHAB}
\label{sec:openhab}
\subsection{Konzept}
\subsection{Architektur}
\subsection{Ziele und Schwerpunkte}
\subsection{Stärken und Schwächen}

%\section{Requirements Engineering}
\label{sec:requirementsengineering}
    In der heutigen Softwareentwicklung ist zu Anfang jedes Projekts die Frage offen, welche Anforderungen soll das 
    Produkt erfüllen, sodass daraus eine gute, stabil und effiziente Software entsteht und der Umfang und das Ziel 
    klar gestaltet sind. Mit dem \acl{RE} wird der Gedanke verfolgt, die Bedürfnisse des Kunden, bzw. des Stakeholders 
    zu erfüllen und eine erfolgreiche, den Kunden zufriedenstellende Entwicklung des Systems zu erzielen. Ein Stakeholder 
    ist eine Person oder Organisation mit Einfluss auf die Anforderungen des Systems oder die Auswirkungen auf das System 
    hat \cite{pohl2021basiswissen}.
    Diese Anforderungen werden durch das \acl{RE} ermittelt und dokumentiert. 
    \\
    \linebreak
    Unter dem Begriff \ac{RE} ist folgendes zu verstehen: 
    \begin{quote}
        Das Requirements Engineering ist ein systematischer und disziplinierter Ansatz zur Spezifikation und zum 
        Management von Anforderungen mit dem Ziel, die Wünsche und Bedürfnisse der Stakeholder zu verstehen und 
        die Gefahr zu minimieren, ein System auszuliefern, das diese Wünsche und Bedürfnisse nicht erfüllt. \cite{pohl2021basiswissen}
    \end{quote}
    Oft wird darunter auch ein kooperativer und inkrementeller Prozess verstanden, dessen Ziele die Gewährleistung 
    folgender Punkte darstellt:
    \begin{itemize}
        \item Alle Anforderungen sind bekannt und werden auch in dem erforderlichen Detaillierungsgrad verstanden.
        \item Alle involvierten Stakeholder haben eine ausreichende Übereinstimmung über die bekannten Anforderungen erzielt.
        \item Alle Anforderungen sind zu den Dokumentationsvorschriften konform, bzw. zu den Spezifikationsvorschriften konform spezifiziert.
    \end{itemize}
    Damit ist von System-Design, Architekturen oder die nachfolgenden Tests zu differenzieren. 
    \\
    Dieser Prozessschritt deckt die Formalitäten ab, um allen beteiligten eine grundlegende Übersicht zu geben, welche 
    Ziele verfolgt werden, bzw. welche Anforderungen erfüllt werden sollen. Auch ist das \acl{RE} ein wesentlicher 
    Bestandteil des Entwicklungsprozesses nach Scrum. Unter dem Begriff Scrum ist ein Vorgehensmodell zu verstehen, 
    welches zum Projekt- und Produktmanagement in agilen Softwareentwicklungen eingesetzt wird. 
    \\
    \linebreak
    Innerhalb der Anforderungen, die im \acl{RE} ermittelt werden, wird zwischen drei Arten von Anforderungen unterschieden: 
    \begin{itemize}
        \item Funktionale Anforderungen: Diese legen die Funktionalität fest, die durch die Entwicklung des Systems 
            erreicht werden soll. Typischerweise werden diese in Funktions-, Verhaltens- und Strukturanforderungen unterteilt.
        \item Qualitätsanforderungen: Diese legen die Qualitäten des Systems fest und beeinflussen die Gestaltung der 
            Systemarchitektur in größerem Maß als die funktionalen Anforderungen.
        \item Randbedingungen (Constraints): Diese legen die Randbedingungen des Systems fest.
    \end{itemize}
\chapter{Stand der Technik}
\label{chap:technikStand}
    Zur Analyse des aktuellen Standes der Technik und Forschung in Bezug auf die Konzeption von Software-Lösungen, mit denen 
    die formalisierten Interaktionen der Softwareentwickler vereinfacht werden können, erfolgt in diesem 
    Kapitel ein systematisches Literaturreview. Die Literaturprüfung wird gemäß den Richtlinien, die in der Publikation 
    von \cite{Kitchenham2007} vorgeschlagen werden, durchgeführt. 

    \section{Systematisches Literaturreview}
    \label{subsec:systematischesLiteraturReview}
        Die Thematik des systematischen Literaturreviews wurde bereits in den einleitenden Kapiteln erwähnt. 
        Dieser Abschnitt widmet sich ausschließlich der Anwendung der Richtlinien und dem daraus abgeleiteten Stand der Technik 
        abhängig zu dem in der Arbeit behandelten Thema. 

    \subsection{Ziele des Systematischen Literaturreviews}
        Das Ziel dieses systematischen Literaturreviews ist es, die aktuellen Fortschritte von Smart Home Plattformen und 
        Gateways in Richtung der entwicklerseitigen Benutzerfreundlichkeit zu recherchieren. Dabei liegt der Schwerpunkt 
        auf der Usability und der einfachen Handhabung der formalisierten Interaktionen der Softwareentwickler. Es gilt zu 
        analysieren, ob es in diesem Themenbereich bereits Publikationen und Forschungen gibt und welche Entscheidungen 
        getroffen werden müssen, um die Weiterentwicklung eines Systems je nach hinzukommender Funktionalität oder 
        auch Bedingung übersichtlich zu halten. Die Ergebnisse dieses systematischen Literaturreviews sollen 
        als Grundlage der Konzeption einer solchen Plattform dienen und mit einfließen. 

    \subsection{Suchstrategie- und anfragen}
        Dieser Abschnitt beschreibt die Suchstrategie und die Anfragen zu dem systematischen Literaturreview. Hierbei wird 
        erläutert, anhand welcher Kriterien die Literatur ausgewählt wird.
        \\
        In den ersten Schritten werden anhand der Forschungsfragen in Kapitel (\ref{sec:forschungsfragen}) die Stichpunkte 
        aufgegriffen und als Suchterm formuliert. Die daraus resultierenden Suchterme, die der tabellarischen Darstellung 
        (\ref{tab:slr}) zu entnehmen sind, werden in diversen wissenschaftlichen Fachdatenbanken recherchiert und analysiert. 
        Die Ergebnisse werden nach ihrem Titel und der Zusammenfassung sortiert. Gibt es Publikationen mit vergleichbaren 
        Inhalten, so sind diese in weiteren Schritten näher zu betrachten. Damit die Literaturrecherche weitere Ergebnisse 
        erzielt, wird beim studieren der Publikationen die Schneeballsuche angewendet. Dabei wird in den Quellen der jeweiligen 
        Literatur auf weitere Verweise geschaut, die ebenso potentielle Inhalte bearbeiten. Die Kombination beider 
        Literaturergebnisse bilden die Grundlage der zu analysierenden Quellen und geben so den Stand der Technik wieder.
        \begin{figure}[hbt!]
            \centering
            \includegraphics[width=13cm,height=13cm,keepaspectratio]{images/slr_walkthrough.png}
            \caption{Strategie der Literatursuche}
            \label{fig:slr}
        \end{figure}
        \\
        Die generierten Suchterme, die daraus resultierenden Literaturergebnisse und der damit einhergehende Suchverlauf in Auszügen ist 
        zur Nachvollziehbarkeit tabellarisch aufgeführt:
        \\
        \linebreak
        \pagebreak
        \begin{table}[hbt!]
            \begin{center}
                \begin{tabular}{| p{2.9cm} | p{1.9cm} | p{1.6cm} | p{1.9cm} | p{1.9cm} | p{1.8cm} | p{1.8cm} | }
                    \hline
                        \textbf{Suchanfrage} & \textbf{Datum} & \textbf{Filter} & \textbf{Plattform} & \textbf{Ergebnisse} & \textbf{Gesehene} & \textbf{Relevant} \\
                    % \hline
                        % formalized interactions software development & 08.04.2022, 03.05.2022 & Nein & IEEExplorer, WebOfScience & 55, 106 & 5, 4 & 0, 0 \\ 
                    \hline
                        formalized interactions software development architecture  & 11.04.2022, 03.05.2022 & Nein & IEEExplorer, WebOfScience & 13, 26 & 6, 4 & 0, 0 \\ 
                    \hline
                        usability AND formalized interactions AND architecture & 11.04.2022, 03.05.2022 & Nein & IEEExplorer, WebOfScience & 3, 7 & 1, 3 & 0, 1 \\ % A Conceptual Model of Service Customization and Its Implementation
                    \hline
                        usability AND formalized interactions AND architecture AND smart home & 01.05.2022 & Nein & IEEExplorer, WebOfScience & 0, 0 & 0, 0 & 0, 0 \\
                    \hline
                        usability AND formalized interactions AND software development AND architecture & 02.05.2022 & Nein & IEEExplorer, WebOfScience & 1, 1 & 1, 1 & 0, 1 \\
                    \hline
                    %    usability AND formalized interactions AND software development AND architecture AND smart home & 02.05.2022, 03.05.2022 & Nein & IEEExplorer, WebOfScience & 0, 0 & 0, 0 & 0, 0 \\
                    %\hline
                        usability AND architecture AND gateway AND smart home & 06.05.2022 & Nein & IEEExplorer, WebOfScience & 2, 4 & 1, 1 & 1, 1 \\
                    \hline
                        usability AND formalized interaction AND (architecture OR smart home OR software developer OR gateway) & 09.05.2022 & Nein & IEEExplorer, WebOfScience & 3, 10 & 1, 2 & 0, 0 \\
                    \hline
                        usability AND architecture AND smart home AND (formalized interaction OR software developer OR iot) & 09.05.2022 & Nein & IEEExplorer, WebOfScience & 13, 22 & 3, 3 & 0, 0 \\
                    \hline
                \end{tabular}
            \end{center}
            \caption{Suchprotokoll des Systematischen Literaturreviews}
            \label{tab:slr}
        \end{table}
    \subsection{Datenextraktion und Synthese}
        Zu der zielgerichteten Datenextraktion werden in Anlehnung an die Richtlinien des systematischen Literaturreviews die folgenden 
        Einschluss- und Ausschlusskriterien definiert, die dabei helfen, die relevanten Publikationen zu finden:
        \begin{table}[hbt!]
            \centering
            \begin{tabular}{p{0.125cm} p{15cm}}
                    \textbf{\#} & \textbf{Einschluss-Kriterien}\\ 
                \hline
                    1  & Die Literatur ist in den wissenschaftlichen Fachdatenbanken veröffentlicht, darunter: WebOfScience, IEEExplorer, SpringerLink, Elsevier, Addison-Wesley, dpunkt-Verlag, ACM und Google Scholar  \\ 
                \hline
                    2  & Der Beitrag wurde nach 2010 veröffentlicht \\ 
                \hline
                    3  & Die Veröffentlichung ist in deutscher oder englischer Sprache \\
                \hline
                    \textbf{\#} & \textbf{Ausschluss-Kriterien}\\ 
                \hline
                    1  & Die Veröffentlichung beinhaltet, bzw. behandelt nicht die Schlagworte usability, architecture, formalized interaction oder iot \\ 
                \hline
                    2  & Die Publikation gehört zu der Literatur der Grauzone \\ 
                \hline
                    3  & Die Veröffentlichung hat weniger als 5 Zitationen \\
                \hline
                    4  & Die Literatur hat weniger als 5 Seiten Inhalt \\
            \end{tabular}
            \caption{Einschluss- und Ausschlusskriterien des systematischen Literaturreviews}
            \label{tab:slr_criteria}
        \end{table}
        \\
        Anhand der Forschungsfrage, den Suchtermen und den Einschluss- und Ausschlusskriterien zu der Quellenrecherche werden während der Anwendung
        des systematischen Literaturreview-Templates und deren Richtlinien die erzielten Ergebnisse synthetisiert und zusammengefasst. Im Rahmen 
        dieser Forschungsfrage und der Auslegung der Suchterme, die der Tabelle (\ref{tab:slr}) zu entnehmen sind, ergab sich nicht die Menge an Literatur, 
        die notwendig ist, um ein umfangreiches Literaturreview im Stil des Templates durchzuführen. Demnach ist kein vollständiges systematisches Literaturreview 
        möglich. Die wenigen relevanten Ergebnisse des Literaturreviews geben einen Einblick in Bereiche, die als Gedankenanstoß und zum Transferieren von 
        Wissen, Gedanken und Ideen geeignet sind. 
        \\
        \linebreak
        Das Resultat des durchgeführten Literaturreviews beweist, dass es in diesem Bereich bisher wenige Publikationen gibt und sich dadurch eine neue Nische bilden könnte. 
        Um dennoch eine fundierte wissenschaftliche Grundlage zu repräsentieren, wird in 
        folgendem Abschnitt auf Referenzen und Literatur eingegangen, die Teile der Forschungsfrage abdecken und mehr als Gedankenanstoß 
        anzusehen sind, beziehungsweise auch einen partiellen Einblick in den Stand der Technik bietet. 
%\pagebreak
\section{Zusammenfassung} 
    Speziell auf die Forschungsfrage in Kapitel (\ref{sec:forschungsfragen}) gibt es in dem heutigen Stand der Technik keine explizite 
    Literatur, welche den Gedanken aufgreift und behandelt. Die Literatur, die bei dem systematischen Literaturrecherche erarbeitet wurde, 
    deckt trotz dessen manche Aspekten ab, die interessante Anhaltspunkte und Ideen thematisieren. Diese geben Impulse und helfen bei den weiteren 
    Schritten in dieser Arbeit hinsichtlich der Konzeption. 
    
    \subsection{Publikationen}
    \label{subsec:publications}
        Im Folgenden wird die relevante Literatur aufgegriffen und zusammengefasst, sodass ein Einblick in den Prozess der 
        Informationserhebung, der Sammlung von Erfahrungswerten und Ideen gewährleistet wird.
        
        \subsubsection*{Design and Realization of a Framework for Human-System Interaction in Smart Homes}
            In dem Artikel von \cite{Wu2012} wird zu Anfang die Beziehung zwischen Benutzern, Räumlichkeiten und 
            Diensten analysiert. Basierend auf den daraus gewonnenen Erkenntnissen wird ein Framework und ein 
            entsprechender Algorithmus vorgestellt, welcher die Interaktionsbeziehungen modelliert. Basierend auf den  
            Ergebnissen wird ein Framework entwickelt, welches die Interaktionsanforderungen abdeckt. Hauptmerkmale 
            waren dabei Komfort, Bequemlichkeit und Sicherheit. Zur abschließenden Überprüfung des Designkonzeptes und 
            der Implementierung wurden Probanden zum Testen der Anwendung ausgewählt und anschließend ein Interview 
            durchgeführt. Das Evaluierungsergebnis zeigt, dass das Framework eine gute Einführung in die Verbesserung 
            der Mensch-System-Interaktionen darstellt. 

        \subsubsection*{Seamless Integration of Heterogeneous Devices and Access Control in Smart Homes}
            Der Artikel von \cite{Kim2012} erarbeitet einen Vorschlag einer ganzheitlichen und erweiterbaren 
            Softwarearchitektur, welches Dienste und heterogene protokoll- und herstellerspezifische Geräte 
            nahtlos integrieren lässt und Sicherheit über das Internet gewährleistet. Grundlegend wird hierbei auf das 
            \acs{OSGI} Framework gesetzt. Dadurch wird die semantische Interoperabilität hervorgehoben. Dies ist die Fähigkeit, 
            neue Anwendungen und Treiber zur Laufzeit in das bereitgestellte System zu integrieren \cite{Kim2012}. 
            Zusätzlich zu dem System wird im Rahmen dieser Publikation ein Zugangskontrollmodell für spezielle \acl{SH} 
            Szenarien integriert. Zur Beweisführung wird das Konzept anhand von semantischen Erkennungen von Heimgeräten zur 
            Laufzeit demonstriert. Dafür werden mehrere Protokolle, darunter X10, ZigBee und Insteon, in einen realen Test 
            integriert. Die Arbeit behandelt die folgenden Schwerpunkte \cite{Kim2012}:
            \begin{itemize}
                \item Analyse einer Reihe von Heimnetzprotokollen hinsichtlich ihrer Erkennungs- und Integrationsanforderungen.
                \item Eine erweiterbare Home-Gateway-Architektur, die es ermöglicht, heterogene Geräte während der Laufzeit flexibel zu installieren, zu verwalten und darauf zuzugreifen.
                \item Ein neuartiger Zugangskontrollmechanismus speziell für Smart-Home-Systeme.
                \item Die Umsetzung des vorgeschlagenen Konzepts, indem gezeigt wird, wie verschiedene Geräte integriert und von Endbenutzern aufgerufen werden können.
            \end{itemize}
            Die Architektur und die damit eingesetzte semantische Abstraktionsschicht unterstützt laut den Ergebnisse 
            erheblich die Anwendungsentwicklung. 
            \\
            Mit der Zugriffskontrollrichtlinie wird den Hausbesitzern eine stabile Kontrolle darüber gegeben, mit der 
            die Benutzer auf die smarte Geräte zugreifen können. Das Resultat dieser Arbeit scheint zu zeigen, dass damit 
            die Barriere für \acl{SH} Systeme gesenkt wird. 

        \subsubsection*{Wireless Architectures for Heterogeneous Sensing in Smart Home Applications: Concepts and Real Implementation}
            Dieser Beitrag von \cite{Viani2013} diskutiert die aktuellen Trends von drahtlosen Architekturen für Anwendungen 
            im \acl{SH} Bereich. Aus der Diskussion wurden Vorteile erarbeitet, die anschließend über die Verwendung der 
            drahtlosen Architektur analysiert wurden. Schwerpunkt dabei lag jedoch auf der Schätzung des Benutzerverhaltens. 
            \\
            \linebreak
            Allgemein behandelt der Artikel die Vorstellung einer drahtlosen Architektur für intelligentes Energiemanagement 
            und die Überwachung von älteren Menschen, indem die Anwesenheit, die Bewegung und das Verhalten der Bewohner 
            analysiert wird \cite{Viani2013}. 
            \\
            Interessant dabei ist das entstandene Konzept der konkreten Softwarearchitektur.

        \subsubsection*{My House, My Rules: A Private-by-Design Smart Home Platform}
            Dieses Whitepaper von \cite{Zavalyshyn2020} stellt eine \textit{Private-by-Design-IoT-Plattform} für \acl{SH} 
            Umgebungen vor. Mit dem Konzept wird eine typische Architektur für bestehende \acs{IoT} Plattformen als Grundlage 
            verwendet, die über ein alternatives Design mehr Sicherheit und Kontrolle für den Hauseigentümer bietet. 
            Genutzt wird dabei die von Intel entwickelte \ac{SGX}\footnote{Eine hardwarebasierte Verschlüsselung von Speicherinhalten, die bestimmten Programmiercode und Daten im Speicher isoliert. \url{https://www.intel.de/content/www/de/de/architecture-and-technology/software-guard-extensions.html} Abgerufen am 15.05.2022.}. 
            Diese Erweiterung ermöglicht es, eine intuitive Sicherheitsabstraktion einzuführen, die den unbefugten Zugriff auf Daten 
            durch nicht vertrauenswürdige \acs{IoT} Cloud-Anbieter verhindert. Das Konzept wurde in einen Prototyp umgesetzt und 
            evaluiert. Dabei wurden auch eine quantitative Forschung durchgeführt, die aus mehr als 40 Probanden bestand, welche 
            den Prototypen verwendet und anschließend bewertet und beurteilt haben. Die Mehrzahl der Teilnehmer der Feldstudie 
            hielten die Softwareplattform als benutzerfreundlich und die unterstützen Richtlinien durch die Sicherheitsabstraktion 
            für nützlich \cite{Zavalyshyn2020}. Mit den Richtlinien kann verstärkt die Privatsphäre der Anwender in ihrem Wohnraum 
            geschützt werden. Der Sicherheitsmonitor der Software ermöglicht Endbenutzern eine granulare Kontrolle und Überwachung 
            der Datenflüsse, die durch die \acs{IoT} Geräte generiert werden, sowie die Verhinderung von potentiellen 
            Datenschutzverletzungen der Hersteller durch die Verwendung einer Datenschutzrichtlinie. 

        \subsubsection*{Fast-prototyping Approach to Design and Validate Architectures for Smart Home}  
            Inhalt des Artikels von \cite{Montanaro2021} ist das Entwickeln eines komplexen \acl{SH} Systems. 
            Hintergrund dafür ist die kontinuierliche Entwicklung und Kommerzialisierung sämtlicher \acs{IoT} Geräte und 
            die damit einhergehende Änderung oder Anpassung der Nutzeranforderungen. Dadurch benötigt die Community eine 
            schnelle Lösung, bzw. einen Prototypen, um die Anforderungen der Nutzer zu erfüllen und auf schnell 
            entstehenden Notwendigkeit vorbereitet zu sein. 
            \\
            Grundlage für die Entwicklung der Plattform ist eine aktuelle und solide Studie, die ebenso im Rahmen des 
            Artikels durchgeführt wurde. Die Benutzeranforderungen wurden aus der Studie extrahiert und sind bei der 
            Planung des Konzeptes mit eingeflossen. Bestandteil dieser Arbeit ist unter anderem die Verwendung von Node-RED\footnote{Ein von IBM entwickeltes grafisches Entwicklungswerkzeug mit Baukastenprinzip. Funktionsbausteine können per Verbindungen miteinander verknüpft und als Prozess bearbeitet werden. \url{https://nodered.org/} Abgerufen am 15.05.2022.} und 
            dem \acs{MQTT} Kommunikationsprotokoll. 
            \\
            \linebreak
            Die in der Arbeit vereinfachten formalisierten Interaktionen werden speziell dem Anwender gegenüber durch 
            Node-RED visualisiert und die komplexe Logik somit vereinfacht dargestellt. Die Idee mit dem Verwenden von Node-RED 
            ist ein guter Gedanke der formalisierten Interaktionen und kann auch dem Entwickler programmatisch Schritte 
            erleichtern. Da selbst das Framework verstanden werden muss, ist eine gewisse Komplexität nicht zu verhindern. 
            Dennoch kann der Gedanke in weiteren Forschungsschritten weitergedacht oder optimiert werden. Mit diesem 
            Grundsatz sind gute Lerneffekt zu erzielen, indem diese Technologie zu praktischen Arbeiten im Forschungs- und 
            Bildungsbereich eingesetzt wird. 
    \\
    \linebreak
    Alle oben aufgeführten Artikel und Forschungsarbeiten sind keine explizite Referenz, bzw. stehen in keinem vollumfänglichen 
    Verhältnis zu der in dieser Arbeit gestellten Forschungsfrage, da mit dem systematischen Literaturreview keine Literatur 
    gefunden werden konnte, die sich ausschließlich dem zu behandelnden Thema widmet. Die zusammengefassten Arbeiten dienen 
    lediglich als stichpunktartige Übersicht, um einen Einblick zu gewährleisten, welche Forschungsarbeiten mit den 
    Suchtermen (siehe Tabelle \ref{tab:slr}) erzielt wurden. Basierend auf diesen Grundlagen und dem damit vermittelten Wissen 
    konnten unter anderem die Anforderungen verfeinert und gefestigt werden, sowie Ideen für das folgende Konzept in Kapitel 
    (\ref{chap:konzept}) aufgegriffen werden.
    \\ 
    Durch das systematische Literaturreview wird deutlich, dass der Stand der Technik in die Richtung, in die die Arbeit 
    abzielt, keine aussagekräftigen Ergebnisse zeigt. Mittels weiteren Experten Interviews wird deutlich, dass viele mit der 
    Thematik des \acl{SH} vertraut sind, jedoch überwiegend nur als Benutzer bestehender Plattformen und Geräte gelten.
    \\
    Die mit den Experten Interviews erlangten Informationen sind nicht repräsentativ, lediglich eine Erkenntnis im Rahmen 
    dieser Arbeit.
    \\
    Um einen Gesamtüberblick bezüglich dem Technikstand rundum \acl{SH} zu vermitteln, 
    wird in folgendem aus einer etwas pragmatischeren Sichtweise der Stand der Technik skizziert.
    
    %\pagebreak
    \subsection{Stand der Technik aus Nutzer- und Produktsicht}
        Der aktuelle Markt bietet bereits viele Möglichkeiten und Alternativen der Umsetzung eines intelligenten Zuhauses. 
        Darunter z.B. kommerzielle in sich geschlossene Systeme, die ausschließlich dem Nutzer als Anwendung verkauft werden, 
        einzelne Intelligente Geräte, die nichts zwangsläufig eine zentrale Plattform benötigen und Plattformen, wie bspw. 
        Home Assistant und openHAB, die eine gewisse Technikaffinität erfordern und voraussetzen, um diese nach belieben 
        einsetzen und konfigurieren zu können. Viele große Unternehmen, darunter Bosch, Telekom, Siemens, Samsung und viele mehr bieten Geräte 
        und Softwarelösungen an, die der Nutzer anwenden kann und wenig Möglichkeiten hat, einzelne Komponenten zu individualisieren. 
        Diese geben keinen Einblick in ihre Software 
        und deren Umsetzung, daher sind diese zum Vergleich nicht weiter zu berücksichtigen. 
        \\
        \linebreak
        Neben den bekanntesten, frei verfügbaren Plattformen \acs{OPENHAB} und Home Assistant gibt es weitere Lösungen, die eine 
        Plattform zur Verfügung stellen\footnote{Aufstellungen von verfügbaren Open Source Smart Home Plattformen im Jahr 2020. \url{https://ubidots.com/blog/open-source-home-automation/} Abgerufen am 16.05.2022}:
        \begin{itemize}
            \item OpenMotics
            \item Jeedom
            \item ioBroker
            \item AGO Control 
            \item Domoticz
            \item Homebridge.io
        \end{itemize}
        Neben den aufgelisteten Produkten gibt es viele mehr, die jedoch nicht weiter aufgeführt werden. 
        Durch die Vielzahl der Angebote und den immer größer werdenden Mehrwert, den eine solche Plattform bietet, 
        nimmt die Forschung und Entwicklung in dem Bereich des \acs{IoT} zu. Dadurch können immer mehr 
        Anwendungsfälle abgedeckt und realisiert werden, wodurch dem Nutzer die Einsatzmöglichkeiten deutlicher aufgezeigt werden und 
        effektiver genutzt werden können. 
        \\ 
        \linebreak
        Für das Abbilden und Verwalten von Prozessen und Automatisierungen gibt es Systeme, sogenannte Regelwerke. 
        Diese sind dafür entwickelt, damit bestimme Regel, Prozesse und Sachverhalte definiert und durchlaufen werde können. 
        Einsatz finden die Frameworks in \ac{BRMS}, in Systemen, die eine bestimmte Logik über 
        Regeln abbilden und in \acl{SH} Plattformen. Durch die Definition von Regel und Automatisierungen im Bereich des 
        \acl{SH} ist der Einsatz von Regelwerken auch Bestandteil heutiger Plattformen. Ursprünglich war der Ansatz des Regelwerks 
        für die effiziente Ausführung von Geschäftsregeln entwickelt. Es gibt eine Vielzahl an Anbietern, die solche Systeme und Frameworks 
        in unterschiedlicher Implementierung bereitstellen. Die im Java-Umfeld bekanntesten sind unter anderem Drools, OpenL Tablets, 
        Easy Rules und RuleBook\footnote{Regelwerke im Java-Umfeld. \url{https://www.baeldung.com/java-rule-engines} Besucht am 16.05.2022}
        \\
        Im folgenden Kapitel ist eine Marktanalyse (siehe Abschnitt \ref{sec:marktanalyse}) im Bereich \acl{SH} dargelegt, die auf die Fortschritte 
        und Prognosen der nächsten Jahre eingeht.
        %Dies wird in Kapitel 
        %(\ref{chap:anforderungsanalyse}) durch die konkrete Marktanalyse (siehe Abschnitt \ref{sec:marktanalyse}) deutlich. 
        % !!! Ergänzung um CommandLine Tools, die dem Entwickler helfen auf zentraler Ebene (ng generate, GitHub Copilot, 
        % Code completion, code generation etc.)
\chapter{Anforderungsanalyse}
\section{Use Cases}
\subsection{Check in mit Temi}
\subsection{Notfallevakuierung mit Temi}

\begin{landscape}
  \AddToShipoutPictureBG*{%
  \AtPageLowerLeft{%
    \raisebox{\dimexpr.5\paperheight-.5\height}{%
      \makebox[.9\paperwidth][r]{%
        \rotatebox{90}
        {\thepage}
      }% \makebox
    }% \raisebox
  }% \AtPageCenter
}% \AddToShipoutPictureBG*
  \includepdf[scale=0.75, pages=-, angle=90, fitpaper=true, ]{images/UC1_Aktionsdiagramm_Check-in.pdf}
\end{landscape}
\begin{landscape}
  \AddToShipoutPictureBG*{%
  \AtPageLowerLeft{%
    \raisebox{\dimexpr.5\paperheight-.5\height}{%
      \makebox[.9\paperwidth][r]{%
        \rotatebox{90}
        {\thepage}
      }% \makebox
    }% \raisebox
  }% \AtPageCenter
}% \AddToShipoutPictureBG*
  \includepdf[scale=0.75, pages=-, angle=90, fitpaper=true, ]{images/UC1_Sequenzdiagramm_Check-in.pdf}
\end{landscape}
\subsection{Notfall-Evakuierung mit einem Service-Roboter}
\label{subsec:evacuation}
    Mit dem Szenario der Notfall-Evakuierung wird ein weiterer Anwendungsfall definiert, welcher dabei helfen soll, die 
    Anforderungen zu identifizieren, die für die Bereitstellung des Frameworks der Steuerzentrale benötigt werden.
    \\
    \linebreak
    Objektiv betrachtet gibt es bei diesem Fall ebenso einen Auslöser, bspw. einen Rauch- 
    oder Gasmelder, der eine \acs{MQTT}-Nachricht veröffentlicht, die über die Steuerzentrale konsumiert, verarbeitet und dadurch die 
    weiteren notwendigen Schritte eingeleitet werden. Nachdem das \acs{MQTT}-Topic mit der Nachricht von der Steuerzentrale erhalten wurde, wird 
    basierend auf dem Nachrichteninhalt die Regel und die darauffolgende Aktion angestoßen. Die Steuerzentrale muss über die \acs{API} 
    Schnittstelle des internen Büroplatzbuchungssystems alle eingecheckten Personen und Platzbuchungen abfragen und 
    zwischenspeichern. Die Informationen werden genutzt, um den Service-Roboter an die Arbeitsplätze der jeweilig 
    eingecheckten Personen zu schicken, sodass dadurch diese über den Notfall informiert, bzw. gebeten werden, das 
    Gebäude zu verlassen. Wird eine Person an einem Arbeitsplatz erkannt, so kann diese über den Sachverhalt informiert 
    werden. Ist jedoch der Arbeitsplatz leer, soll der Service-Roboter den nächsten Arbeitsplatz ansteuern. Nachdem alle 
    Plätze von dem Service-Roboter abgefahren wurden, soll dieser über die restliche Bürofläche fahren und nach Personen 
    Ausschau halten. Die Erkennung der Person wird durch die Kamera des Roboters durchgeführt. Abschließend, wenn alle Plätze und die 
    Bürofläche abgefahren wurden, soll der Roboter an eine zentrale Stelle im Büro fahren und ohne Unterbrechung eine 
    Durchsage starten und diese solange wiederholen, bis eine Person den Vorgang manuell beendet. Mit der Beendung der Dauerschleife 
    ist das Szenario abgeschlossen und der Roboter kann an seine Ausgangsposition zurückgeführt werden, sofern dies 
    umgebungsbedingt noch möglich ist.
    \\
    \linebreak
    Die Anwendungsfälle wurden so gewählt, da diese eine gewisse Komplexität mit sich bringen, die es mit der Steuerzentrale 
    abzudecken gilt.
    \\
    \linebreak
    Zur Stützung der textuellen Schilderung des zweiten Anwendungsfalls werden nachfolgend Diagramme dargestellt, die das Szenarien 
    widerspiegeln. Im Rahmen des \acs{RE} wurden hierfür ebenso eine konkrete Aufgabenbeschreibung, sowie eine User Story definiert. Diese sind dem 
    Anhang (siehe Anhang \ref{appendix:user-story-uc2}) zu entnehmen. Die folgenden Abbildungen stellen sich aus einem 
    Aktivitätsdiagramm, einem Sequenzdiagramm und einem Ablaufdiagramm zusammen: 
    %\\
    %Aus dem Kontext beider Anwendungsfälle, der vorangestellten Zielgruppenanalyse und den Experteninterviews werden in Abschnitt 
    %(\ref{sec:requirementsFinal}) die daraus identifizierten Anforderungen für die Steuerzentrale aufgeführt.
    
\begin{landscape}
  \AddToShipoutPictureBG*{%
  \AtPageLowerLeft{%
    \raisebox{\dimexpr.5\paperheight-.5\height}{%
      \makebox[.9\paperwidth][r]{%
        \rotatebox{90}
        {\thepage}
      }% \makebox
    }% \raisebox
  }% \AtPageCenter
}% \AddToShipoutPictureBG*
  \includepdf[scale=0.75, pages=-, angle=90, fitpaper=true, ]{images/UC2_Aktionsdiagramm_Notfall.pdf}
\end{landscape}
\begin{landscape}
  \AddToShipoutPictureBG*{%
  \AtPageLowerLeft{%
    \raisebox{\dimexpr.5\paperheight-.5\height}{%
      \makebox[.9\paperwidth][r]{%
        \rotatebox{90}
        {\thepage}
      }% \makebox
    }% \raisebox
  }% \AtPageCenter
}% \AddToShipoutPictureBG*
  \includepdf[scale=0.75, pages=-, angle=90, fitpaper=true, ]{images/UC2_Sequenzdiagramm_Notfall.pdf}
\end{landscape}
\section{Experteninterview}
\label{sec:experteninterviewReqirements}
    Zur Analyse und Erhebung von Anforderungen, die sich an das System richten, werden Experteninterviews durchgeführt. 
    Dabei wird sich, wie in dem Abschnitt (\ref{subsec:experteninterview}) beschrieben, an dem unstrukturierten Ansatz der 
    Führung eines solchen Interviews orientiert \cite{robson2002real}. Infolgedessen werden keine konkreten offenen oder 
    geschlossenen Fragen gestellt. Der Aufbau des Interviews wird in weiteren Schritten aufgegriffen. 
    \\
    Die Ergebnisse der Interviews sind nicht repräsentativ, da sie im Rahmen dieser Arbeit keine große Masse abdecken. Diese 
    dienen lediglich der weiträumigeren Informationsgewinnung und dem Sammeln mehrerer Meinungsbilder, um aus allen 
    Ideen, Gedankenanstößen und Meinungen ein objektives Gesamtbild zu erzeugen und daraus viele Anforderungen zu gewinnen 
    und abzuleiten. 
    %Die Experten Interviews sind eine Möglichkeit, die Anforderungen zu erfassen, die durch die Entwicklung des Systems 
    %nicht durch eine einfache Marktanalyse erfasst werden konnten. Diese Interviews sind nicht repräsentativ und dienen lediglich 
    %der weiträumigeren Informationsgewinnung.
\subsection{Ziel des Experteninterviews}
    % Diese helfen dabei, mehrere Sichten, Meinungen und Erfahrungen einzuholen, um ein 
    Ziel des Experteninterview ist die Informationsgewinnung aus bestimmten Sachverhalten, die nicht leicht auf anderem Wege 
    zu beschaffen sind. Der Experte dient als Informationsträger und kann dabei helfen, seine Sicht auf den Sachverhalt 
    wiederzugeben. Dadurch können Meinungen eingeholt und Erfahrungen ausgetauscht werden. Diese sind oft wichtige 
    Informationen zur Erhebung von Anforderungen zu einem bestimmten Sachverhalt. % gegenüber.

\subsection{Aufbau des Experteninterviews}
    Die Experteninterviews wurden im Gesamten als unstrukturierte Interviews durchgeführt. Lediglich die Rahmenbedingungen sowie 
    der Einstieg in das Interview waren vorgegeben und bei jeder interviewten Person ähnlich. Zu Anfang des Gesprächs wurde 
    der Kontext und die Intension erläutert, damit der Experte die Situation und die Absichten kennen lernt und die 
    eigentliche Herausforderung %Problemstellung%
    erkennt. Grundlage dafür war die Erläuterung der Zielsetzung (\ref{sec:zielsetzung}) der Arbeit, 
    um so die Intension zu verdeutlichen. Mit den identifizierten Anwendungsfällen (\ref{sec:usecases}), die als Basis 
    zur Extraktion von Anforderungen und als potentiell umsetzbare Funktionalitäten gelten, wurden Szenarien veranschaulicht, 
    die dabei halfen, das Anwendungsumfeld zu konkretisieren.  
    Nach der Schilderung des Kontextes und der zugrundeliegenden Ausgangspunkten wurde das Gespräch in Richtung Anforderungen 
    gelenkt. Hierbei lag der Fokus auf der Erhebung der Ideen, Sichtweisen und Meinungen, die als Grundlage für Anforderungen 
    dienten oder gar direkte Anforderungen an das System ergaben. Dabei war der Ausgang des Gesprächs offen. Falls während 
    eines Gespräches der Fokus verloren ging, bzw. Exkurse ein zu großes Ausmaß annahmen, wurde wieder auf die 
    vorliegende Sachlage aufmerksam gemacht und der Fokus erneut auf die Anforderungen geworfen. 
    Schwerpunkte bei den Interviews waren zum einen, welche Anforderungen gelten, um eine einfache Handhabung der 
    formalisierten Interaktionen für den Softwareentwickler zu gewährleisten und zum anderen die Funktionalitäten, die der 
    Experte dem System gegenüber sieht, um ein Regelwerk für ein intelligentes Büro zu erstellen. 
    Ebenso wurden nichtfunktionale Anforderungen, die ein solches System mit sich bringen soll, adressiert. 
    Die zusammengefassten Anforderungen und daraus abgeleiteten Bedingungen sind dem Abschnitt (\ref{sec:requirementsFinal}) 
    zu entnehmen.
    % Zum Einstieg des Interviews wird der Kontext erläutert (Welche Software entwickelt werden soll) UND 
    % dass der Fokus auf der einfachen Handhabung der formalisierten Interaktionen für SOFTWAREENTWICKLER liegt. 
    % Welche Anforderungen der Experte in der Funktionalität als auch Anforderungen dem System gegenüber sieht. 
     
\subsection{Zusammenfassung der Experteninterviews}
    In Summe wurden insgesamt fünf Experteninterviews durchgeführt. Jedes Gespräch war individuell und hatte dementsprechend 
    einen anderen Verlauf, bzw. ein anderes Ergebnis. Dennoch konnte der inhaltliche Fokus gewahrt und verschiedene 
    Meinungsbilder eingeholt werden. Jeder befragte Experte konnte zu dem anliegenden Sachverhalt seine Meinung äußern und 
    wichtige Informationen und Sichtweisen mitteilen. Die erhobenen Informationen wurden analysiert, aufbereitet und 
    als Anforderungen dokumentiert. Die dabei entstandenen Informationen und Anforderungen werden in 
    folgendem Abschnitt aufgezeigt. 
    % Interessante Gespräche wurden geführt. Bereits die Schilderung des Kontextes und der Idee gab es Diskussionsbedarf und 
    % Anforderungen, die sehr interessant waren und als Anforderungen so mit aufgenommen worden sind. 
    % Aus den Interviews ergaben sich Anforderungen, die in die Konzeption mit einfließen. Diese werden im folgenden 
    % Abschnitt thematisiert.
    
%\pagebreak
\section{Anforderungen}
\label{sec:requirementsFinal}
    In Folge der vorangestellten Literaturrecherche, der Markt- und, Zielgruppenanalyse, der entwickelten Anwendungsfälle und 
    der durchgeführten Experteninterviews ergaben sich Anforderungen an das System als auch Bausteine, die in der 
    Konzeption und der anschließenden prototypischen Implementierung der Software essentiell sind. Da der Fokus dieser 
    Arbeit auf der einfachen Handhabung der formalisierten Interaktionen für den Softwareentwickler liegt, werden 
    spezifische Anforderungen, die diese Richtung konkret adressieren, höher priorisiert als allgemeingültige Anforderungen, die mehr das 
    Verhalten einer Software beschreiben. Zur Dokumentation der nicht funktionalen Anforderungen wird sich an dem 
    \textit{Software Produkt Qualitätsmodell} nach der ISO Norm 25010\footnote{Die acht Qualitätscharakteristiken nach ISO 25010. \url{https://iso25000.com/index.php/en/iso-25000-standards/iso-25010} Abgerufen am 24.06.2022.} 
    orientiert. 
    \\
    \linebreak
    Der folgenden Tabelle (\ref{tab:functionalRequirements}) sind die funktionalen Anforderungen (FA) zu entnehmen: 
    \pagebreak
    \begin{table}[hbt!]
        \begin{center}
            \begin{tabular}{ | p{0.6cm} | p{9.5cm} | p{1.6cm} | p{3.1cm} | }
                \hline
                    \textbf{} & \textbf{Anforderung (User Story)} & \textbf{Priorität} & \textbf{Quelle} \\
                \hline
                    FA1 & Als Softwareentwickler möchte ich mit einer vorgegebenen Struktur Regeln definieren, die zur Laufzeit ausgeführt werden können, wenn das zutreffende Ereignis eintrifft. & Essentiell & Zielsetzung der Arbeit \\
                \hline
                    FA2 & Als Softwareentwickler möchte ich Auslöser aktivieren, bzw. definieren können, nach denen die Regeln ausgelöst werden. & Essentiell & Experteninterview \\
                \hline
                    FA3 & Als Softwareentwickler möchte ich einen Zustandsraum erstellen, der nach meinen Bedürfnissen und Anwendungsfällen variabel implementiert werden kann. & Essentiell & Experteninterview \\ 
                \hline
                    FA4 & Als Softwareentwickler möchte ich Bedingungen definieren und abfragen können, die basierend auf dem Ereignis die dazugehörige Regel starten und ausführen lassen. & Essentiell & Experteninterview \\ 
                \hline
                    FA5 & Als Softwareentwickler möchte ich Komponenten anlegen können, die reelle Gegenstände und dessen Zustände abbilden. & Hoch & Experteninterview \\
                \hline
                    FA6 & Als Softwareentwickler möchte ich das vorgegebene Framework individuell nutzen können, indem ich bei der Definition der Regeln und der Informationsbeschaffung nicht eingeschränkt bin. & Essentiell & Zielsetzung der Arbeit, Experteninterview \\
                \hline
                    FA7 & Als Softwareentwickler möchte ich, dass alle Regeln, die ich definiert habe, zu bestimmtem Auslöser gestartet werden. & Essentiell & Experteninterview \\ 
                \hline
            \end{tabular}
        \end{center}
        \caption{Funktionale Anforderungen (FAs)}
        \label{tab:functionalRequirements}
    \end{table}
    \\
    Nach der Aufstellung der funktionalen Anforderungen folgt die der nicht funktionalen Anforderungen (NFA):
   
\begin{table}[hbt!]
    \begin{center}
        \begin{tabular}{ | p{1.0cm} | p{9.7cm} | p{1.6cm} | p{2.6cm} | }
            \hline
                \textbf{} & \textbf{Anforderung} & \textbf{Priorität} & \textbf{Quelle} \\
            \hline
                NFA1 & Benutzerfreundlichkeit (BF): Der Anwender soll nach Implementierung, bzw. Definition einer Regel, diese mit weniger als 3 Schritten in das Regelwerk einfügen können. & Hoch & Experten-interview \\
            \hline
                NFA2 & BF: Mehr als 90\% der Anwender sollten bei erster Verwendung des Frameworks in der Lage sein, Regeln anzulegen und diese dem Regelwerk hinzuzufügen. & Essentiell & Zielsetzung der Arbeit, Experteninterview \\ 
            \hline
                NFA3 & BF: Der Anwender soll die Möglichkeit haben, zu dem Zeitpunkt der Verwendung des Frameworks alle verfügbaren Funktionen nutzen und anwenden zu können. & Mittel & Experten-interview \\ 
            \hline
                NFA4 & BF: Der Anwender braucht sich zu keinem Zeitpunkt um das Management, bzw. um den Ablauf und Durchführung von Regeln kümmern. Er muss nur sicherstellen, dass die Regeln valide sind und dem Kontext passende Bedingungen und Regelausführungen festlegt. & Essentiell & Experten-interview \\ 
            \hline
                NFA6 & Zuverlässigkeit (Z): Zeitlich gebundene Regeln sollen immer zu der vorgegebenen Uhrzeit ausgelöst werden. & Hoch & Experten-interview \\
            \hline
                NFA5 & Z: Als Anwender möchte ich, dass zu 100\% die Regel ausgeführt wird, die der Auslöser ansteuert. (Bei sequenzieller Ausführung soll immer die Regel gestartet werden, die durch den Auslöser ausgelöst wurde.) & Essentiell & Experten-interview \\
            \hline
                NFA7 & Performanz (P): Ausgelöste Regeln, die beide voneinander unabhängige Ressourcen, bzw. einen Wert im Zustandsraum belegen, sollen parallel ausgeführt werden. & Hoch & Anwendungsfall, Experten-interview \\ 
            \hline
                NFA8 & P: Regeln, die über Kommunikationsprotokolle ausgelöst werden, sollen unter einer Sekunde starten. Die Zeit der Durchführung ist abhängig von der Regel selbst. & Mittel & Anwendungsfall, Experten-interview \\ 
            \hline
                NFA9 & Verfügbarkeit (V): Die Steuerzentrale muss eine Verfügbarkeit von 99,9\% vorweisen. (Zum aktuellen Zeitpunk ist diese Anforderung zu vernachlässigen.) & Niedrig & Experten-interview \\ 
            \hline
                NFA10 & V: Die Steuerzentrale muss während der Arbeitszeiten zwischen 7-18 Uhr ohne Ausfälle verfügbar sein. (Zum aktuellen Zeitpunk ist diese Anforderung zu vernachlässigen.) & Niedrig & Experten-interview \\ 
            \hline
                NFA11 & Fehlertoleranz: Syntaktisch oder inhaltlich fehlerhafte Regeln sollen nicht zum Absturz der Steuerzentrale führen. & Mittel & Experten-interview \\ 
            \hline  
                NFA12 & Kontrollierbarkeit, Beobachtbarkeit: Prozesse und Zustände der Steuerzentrale müssen eingesehen werden können, bspw. durch Monitoring oder über Oberflächen. (Zum aktuellen Zeitpunkt zu vernachlässigen.) & Niedrig & Experten-interview \\
            \hline
        \end{tabular}
    \end{center}
    \caption{Nicht funktionale Anforderungen (NFAs)}
    \label{tab:notfunctionalRequirements}
\end{table}
\subsubsection*{}
Eine Auswahl an zusätzlichen Anforderungen, die ergänzend zu den bereits Bestehenden aus der Zielgruppenanalyse und den Anwendungsfällen entstanden sind und sowohl die funktionalen 
(\ref{tab:functionalRequirements}) als auch die nicht funktionalen Anforderungen (\ref{tab:notfunctionalRequirements}) erweitern, ist der folgenden 
Tabelle zu entnehmen. Diese sind als zusätzliche Anforderungen (ZAF) gekennzeichnet: 
\begin{table}[hbt!]
    \begin{center}
        \begin{tabular}{ | p{1.0cm} | p{9.2cm} | p{1.6cm} | p{3.1cm} | }
            \hline
                \textbf{} & \textbf{Anforderung} & \textbf{Priorität} & \textbf{Quelle} \\
            \hline
                ZAF1 & Zur Entwicklung des Frameworks soll Java als Programmiersprache verwendet werden. & Hoch & Experteninterview, Zielgruppenanalyse \\ 
            \hline
                ZAF2 & Die Abbildung von Komponenten soll über einen Zustandsraum dargestellt werden. & Niedrig & Anwendungsfall, Experteninterview \\ 
            \hline
                ZAF3 & Zur parallelen Ausführung von Regeln soll ein Thread Pool eingesetzt werden, der die Regeln jeweils als eigenen Thread laufen lässt. & Hoch & Anwendungsfall, Experteninterview \\ 
            \hline
                ZAF4 & Regeln sollen über zeitbasierte oder MQTT basierte Auslöser gestartet werden, nachdem die dafür vorgesehene Bedingung zutrifft. (Im Rahmen der Anwendungsfälle wird MQTT als Kommunikationsprotokoll und Auslöser gewählt.) & Hoch & Anwendungsfall \\
            \hline
                ZAF5 & Regeln und damit einhergehende Aktionen dürfen erst nach aktivieren eines bestimmten Auslösers (Triggers) ausgeführt werden. & Essentiell & Experteninterview \\
            \hline
                ZAF6 & Der Zustandsraum muss zur Laufzeit zur Verfügung stehen. & Hoch & Experteninterview \\ 
            \hline
                ZAF7 & Die Kommunikation erfolgt überwiegend durch MQTT. & Hoch & Anwendungsfall \\ 
            \hline
                ZAF8 & Sicherheit: Die Kommunikation soll über einen MQTT Broker im eigenen Netzwerk erfolgen. Das System soll nach außen nicht erreichbar sein. & Hoch & Experteninterview \\
            \hline
                ZAF9 & Bereitstellung eines \ac{SPC} (Implementierung, Anpassung und Erweiterung der Regel und der Logik). & Hoch & Zielgruppenanalyse, Experteninterview \\ 
            \hline
            %    ZAF10 & Die Zustände der Komponenten sollen persistiert werden, damit bei einem Systemausfall die letzten Transaktionen und Änderungen nachvollzogen werden können. (Ist zur Konzeption zu berücksichtigen, Priorität aus dem Proof of Concept (PoC) jedoch gering.) & Niedrig & Experteninterview \\ 
            %\hline 
        \end{tabular}
    \end{center}
    \caption{Zusätzliche Anforderungen}
    \label{tab:furtherRequirements}
\end{table} 
\\
Alle relevanten Anforderungen, die mithilfe den genannten und durchgeführten Erhebungsmethoden eruiert wurden, fließen in die Konzeption 
des Frameworks mit ein. Diese bilden die Grundlage des Konzeptes. 

\chapter{Konzeption}
\label{chap:konzept}
    In diesem Kapitel wird das erarbeitete Konzept dargelegt. Basierend auf den 
    Anforderungen, die aus den Anwendungsfällen, den Experteninterviews und der Zielgruppenanalyse 
    erhoben wurden, werden die daraus generierten Überlegungen und Entscheidungen transparent 
    dargestellt. Durch die bereits erfolgte Anforderungsanalyse (siehe Kapitel \ref{chap:anforderungsanalyse})
    sind erste Schritte der Konzeption abgeschlossen. 
    \\
    Zu Anfang des Kapitels wird das allgemeine Ziel eines Konzeptes, sowie die konkrete Absicht hinter dieser Arbeit 
    erläutert. Anschließend 
    wird auf das Anwendungsumfeld des Systems (\ref{sec:anwendungsumfeld}) eingegangen, darunter die
    zusätzlichen Komponenten, die notwendig sind, um das Framework in einer dafür vorgesehenen Umgebung sinnvoll 
    einzusetzen. Darauffolgend wird anhand der zugrundeliegenden Informationen und Anforderungen das 
    Architekturkonzept (\ref{sec:architekturkonzept}) ausgeführt. 
    Dabei wird auf verschiedene kontextabhängige Aspekte, sowie auf die Sichtweisen des Anwenders und des Frameworks eingegangen.


\section{Ziel der Konzeption}
\label{sec:konzeptziele}
    Das Ziel einer Konzeption ist die Veranschaulichung von abstrakten Ideen und Entwürfen. %und Leitideen. 
    Hierbei werden aus den zugrundeliegenden Problemstellungen, Szenarien und Anforderungen Entwürfe und 
    Lösungsmöglichkeiten erarbeitet und identifiziert. Diese helfen bei der Aufstellung von notwendigen Schritten 
    und dienen als Grundlage zur Untermauerung und Darlegung von Entscheidungen. Somit wird Dritten der Kontext, die 
    Domäne und das zu lösende Problem, bzw. die Lösung dargestellt. 
    \\
    \linebreak
    Das Konzept dieser Thesis 
    erarbeitet eine Lösung zur Implementierung einer Anwendung, die es ermöglicht, Automatisierungen, Regeln und Prozesse innerhalb eines 
    Firmenbüros zu koordinieren. Der Fokus liegt dabei verstärkt auf der einfachen Nutzung des Frameworks für den Anwender und 
    die uneingeschränkte und schnell umzusetzende Ausprägungsvielfalt von Regelprozessen. Dadurch kann der Anwender die vorgegebene Struktur nutzen, um individuelle 
    Sachverhalte zu realisieren, die in seinem Umfeld abzudecken sind. Im Rahmen dieser Arbeit werden verstärkt Anwendungsfälle mit einem Service-Roboter 
    umgesetzt.
    %\\
    Hierfür wird der allgemeine Aufbau der Architektur skizziert und demonstriert, wie eine solche Lösung aussehen kann. 
    Unter Berücksichtigung der Forschungsfrage (siehe Abschnitt \ref{sec:forschungsfragen}) wird eine Möglichkeit offengelegt, mit der 
    ein Softwareentwickler neue Regeln entwickeln und dem System hinzufügen kann, ohne ein weiteres zu erlernendes Framework zu verwenden. 
    Dabei sollen die notwendigen Schritte und Interaktionen formalisiert und für den Entwickler vereinfacht werden. 
    %\\
    %\linebreak
    %In folgender Darlegung wird nochmals konkreter auf den Kontext als auch auf die Intension der Arbeit eingegangen. 
    % Das Ziel des Konzeptes ist es, dem Entwickler den Aufwand zur Erweiterung des Systems zu minimieren durch weitere 
    % Regeln und Abdeckung von Anwendungsfällen (Use Cases) und eine Struktur vorgeben. (ToDo's, Flexibilität in der 
    % Umsetzung (nicht wie Home Assistant und openHAB eher eingeschränkt)) 

%\section{Abzudeckende Funktionalitäten}
%\label{sec:konzeptfunktionalitaet}
    % Was soll der Entwickler machen können? 
    % Welche Grundlagen braucht er, um eine Regel implementieren zu können?
    % Welche Funktionen müssen gegeben sein, um die Struktur vorzugeben? 
    % Reicht ein Hinweis weöche Stellen angepackt werden müssen, um eine Regel hinzuzufügen? 
    
    %%%%%%%%%%%%%%%%%%%%%%%%%%%%%%%%%%%%%%%%%%%%%%%%%%%%%%%%%%%%%

    % ZIEL DES KONZEPTES: Ein Framework für Entwickler bereitzustellen, welches die Mächtigkeit für den Entwickler offen lässt, nicht einschränkt 
    % und dennoch Konfiguration und Ausführung umsetzt. Der Entwickler muss lediglich den Zustandsraum, die MQTT-Topics und die Regeln definieren.
    % Der Entwickler bekommt ein Framework an die Hand, welches die Umsetzung von Prozessen in einem smarten Büro ermöglicht. Das Framework kümmert sich um die 
    % Organisation und die Ausführung der Regeln. Die Richtigkeit der Regeln und des Zustandsraumes muss der Entwickler sicherstellen. 
    % Die Kommunikation über MQTT ist nur eine Möglichkeit. Das Setup wird wegabstrahiert 

    %%%%%%%%%%%%%%%%%%%%%%%%%%%%%%%%%%%%%%%%%%%%%%%%%%%%%%%%%%%%%

\section{Anwendungsumfeld}
\label{sec:anwendungsumfeld}
    Grundsätzlich ist der Einsatzort des Frameworks variabel, da die eigentliche Implementierung und Nutzung der Regeln und Prozesse 
    abhängig vom Anwender und dessen Umfeld sind. Dadurch kann sowohl im privaten \acl{SH} Umfeld als auch in Büroräumen ein System mithilfe des 
    Frameworks aufgebaut werden. Basierend auf den vorangestellten Tätigkeiten, darunter die Anforderungsanalyse, und der Eingrenzung auf den 
    Einsatz im Smart Office liegt darauf der Fokus. % Schwerpunkt der Arbeit auf dem Einsatz in einem smarten Büro. 
    \\
    \linebreak
    Stützend auf den vorab ermittelten Anwendungsfällen (siehe Abschnitt \ref{subsec:checkin} und \ref{subsec:evacuation}) sind unabhängige 
    Komponenten, darunter bspw. ein Service-Roboter und weitere einsatzfähige Geräte, sowie ein \acs{MQTT}-Broker notwendig. Ein \acs{MQTT}-Broker 
    wird im Zuge der Konzeption von dem Framework selbst nicht bereitgestellt, lediglich die Client-Anbindung wird gegeben. Der Anwender muss sich selbstständig um den verfügbaren Broker kümmern. 
    Dadurch kann die Anforderung einer Kommunikation mittels \acs{MQTT} umgesetzt werden. Eine denkbare infrastrukturelle 
    Architektur kann wie folgt aussehen: 
    \begin{figure}[hbt!]
        \centering
        \includegraphics[width=14cm,height=11cm,keepaspectratio]{images/Systemarchitektur.png}
        \caption{Infrastruktur des Anwendungsumfeldes der Steuerzentrale}
        \label{fig:infrastructure}
    \end{figure}

\section{Konzept}
\label{sec:concept}
    Die Intension, die hinter der Ausarbeitung dieses Konzeptes steht, ist zum einen die einfache Handhabung der 
    formalisierten Interaktionen für Softwareentwickler %während
    unter der Verwendung des Frameworks und zum anderen die 
    offene Gestaltung von Regelprozessen. Dadurch ist der Anwender bei der Implementierung von Regeln, Aufgaben und Automatisierungen 
    für ein intelligentes Büro nicht eingeschränkt und kann mit der Regeldefinition flexibel variieren.
    %einzugrenzen zu beschränken. 
    Es soll lediglich ein Muster vorgegeben werden, damit Regeln einheitlich als solche von dem Framework 
    verarbeitet und genutzt werden können. 
    \\ 
    \linebreak
    Bei vergleichbaren Softwareprodukten, die im Rahmen dieser Arbeit erläutert 
    wurden (siehe Kapitel \ref{sec:homeassistant} und \ref{sec:openhab}), ist die Vielfalt der Regelausprägung auf 
    den Kontext des Systems eingeschränkt und benötigt mehrere Schritte, um sie zu realisieren. Dies bedeutet, dass 
    Regeln und Prozesse nur mit Komponenten und Informationen innerhalb des Systems arbeiten können, bzw. benötigte 
    Informationen erst durch eine systemseitige Erweiterung durch Plugins verfügbar sind, auf die der Nutzer keinen direkten 
    Einfluss nehmen kann. Mit den bestehenden Lösungen wird versucht, die Regeldefinition für Endnutzer so 
    einfach wie möglich zu gestalten, wodurch eventuell erst zahlreiche Selektionen über die Benutzeroberfläche zum gewünschten Ziel führen.   
    \\
    \linebreak
    Um dennoch dem Anwender eine Struktur vorzugeben, mit der Regeln definiert und zur Laufzeit der Anwendung ausgeführt 
    werden können, soll mit diesem Konzept ein Framework zum Lösen dieser Herausforderungen erarbeitet werden. Hierfür soll 
    der Anwender mit der Komplexität der Regelverwaltung und deren Durchführung nicht konfrontiert werden. Dieser ist lediglich 
    in der Verantwortung, die für ihn notwendigen Regeln und Prozesse zu definieren und dem Framework bereitzustellen. 
    Dadurch soll dem Framework-Verwender die Möglichkeit geboten werden, das zur Verfügung gestellte System 
    mit individuellen Regeln, dafür vorgesehenen Bedingungen, Komponenten und deren Zustände zu implementieren. Beispielsweise 
    können bei Regeldefinitionen Informationen direkt von Datenpunkten über \acs{HTTP}-Abfragen bezogen werden, bzw. Ressourcen 
    und Inhalte individuell nachgezogen werden.
    \\ 
    Das Konzept beinhaltet die in folgendem Abschnitt dargelegten grundsätzlichen Komponenten. %Aus denen bildet sich der grundlegende Aufbau 
    %des Frameworks.

    \subsection{Konzeptkomponenten}
    \label{subsec:conceptcomps}
        Für die Erläuterung des Konzeptes werden vorab die einzelnen Konzeptkomponenten dargelegt, da diese die Anhaltspunkte 
        für den weiteren Verlauf und den Kern des Frameworks darstellen: % . Das Framework baut auf folgenden Bausteinen auf:
        \begin{itemize} 
            \item Regel (Rule): Eine Regel ist ein Konstrukt, welches bei bestimmten Events die darin enthaltenen Aktionen ausführen soll. 
            Der grobe Aufbau einer Regel ist immer gleich. Diese beinhaltet einen Auslöser, eine Bedingung, einen Prozess und einen eindeutigen 
            Namen. Die Inhalte der Regel kann der Anwender ja nach Bedarf beliebig ausprägen und ergänzen. 
            \item Komponente (Component): Eine Komponente soll einen reellen Gegenstand abbilden, der bei Benutzung durch eine Regel unter anderem 
            für weitere gesperrt und danach freigegeben werden kann. Diese Komponente kann ebenso sämtliche Zustände beinhalten und als Objekt in den Zustandsraum aufgenommen werden.
            \item Zustandsraum (State): Der Zustandsraum, das sog. Zustandsobjekt, soll alle Zustände von Komponenten und weiteren Geräten abbilden. Dieser repräsentiert die Zustände der Realität.
            \item Auslöser (Trigger): Ein Auslöser löst im Allgemeinen eine Zustandsänderung aus. Im Kontext des Frameworks gibt es zwei Ausprägungen von 
            Auslösern. Zum einen werden eingehende Aktionen entgegengenommen und über die Transformation zu einer Zustandsänderung verarbeitet, zum anderen können durch Regelprozesse 
            wiederum weitere Schritte und Zustandsänderungen ausgelöst werden. Dadurch werden ausgehende Aktionen gesteuert. 
            %Der Auslöser ist unter anderem ein von außen einwirkendes Objekt, das Zustandsänderungen hervorruft. Beispielsweise ist ein 
            %Auslöser das eintreffen eines Events über eine Kommunikationsschnittstelle. Ebenso können durch Regelprozesse Events ausgelöst werden. 
            \item Transformation (Transformer): Mit der Transformation werden eingehenden Zustandsänderungen, die durch ein Event oder Auslöser ausgelöst werden, 
            auf das eigentliche Zustandsobjekt übertragen. 
            %, die eine Zustandsänderung bewirken, die durch einen Auslöser hervorgerufen 
            %werden und eine Zustandsänderung bewirken
        \end{itemize}
        %Diese Komponenten bilden den Kern des Frameworks.
        %In folgendem Abschnitt wird nochmals auf die Sichten eingegangen und anhand denen der Ablauf des Frameworks als auch die 
        %Aufgaben des Anwenders erläutert. %Der Ablauf des Prozesses von Auslöser bis zur Ausführung der Regel wird anhand eines Programmablaufdiagramms gestützt.
        %\\
        %\linebreak
        Das Konzept wird aus zweierlei Sichtweisen betrachtet, die in folgendem Abschnitt aufgegriffen werden und für die weitere 
        Konzepterläuterung notwendig sind.
    
    \subsection{Sichtweisen}
    \label{subsec:sichtweisen}
        Wie bereits aus dem Kontext hervorgeht, wird das Konzept auf zwei Sichtweisen aufgeteilt. Zum einen auf die 
        Anwendersicht, die der Softwareentwickler als entscheidende Kraft einnimmt, indem dieser das Aufsetzen und das in 
        Betrieb nehmen des Systems, sowie das Definieren und Implementieren von Regeln übernimmt. Zum anderen die 
        Bereitstellungssicht, die das Framework als ausführende Kraft besetzt. Dieses sorgt für die Ausführung der vom 
        Anwender definierten Regeln. Ebenso stellt es Funktionen bereit, die das Empfangen von Events und das starten 
        von Regeln ermöglicht. Das Management zum Starten von Regeln und Prozessen wird in Gänze vom Framework übernommen. 
        Das Konzept widmet sich grundlegend dem initialen Aufbau des Frameworks. %Zum besseren Verwalten und Starten von 
        %Prozessen können weitere Konzepte und Managementroutinen ergänzt werden. Diese sind nicht Teil dieses Konzeptes und 
        %werden im Ausblick zu möglichen Erweiterungsschritten nochmals aufgegriffen.
        \\
        \linebreak
        Zusammenfassend wird zwischen den beiden Sichtweisen differenziert, da diese Trennung ein elementarer Schnitt des Konzeptes 
        aufzeigt. Die Sichten werden konkretisiert, indem die Aufgaben des Anwenders als auch der Ablauf innerhalb des Frameworks 
        andeutungsweise erläutert wird.
        % dafür, dass für das Setup benötigte Funktionen bereitstehen und vom Anwender definierte Regeln 
        %ausgeführt werden. 
        \\
        \linebreak
        Der Anwender hat das initiale Setup des Frameworks zur Aufgabe. Zuerst sollte der Zustandsraum erstellt werden. 
        Darin sind alle notwendigen Zustände als Attribute und Objektfelder abzubilden. Diese können konkret Geräte sein, die im Rahmen des 
        intelligenten Büros verwendet werden, bspw. ein Service-Roboter, ein Türöffner und weitere steuerbare Geräte, die 
        ihren Zustand ändern können. Nachdem das Zustandsobjekt 
        definiert wurde, geht es an die Definition und Implementierung der Regeln. Diese sind individuell, je nach 
        Anforderungen des Anwenders zu erstellen. Jedoch sind bestimmte Funktionen und Vorgaben bezüglich Bedingungsprüfung 
        und Regeldurchlauf einzuhalten. Anforderungen dabei sind das Implementieren von drei unabdingbaren Funktionen. 
        Die erste Funktion ist die Zuordnung des Auslösers, damit klar wird, durch welches Event eine Regel ausgelöst werden kann. 
        Die zweite zu implementierende Funktion ist die Prüfung von Werten im Zustandsraum, sodass beim eintreffen spezifischer 
        Bedingungen und Zuständen bestimmte Regelprozesse ausgeführt werden können. Die dritte vorgegebene Funktion ist die 
        des eigentlichen Regeldurchlaufes. Die darin enthaltenen Aufgaben werden nach zutreffender Bedingung abgearbeitet.
        \\
        Nach Fertigstellung einer Regel, muss diese dem System übergeben werden. Anschließend ist die 
        konkrete Transformation zu definieren. Darüber wird festgelegt, welches eingehende Event, bzw. \acs{MQTT}-Topics 
        eine bestimmte Zustandsänderung bewirkt, da erst nach der Änderung des Zustandes die Überprüfung und Ausführung 
        von Regeln ausgelöst wird. Das konkrete Konstrukt der Transformation wird im Kapitel der Umsetzung (siehe Abschnitt \ref{subsec:transformation}) 
        erläutert. Abschließend sind die Kommunikationswege zu aktivieren. 
        Da \acs{MQTT} zum aktuellen Zeitpunkt und zur Abdeckung der Anforderungen die einzige Kommunikationsschnittstelle 
        abbildet, ist die Konfiguration des \acs{MQTT}-Clients einzustellen. Hierfür muss der Entwickler 
        dem Framework den Host des \acs{MQTT}-Brokers, sowie den Nutzername des Clients und dessen Passwort übergeben. Über das 
        Framework wird dann das Setup durchgeführt, sodass Topics (Themen), die über den Broker veröffentlicht werden, konsumierbar 
        sind. In Zuge dessen muss der Anwender die Topics innerhalb der jeweiligen Transformation definieren. Dadurch wird gewährleistet, dass 
        die Steuerzentrale nur auf die Topics hört, die der Anwender bekannt gegeben hat. Durch die Transformation erfolgt ebenso die 
        Zuordnung, welches Zustandsattribute mittels eines bestimmten Topics geändert werden soll. Hierfür ein Beispiel:  
        \\
        \linebreak
        Sobald eine Person an der Tür authentifiziert wurde, wird ein Thema mit dem Namen derjenigen Person übergeben. Basierend auf 
        dem eingehenden Topic wird dann für dieses Ereignis der Wert im Zustandsraum auf den Namen der Person geändert. Somit kann 
        die Steuerzentrale mit dem neuen Wert und der Änderung des Zustandsraumes arbeiten. Durch die Zustandsänderung wird dann 
        geprüft, ob eine Regel für diese bestimmte Änderung definiert wurde. Trifft diese Bedingung zu, so wird der dazugehörige Regelprozess 
        durchlaufen.  
        \\
        \linebreak
        Sind diese Schritte abgearbeitet, so sind die Pflichten des Anwenders erfüllt und die Steuerzentrale kann gestartet werden. 
        \\
        Hierfür wird der Durchlauf von dem Eingang einer \acs{MQTT}-Nachricht bis zur Ausführung einer Regel aus Sicht des Frameworks 
        erläutert.
        \\
        \linebreak
        Nach erfolgreichem Einstellen und Starten des Systems soll der \acs{MQTT}-Client hochgefahren werden. Dieser baut daraufhin eine 
        Verbindung zu dem \acs{MQTT}-Broker auf und startet das Zuhören auf eingehende Themen. Dabei soll bereits auf die Topics,  
        die der Nutzer über die Einstellungen im Rahmen der Transformation übergibt, gefiltert werden. Sobald eine der bekannten 
        \acs{MQTT}-Nachrichten konsumiert wird, soll mittels des Transformers der Zustandsraum, bzw. das Zustandsobjekt 
        auf die in dem Topic enthaltenen Informationen abgeändert werden. Der Transformer stellt den Ausgangspunkt für die Transformation 
        des Inhaltes der Nachricht zur Änderung des Zustandsobjektes dar. Innerhalb dieser Komponente findet die Zuordnung 
        des Topics zu den darauf adressierten Feldern des Zustandsobjektes statt. Mittels der Nachricht sollen die Informationen dieser 
        in den Attributwert der durch den Entwickler spezifizierten Variable übertragen werden. Die Änderung des 
        Zustandsraumes löst darauf hin den weiteren Ablauf aus. 
        \\
        Das aktuelle Zustandsobjekt wird zum Zeitpunkt der Überschreibung für weitere Lese- und Schreiboperationen gesperrt, sodass 
        die Gültigkeit des Zustandsraumes nicht erlischt und nach wie vor die realen Zustände wiedergibt. Anschließend erfolgt die 
        Erstellung einer Kopie des Zustandsobjektes, mit der 
        darauf folgend weiter verfahren wird. Nachdem das Zustandsobjekt geklont wurde, wird der Sperrvorgang aufgehoben. 
        Nachdem der Lock-Mechanismus entsperrt wurde, wird über alle Regeln, die dem Framework bekannt sind, iteriert. In diesem Prozess 
        wird jeweils nochmals eine Kopie des Zustandsobjektes erzeugt, mit dem bei der Iteration die Regel-Bedingungen überprüft werden. 
        Bei zutreffender Bedingung wird die entsprechende Regel der aktuellen Iteration durchgeführt. Zur Abarbeitung mehrerer Prozesse 
        gleichzeitig, soll durch Asynchronität die Bedingungsprüfung und Durchführung der Regel in einen separaten Thread ausgelagert werden. 
        %Für die Zeitspanne der Zustandsänderung und der 
        %nachfolgenden Kopie soll das Zustandsobjekt kurzzeitig für weitere Transaktionen und Änderungen, genauer gesagt für Lese- und 
        %Schreiboperationen durch einen Lock-Mechanismus gesperrt werden. Dadurch kann anschließend das Zustandsobjekt mit der vorher 
        %getätigten Änderung kopiert werden, um die anschließende Regel-Iteration durchzuführen. 
        Wichtig an dieser Stelle zu 
        erwähnen ist, dass die Prüfung der Regel-Bedingungen mit der Kopie durchgeführt werden soll.
        Durch den Sperrvorgang wird gewährleistet, dass eine Zustandsänderung sauber durchgeführt wird und darauf hin eine Regel mit der 
        Kopie des Zustandsobjektes ausgelöst werden kann. Hierdurch ist sichergestellt, dass jeweils eine Zustandsänderung zu einem 
        Zeitpunkt stattfindet und zu dieser die dafür definierte Regel ausgeführt wird, sofern die Bedingung im Vorfeld ebenso zutrifft. 
        \\
        Wird kurz darauf eine weitere Nachricht konsumiert, so kann das aktuelle Zustandsobjekt dahingehend überschrieben und erneut 
        kopiert werden. Die Kopie wird wiederum für die Regel-Iteration verwendet und tangiert die vorherige Kopie nicht. Somit soll 
        zu jedem Zeitpunkt der aktuelle Zustandsraum repräsentiert werden. Davon ausgehend entspricht dieser immer der Wahrheit.
        \\
        \linebreak
        Wird eine Regel durchgeführt, so wird der vom Anwender definierte Prozess abgearbeitet. Über Regelprozesse selbst kann wiederum eine 
        Zustandsänderung erfolgen, die daraufhin eine weitere Regel ausführt. Dies soll durch die inverse Funktion der Transformation erfolgen, indem der 
        Anwender in dem Regelprozess lediglich die Zustandsänderung beschreibt. Das Framework nutzt auf dessen Grundlage die vorab durch die Transformation 
        übergebenen Informationen, damit zum Zeitpunkt der Regelimplementierung dem Anwender nicht zwangsweise das Topic für die Kommunikation über \acs{MQTT} 
        bekannt sein muss. Die Kenntnis über das Topic und die darauf zugewiesene Zustandsänderung ist nur bei initialer Definition erforderlich. 
        \pagebreak
        \\
        \linebreak
        Dem folgenden Diagramm ist der abstrakte Programmablauf des Frameworks, welcher soeben skizziert wurde, zu entnehmen:
        \begin{figure}[hbt!]
            \centering
            \includegraphics[width=14cm,height=11cm,keepaspectratio]{images/Programmablauf_Framework.png}
            \caption{Grober Programmablauf des Frameworks}
            \label{fig:programmablauf_framework}
        \end{figure}
        \\
        Aspekte, die nicht aus dem Ablaufdiagramm hervorgehen, sind zum einen die parallele Ausführung von zwei voneinander unabhängigen 
        Zustandsänderungen und zum anderen die erneute Änderung des Zustandes durch eine Regel selbst. Sofern die Änderungen unabhängig 
        sind, sollen Prozesse auch asynchron ablaufen, bzw. Regeln auch neue Zustandsänderungen hervorrufen können. Die genauere Ausprägung des 
        Konzeptes wird im Rahmen des Architekturkonzeptes (\ref{sec:architekturkonzept}), sowie dem Kapitel der Umsetzung 
        (\ref{chap:umsetzung}) aufgegriffen. 
        \\
        \linebreak
        Zur Veranschaulichung der abgeschlossenen textuellen Erläuterung des Konzeptes hilft die nachfolgende Abbildung: 
        
\begin{landscape}
  \AddToShipoutPictureBG*{%
  \AtPageLowerLeft{%
    \raisebox{\dimexpr.5\paperheight-.5\height}{%
      \makebox[.9\paperwidth][r]{%
        \rotatebox{90}
        {\thepage}
      }% \makebox
    }% \raisebox
  }% \AtPageCenter
}% \AddToShipoutPictureBG*
  \includepdf[scale=0.75, pages=-, angle=90, fitpaper=true, ]{images/konzept_MA.pdf}
\end{landscape}
\section{Architekturkonzept}
\label{sec:architekturkonzept}
    Das Framework stellt die Kern-Funktionalität zur Steuerung von implementierten Regeln und Prozessen innerhalb eines 
    Gebäudes, bspw. eines Büroraumes, dar. Mit dem Architekturkonzept wird der grundlegende Aufbau, sowie die Funktionalität des Frameworks erläutert. 
    Potentielle Erweiterungen und Adaptionen werden im Ausblick (\ref{chap:ausblick}) aufgegriffen.
    \\ 
    \linebreak
    Das System soll in drei Schichten aufgeteilt werden. Die oberste Schicht stell die Kommunikationsschicht dar, die mittlere Ebene 
    handelt um die Logik- und Prozessschicht und der unterste Block repräsentiert die Persistenzschicht. Im Rahmen der Arbeit wird 
    die dritte Schicht nicht detailliert beschrieben, da diese zu aktuellem Zeitpunkt keinen Nutzen generiert, bzw. für keine weitere Verarbeitung 
    genutzt wird. Die Ausprägung der Schicht ist dennoch ohne weiteres möglich und wird auch grob skizziert. Der Abbildung 
    (\ref{fig:schichtenarchitektur}) ist die Aufteilung der Schichtenarchitektur zu entnehmen. In der Darstellung unterscheidet sich die 
    Persistenzschicht von den anderen, um erkenntlich zu machen, dass diese in aktuellem Konzept keine konkrete Implementierung erfährt. 
    Durch die Darstellung in übergreifender Form wird deutlich, das bereits auf dieser Ebene die Trennung der Zuständigkeiten, engl. 
    \textit{seperation of concerns}, greift. Durch die gezielte Abstraktion von Komponenten und Informationen kann die Komplexität 
    eines Systems gesteuert werden. Die Kommunikationsschicht ermöglicht die Anbindung von verschiedensten Kommunikationsprotokollen, die 
    dadurch schnell adaptiert werden können. Im Rahmen des Konzeptes wird ausschließlich Gebrauch des \acs{MQTT}-Protokolls gemacht. 
    Die Logikschicht nimmt alle eingehenden Events, die jeweils eine Zustandsänderung erzwingen, der Kommunikationsschicht entgegen 
    und durchläuft den Prozess des Frameworks, um auf die Zustandsänderung die passende Regel auszuführen. Die 
    Persistenzschicht ist für die Speicherung erzeugter Daten zuständig, bspw. wenn eine Transaktionshistorie von Zustandsänderungen 
    persistiert werden soll. Ebenso können Informationen gespeichert werden, die anderweitig zur Verfügung stünden. Das Vorantreiben der Speicherung 
    von Informationen kann durch den agilen Entwicklungsprozess außerhalb des Thesis-Rahmens erfolgen. 
    \begin{figure}[hbt!]
        \centering
        \includegraphics[width=14cm,height=10cm,keepaspectratio]{images/Schichtenarchitektur.png}
        \caption{Schichtenarchitektur}
        \label{fig:schichtenarchitektur}
    \end{figure}
    \\
    %\pagebreak
    Für die Konzeption von wartbarer und erweiterbarer Software, wird von den \textit{SOLID}-Prinzipien, sowie von nützlichen 
    und für das Framework verwendbaren Programmiermustern Gebrauch gemacht. Diese werden an geeigneter Stelle und auf den Nutzen bezogen erläutert.
    \\
    \linebreak
    Das \textit{SOLID} wurde von Robert C. Martin\footnote{Software Engineer, Lehrer und Autor Robert C. Martin. \url{https://en.wikipedia.org/wiki/Robert_C._Martin} Besucht am 10.07.2022} 
    geprägt und soll zum Design guter objektorientierter Software beitragen. Dadurch soll 
    ein Softwaresystem einer höheren Wartbarkeit unterliegen. Es setzt sich aus fünf Prinzipien zusammen, die wie folgt definiert sind.
    \begin{itemize}
        \item S - Single-Responsibility-Prinzip: „Das Single-Responsibility-Prinzip (SRP; Eine-Verantwortlichkeit-Prinzip; ...) fordert, dass eine Klasse oder ein Modul einen und nur einen \textit{Grund zur Änderung} haben sollte.“ \cite{cleancode2009}
        \item O - Open-Closed-Prinzip: „Module sollten sowohl offen (für Erweiterungen) als auch verschlossen (für Modifikation) sein.“ \cite{ocpmeyer1988}
        \item L - Liskovsches Substitutionsprinzip: „... Sei q(x) eine beweisbare Eigenschaft von Objekten x des Typs T. Dann soll q(y) für Objektes des Typs S wahr sein, wobei S ein Untertyp von T ist.“ \cite{liskov2001behavioral} - Das (Ersetzungs-)Prinzip sagt somit aus, dass eine 
        Subklasse stets die Eigenschaften der Superklasse erfüllt und immer als Objekt der Superklasse anwendbar sein muss. Eine Subklasse darf dadurch erweitert werden, aber 
        keine grundlegende Änderung erfahren. 
        \item I - Interface-Segregation-Prinzip: „Clients sollen nicht dazu gezwungen werden, von Interfaces abzuhängen, die sie nicht verwenden.“ \cite{martin1996interface}
        \item D - Dependency-Inversion-Prinzip: „A. Module hoher Ebenen sollten nicht von Modulen niedriger Ebenen abhängen. Beide sollten von Abstraktionen abhängen. B. Abstraktionen sollten nicht von Details abhängen. Details sollten von Abstraktionen abhängen.“ \cite{martin2003agile}
    \end{itemize}
    Diese Prinzipien werden bei der Konzeption des Frameworks berücksichtigt. Da es sich bei diesem System nicht um eine klassische 
    Webanwendung handelt und keine direkte Kommunikation unter anderem über \acs{REST} \acs{API}s stattfindet, wird ein allgemeines 
    Programmiermuster, wie bspw. das \ac{MVC}\footnote{Architekturmuster für Webapplikationen. \url{https://de.wikipedia.org/wiki/Model_View_Controller} Besucht am 10.07.2022} 
    oder das \ac{MVVM}\footnote{Software-Architekturmuster. \url{https://en.wikipedia.org/wiki/Model–view–viewmodel} Besucht am 10.02.2022 }, nicht 
    in Betracht gezogen. Verwendet werden allgemeine Architekturmuster und -prinzipien, die dazu beitragen die Software wartbar 
    zu gestalten, die Komplexität einzelner Funktionen zu reduzieren und die Lesbarkeit und Überschaubarkeit zu erhöhen. 
    
    \subsection{Aufbau des Frameworks}
        Nach Veranschaulichung der Schichten-Modellierung wird übergreifend der Komponenten Aufbau, sowie die dafür einzusetzenden Architekturmuster aufgegriffen. 
        \\
        Die Konzeptkomponenten (\ref{subsec:conceptcomps}) finden sich im umfassenden Architekturschaubild (siehe Abbildung \ref{fig:patternarchitektur}) wieder. 
        Die Regelstruktur 
        soll die Funktion einer Bedingungsprüfung und der darauffolgenden Aktion vorgeben, damit bestimmte Anhaltspunkte geschaffen sind, die das Framework überprüfen 
        und anwenden kann. Um zu gewährleisten, dass dem Anwender diese Methoden bei Implementierung einer spezifischen Regel zur Verfügung stehen, wird mittels dem 
        \textit{Template Method} Pattern \cite{gamma1995template} ein Leitfaden vorgegeben. Dieses gehört zu den Verhaltensmustern, engl. behavioral patterns. 
        Ziel des Schablonenmethoden-Entwurfsmusters ist die Delegation der konkreten Ausformung einzelner Methoden und Funktionen zu deren Unterklassen. Innerhalb der 
        abstrakten Klasse wird das Skelett des Objektes definiert. Die einzelnen Schritte können so in den darunter liegenden Klassen überschrieben oder ergänzt 
        werden, ohne dass die grundlegende Struktur der übergreifenden Klasse modifiziert werden muss. Die Unterklassen, die konkrete Definition einer Regel, sollen 
        unter Verwendung des \textit{Singleton} Patterns \cite{gamma1995singleton} realisiert werden. Das zu den Erzeugungsmustern 
        zugehörige Pattern dient zur 
        Sicherstellung, dass eine Klasse, in dem Fall die einzelne Regel, zur Laufzeit ausschließlich eine einzige Instanz erzeugt und diese global erreichbar ist. 
        Dadurch wird von jeder Regel immer nur ein Objekt erzeugt. 
        \\
        \linebreak
        Ähnlich zu dem Aufbau des Regelwerkes, sollen die Komponenten, die reale Gegenstände abbilden, ebenso Gebrauch des Schablonenmethoden-Entwurfsmusters machen. 
        Unterschied zu der Regel-Schablone ist lediglich die Funktion der Sperrung einer Komponenten für weitere Aufgaben. Diese soll bereits in der abstrakten Klasse 
        definiert sein. Konkrete Komponenten-Eigenschaften und Zustandsattribute können von dem Anwender selbst implementiert werden.
        \\
        \linebreak
        Die durch den Anwender implementierte Transformation, die unter anderem basierend auf einem eingehenden \acs{MQTT}-Topic das Zustandsobjekt verändert, soll %als Vorgabe 
        %stützend 
        das {Template Method} Pattern verwenden. Dadurch ist die Struktur der Transformation vorgegeben, die bei der Definition eines neuen Transformationsobjektes die zu 
        implementierenden Funktionen bereitstellt. %anhand dessen der Anwender neue Transformationsobjekte 
        %je nach Anwendungsfall erzeugen kann. 
        Innerhalb dieser Schablone sollen alle Attribute vorgegeben werden, die für die Transformation notwendig sind und von dem Softwareentwickler implementiert werden sollen. 
        \\
        \linebreak
        Damit zukünftig weitere Kommunikationsschnittstellen hinzugefügt werden können, soll für die Transformation zusätzlich das 
        \textit{Adapter} Pattern verwenden werden. Dadurch können Events aus unterschiedlichen Services und über mehrere Kommunikationswege 
        entgegengenommen werden, ohne dass die eigentliche Schnittstelle modifiziert werden muss. Das \textit{Adapter} Pattern gehört 
        zu den Strukturmustern und ist wie folgt definiert: 
        Die Konvertierung einer Schnittstelle einer Klasse in eine andere Schnittstelle, die vom Client erwartet wird. Dadurch ist die Zusammenarbeit von Klassen 
        gewährleistet, die sonst aufgrund der inkompatiblen Schnittstelle nicht zusammenarbeiten könnten. \cite{gamma1995adapter}
        \pagebreak
        \\
        Das Zustandsobjekt ist aus Anwendersicht als ein einfaches \ac{POJO}, ein ganz normales Java Objekt, zu implementieren. Das Framework soll sich um die korrekte Nutzung 
        des Objektes kümmern. 
        \\
        \linebreak
        Der übergreifende Aufbau unter Verwendung von Entwurfsmustern ist folgender Abbildung zu entnehmen:
        \begin{figure}[hbt!]
            \centering
            \includegraphics[width=14cm,height=10cm,keepaspectratio]{images/final_architecture_with_patterns.png}
            \caption{Architektur mit verwendeten Entwurfsmustern}
            \label{fig:patternarchitektur}
        \end{figure}
        \\
        Der soeben grob skizzierte Aufbau wird in den folgenden Abschnitten konkretisiert.

    % ALLGEMEIN MIT PATTERNS BESCHREIBEN (KERN-KOMPONENTEN UND DARAUF ANGEWENDETE PATTERN)
    \subsection{Kommunikationsschicht}
    Damit das Framework einer hohen Wartbarkeit und Erweiterbarkeit unterliegt, soll bereits in der Kommunikationsebene mithilfe 
    des genannten \textit{Adapter} Musters eine Abstraktionseben geschaffen werden. Diese erleichtert das Adaptieren 
    hinzukommender Kommunikationsschnittstellen. Im Rahmen der Arbeit und unter Verwendung des \acs{MQTT}-Kommunikationsprotokolls kann die 
    Konzeption der Kommunikationsschicht mithilfe des \textit{Adapter} Patterns dem Klassendiagramm der Abbildung (\ref{fig:patternarchitektur}) entnommen werden. 
    Der Client, in dem Fall der \acs{MQTT}-Topic-Subscriber, wird mit einem Transformationsadapter versehen. Hierbei werden die 
    Informationen, die über den Subscriber generiert werden, über eine Funktion an das Framework übergeben, sodass dieses mit den Eingangsinformationen arbeiten kann. 
    Wird eine weitere Kommunikationsschnittstelle hinzugefügt, so muss lediglich ein dafür vorgesehener \textit{Adapter} implementiert werden, der 
    alle für das Framework notwendigen Informationen übergibt. So bleibt die interne Arbeitsweise des Frameworks unverändert.
    \begin{figure}[hbt!]
        \centering
        \includegraphics[width=14cm,height=10cm,keepaspectratio]{images/Kommunikationsschicht_final.png}
        \caption{Klassendiagramm der Kommunikationsschicht}
        \label{fig:patternkommunikation}
    \end{figure}
    \pagebreak
    %\\
    Der interne Ablauf einer Nachrichtenkonsumierung kann dem Sequenzdiagramm in Abbildung (\ref{fig:kommunikationsequenz}) entnommen werden. Mit der Transformation 
    wird das eingehende \acs{MQTT}-Topic mit den vom Anwender definierten Topics überprüft. Gibt es dabei eine Übereinstimmung, so wird anhand des Topics die mitgelieferte 
    Nachricht konsumiert und dem jeweiligen Wert im Zustandsraum zugeordnet und dementsprechend verändert. Mit der Zustandsänderung wird die 
    Prozessschicht des Frameworks aufgerufen, die daraufhin die dem Framework bekannten Regeln überprüft und zutreffende startet. Der Durchlauf wird in dem Abschnitt (\ref{subsec:logikschicht}) 
    explizit dargestellt. 
    \begin{figure}[hbt!]
        \centering
        \includegraphics[width=14cm,height=10cm,keepaspectratio]{images/Kommunikationsschicht_Sequenz_Final.png}
        \caption{Sequenzdiagramm zur Kommunikationsschicht}
        \label{fig:kommunikationsequenz}
    \end{figure}
    % NOCHMALS DIE SCHICHTEN MIT:
    % KOMMUNIKATIONSSCHICHT UND AUFBAU -> \subsection{}?
    % LOGIKSCHICHT UND AUFBAU -> \subsection{}?
    \subsection{Logik- und Prozessschicht}
    \label{subsec:logikschicht}
    Die Logik- und Prozessschicht wird aktiv, wenn durch eine bestimmte Kommunikationsschnittstelle, im Rahmen der Arbeit ausschließlich durch \acs{MQTT} realisiert, ein eingehendes 
    Event, bzw. eine eingehende Nachricht konsumiert und daraufhin das Zustandsobjekt geändert wurde. Die in erster Linie dafür benötigten Objekte sind dem 
    Klassendiagramm (\ref{fig:patternlogik}) zu entnehmen. Die an zentraler Stelle stehende Klasse des \textit{LogicHubState} übernimmt die Koordination 
    der Zustandsänderung, sowie das nachträgliche Prüfen der Regelbedingungen, sowie das Ausführen des Regelprozesses. 
    \\
    Darin wird die Kopie des Zustandsobjektes inklusive der Änderung vorgenommen werden, damit in folgenden Prozessen die Repräsentation des Objektes nicht verfälscht wird. 
    Somit wird verhindert, dass eine durch Regeln erzeugte Zustandsänderung das eigentliche Objekt direkt manipuliert. Dies ist eine getroffene Maßnahme, da die Regelprozesse asynchron abgearbeitet werden 
    und so mehrere Zustandsänderungen erfolgen könnten. Zusätzlich ist der Kopiervorgang zu sperren, dass keine Inkonsistenz des Zustandes entsteht. 
    Dieser Vorgang wird in der Umsetzung (siehe Kapitel \ref{chap:umsetzung}) erläutert.
    \begin{figure}[hbt!]
        \centering
        \includegraphics[width=14cm,height=10cm,keepaspectratio]{images/Logikschicht_final.png}
        \caption{Klassendiagramm zur Logik- und Prozessschicht}
        \label{fig:patternlogik}
    \end{figure}
    \\
    Der initiale Ablauf von der Erstellung einer Kopie, bis hin zur Prüfung der Regelbedingung, sowie das anschließende Durchführen gestaltet sich wie folgt: 
    \\
    \pagebreak
    \begin{figure}[hbt!]
        \centering
        \includegraphics[width=14cm,height=14cm,keepaspectratio]{images/Logikschicht_Sequenz_final.png}
        \caption{Sequenzdiagramm zur Logikschicht}
        \label{fig:logiksequenz}
    \end{figure}
    \\
    \linebreak
    Dem Sequenzdiagramm ist ebenso die Transformation zu entnehmen, die erzielt wird, wenn durch eine Regel eine erneute Zustandsänderung stattfindet, die bspw. Befehle über \acs{MQTT} 
    veröffentlicht, um Komponenten steuern zu können. Auf die konkrete Umsetzung der soeben architektonisch beschriebenen Komponenten wird in folgendem Kapitel (siehe Umsetzung \ref{chap:umsetzung}) 
    eingegangen.
    \\
    Die Ausprägung der Persistenzsschicht wurde im Rahmen dieser Arbeit nicht weiter detailliert, da zu aktuellem Zeitpunkt 
    kein Bedarf an der Speicherung von Historien oder Zustandsänderungen besteht. Dennoch wird an dieser Stelle erwähnt, dass die Einbindung einer 
    Datenbank, sowie die Modellierung der Datenobjekte und deren Speicherung ohne weiteres möglich ist und dem keine Grenzen gesetzt sind.  
    %\subsection*{Persistenzschicht}
    % EVTL.: PERSISTENZSCHICHT UND POTENTIELLER AUFBAU -> \subsection{}?
    % Es wird alles abgebildet über einen Zustandsraum, der sich aus den Dingen (Gegenständen) und Zuständen der Anwendung ergibt.
    % Der Zustandsraum wird verändert, wenn eine Aktion durchgeführt wird, bzw. durch eine Trigger angestoßen. 
    % Bzw. speichert den aktuellen Zustand des Gegenstandes 
    % (lightBulb = true/false, personOnDoor = null/Mikka, booking = stringBooking, temiAktive = true/false, 
    % temiPosition = stringKoordinates)
    % Zustandsraum muss von dem Entwickler definiert werden. 
    % MQTT Broker über Home Assistant, bzw. losgelöster Broker
    % Anbindung von APIs auch Entwickler-Sache. Kann ich das vereinfachen, sodass die Integration einfacher wird?
    %\subsection{Überlegungen, Anstöße und Herausforderungen}
    % Regeln über Thread abbilden? Ja, Nein? - Nein, wieso? Da Durch die MQTT Message mehrere Regeln ausgeführt 
    %werden können. -> Lediglich den Zustand der Komponenten locken.
    % KEIN THREAD (wird schon abgebildet durch die Services und die Auslöser durch MQTT), Falls eine Komponente 
    %  doppelt beansprucht wird, ist der Zustand der Komponenten zu locken und ein 
    % Thread.sleep einzurichten. Abfrage, ob der Wert, bzw. die Komponente wieder freigegeben wurde. 
    % Zustandsraum -> Abbildung aller notwendigen Komponenten 
    % Bei Bearbeitung einer Regeln die Komponenten Locken, sodass nur die einzelne Komponenten (deren Zustand) gelockt ist 
    % und nicht der ganze Zustandsraum, somit können mehrere Komponenten und Aktionen ausführen zu können. 
    %Was brauche ich für Funktionen und Werte in einer Regel?
    % Ein Zustandsraum (Objekt) für alles oder ein Globales, welches die die Komponenten enthält? - Begründung für die Auswahl.
    %\subsection{Schnittstellen}
        % Kommunikation mit API's je nach Use Case und Gebrauch zur Datenabfrage
    %\subsection*{Datenbanken}
    %   Anbindungen an Datenbanken sind je nach hinzukommendem Anwendungsfall ohne weiteres möglich. Dies ist eine Konzeptvertiefung, die 
    %    im Rahmen dieser Arbeit nicht konkretisiert wird, da der Fokus auf der Kernfunktionalität, sowie auf dem 
        % Datenbanken je nach Use Case und Gebrauch zur Datenabfrage
    %\subsection{Parallelität}
    %HERAUSFORDERUNG MIT DEM QUEUEING ANTEASERN.
%\subsection{Softwarekonzept}
%\label{subsec:softwarekonzept}
\\
%\linebreak
Für die Generierung mehrerer konkurrierender Sichten, wurde unter Verwendung des \textit{4+1 Sichtenmodells} von Philippe Kruchten \cite{Kruchten1995}
bereits bei der Anforderungsanalyse (\ref{chap:anforderungsanalyse}) die Szenarien beschrieben. Die im Modell dargestellten Sichten wurden mit 
Klassen- und Sequenzdiagrammen %-, Aktivitäts- und Komponentendiagrammen 
im Konzept veranschaulicht. Nach Beendigung der Konzeption wird in folgendem Kapitel die Umsetzung des Konzeptes und die 
Implementierung des Prototypen vertieft. 

\chapter{Umsetzung}
\label{chap:umsetzung}
\section{Implementierung}
\subsection{Aufbau der Architektur}
\subsection{Einbindung der Funktionen abgeleitet von der Konzeption}
\section{Ergebnis}
%\chapter{Ergebnis}
\label{chap:ergebnis}
\chapter{Evaluation}
\label{chap:evaluation}
In diesem Abschnitt wird das erzielte Ergebnis des erarbeiteten Konzeptes (siehe Kapitel \ref{chap:konzept}) und des 
Prototypen analysiert. Anhand von ausgewählten Forschungsmethoden werden die vorab identifizierten Anforderungen überprüft 
und evaluiert. Mithilfe der erhobenen Informationen wird zum Ende des Kapitels die Forschungsfrage sowie das
Ziel der Arbeit abschließend betrachtet. 

\section{Usability-Test}
\label{sec:usabilitytest}
    Im Rahmen der Evaluation wurde ein Usability-Test durchgeführt, um bestimmte, für die Nutzbarkeit aufgestellte Anforderungen zu 
    überprüfen.
    Das angestrebte Ziel und die Ergebnisse der Tests werden in dem folgenden Abschnitt erläutert:
    
    \subsection{Ziel des Usability-Tests}
        Anhand des Usability-Tests sind die Anforderungen zu prüfen, die den Bereich der Nutzbarkeit abdecken. 
        Hierfür werden die Anforderungen, die den Tabellen (\ref{tab:functionalRequirements}, \ref{tab:notfunctionalRequirements} und \ref{tab:furtherRequirements}) 
        zu entnehmen sind, betrachtet. Die dafür berücksichtigten Anforderungen sind konkret die Folgenden: NFA1, NFA2, NFA3 und NFA4 (siehe \ref{tab:notfunctionalRequirements}).
        \\
        Damit die Erfüllung der Anforderungen überprüfbar ist, wird ausgewählten Anwendern der Zielgruppe %im Rahmen der Nutzbarkeit 
        eine Aufgabe gestellt, mit der das Framework getestet wird und dabei die oben genannten Anforderungen genauestens betrachtet werden. 
        Ziel dabei ist die Überprüfung der Erfüllung der oben genannten Anforderungen an das System bzw. Verbesserungen zu eruieren. 

    \subsection{Durchführung des Usability-Tests}
        Für die Durchführung der Tests wurde vorab eine Aufgabe erstellt. Diese 
        dient als Grundlage zur Identifizierung der Usability-Anforderungen. Der Aufbau und das Vorgehen der Testung wurde für jeden Teilnehmer 
        identisch gestaltet. Vor Beginn wurde der Kontext sowie das Framework und dessen Kernkomponenten, die Rahmenbedingungen und die Aufgabenstellung selbst erläutert. 
        Die Aufgabe war so zu erstellen, dass sie umgebungsunabhängig erfüllt werden konnte. 
        Da die Tests ausschließlich auf die Nutzung des Frameworks abzielten, war die Umgebung sowie die 
        Anbindung verfügbarer Geräte zu vernachlässigen. Voraussetzung war die Verfügbarkeit eines \acs{MQTT}-Brokers. 
        Durch die fehlende Anbindung von reellen Geräten wurden die ausgehenden 
        \acs{MQTT}-Nachrichten durch eine Simulation, konkret die eines Service-Roboters, verarbeitet. Eingehende Nachrichten 
        wurden simuliert, indem \acs{MQTT}-Nachrichten manuell veröffentlicht wurden. 
        \\
        Die von den Anwendern durchzuführende Aufgabe ist wie folgt definiert: 
        \\
        \linebreak
        Es soll ein Anwendungsfall implementiert werden, bei dem ein Service-Roboter einen Mitarbeiter oder Gast, der an der Tür steht, 
        empfangen und begrüßen soll. Die Eingangsbedingung ist die Authentifizierung über eine Kamera. Der ganze Vorgang wird simuliert, indem die \acs{MQTT}-Nachricht manuell erzeugt und ausgelöst wird. 
        Auch die Definition des \acs{MQTT}-Topics ist vorgegeben. 
        Danach soll über die Steuerzentrale eine Regel ausgeführt werden, anhand derer der simulierte 
        Service-Roboter gesteuert wird. Dafür sind folgende Anforderungen zu erfüllen: 
        \begin{itemize}
            \item Nach eingehendem Topic soll der Service-Roboter angesteuert und an die Tür geschickt werden. 
            (Die Ansteuerung des Service-Roboters erfolgt ebenso über \acs{MQTT}. Da in den meisten Fällen kein Roboter verfügbar ist, wird auch 
            diese Kommunikation mittels \acs{MQTT} simuliert.)
            \item Ist der Roboter an der Tür, soll er die Begrüßung starten. (Die folgende Interaktion wird im Rahmen des Tests ebenso mittels der Simulation durchgeführt. 
            Wird die gesendete Nachricht über den Service-Roboter-\textit{Mock} auf der Konsole ausgegeben, so gilt die Aufgabe als erledigt.)
        \end{itemize}
        Für die Aufgabe sind folgende Punkte zu erfüllen:
        \begin{itemize}
            \item Die Kenntnis über \acs{MQTT}-Topics, die für die Kommunikation benötigt werden.
            \item Ein Zustandsraum, der alle benötigten Komponenten abbildet.
            \item Der Service-Roboter als Komponente, sodass dieser bei Ausführung einer Aufgabe für weitere Aufgaben gesperrt werden kann. 
            \item Der Auslöser, welcher durch ein \acs{MQTT} (PlugIn) abgebildet wird.
            \item Die Transformation der eingehenden \acs{MQTT}-Topics zu Zustandsänderungen, auf die Regeln ausgeführt werden sollen.
            \item Die Regel, die den Vorgang der Begrüßung und des Check-ins vorgibt.
        \end{itemize}
        Dem Anhang (siehe \ref{appendix:usabilitytestpaper}) ist das Dokument zur Durchführung des Usability-Test beigefügt. 
        \\
        \linebreak
        Nach dem Abschluss der Aufgabe wurden die Probanden um ihre Meinung mithilfe eines Fragebogens, der sich an dem \textit{System Usability Scale (SUS)} 
        Template orientiert, gebeten. Dieser Fragebogen ist ebenso dem Anhang (siehe \ref{appendix:usabilitytestpaper}) zu entnehmen. Auf die Auswertung wird in der 
        Zusammenfassung (\ref{subsec:usabilityFazit}) des Usability-Tests eingegangen. 

    \subsection{Fazit}
    \label{subsec:usabilityFazit}
        Der Usability-Test konnte erfolgreich durchgeführt werden.
        Das Ergebnis der quantitativen Forschung kann nicht als repräsentativ eingestuft werden, da es sich um eine kleine Auswahl an Experten handelt. Es zeigt dennoch eine Tendenz, 
        welche Anforderungen abgedeckt sind bzw. welche Verbesserungspotenziale in dem Framework stecken.  
        Das Resultat der Aufgabe ähnelte sich bei den Probanden mit Ausnahme der Namensgebung der Attribute im Zustandsraum. 
        \\
        Folgender Auflistung ist der durchschnittliche Wert der Antworten zu entnehmen:
        \begin{enumerate}
            \item \textit{Ich denke, dass ich dieses System gerne öfter nutzen würde.} (80 Punkte)
            \item \textit{Ich fand das System unnötig komplex.} (8 Punkte)
            \item \textit{Ich fand das System einfach zu bedienen.} (76 Punkte)
            \item \textit{Ich denke, dass ich die Unterstützung einer technischen Person benötigen würde, um dieses System nutzen zu können.} (9 Punkte)
            \item \textit{Ich fand, dass die verschiedenen Funktionen in diesem System gut integriert waren.} (86 Punkte)
            \item \textit{Ich dachte, es gäbe zu viele Inkonsistenzen in diesem System.} (0 Punkte)
            \item \textit{Ich könnte mir vorstellen, dass die meisten Leute sehr schnell lernen würden, dieses System zu benutzen.} (89 Punkte)
            \item \textit{Ich fand das System sehr umständlich zu bedienen.} (5 Punkte)
            \item \textit{Ich fühlte mich sehr sicher mit dem System.} (76 Punkte)
            \item \textit{Ich musste viele Dinge lernen, bevor ich mit diesem System loslegen konnte.} (19 Punkte)
        \end{enumerate}
        Die obigen Fragen wurden aus dem \textit{SUS} \cite{brook1995} (siehe Kapitel \ref{subsec:usabilitytests}) verwendet und ins Deutsche übersetzt. 
        \\
        Im Rahmen der Bewertung, bei der 100 die volle Zustimmung und 0 die Ablehnung darstellt, ist das Ergebnis 
        sehr positiv und zeigt das Potenzial des Frameworks. 
        \\
        Nachdem der Test abgeschlossen und der Fragebogen beantwortet war, wurde die befragte Person abschließend noch interviewt, damit 
        Eindrücke über den Test hinaus gesammelt werden konnten. Diese Interviews werden im folgenden Abschnitt resümiert. 
        %1.: - 80 = 100, 75, 60, 80, 85 
        %2.: - 8 = 0, 15, 10, 10, 5 
        %3.: - 76 = 100, 80, 75, 75, 50 
        %4.: - 9 = 0, 5, 5, 10, 25 
        %5.: - 86 = 100, 90, 80, 85, 75 
        %6.: - 0 = 0, 0, 0, 0, 0 
        %7.: - 89 = 100, 100, 90, 85, 70
        %8.: - 5 = 0, 0, 5, 5, 15 
        %9.: - 76 = 85, 80, 75, 90, 50
        %10.: - 19 = 0, 15, 15, 20, 45 

\section{Experteninterview}
        Zur umfangreichen Informationsgewinnung wurde anschließend zu dem Usability-Test ein Interview mit dem jeweiligen Probanden 
        durchgeführt. Hierbei wurden sowohl die gesammelten Erkenntnisse und Erfahrungen während der Aufgabe erfragt sowie auf die Anforderungen 
        (siehe Kapitel \ref{sec:requirementsFinal}) eingegangen. Ein Teil der Probanden wurde dafür bereits 
        bei der Erhebung von Anforderungen befragt. Somit konnte sich ein Bild verschafft werden, wie die Anforderungen 
        umgesetzt bzw. diese von den Experten empfunden wurden.
    
    \subsection{Ziele des Experteninterviews}
        Das Ziel des abschließenden Experteninterviews ist es, die Erfahrungswerte der Probanden zu sammeln, um diese den Anforderungen gegenüberzustellen. Bei 
        Experten, die bereits zu der Anforderungserhebung interviewt wurden, konnte konkret auf die Umsetzung der Kriterien und Anforderungen eingegangen werden. 
        Ebenso ist ein abschließendes Feedback wünschenswert sowie ein Identifizieren der Herausforderungen für den Anwender und der eventuellen Schwachstellen 
        des Systems. 

    \subsection{Fazit}
        Das Experteninterview wurde, vergleichbar zu dem vorherigen Interview (siehe \ref{subsec:experteninterview}), mit einem semi-strukturierten Ansatz durchgeführt. 
        Anfangs wurde an den Fragebogen angeknüpft, um eine nachträgliche Zusammenfassung der Erkenntnisse jedes einzelnen zu erfahren. Sofern die Zusammenfassung bereits 
        die notwendigen Informationen enthielt, wurde ein unstrukturierter Ansatz verfolgt und ein offener Dialog geführt. War die Zusammenfassung jedoch nicht 
        aufschlussreich, so wurden anschließend konkret die 
        folgenden Fragen gestellt: 
        \begin{enumerate}
            \item \textit{„Wie ist Ihr erster Eindruck des Systems?“}
            \item \textit{„Gibt es aus Ihrer Sicht (Sicht des Anwenders) Mängel, die Ihnen die Nutzung des Frameworks erschweren?“}
            \item \textit{„Gab es Unklarheiten während der Anwendung des Frameworks? - Wenn ja, welche?“}
            \item \textit{„Haben Sie eine Idee oder Lösung, um diese Unklarheit zu beseitigen?“}
            \item \textit{„Gibt es allgemein Anregungen, bzw. Vorschläge für Verbesserungen?“}
            \item \textit{„Was gefällt Ihnen an dem Framework am meisten? - Was gefällt Ihnen nicht?“}
        \end{enumerate} 
        Rekapitulierend wurde ein aufschlussreiches und erfreuliches Ergebnis erzielt. Die Antworten gaben ein klares Meinungsbild von der Einfachheit der 
        Handhabung der formalisierten Interaktionen der Steuerzentrale wieder und bestätigen die Erfüllung des Ziels dieser Arbeit. 
        Aus Sicht des Anwenders gab es wenige Anmerkungen zum Aufbau des Frameworks. 
        \\
        Eine erwähnenswerte Anmerkung war folgende:
        \begin{quote}
            „Wenn man wenig bis keine Erfahrung mit der Java Entwicklung hat, gibt es klar die ein oder andere Wissenslücke, die zuerst beseitigt werden muss. 
            Durch die wenigen Handlungsschritte, die jedoch viel Interpretationsfreiheiten bieten, besonders bei der Regeldefinition, ist dies doch überschaubar. 
            Bekommt man dementsprechend eine Hilfestellung, bzw. eine detailliertes Beispiel an die Hand, so können kleinere Schritte und Anwendungsfälle 
            relativ schnell umgesetzt werden.“
        \end{quote}
        Ein Ergänzungs-, bzw. Verbesserungsvorschlag, der während eines Interviews erwähnt wurde, ist die Optimierung des Anlegens von sämtlichen Objekten. 
        Hierfür wurde vorgeschlagen, eine Art \textit{\ac{CLI}}, wie bspw. die von Angular\footnote{TypeSkript basiertes Web Application Framework. \url{https://angular.io/} Besucht am 05.08.2022}, 
        zu implementieren. Die Idee war Befehle über die Kommandozeile geben zu können, bspw. das Anlegen von Attributen im Zustandsobjekt, das Erzeugen leerer, bzw. mit den dafür 
        vorgesehenen Parameterwerten befüllter Transformationen, und das Erstellen eines leeren Regelkonstruktes. Dadurch würde der notwendige Code generiert und dem Entwickler zur Verfügung gestellt werden, wodurch dieser 
        sich ausschließlich um das Implementieren der Regeln kümmern muss. 
        \\
        \linebreak
        Abschließend lässt sich anmerken, dass mit der konkreten Anweisung, welcher Anwendungsfall umgesetzt werden soll, bzw. welche Randbedingungen 
        dafür gelten, schnell ein Ergebnis erzielt werden kann. 
        
\section{Evaluation der Anforderungen}
    Im Rahmen der Arbeit werden die Anforderungen (siehe Kapitel \ref{sec:requirementsFinal}) zum Abschluss anhand von dafür vorgesehenen Praktiken evaluiert. 
    Die Reihenfolge entspricht der tabellarischen Aufstellung der Anforderungen. Zuerst erfolgt die Abarbeitung der \textit{funktionalen Anforderungen}, 
    anschließend die der \textit{nicht funktionalen Anforderungen} und zum Ende hin die der \textit{zusätzlichen Anforderungen}. 
    
    \subsection*{Funktionale Anforderungen}
        Während der Konzeption und Entwicklung des Frameworks wurden die funktionalen Anforderungen (FA1 - FA6, siehe Tabelle \ref{tab:functionalRequirements}) 
        berücksichtigt. Sie stellen den wichtigsten Kernpunkt dar. Deren Erfüllung konnte durch die abschließenden 
        Experteninterviews identifiziert werden. Die Resonanz der Experten gab auch deutlich wieder, dass auf Grundlage der Anforderungen der Prototyp aufbaut. Die Struktur 
        des Regelobjektes (FA1) wird gegeben, indem eine Klasse von der Oberklasse erbt und diese erweitert. Umgesetzt wurde diese Anforderung mithilfe des 
        \textit{Template Method} Architekturmusters, auf das auch bereits in der Konzeption (siehe Kapitel \ref{chap:konzept}) eingegangen wurde. 
        \\
        \linebreak
        Die Anforderung (FA2) ist 
        umgesetzt, indem der Regel mit der \textit{ruleTrigger}-Methode ein Wert eines \textit{Enum Types}\footnote{Spezieller Datentyp. \url{https://docs.oracle.com/javase/tutorial/java/javaOO/enum.html} Besucht am 05.08.2022} 
        zugeordnet wird. Ein essentieller Teil des Frameworks ist der vom Anwender zu implementierende Zustandsraum, welcher über eine separate Klasse vorgegeben wird. Dadurch 
        ist auch die Anforderung (FA3) als umgesetzt anzusehen. Die Individualität der Implementierung des Zustandsraume ist durch die frameworkseitige Verwendung der 
        Java \textit{Generics} gegeben. Demzufolge arbeitet das Framework mit dem vom Entwickler übergebenen Klassenobjekt. 
        \\
        \linebreak
        Die Ausprägungen der Bedingungen zur Ausführung der Regelprozesse sind unverzichtbar. Sie müssen vom Anwender aufgestellt werden (FA4). Damit eine Aufstellung erfolgen kann, ist in der Regelklasse eine 
        Funktion gegeben, die das Implementieren der Bedingung erzwingt. 
        Die darauffolgende Forderung (FA5) ist prinzipiell gegeben, wird allerdings durch die Einschränkung im 
        Zustandsraum aktuell nicht als geeignet angesehen. Die Nutzung von Java \textit{Reflection}, bzw. die Umsetzung der inversen Transformation erschwert dies. Derzeit wird von der Komponenteneinbindung 
        in das Zustandsobjekt abgeraten. Das Erstellen von Komponenten ist dennoch möglich und kann im Regelkontext angewendet werden, sofern bestimmte Eigenschaften für eine Regel relevant sind. 
        \\
        \linebreak
        Der letzte Punkt (FA6) ist ebenso umgesetzt, da innerhalb der Regel bis auf die Voraussetzung der Implementierung der Regelbedingung sowie des -prozesses keine Einschränkungen bestehen. 
        Zusätzlich können bspw. sämtliche \acs{HTTP}-Abfragen getätigt werden. Im Falle des Use Cases "Check-in mit einem Service-Roboter" werden \acs{HTTP}-Abfragen ausgeführt, damit 
        Informationen der Büroplatzbuchungssoftware abgerufen werden können. 

    \subsection*{Nicht funktionale Anforderungen}
        Zur Überprüfung der \textit{nicht funktionalen Anforderungen} wurden Methoden für deren Überprüfung verwendet. Zusätzlich zu den 
        verifizierten \textit{User Stories} mittels der Usability-Tests konnten ebenso die aufgestellten Forderungen, die allgemein 
        die Benutzerfreundlichkeit (BF) umfassen, überprüft werden. Der durch die Experteninterviews aufgestellte Anspruch (NFA1) konnte 
        durch die Usability-Tests erfüllt werden, da die Probanden selten mehr als eine Aktion benötigten, um die bereits erstellte Regel 
        dem Framework zu übergeben. Grund dafür ist die Funktion der Steuerzentrale \textit{addRule(...)}. Der Anwender kann die Aufgabe der Funktion 
        schnell zuordnen und bekommt über die Entwicklungsumgebung die erwarteten Übergabeparameter mitgeteilt. 
        \\
        \linebreak 
        Im Rahmen des Usability-Tests waren alle Teilnehmer in der Lage eine Regel zu erstellen 
        und diese dem Regelwerk hinzuzufügen (NFA2). Ausgehend von dieser Anforderung wurden 
        während der Experteninterviews Anregungen angebracht, die gegebenenfalls eine weitere Vereinfachung für den Entwickler darstellen. 
        Diese werden im Ausblick (siehe Kapitel \ref{chap:ausblick}) aufgegriffen. 
        \\
        Durch die rein programmatische Anwendung des Frameworks kann der Entwickler zu jedem Zeitpunkt auf die Funktionen der Steuerzentrale zugreifen (NFA3). 
        Dies wurde ebenso von den Testteilnehmern im anschließenden Experteninterview bestätigt. Zum aktuellen Zeitpunkt sind die nutzbaren Funktionen sehr 
        überschaubar und decken die notwendigsten Funktion ab, darunter das Hinzufügen von Regeln, 
        Erstellen aller wesentlichen Objekte und Komponenten, Aufsetzen der erforderlichen \acs{MQTT}-Konfiguration, Ergänzen von Transformationen, Starten 
        des \acs{MQTT}-Clients und das Erweitern um zeitbasierte Regelauslöser. 
        \\
        \linebreak 
        Die Anforderung (NFA4) ist ebenso gegeben, da der Anwender sich 
        ausschließlich um die Definitionen der Regeln, der Transformationen und des Zustandsraumes kümmern muss. Dennoch gibt es zur weiteren 
        Vereinfachung der Regeldefinition zusätzliche Möglichkeiten, die im Rahmen dieser Arbeit nicht umgesetzt wurden. Diese werden im Ausblick 
        aufgegriffen, um weitere Potenziale aufzuzeigen. 
        \\
        \linebreak
        Ein nennenswerter Aspekt ist die allgemeine Zuverlässigkeit des Systems. Zur Überprüfung der Zuverlässigkeitsanforderung (NFA5) wurden im Rahmen der 
        Entwicklung \textit{JUnit-Tests}\footnote{Test Framework für Java. \url{https://junit.org/junit5/} Besucht am 08.08.2022} implementiert, die 
        einen bestimmten Sachverhalt auslösen und überprüfen. Demnach wird ein Zustand erwartet, der nach dem Auslösen einer Regel eintreffen sollte. Anschließend 
        wird der Ist-Zustand mit dem Soll-Zustand abgeglichen und somit die Zuverlässigkeit überprüft. 
        Dennoch gibt es eine starke Abhängigkeit zur Regelbedingung, die der Anwender definiert. Ist diese 
        nicht deutlich genug, kann es passieren, dass Regeln ungewollt ausgeführt werden. Deshalb ist die Validierung der Regelbedingung durch den 
        Entwickler essentiell. Somit ist die Zuverlässigkeit abhängig von den Interaktionen und der Implementierung des Anwenders. Grundsätzlich gibt es 
        systemseitig keine negativen Auswirkungen von mehrdeutigen Bedingungen, die fälschlicherweise die Zuordnung und Ausführung einer Regel gefährden. Wird jedoch durch die 
        Regel der Zustand an einer anderen Stelle erneut geändert, ohne dass sich der vorherige Wert und dessen Bedingung wieder negiert hat, so würde das 
        Framework eine Endlosschleife und Rückkopplung bilden und dadurch einen \textit{StackOverflowError} erzeugen. Demnach ist die Auswahl der Bedingung einer der wichtigsten 
        Punkte, die zu berücksichtigen sind.
        \\
        \linebreak
        Durch die Verwendung von \textit{Threads} werden Regeln, deren Bedingungen im Vorfeld zutreffen, asynchron ausgeführt. Es können dadurch voneinander 
        unabhängige Regel parallel abgearbeitet werden. Somit ist die Performanz des Systems erhöht (NFA6). Die Reaktionszeit und die Performanz der 
        Kommunikation ist sowohl abhängig von den jeweiligen Protokollen und Technologien, die eingesetzt werden sowie von der dafür verwendeten Hardware 
        und Infrastruktur (NFA7). Eine konkrete Überprüfung der Anforderungen wurde im Rahmen der Arbeit nicht durchgeführt, da diese zum aktuellen Zeitpunkt keine 
        Auswirkungen auf das System selbst haben. Dennoch ist unter Verwendung der derzeitigen Technologien eine gewisse Performanz vorauszusetzen und zu erwarten. 
        \\
        \linebreak
        Die \textit{nicht funktionalen Anforderungen}, die allgemein die Verfügbarkeit des Systems beschreiben (NFA8 \& NFA9), sind aktuell zu vernachlässigen und wurden im Rahmen dieser 
        Arbeit nicht überprüft. Das Framework selbst produziert von sich aus keine Downtime. Entscheidend sind dabei die Richtigkeit der Regeldefinition, 
        Verbindungen zum \acs{MQTT}-Broker und Internet sowie die Stromversorgung, da das Framework lokal auf einem \textit{Raspberry Pi} betrieben wird. 
        \\
        \linebreak
        Zur Überprüfung der Fehlertoleranz (NFA10) wurden die \textit{JUnit-Tests} um weitere Testfälle ergänzt. Hierbei wurde eine fehlerhafte Regel definiert, die innerhalb des 
        Tests ausgelöst und ausgeführt wurde. Die Resultate erzielten eine gewisse Fehlertoleranz, die durch eine Weiterentwicklung stets verbessert werden kann.
        \\
        \linebreak
        Der letzte Punkt (NFA11) der Tabelle (\ref{tab:notfunctionalRequirements}) zählt als Erweiterung des Frameworks, sodass ein ständiger Informationsaustausch stattfindet und der 
        Entwickler das System überwachen kann und zu jedem Zeitpunkt einen Einblick in die Prozesse erlangt. Die Integration solch einer Lösung in das System ist ohne weiteres möglich.
    
    \subsection*{Zusätzliche Anforderungen (Randbedingungen)}
        Die Tabelle (\ref{tab:furtherRequirements}) beinhaltet weitere Anforderungen und Randbedingungen, die mithilfe der verwendeten 
        Erhebungstechniken identifiziert wurden. 
        \\
        Die Randbedingung (ZAF1), Nutzung der Programmiersprache Java, wurde eingehalten. Das Framework ist ausschließlich in Java entwickelt. 
        \\
        Für den Zustandsraum ist eine gewissen Modellierungsempfehlung vorgegeben, die keine Objekte als Felder im Zustandsraum zulässt. Die 
        Abbildung von reellen Zuständen findet 
        ausschließlich über Attribute statt. Somit ist die Mindestanforderung für (ZAF2) erfüllt. 
        \\
        \linebreak
        Die Anforderung (ZAF3) ist abgedeckt, da die Regelprüfung uns -ausführung über einen \textit{Thread Pool} in einen eigenen \textit{Thread} ausgelagert wird. Ebenso 
        ist (ZAF4) erfüllt, da die Kommunikation zum aktuellen Zeitpunkt ausschließlich über \acs{MQTT} erfolgt. Das Auslösen von Regeln über definierte Zeitpunkte kann durch 
        \textit{Cronjobs} oder durch die Annotation (\textit{\@ Scheduled}) des Spring Frameworks stattfinden. Demnach stellen diese beide Möglichkeiten die einzigen Optionen eines 
        Auslösers dar. In Folge dessen gilt auch die Forderung (ZAF5) als erledigt.
        \\
        \linebreak
        Durch die Verwendung der generischen Programmierung in Java ist sichergestellt, dass das Zustandsobjekt zur Laufzeit dem Framework übergeben wird (ZAF6). %Zusätzlich zum generischen 
        %Typ wird ebenfalls eine Objektinstanz erzeugt, die den Zustandsraum wiedergibt. 
        Somit kann das Framework mit dem Objekt arbeiten, unabhängig von der Struktur der Klasse.
        \\
        \linebreak
        Anhand der Anforderung (ZAF7) findet die Kommunikation überwiegend über \acs{MQTT} statt. Ein Punkt, der nicht vom Framework abgedeckt ist, ist die Sicherheit der 
        Kommunikation. Der Anwender muss sicherstellen, dass dementsprechende Vorkehrungen getroffen sind, damit der \acs{MQTT}-Broker und die darüber laufende Kommunikation abgeschirmt und für 
        Dritte nicht zugänglich sind (ZAF8). Im Rahmen der Arbeit ist der \acs{MQTT}-Broker ausschließlich im lokalen Netzwerk verfügbar und erlaubt den Zugriff nur für bestimmte Nutzer und Kommunikationsteilnehmer. 
        \\
        \linebreak
        Der \textit{\acl{SPC}} ist gegeben, da die Einstellungen und das Hinzufügen von Regeln und Transformationen ausschließlich über eine einzige Klasse funktionieren, die \textit{InnovationLogicHubApplication}. 
        \\
        \linebreak
        Nachdem die aufgestellten Anforderungen evaluiert wurden, wird im folgenden Abschnitt auf die Evaluation der Forschungsfrage eingegangen.
        \pagebreak

\section{Evaluation der Forschungsfrage}
    %Wie kann man die Usability von Prozessautomatisierungen und Regeldefinitionen von SmartHome-Plattformen optimieren, sodass die formalen Interaktionen für den Softwareentwickler einfacher in der Handhabung sind?
    Der in dieser Arbeit verfolgte Ansatz bietet dem Softwareentwickler eine Möglichkeit, mit wenigen Schritten eine 
    Prozessautomatisierung oder Regeldefinition basierend auf einem Anwendungsfall im Bereich des intelligenten 
    Büros, bzw. \textit{Smart Office} oder \textit{\acl{SH}}, umzusetzen. Hierbei wird im Vergleich zu anderen 
    Lösungen, darunter \acs{OPENHAB} und Home Assistant, nicht der Weg über eine Integration angestrebt. 
    Die Gestaltung solcher Komponenten, die über Integrationen abgebildet werden, findet in dem konzipierten 
    Framework über den Zustandsraum statt. Er lässt dem Entwickler die Ausprägung offen. Zusätzlich 
    wird darauf verzichtet eine Benutzeroberfläche bereitzustellen, mit der die Interaktionen gegebenenfalls 
    aufwändig, bzw. mehrschrittig sein könnten. Durch die alleinige Nutzung der Programmiersprache Java ist innerhalb des 
    Frameworks kein Kontextwechsel, bzw. Syntaxwechsel erforderlich, wodurch die Interaktionen ebenso vereinfacht bleiben. 
    Mithilfe der reduzierten Interaktionspunkte für den Entwickler ist gleichermaßen eine Formalisierung der Interaktionen 
    gewährleistet. So ist lediglich pro Anwendungsfall eine klar vorgegebene Struktur zu implementieren. Diese besteht aus mindestens einer 
    Transformation, einer Regel und einem Attribut im Zustandsraum. 
    \\
    \linebreak
    Die Usability-Tests sowie die Experteninterviews zeigen das Potenzial dieses Systems, welches bei Weitem nicht ausgeschöpft ist 
    und in Zukunft vorangetrieben werden kann. 
    \\
    \linebreak
    Durch die einfache Erweiterbarkeit, die durch die Architektur gewährleistet ist, können nützliche Funktionen hinzugefügt 
    werden. 
    \\
    In der Gesamtheit wurde das System als überaus hilfreich und entwicklungsfähig eingestuft. In dieser Arbeit stehen ausschließlich die 
    Konzeption, Grundlagenschaffung und prototypische Entwicklung zur Erreichung der Zielsetzung im Fokus. Die Grundlagen und -funktionen 
    wurden implementiert und zur Verfügung gestellt.
    \\
    \linebreak
    Das Framework kann dennoch nicht direkt mit den bestehenden Softwarelösungen, bspw. Home Assistant und \acs{OPENHAB}, verglichen 
    werden, da diese weitaus mehr Funktionalitäten anbieten, die im Rahmen dieser Arbeit nicht vorgesehen waren. 
    \\
    \linebreak
    Zusammenfassend lässt sich sagen, dass die Usability von Prozessautomatisierungen und Regeldefinitionen durch das Framework 
    optimiert werden können. Dieses bietet klare Strukturen und zeigt alle Handlungsschritte auf. Die Softwareentwickler erfahren 
    durch das Framework die einfache Handhabung der formalisierten Interaktionen. Das System bietet einen möglichen Lösungsweg, 
    der die Forschungsfrage beantwortet. 
\chapter{Fazit}
\label{chap:fazit}
    Ziel dieser Arbeit war die Konzeption und prototypische Umsetzung einer Steuerzentrale für ein 
    smartes Büro mit dem Fokus einer einfachen Handhabung der formalisierten Interaktionen für Softwareentwickler. 
    Das in diesem Rahmen entstandene Framework bietet eine klare Struktur zur Implementierung von Anwendungsfällen, die der Anwender 
    im Umfeld eines smarten Büros umsetzen möchte. Durch den rein programmatischen Ansatz ist somit 
    dem Softwareentwickler ein System gegeben, dass ausschließlich mit Java bedient wird. Mithilfe von verwendeten 
    Softwarearchitekturmustern kann eine Ordnung geschaffen werden, die dennoch dem Entwickler in der 
    Ausprägung des jeweiligen Anwendungsfalls alle Freiheiten offen lässt. 
    \\
    \linebreak 
    Für die Erreichung des Ziels wurden unter anderem tiefgreifende Recherchen und Analysen durchgeführt, die den Stand 
    der Technik hervorbrachten, gefolgt von der Testung derzeitiger Softwarelösungen, um deren Möglichkeiten und Grenzen zu erfahren. 
    Auf dieser Grundlage wurden weitere Schritte eingeleitet, damit gezielte Anforderungen definiert werden konnten. Hierfür wurde 
    eine Markt- sowie Zielgruppenanalyse durchgeführt und konkrete Anwendungsfälle definiert. Mit diesen Hintergründen wurden zur 
    Auffassung möglichst vieler Anforderungen zusätzlich Experteninterviews durchgeführt, wodurch die Konzeption und Umsetzung 
    durchgeführt werden konnte. 
    \\
    \linebreak
    Unter Berücksichtigung der zu gewährleistenden Modularisierung, Erweiterbarkeit und Wartbarkeit wurden dementsprechende 
    Architekturentwurfsmuster verwendet. Zudem wurden einzelne Schichten des Systems entworfen, darunter zählt die Kommunikations-, sowie 
    Logik und Persistenzschicht, wobei letztere nicht konkret konzipiert wurde.     
    \\
    Damit der Anwender stets eine klare Struktur zur Regeldefinition auffindet und so die einfache Handhabung der formalisierten Interaktionen gefördert wird, wurde 
    ein konkretes Konstrukt verwendet, das diesen Leitfaden vorgibt. Darüber hinaus wurde zu der Regeldefinition eine Annotation ergänzt, die eine Regel als solche 
    deutlich markiert.
    \\
    \linebreak
    Zur Überprüfung und Bestätigung der Anforderungen, sowie des gestellten Ziels wurden anschließend zu der Umsetzung Usability-Tests und zusätzliche 
    Experteninterviews durchgeführt. Diese halfen dabei, den erarbeiteten Prototypen und dessen Konzept zu evaluieren. Alle hochpriorisierten 
    Anforderungen konnten erfüllt werden. Das Ergebnis der Usability-Tests, sowie der Interviews ergaben ein überaus positives Feedback. Dennoch 
    gibt es Verbesserungs-, bzw. Anpassungspotentiale, die in folgendem thematisiert werden. Erweiterungen und mögliche Weiterentwicklungen werden 
    im Ausblick (siehe Kapitel \ref{chap:ausblick}) aufgegriffen. 
    \\
    \linebreak 
    Im aktuellen Stand des Konzeptes ist die Verwendung von Komponenten nicht vorgesehen, da diese im Zustandsraum derzeit keine Verwendung haben. Die 
    Gründe dafür sind unter anderem die Entscheidung der Verwendung der Java Reflection, sowie die aktuelle inverse Transformation, die jedoch in weiteren 
    Iterationen angepasst werden kann. ZUsätzlich gibt es mögliche Optimierungen der Regelausführung, bzw. dessen Management, um die Performanz 
    zu steigern und eine strukturierte Abfolge zu gewährleisten.  
    \\
    %\linebreak
    Dennoch liefert das aktuelle Konzept, sowie der aktuelle Prototyp alle Grundvoraussetzungen, um das gewünschte Ziel zu 
    erreichen, bzw. Prozesse im intelligenten Büro zu automatisieren. Mit der Berücksichtigung der Modellierungsgrenzen, als 
    auch den -vorgaben ist die Realisierung von komplexen Anwendungsfällen, bspw. die Verwendung eines 
    Service-Roboters zur Interaktion mit Menschen, ohne weiteres möglich. 
    \\
    \linebreak
    Das Ziel dieser Master-Thesis, eine Steuerzentrale für ein intelligentes Büro zu konzipieren und entwickeln, welches den Fokus der 
    einfachen Handhabung der formalisierten Interaktionen für Softwareentwickler besitzt, wurde demnach erreicht: das hier erarbeitete 
    und vorgestellte Framework optimiert die Usability in Form der Programmierung und ermöglicht das Umsetzen und Ausführen von Regeln. 

\chapter{Ausblick}
\label{chap:ausblick}
    Das Framework liefert aus Sicht der Usability gute Ergebnisse und gibt dem Anwender die zu implementierenden Elemente klar vor. 
    Vorausgesetzt werden hierbei jedoch Grundkenntnisse der Programmierung, sowie die Kenntnis über die zu implementierenden Elemente des Systems. 
    Zukünftige Weiterentwicklungen des im Rahmen dieser Arbeit vorgestellten Frameworks sollten demnach zum Ziel haben, das entworfene Konzept, 
    sowie den aktuellen Prototypen entsprechend zu erweitern. 
    \\
    \linebreak
    Für die Optimierung des Systems bietet primär der Algorithmus der Regelprüfung und -ausführung Potential. Dieser ließe sich beispielsweise 
    gut um einen \textit{Queuing}-Mechanismus ergänzen, damit alle eintreffenden Zustandsänderungen berücksichtigt werden, die zum 
    Zeitpunkt der Bedingungsüberprüfung eventuell nicht zutreffen. Diese würden dann zurückgehalten und immer wieder überprüft werden, bis eine 
    Bedingung zutrifft und die Zustandsänderung eine entsprechende Aktion ausführt. Sofern dies beabsichtigt ist. 
    \\
    \linebreak
    Des weiteren kann das derzeit statische \textit{JSON}-Konstrukt der inversen Transformation so optimiert werden, dass beliebige Konstrukte 
    als \acs{MQTT}-Nutzlast veröffentlicht werden können. Dadurch wäre eine Versendung mehrerer Informationen möglich. 
    \\
    Im Zuge dessen wäre eine Optimierung des Kopieralgorithmus angebracht, sodass verwendete Komponenten im Zustandsraum nach dem Kopiervorgang keine neue 
    Objektreferenz zur Programmlaufzeit erhalten. In Kombination dieser Schritte wäre nachfolgend eine Verwendung von Komponenten im Zustandsraum möglich, wobei 
    dies nach wie vor keine konkreten Vorteile aufweist. Dadurch würde lediglich die Gestaltung des Zustandsraumes auf die Komponenten verlagert werden. 
    \\
    Die Java Reflection müsste dahingehen geringfügig angepasst werden, sodass Überprüfungen im Stile des abfolgenden Baumgraphs, entlang der Pfade stattfindet.
    \\
    \linebreak
    Neben den inhaltlichen Optimierungen gibt es ebenso Erweiterungsmöglichkeiten, um die formalisierten Interaktionen der Softwareentwickler zusätzlich zu vereinfachen und 
    weitere Funktionalitäten dem Framework zu übergeben. Diese sind im Rahmen von Ergänzungen anzusehen.
    \\
    Damit der Softwareentwickler zu den Annotationen und der vorgegebenen Regelstruktur dessen Richtigkeit überprüfen kann, wäre ein Debug-Mechanismus für die implementierten Regeln sinnvoll. 
    Mithilfe dessen könnte zusätzlich die Syntax und zu verwendende Struktur der Regel sichergestellt werden. In Folge dessen könnte auch ein Ansatz konzipiert werden, wie ein Testframework 
    auf das System angewendet werden könnte, um die Funktionalität ergänzend zu den JUnit-Tests zu überprüfen. 
    \\
    \linebreak
    Eine weitere Maßnahme mit der die formalisierten Interaktionen der Softwareentwickler künftig noch einfacher zu handhaben sind, wäre die Entwicklung eines \ac{CLI}. Dadurch könnten mithilfe 
    von dynamischer Quellcode Generierung die einzelnen Komponenten für eine Regel, darunter Attribut im Zustandsraum, leeres Transformationsobjekt und Regelklasse, erstellt werden. Als Grundlage dafür 
    würde beispielsweise das \acs{CLI} von \textit{Apache Maven}\footnote{Kommandozeilenwerkzeug aus der Kategorie der Build-Werkzeuge. \url{https://maven.apache.org/index.html} Besucht am 12.08.2022} 
    dienen. Dadurch könnte über einen einfachen Befehl alle Elemente, die für eine Regel notwendig sind erstellt werden. Dadurch entfiele z.B. die 
    eigenständige Erstellung der Regel, wobei zu beachten ist, dass die richtige abstrakte Klasse verwendet wird und alle zu implementierenden Funktionen präsent sind. 
    \\
    Alternativ wäre die Entwicklung eines \ac{IDE} Plug-ins möglich, mithilfe dessen das Regelkonstrukt darüber erzeugt werden kann. 
    \\
    \linebreak
    Zum stetigen Ausbau des Frameworks wäre eine Erweiterung zur Nutzung zusätzlicher Kommunikationsprotokolle durchaus denkbar. Die notwendigen Vorkehrungen sind durch die Architektur durchaus gegeben. 
    Ähnlich zu Home Assistant wäre die Bereitstellung einer \acs{MQTT}-Broker-Instanz möglich, sodass das Framework autark und unabhängiger verwendet werden kann, bzw. das Management der grundlegenden Funktionen 
    gebündelt wird. 
    \\
    \linebreak
    Zur Steigerung der Integrität und Konsistenz des Zustandsraumes, kann mittels der Persistenzschicht beispielsweise eine Transaktionshitsorie erstellt werden, die jede Zustandsänderung als Datenmodell, bzw. Objekt in eine 
    Datenbank speichert. So kann nach einem Ausfall der konkrete Zustandsraum wieder eingespielt und die eigentlichen Zustände verwendet werden. Aktuell müssten alle Geräte ihren derzeitigen Zustand erneut übermitteln, wodurch 
    die Steuerzentrale diese erneut registrieren kann. Durch das Einspielen der letzten Transaktion würden alle Zustände wieder vorhanden sein. Die Ausprägung dieser Schicht ist für weitere Entwicklungen offengehalten. 
    \\
    \linebreak
    Für die Fähigkeit der Überwachung von aktuell ablaufenden Regeln könnte ein sogenanntes \textit{Monitoring} entwickelt und beliebig ausgeprägt werden, sodass auch Prognosen erhoben und Interaktionspunkte geschaffen 
    werden können. Hierbei könnte beispielsweise über einen separaten Monitor im Büro angezeigt werden, welche Regel durchlaufen wird, bzw. wie die Zustände im Zustandsraum aussehen. 
    \\
    \linebreak
    Der Ansatz bietet allgemein eine Möglichkeit, um in einer Umgebung neue Automationen und Prozesse schnellstmöglich umzusetzen. Durch das stetige Wachstum und die Akzeptanz des Bereiches \acl{SH} und dessen Spezifikation des 
    \textit{Smart Office} wäre das Vorantreiben dieser Idee durch ein Open-Source-Ansatz möglich. Durch die geschaffene Grundlage sind alle Optimierungs-, sowie Erweiterungsmöglichkeiten denkbar und ebenso umsetzbar. 
    %Monitoring
    %Open-source Projekt
    
    %Ergänzungen, bspw. 
        %Regeldebugger (Anwendung eines Testframeworks), 
        %Code Generator (CLI like Angular CLI), 
        %IDE Plugin zur Erstellung von Regelkonstrukten, 
        %Weitere KommunikationsProtokolle, Bereitstellung eigener Kommunikationskomponenten?
        % Transaktionshistorie
        %Monitoring von Prozessen und Regeln, um den Status des Systems überblicken zu können. (Welche Regel wird gerade ausgeführt und wie lange braucht diese?)
    % Optimierungen: 
        %Verbesserung der Regeldurchführung durch Queueing, 
        %inverse Transformation um zu sendendes JSON Konstrukt dynamisch zu erstellen je nach Bedarf, 
        %Nutzung von Komponenten im Zustandsraum 
    % Das Framework als open-source-Projekt pushen und anbieten


% Nutzung von Spring Dynamic Modules for OSGi Service Platforms -> https://de.wikipedia.org/wiki/Spring_(Framework)#Vergleich 
% -> Für die Realisierung eines weitreichenderen Frameworks, bzw. Applikation. 

% -> Framework als open-source projekt anbieten 

\pagenumbering{Roman}
% Ab hier beginnt der Anhang
\appendix

\addcontentsline{toc}{chapter}{Index}
\printindex

%\newpage
%\addcontentsline{toc}{chapter}{Liste der ToDo's}
%\listoftodos[Liste der ToDo's]

%\pagenumbering{roman}
\listoffigures             % Liste der Abbildungen
\listoftables              % Liste der Tabellen
\lstlistoflistings         % Liste der Listings
%\listofequations           % Liste der Formeln

%%%%%%%%%%%%%%%%%%%%%%%%%%%%%%%%%%%%%%%%%%%%%%%%%%%%%%%%%%%%%%%%%%%%%%%%%%%%%%
%% Descr:       Vorlage für Berichte der DHBW-Karlsruhe, Datei mit Abkürzungen
%% Author:      Prof. Dr. Jürgen Vollmer, vollmer@dhbw-karlsruhe.de
%% $Id: abk.tex,v 1.4 2017/10/06 14:02:03 vollmer Exp $
%% -*- coding: utf-8 -*-
%%%%%%%%%%%%%%%%%%%%%%%%%%%%%%%%%%%%%%%%%%%%%%%%%%%%%%%%%%%%%%%%%%%%%%%%%%%%%%%

\chapter*{Abkürzungsverzeichnis}                   % chapter*{..} -->   keine Nummer, kein "Kapitel"
						         % Nicht ins Inhaltsverzeichnis
% \addcontentsline{toc}{chapter}{Akürzungsverzeichnis}   % Damit das doch ins Inhaltsverzeichnis kommt

% Hier werden die Abkürzungen definiert
\begin{acronym}[DHBW]
  % \acro{Name}{Darstellung der Abkürzung}{Langform der Abkürzung}
 \acro{Abk}[Abk.]{Abkürzung}
 \acro{IoT}[IoT]{Internet of Things}
 \acro{IdD}[IdD]{Internet der Dinge}
 \acro{SM}[SM]{Smart Home}



 %%%%%%%%%%%%%%%%%%%%%%%%%%%%%%%%%%%%%%%
 %OLD
 %%%%%%%%%%%%%%%%%%%%%%%%%%%%%%%%%%%%%%%
 \acro{API}[API]{Application Programming Interface}
 \acro{AR}[AR]{Augmented Reality}
 \acro{DBMS}[DBMS]{Datenbank-Management-System}
 %\acro{ER}[ER]{deutsch erweiterte Realität}
 \acro{EKF}[EKF]{Erweiterter Kalman Filter}
 \acro{ERM}[ERM]{Entity-Relationship-Modell}
 \acro{Fraunhofer IOSB}[IOSB]{Fraunhofer Institut für Optronik, Systemtechnik und Bildauswertung IOSB}
 \acro{GPS}[GPS]{Global Positioning System}
 \acro{GUI}[GUI]{Graphical User Interface, dt. Benutzeroberfläche}
 \acro{HMD}[HMD]{Head-Mounted Display}
 \acro{ICP}[ICP]{Iterativ Closest Point}
 \acro{IDE}[IDE]{Integrated Development Environment}
 \acro{ID}[ID]{Identifikation}
 \acro{IMU}[IMU]{Inertial Measurement Unit}
 \acro{INS}[INS]{Inertial Navigation System}
 %\acro{IoT}[IoT]{Internet of Things, dt. Internet der Dinge}
 \acro{KIT}[KIT]{Forschungszentrum Karlsruhe}
 \acro{MEMS}[MEMS]{Micro-Electro-Mechanical Systems}
 \acro{MR}[MR]{Mixed Reality}
 \acro{MVC}[MVC]{Model View Controller}
 \acro{MVVM}[MVVM]{Model View ViewModel}
 \acro{SDK}[SDK]{Software Development Kit}
 \acro{SLAM}[SLAM]{Simultanious Localization And Mapping}
 \acro{SQL}[SQL]{Structured Query Language}
 \acro{TOF}[TOF]{Time of flight}
 \acro{UX}[UX]{User-Experience, dt. Nutzererfahrung und Nutzererlebnis}
 \acro{UC}[UC]{Use Case}
 \acro{UCs}[UCs]{Use Cases}
 \acro{UI}[UI]{User Interface}
 \acro{VR}[VR]{Virtual Reality}
 \acro{WMR}[WMR]{Windows Mixed Reality}

 % Folgendes benutzen, wenn der Plural einer Abk. benöigt wird
 % \newacroplural{Name}{Darstellung der Abkürzung}{Langform der Abkürzung}
 \newacroplural{Abk}[Abk-en]{Abkürzungen}


 \acro{H2O}[\ensuremath{H_2O}]{Di-Hydrogen-Monoxid}

 % Wenn nicht benutzt, erscheint diese Abk. nicht in der Liste
 \acro{NUA}{Not Used Acronym}
\end{acronym}
              % Abkürzungsverzeichnis

\addcontentsline{toc}{chapter}{Literaturverzeichnis}

% Haben Sie das "biblatex"-Paket nicht installiert, benutzen Sie folgendes:
% Ohne das "biblatex"-Paket (s. bericht.sty) produziert folgendes
% "deutsche" Zitate in Literaturverzeichnissen gemaß der Norm DIN 1505,
% Teil 2 vom Jan. 1984.
% Die Zitatmarken werden alphabetisch nach Verfassern
% sortiert und sind durch abgekürzte Verfasserbuchstaben plus
% Erscheinungsjahr in eckigen Klammern gekennzeichnet.

% \bibliographystyle{alphadin}
% \bibliography{bericht}

%%%%%%%%%%%%%%%%%%%%%%%%%%%%%%%%%%%%%%%5
% BIBLATEX
% Benutzt man das "biblatex"-Paket, muß man folgendes schreiben:
\def\refname{Literaturverzeichnis}
\printbibliography

\addcontentsline{toc}{chapter}{Anhang}

\chapter{}
\label{appendix:mqtt-amqp}
  \begin{table}[hbt!]
    \begin{center}
      \begin{tabular}{| p{3.25cm} | p{6cm} | p{6cm} | }
        \hline
          \textbf{Kriterium} & \textbf{MQTT} & \textbf{AMQP}\\
        \hline
          Jahr & 1999 & 2003 \\ 
        \hline
          Architektur & Client/Broker, & Client/Broker Client/Server \\ 
        \hline
          Abstraktion & Publish/Subscribe & Publish/Subscribe Request/response \\ 
        \hline
          Header Größe & 2 Byte,  & 8 Byte \\ 
        \hline
          Nachrichtengröße & max. 265 MB & Abhängig von Broker/Server \\ 
        \hline 
          Semantik / Methoden & Connect, Disconnect, Publish, Subscribne, Unsubscribe, Close & Consume, Deliver, Publish, Get, Select, Ack, Delete, Nack, Recover, Reject, Open, Close \\
        \hline
          Zwischenspeicher (Cache) und Proxy & Teilweise & Ja \\
        \hline
          Standards & OASIS, Eclipse Foundation  & OASIS, ISO/IEC \\ 
        \hline
          Transport-Protokoll & TCP, & TCP, SCTP \\
        \hline
          Sicherheit & TLS/SSL & TLS/SSL \\ 
        \hline
          Standard-Port & 1883/8883 (TLS/SSL) & 5671 (TLS/SSL), 5672 \\
        \hline
          Format & binär  & Binär \\ 
        \hline
          Lizenzmodell & Open Source & Open Source \\
        \hline
          Support & IBM, Facebook, Eurotech, Cisco, Red Hat Amazon (AWS) etc. & Microsoft, JP Morgan, Bank of America, Barclays, Goldman Sachs Credit Suisse \\ 
        \hline
      \end{tabular}
    \end{center}
    \caption{Vergleich von MQTT und AMQP \cite{Naik2017}}
    \label{tab:mqtt-vs-amqp}
  \end{table}

\chapter{}
\label{appendix:comparison}
\begin{table}[hbt!]
  \begin{center}
      \begin{tabular}{| p{3.0cm} | p{6.2cm} | p{6.2cm} | }
          \hline
            \textbf{Attribut} & \textbf{openHAB} & \textbf{Home Assistant} \\
          \hline
            Architektur & Robustheit und Starrheit, sowie die gewissenhafte Entwicklung & Erfordert häufigere Updates, bietet jedoch eine schnelle Entwicklung und eine viel modernere und anspruchsvollere Architektur \cite{sh-uni-comparison}. \\ 
          \hline
            Installation und Wartung & Installation ist genauestens dokumentiert. Initiale Konfiguration muss über die Kommandozeile durchgeführt werden, Updates ebenso. & Ein-Klick-Installationsprozess. Initiale Konfiguration benötigt das Verständnis von YAML-Dateien. Updates können über die UI verwaltet werden \cite{sh-uni-comparison}. \\ 
          \hline
            Unterstützte Geräte und Verknüpfungen & Durch die gängigen Smart Home Protokolle werden über 1000 Komponenten unterstützt. & Home Assistant unterstützt ebenso über 1000 Komponenten, wobei die Benutzerfreundlichkeit beim Anlegen und Verwalten bei Home Assistant deutlich angenehmer und leichter erscheint \cite{msuttner-comparison}. \\ 
          \hline
            Automatisierungsregeln & Bereitstellung von Automationen durch Xtend. Xtend ist ein flexibler und ausdrucksstarker Java-Dialekt, der sich in Quellcode basierend auf Java 8 kompilieren lässt. Ebenso können über Blockly Automatisierungsregeln erstellt werden. Blocky ist eine clientseitige JavaScript Bibliothek, die visuelle Blöcke mit Anweisungen, Bedingungen uvm. erstellen und editieren lässt. Eine weitere Alternative ist die Verwendung von Node-RED.  & Home Assistant bietet die Möglichkeit, Automatisierungen per YAML zu erstellen. Alternativ mit Node-RED. YAML-Dateien sind flexibler, jedoch nicht unbedingt benutzerfreundlich. \\ 
          \hline
            User Interface & openHAb hat in den User Interface viele Duplikate und zu viele ähnliche Optionen zur Verfügung bei unterschiedlicher Oberfläche. & Die Benutzeroberfläche der Home Assistant Plattform ist für Einsteiger und Anfänger deutlich benutzerfreundlicher und weniger komplex als openHAB. \\
          \hline
            Mobile Anwendungen & openHAB bietet eine dedizierte App für iOS und Android. Diese ist mit innovativen Optionen ausgestattet.  & Home Assistant bietet ebenso eine App für iOS und Android, jedoch bei weitem nicht so stark entwickelt als die von openHAB. Die Benachrichtigungsdienste sind jedoch mit Home Assistant besser genutzt. \\ 
          \hline
            Community und Dokumentation & openHAB bietet eine sehr starke und repräsentative Community. Die Dokumentation ist ausführlich und beschreibt die Prozesse, Konzept und Architektur ausführlich. & Home Assistant unterscheidet sich in diesen Punkten nicht von openHAB. Eine starke Community und eine solide Dokumentation der Plattform.  \\
          \hline
            Vergleich von Xtend \& YAML &  &  \\ 
          \hline 
      \end{tabular}
  \end{center}
  \caption{Vergleich der Plattformen \cite{sh-uni-comparison} \cite{msuttner-comparison} \cite{barclay-comparison}}
  \label{tab:comparisonTableHAOS-openHAB}
\end{table}

\chapter{}
\label{appendix:brandings}
\begin{figure}[hbt!]
  %\centering
  \includepdf[scale=0.60, pages=1, pagecommand={}]{chapter/9Anhang/beliebteste-smart-home-marken-in-deutschland-2022.pdf} %\includegraphics[width=13cm,height=13cm,keepaspectratio]{images/smart_home_connection.png}
  %\caption{Umfrage \url{https://de.statista.com/prognosen/999896/deutschland-beliebteste-smart-home-marken} Abgerufen am 22.05.2022}
  %\label{appendix:brandings}
\end{figure}

\chapter{}
\label{appendix:persona}
\pagebreak
%Personas, die im Rahmen der Zielgruppenanalyse erstellt wurden:
\begin{figure}[hbt!]
  \includepdf[scale=1.00, pages=1, pagecommand={}]{chapter/9Anhang/Persona-1_MA.pdf}
  \caption{Persona 1 zu der Zielgruppenanalyse}
  \label{appendix:persona1}
\end{figure}
\pagebreak
\begin{figure}[hbt!]
  \includepdf[scale=1.00, pages=1, pagecommand={}]{chapter/9Anhang/Persona-2_MA.pdf}
  \caption{Persona 2 zu der Zielgruppenanalyse}
  \label{appendix:persona2}
\end{figure}
\pagebreak
\begin{figure}[hbt!]
  \includepdf[scale=1.00, pages=1, pagecommand={}]{chapter/9Anhang/Persona-3_MA.pdf}
  \caption{Persona 3 zu der Zielgruppenanalyse}
  \label{appendix:persona3}
\end{figure}

\chapter{}
\label{appendix:user-story-uc1}
\section*{Use Case 1 - Check-in mit einem Service-Roboter}
\subsection*{Aufgabenbeschreibung UC 1}
\begin{table}[hbt!]
    \begin{center}
        \begin{tabular}{| p{3cm} | p{12.75cm} | }
            \hline
                \textbf{Task:} & \textbf{Begrüßung / Check-In von Member und Gästen} \\
            \hline
                \textbf{Ziel:} & Einen Member (Mitarbeiter) oder einen Gast in der Lokation mit einem Service-Roboter zu begrüßen, wenn dieser eintreten möchte. Nach der Begrüßung findet der Check-In statt, bei dem der Member seinem Platz zugewiesen und der Gast zu seinem Ansprechpartner geleitet wird. \\
            \hline
                \textbf{Eingriffs-möglichkeiten:} & Auswahl der Situation, Interaktion mit dem Service-Roboter (Informieren, Begleiten). \\
            \hline
                \textbf{Ursachen:} & Terminbuchung, Anmeldung, Klingeln an der Tür, Authentifizierung / Authentisierung an der Tür mittels Biometrische Eingangskontrolle. \\
            \hline
                \textbf{Priorität:} & Hoch \\
            \hline
                \textbf{Durchführungs-profil:} & Nach Anmeldung oder Eintritt eines Gastes oder Members, Unterbrechungen möglich (Verbindungsfehler, Fehlerhafte Abfrage der Verfügbarkeiten von Platzreservierungen), mittlere Komplexität. \\ 
            \hline
                \textbf{Vorbedingung:} & Platzbuchung / Reservierung und Einbuchung für ein Meeting fand im Vorfeld statt, Einverständniserklärung von den Personen liegt vor, Authentisierung über die Eingangskontrolle hat funktioniert / stattgefunden. \\
            \hline 
                \textbf{Info-In:} & Die Person, die eintreten will, ist durch die Buchung oder Authentisierung bekannt, Buchung ist registriert und kann abgerufen werden. \\
            \hline
                \textbf{Info-Out:} & Begrüßung Standardprotokoll mit Interaktionsmöglichkeiten und Kenntnis des Namens der Person, Check-In Informationen für die einzucheckende Person. \\
            \hline
                \textbf{Ressourcen:} & Service-Roboter, Eintrittskontrolle per Biometrischer Authentisierung, Buchung der Person abrufbar über die API. \\
            \hline
        \end{tabular}
    \end{center}
    \caption{Aufgabenbeschreibung UC 1 - Begrüßung / Check-in}
    \label{tab:checkIn}
\end{table}

%\subsection*{UC 1 - Beschreibung}
%    Steuerzentrale:
%    \begin{itemize}
%        \item Input: MQTT Nachricht von UC 1 (Verifizierung ok?, Name des Gastes / Members und Prozessanstoß (Check In))
%        \item Input: Terminregistrierung durch DoorJames (Übertragung der Termindaten) 
%        \item Output: Zu durchlaufender Prozess (CheckIn) und Name des Gastes / Members
%        \subItem (CheckIn) Genauer Begrüssungsprozess? Aktiv / Passiv 
%        \subsubItem Zuständigkeit Service-Roboter / Member / Gast / Steuerzentrale (Zentral / dezentral)
%        \subsubItem Begrüßungsprozess durch Service-Roboter selbst (Möglichkeiten sind im Service-Roboter implementiert) Steuerzentrale dient nur als Event-Auslöser 
%    \end{itemize}
	
%    Service-Roboter: 
%    \begin{itemize}
%        \item Input: MQTT (Prozess-Trigger durch Steuerzentrale und Name des Gastes / Members)
%        \item Input: Mitarbeiter lädt Kunde ein / Prepare / Nachweis der Platzbuchungssoftware
%        \item Output: (An Steuerzentrale) – Prozess erfolgreich abgeschlossen
%        \subItem Event-Möglichkeiten werden von Service-Roboter direkt verarbeitet. (Interaktion mit Gast, Ausrichten mit Hilfe Objekterkennung zur Person) 
%        \subItem Event-Möglichkeiten: 
%        \subsubItem Kunde: 
%        \subsubsubItem Begrüßen von Kunde und einchecken mit QR / Wenn eingeladen, Service-Roboter wartet schon vorher 5-15min vor der Tür
%        \subsubsubItem Zum Einladenden führen oder einen Call starten
%        \subsubItem Mitarbeiter: 
%        \subsubsubItem Anbieten eines Kaffees (zum Arbeitsplatz?) / oder zur Kaffee Maschine führen
%    \end{itemize}

\subsection*{UC 1 - User Story}
\textbf{Gast:}
\\
Als Gast möchte ich für den Zugang in das Gebäude ein Bild für den Anmeldeprozess hinterlegen können. Vor Ort möchte ich über die Kamera 
am Eingang mein Gesicht Authentisieren, bei nicht Hinterlegung eines Bildes bei der Anmeldung, möchte ich klingeln. Nach der 
Authentisierung bzw. nach Betätigung der Klingel soll sich die Tür öffnen. Beim Hineintreten in das Gebäude nehme ich einen Service-Roboter 
(Service-Roboter) wahr. Dieser begrüßt mich und startet eine Konversation mit mir. Er bietet mir an die Registrierung bzw. das Antreffen im Gebäude 
per QR Code über den Service-Roboter-Bildschirm durchzuführen. Somit werde ich als anwesend markiert und der Mitarbeiter, mit dem ich einen Termin 
habe, wird informiert. Nach erfolgreicher Registrierung führt mich der Service-Roboter zum Mitarbeiter mit dem ich einen Termin vereinbart habe.
\\
\linebreak
\textbf{Member:}\\

Als Member möchte ich die Buchung eines Arbeitsplatzes wahrnehmen können. Um in das Gebäude eintreten zu können, ist eine Authentifizierung durch 
die Gesichtserkennung oder das Abscannen der Chipkarte notwendig. Nach erfolgreicher Autorisierung werde ich von Service-Roboter bei Eintritt des Gebäudes 
begrüßt. Service-Roboter gibt mir den Arbeitsplatz bekannt und fragt, ob ich zum Platz begleitet werden soll. 
(oder Service-Roboter mir einen Kaffee an den Arbeitsplatz bringt.) -> (Erweiterung des Use Cases in Planung)
%Kunden zum Ausgangführen/Checkout anderer use case

\subsubsection*{Akzeptanzkriterien}
\begin{itemize}
    \item AK-01: Eine MQTT-Message mit dem Namen der Authentisierten Person wird empfangen.
    \item AK-02: Die MQTT-Message wird gelesen und mittels den Informationen können Abfragen an die DoorJames API erfolgen.
    \item AK-03: Informationen der DoorJames API werden empfangen und können weiterverarbeitet werden. 
    \item AK-04: Die benötigten Informationen können an den Service-Roboter (Temi) weitergeleitet werden.
    \item Ak-05: Eine MQTT-Message wird empfangen, sobald der Service-Roboter alle Tasks (ToDo's)  des Prozesses abgearbeitet hat. 
\end{itemize}

\begin{landscape}
  \AddToShipoutPictureBG*{%
  \AtPageLowerLeft{%
    \raisebox{\dimexpr.5\paperheight-.5\height}{%
      \makebox[.9\paperwidth][r]{%
        \rotatebox{90}
        {\thepage}
      }% \makebox
    }% \raisebox
  }% \AtPageCenter
}% \AddToShipoutPictureBG*
  \includepdf[scale=0.75, pages=-, angle=90, fitpaper=true, ]{images/UC1_Ablaufdiagramm_Sichtweisen.pdf}
\end{landscape}

\chapter{}
\label{appendix:user-story-uc2}
\section*{Use Case 2 - Notfall-Evakuierung}
\begin{table}[hbt!]
    \begin{center}
        \begin{tabular}{| p{3cm} | p{12.75cm} | }
            \hline
                \textbf{Task:} & \textbf{Begrüßung / Check-In von Member und Gästen} \\
            \hline
                \textbf{Ziel:} & Einen Notfall zu erkennen und darauf hin alle Member und Gäste, die sich im Büro befinden, informieren und bitten das Gebäude zu verlassen. \\
            \hline
                \textbf{Eingriffs-möglichkeiten:} & Keine, (Evtl. direkte Interaktion mit dem Service-Roboter). \\
            \hline
                \textbf{Ursachen:} & Aktivierung des Sensors durch äußerliche Einwirkung. \\
            \hline
                \textbf{Priorität:} & Hoch \\
            \hline
                \textbf{Durchführungs-profil:} & nach Meldung eines Notfalls durch Sensoren und Melder \\ 
            \hline
                \textbf{Vorbedingung:} & Ein Melder kann Nachrichten veröffentlichen, Topics für die Melder und Empfänger sind definiert. \\
            \hline 
                \textbf{Info-In:} & Meldung eines Sensors (Melder) über einen Notfall (Rauch, Feuer, Wasser, Gas). \\
            \hline
                \textbf{Info-Out:} & Kenntnis der gebuchten Plätze, Interaktionsmöglichkeiten mit dem Service-Roboter, Benachrichtigung bei Beendung des Prozesses. \\
            \hline
                \textbf{Ressourcen:} & Service-Roboter, Sensor (Rauch-, Feuer- oder Gas-Melder), Information über aktuelle Buchungen, Zuordnung von Positionen von Büroplätzen. \\
            \hline
        \end{tabular}
    \end{center}
    \caption{Aufgabenbeschreibung UC 2 - Notfall-Evakuierung}
    \label{tab:evacuation}
\end{table}

%\subsection*{UC 2 - Beschreibung}
%%    Steuerzentrale:
%    \begin{itemize}
%        \item Input: MQTT Nachricht angestoßen von einem Member (Aktion; Uhrzeit)
%        \item Input: DoorJames API - Gebuchte Büroplätze und Personen die einen Termin vor Ort registriert haben
%        \item -	Output: Zu durchlaufender Prozess (Evakuierung) und Name des Gastes / Members und Position der Plätze und Räume, die gebucht wurden. 
%    \end{itemize}
	
%    Service-Roboter: 
%    \begin{itemize}
%        \item Input: MQTT (Prozess-Trigger durch Steuerzentrale und Position gebuchter Plätze und Räume
%        \item Input: Objekterkennung (Personenerkennung) durch die Kamera
%        \item Output: (An Steuerzentrale) – Status / Prozess (erfolgreich/nicht erfolgreich) abgeschlossen
%        \item -> Event-Möglichkeiten werden von Service-Roboter direkt verarbeitet. (Interaktion mit Person, Anweisungen an die Person) 
%        \item -> Event-Möglichkeiten: 
%        \item --> Kunde, Gast & Member: 
%        \item ---> Alarmieren und bitten das Gebäude auf schnellstem Weg zu
%    \end{itemize}

\subsection*{UC 2 - User Story}
\textbf{Gast:}
\\
    Als Gast, Member und Manager möchte ich bei einem Notfall alarmiert werden. Bei eintretendem Fall möchte ich 
    Anweisungen, die zu beachten und umzusetzen sind, erhalten. Ebenso ist ein Hinweis zu den Notausgängen 
    sinnvoll und hilft bei der Flucht, bzw. beim Verlassen des Gebäudes. Damit keine Personen im Büro 
    bleiben, ist eine Information über Anwesende von belangen. Der Member sollte die Möglichkeit haben, den 
    Service-Roboter aufzufordern nach Leuten zu rufen, die nicht in Sichtweite sind. Sobald alle Plätze 
    abgefahren sind, bzw. durch den Service-Roboter keine Personen mehr erkannt werden, soll eine Benachrichtigung 
    erfolgen, dass alle aus dem Gebäude sind, bzw. keine mehr gesichtet wurden. Diese Information hilft bei der 
    anschließenden Kontrolle über am Treffpunkt anwesender Personen. Abschließend soll ggf. Eine Information über die 
    Platzbuchungen gezeigt werden, damit bei der Kontrolle keine Fehler unterlaufen und keine Personen, die nicht 
    gesehen werden, vergessen werden.

\subsection*{Akzeptanzkriterien}
\begin{itemize}
    \item AK-01: Eine MQTT-Message des Rauchmelders (Sensors) wird von der Steuerzentrale empfangen.
    \item AK-02: Die MQTT-Message wird gelesen und Abfragen an die DoorJames API erfolgen (Abfrage der Personen zum Zeitpunk im Bürogebäude)
    \item AK-03: Informationen der DoorJames API werden empfangen und können weiterverarbeitet werden. 
    \item AK-04: Die benötigten Informationen können an den Service-Roboter (Temi) weitergeleitet werden.
    \item AK-05: Temi wird losgeschickt, um nach Personen im Gebäude zu suchen.
    \item AK-06: Die Steuerzenrale schickt eine SMS, E-Mail an die Gebäudeverantwortliche Person, dass ein Notfall losgetreten wurde.
    \item Ak-07: Eine MQTT-Message wird empfangen, sobald der Service-Roboter alle Tasks (ToDo's)  des Prozesses abgearbeitet hat. 
\end{itemize}
\begin{landscape}
  \AddToShipoutPictureBG*{%
  \AtPageLowerLeft{%
    \raisebox{\dimexpr.5\paperheight-.5\height}{%
      \makebox[.9\paperwidth][r]{%
        \rotatebox{90}
        {\thepage}
      }% \makebox
    }% \raisebox
  }% \AtPageCenter
}% \AddToShipoutPictureBG*
  \includepdf[scale=0.75, pages=-, angle=90, fitpaper=true, ]{images/UC2_Ablaufdiagramm_Sichtweisen.pdf}
\end{landscape}

%%%%%%%%%%%%%%%%%%%%%%%%%%%%%%%%%%%%%%%%%%%%%%%%%%%%%%%%%%%%%%%%%%%%%%%%%%%%%%%

\newpage
\thispagestyle{empty}
\begin{framed}
\begin{center}
\Large\bfseries Erklärung
\end{center}
\medskip
\noindent
Ich versichere, dass ich diese \Was ~mit dem Thema:
\textit{\enquote{\Titel}}
selbstständig verfasst, keine anderen als die angegebenen Quellen und Hilfsmittel benutzt sowie alle wörtlichen oder sinngemäß übernommenen Stellen in der Arbeit gekennzeichnet habe. 
Die Arbeit wurde noch keiner Kommission zur Prüfung vorgelegt und verletzt in keiner Weise Rechte Dritter.
\vspace{3cm}
\noindent
\underline{\hspace{0cm}}\hfill\underline{\hspace{15cm}}\\
Ort, Datum\hfill Unterschrift\hspace{4cm}
\end{framed}

%%%%%%%%%%%%%%%%%%%%%%%%%%%%%%%%%%%%%%%%%%%%%%%%%%%%%%%%%%%%%%%%%%%%%%%%%%%%%%%
\endinput
%%%%%%%%%%%%%%%%%%%%%%%%%%%%%%%%%%%%%%%%%%%%%%%%%%%%%%%%%%%%%%%%%%%%%%%%%%%%%%%


%%%%%%%%%%%%%%%%%%%%%%%%%%%%%%%%%%%%%%%%%%%%%%%%%%%%%%%%%%%%%%%%%%%%%%%%%%%%%%%

\end{document}
